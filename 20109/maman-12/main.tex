\documentclass[11pt, oneside]{article}
\usepackage{geometry}
\geometry{a4paper}
\usepackage[parfill]{parskip}
\usepackage[nodisplayskipstretch]{setspace}
\usepackage{graphicx,titlesec}
\usepackage{amsmath,amssymb,cancel}

% Hebrew Stuff
\usepackage[utf8x]{inputenc}
\usepackage[english,hebrew]{babel}
\usepackage{hebfont}

% Custom commands
\newcommand{\qed}{\R{$\blacksquare$}}
\newcommand{\br}{\\\\\\\\\\\\\\}
\newcommand{\opr}[1]{\xrightarrow[\text{\R{#1}}]{}}
\newcommand{\bidiarrow}[1]{\underset{\text{\R{#1}}}{\leftrightarrow}}
\newcommand{\ueq}[1]{\underset{\text{\R{#1}}}{=}}

% Custom text commands (for Hebrew)
\newcommand{\q}[3]{שאלה #3#2.#1}
\newcommand{\m}[3]{משפט #3#2.#1}
\newcommand{\h}[3]{הגדרה #3#2.#1}
\newcommand{\ms}[3]{מסקנה #3#2.#1}
\newcommand{\mR}{\mathbb{R}}
\newcommand{\mN}{\mathbb{N}}
\newcommand{\mZ}{\mathbb{Z}}
\newcommand{\inv}[1]{#1^{-1}}

% Custom commands for this document
\newcommand{\rvec}[3]{\begin{bmatrix}#1 & #2 & #3\end{bmatrix}}
\newcommand{\cvec}[3]{\begin{bmatrix}#1 \br #2 \br #3\end{bmatrix}}

% Spacing
\titlespacing\subsubsection{0pt}{5pt}{3pt}
\setstretch{0.1}

\title{ממן 21}
\author{יונתן אוחיון}

\begin{document}
\maketitle

%%%%% <Q1> %%%%%
\section{שאלה 1}
\setcounter{subsection}{-1}
\subsection{טענת עזר}
ראשית, נוכיח טענת עזר $A^kB = B^kA$ לכל $AB = BA$ באינדוקציה:
\subsubsection{$n = 1$}
$$A^1B = BA^1 \opr{} AB = BA$$
נכון לפי הנתון.
\subsubsection{נניח שנכון ל$n = k$ ונוכיח ל$n = k + 1$}
$$A^{k + 1}B \ueq{\h{3}{6}{5.}} A^kAB \ueq{הנתון} A^kBA \ueq{הנחת האינדוקציה} BA^kA \ueq{\h{3}{6}{5.}} BA^{k + 1}$$

עכשיו נוכל להשתמש בטענת העזר על מנת להוכיח את הטענה בשאלה.
\subsection{הוכחה באינדוקציה}
\subsubsection{$n = 1$}
$$(AB)^1 = A^1B^1 \opr{} AB = AB$$

\subsubsection{נניח שנכון ל$n = k$ ונוכיח ל$n = k + 1$}
$$(AB)^{k + 1} \ueq{\h{3}{6}{5.}} AB(AB)^k \ueq{הנחת האינדוקציה} ABA^kB^k \ueq{טענת העזר} AA^kB^kB \ueq{\h{3}{6}{5.}} A^{k + 1}B^{k + 1}$$

לכן, לכל $AB = BA$ מתקיים $(AB)^k = A^kB^k$.
\br\qed
\clearpage
%%%%% </Q1> %%%%%

%%%%% <Q2> %%%%%
\section{שאלה 2}
נניח ש$A$ הפיכה ומתקיים $A = \inv{A}$. לפיכך, $A^2 = I$, כלומר נצטרך למצוא את ערכי $k$  שבהם $A^2 = I$. ראשית, נכפיל את המטריצה $A$ בעצמה:
\begin{align*}
& [A^2]_{11} = \rvec{k}{0}{1}\cvec{k}{0}{-1} = k^2 - 1\\
& [A^2]_{12} = \rvec{k}{0}{1}\cvec{0}{1}{0} = 0\\
& [A^2]_{13} = \rvec{k}{0}{1}\cvec{1}{0}{-k} = 0\\
& [A^2]_{21} = \rvec{0}{1}{0}\cvec{k}{0}{-1} = 0\\
& [A^2]_{22} = \rvec{0}{1}{0}\cvec{0}{1}{0} = 1\\
& [A^2]_{23} = \rvec{0}{1}{0}\cvec{1}{0}{-k} = 0\\
& [A^2]_{31} = \rvec{-1}{0}{-k}\cvec{k}{0}{-1} = 0\\
& [A^2]_{32} = \rvec{-1}{0}{-k}\cvec{0}{1}{0} = 0\\
& [A^2]_{33} = \rvec{-1}{0}{-k}\cvec{1}{0}{-k} = k^2 - 1\\
& A^2 = \begin{bmatrix}
k^2 - 1 & 0 & 0\br
0 & 1 & 0\br
0 & 0 & k^2 - 1
\end{bmatrix}
\end{align*}
כלומר, נצטרך למצוא את ערכי ה$k$ הפותרים את המשוואה $k^2 - 1 = 1 \opr{} k^2 = 2$. מכיוון שהמטריצה מעל השדה הסופי $\mZ_7$ נצטרך לבדוק את הריבוע של כל איבר בשדה. לאחר הבדיקה נמצא כי $k \in \{3, 4\}$. לפיכך, כאשר $k \in \{3, 4\}$, המטריצה $A$ הפיכה ומתקיים $\inv{A} = A$.
\br\qed
\clearpage
%%%%% </Q2> %%%%%

%%%%% <Q3> %%%%%
\section{שאלה 3}
\subsection{סעיף א}
הוכחה:
\begin{align*}
& A^2 + AB + I = 0\\
& AB = -A^2 - I\\
& \inv{A} \cdot/ \cancel{A}B = \cancel{A}(-A - \inv{A})\\
& B = -A - \inv{A}\\
& B = (-A^2 - I)\cancel{\inv{A}}\;/\cdot A\\
& BA = -A^2 - I\\
& A^2 + BA + I = 0
\end{align*}
כעת, מכיוון ששתי המשוואות שוות ל$0$ נוכל להשוות אותם ולהגיע ל$AB = BA$:
$$\cancel{A^2} + BA + \cancel{I} = \cancel{A^2} + AB + \cancel{I} \opr{} BA = AB$$
\qed
\clearpage
%%%%% </Q3> %%%%%

\end{document}