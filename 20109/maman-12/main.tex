\documentclass[11pt, oneside]{article}
\usepackage{geometry}
\geometry{a4paper}
\usepackage[parfill]{parskip}
\usepackage[nodisplayskipstretch]{setspace}
\usepackage{graphicx,titlesec}
\usepackage{amsmath,amssymb,cancel}

% Hebrew Stuff
\usepackage[utf8x]{inputenc}
\usepackage[english,hebrew]{babel}
\usepackage{hebfont}

% Custom commands
\newcommand{\qed}{\R{$\blacksquare$}}
\newcommand{\br}{\\\\\\\\\\\\\\}
\newcommand{\opr}[1]{\xrightarrow[\text{#1}]{}}
\newcommand{\bidiarrow}[1]{\underset{\text{#1}}{\leftrightarrow}}
\newcommand{\ueq}[1]{\underset{\text{#1}}{=}}

% Custom text commands (for Hebrew)
\newcommand{\q}[3]{\R{שאלה #3#2.#1}}
\newcommand{\m}[3]{\R{משפט #3#2.#1}}
\newcommand{\h}[3]{\R{הגדרה #3#2.#1}}
\newcommand{\ms}[3]{\R{מסקנה #3#2.#1}}
\newcommand{\mR}{\mathbb{R}}
\newcommand{\mN}{\mathbb{N}}
\newcommand{\mZ}{\mathbb{Z}}
\newcommand{\inv}[1]{#1^{-1}}

% Custom commands for this document
\newcommand{\rvec}[3]{\begin{bmatrix}#1 & #2 & #3\end{bmatrix}}
\newcommand{\cvec}[3]{\begin{bmatrix}#1 \br #2 \br #3\end{bmatrix}}

% Spacing
\titlespacing\subsubsection{0pt}{5pt}{3pt}
\setstretch{0.1}

\title{ממן 21}
\author{יונתן אוחיון}

\begin{document}
\maketitle

%%%%% <Q1> %%%%%
\section{שאלה 1}
\setcounter{subsection}{-1}
\subsection{טענת עזר}
ראשית, נוכיח טענת עזר $A^kB = B^kA$ לכל $AB = BA$ באינדוקציה:
\subsubsection{$n = 1$}
$$A^1B = BA^1 \opr{} AB = BA$$
נכון לפי הנתון.
\subsubsection{נניח שנכון ל$n = k$ ונוכיח ל$n = k + 1$}
$$A^{k + 1}B \ueq{\h{3}{6}{5.}} A^kAB \ueq{\R{הנתון}} A^kBA \ueq{\R{הנחת האינדוקציה}} BA^kA \ueq{\h{3}{6}{5.}} BA^{k + 1}$$

עכשיו נוכל להשתמש בטענת העזר על מנת להוכיח את הטענה בשאלה.
\subsection{הוכחה באינדוקציה}
\subsubsection{$n = 1$}
$$(AB)^1 = A^1B^1 \opr{} AB = AB$$

\subsubsection{נניח שנכון ל$n = k$ ונוכיח ל$n = k + 1$}
$$(AB)^{k + 1} \ueq{\h{3}{6}{5.}} AB(AB)^k \ueq{\R{הנחת האינדוקציה}} ABA^kB^k \ueq{\R{טענת העזר}} AA^kB^kB \ueq{\h{3}{6}{5.}} A^{k + 1}B^{k + 1}$$

לכן, לכל $AB = BA$ מתקיים $(AB)^k = A^kB^k$.
\br\qed
\clearpage
%%%%% </Q1> %%%%%

%%%%% <Q2> %%%%%
\section{שאלה 2}
נניח ש$A$ הפיכה ומתקיים $A = \inv{A}$. לפיכך, $A^2 = I$, כלומר נצטרך למצוא את ערכי $k$  שבהם $A^2 = I$. ראשית, נכפיל את המטריצה $A$ בעצמה:
\begin{align*}
& [A^2]_{11} = \rvec{k}{0}{1}\cvec{k}{0}{-1} = k^2 - 1\\
& [A^2]_{12} = \rvec{k}{0}{1}\cvec{0}{1}{0} = 0\\
& [A^2]_{13} = \rvec{k}{0}{1}\cvec{1}{0}{-k} = 0\\
& [A^2]_{21} = \rvec{0}{1}{0}\cvec{k}{0}{-1} = 0\\
& [A^2]_{22} = \rvec{0}{1}{0}\cvec{0}{1}{0} = 1\\
& [A^2]_{23} = \rvec{0}{1}{0}\cvec{1}{0}{-k} = 0\\
& [A^2]_{31} = \rvec{-1}{0}{-k}\cvec{k}{0}{-1} = 0\\
& [A^2]_{32} = \rvec{-1}{0}{-k}\cvec{0}{1}{0} = 0\\
& [A^2]_{33} = \rvec{-1}{0}{-k}\cvec{1}{0}{-k} = k^2 - 1\\
& A^2 = \begin{bmatrix}
k^2 - 1 & 0 & 0\br
0 & 1 & 0\br
0 & 0 & k^2 - 1
\end{bmatrix}
\end{align*}
כלומר, נצטרך למצוא את ערכי ה$k$ הפותרים את המשוואה $k^2 - 1 = 1 \opr{} k^2 = 2$. מכיוון שהמטריצה מעל השדה הסופי $\mZ_7$ נצטרך לבדוק את הריבוע של כל איבר בשדה. לאחר הבדיקה נמצא כי $k \in \{3, 4\}$. לפיכך, כאשר $k \in \{3, 4\}$, המטריצה $A$ הפיכה ומתקיים $\inv{A} = A$.
\br\qed
\clearpage
%%%%% </Q2> %%%%%

%%%%% <Q3> %%%%%
\section{שאלה 3}
\subsection{סעיף א}
הוכחה:
\begin{align*}
& A^2 + AB + I = 0\\
& AB = -A^2 - I\\
& \inv{A}\!\cdot\!/\cancel{A}B = \cancel{A}(-A - \inv{A})\\
& B = -A - \inv{A}\\
& B = (-A^2 - I)\cancel{\inv{A}}\;/\cdot A\\
& BA = -A^2 - I\\
& A^2 + BA + I = 0
\end{align*}
כעת, מכיוון ששתי המשוואות שוות ל$0$ נוכל להשוות אותם ולהגיע ל$AB = BA$:
$$\cancel{A^2} + BA + \cancel{I} = \cancel{A^2} + AB + \cancel{I} \opr{} BA = AB$$
\qed

\subsubsection{סעיף ב}
נניח ש$A, B$ רגולריות ונגיע לסתירה.
$$AB \ueq{\R{לפי הנתון}} -BA \opr{} |AB| = |-BA| \opr{\m{4}{5}{1.}} (|A||B| = \overbrace{-|B||A|}^{(*)} \bidiarrow{} |A| = 0 \lor |B| = 0)$$
מכיוון שאנו מניחים ששתי המטריצות רגולריות, $|A| \neq 0 \land |B| \neq 0$ והגענו לסתירה.
לפיכך, בהכרח לפחות אחת משתי המטריצות $A, B$ הינה סינגולרית.
יש לציין שב$(*)$ הסימן של הדטרמיננטה תמיד שווה למינוס, שכן הגורם המשותף $(-1)^n$ תמיד בחזקה אי זוגית )לפי הנתון בשאלה(.
\br\qed
\clearpage
%%%%% </Q3> %%%%%

%%%%% <Q4> %%%%%
\section{שאלה 4}
\L{TODO}
\clearpage
%%%%% </Q4> %%%%%

%%%%% <Q5> %%%%%
\section{שאלה 5}
\subsection{$\det{B}$}
ראשית, נפתח את הדטרמיננטה של $B$ בעזרת $R_3$:
\begin{align*}
|B| & = -a_{31}M_{31} + a_{32}M_{32} - a_{33}M_{33} + a_{34}M_{34}\\
& = -0M_{31} + 2M{32} - 0M_{33} + 0M_{34}\\
& = 0 + 2M_{32} + 0 + 0\\
& = 2M_{32}
\end{align*}
לפיכך, נצטרך רק לחשב את המינור $M_{32}$ ולהכפילו ב$2$ ע"מ לקבל את הדטרמיננטה של $B$.\br
כעת, נחשב את $M_{32}$:
\begin{align*}
M_{32} & = \begin{vmatrix}
1 & 4 & 1\br
2(a + 3) & 2(a - 1) & 2(b + 2)\br
5 - 3b & 2 - 3b & 8 - 3a
\end{vmatrix}
\ueq{$R_3 \opr{} R_3 - 2R_1$} \begin{vmatrix}
1 & 4 & 1\br
2(a + 3) & 2(a - 1) & 2(b + 2)\br
3 - 3b & -6 - 3b & 6 - 3a
\end{vmatrix}\br
& \ueq{\m{4}{3}{3.}} 2\begin{vmatrix}
1 & 4 & 1\br
a + 3 & a - 1 & b + 2\br
3 - 3b & -6 - 3b & 6 - 3a
\end{vmatrix}
= 2\begin{vmatrix}
1 & 4 & 1\br
a + 3 & a - 1 & b + 2\br
-3(b - 1) & -3(b + 2) & -3(a - 2)
\end{vmatrix}\br
& \ueq{\m{4}{3}{3.}} -6\begin{vmatrix}
1 & 4 & 1\br
a + 3 & a - 1 & b + 2\br
b - 1 & b + 2 & a - 2
\end{vmatrix}
\ueq{$R_1 \bidiarrow{} R_2$} 6\begin{vmatrix}
a + 3 & a - 1 & b + 2\br
1 & 4 & 1\br
b - 1 & b + 2 & a - 2
\end{vmatrix}\br
& \ueq{\m{4}{3}{1.}} 6\begin{vmatrix}
a + 3 & 1 & b - 1\br
a - 1 & 4 & b + 2\br
b + 2 & 1 & a - 2
\end{vmatrix}
\ueq{$|A| = \frac{1}{3}$} 6\cdot\frac{1}{3}
= 2
\end{align*}
לפיכך, $M_{32} = 2$ ו$\det{B} = -4$.
\subsection{$\det{\inv{2B}}$}
נחשב:
\begin{align*}
|\inv{2B}| \ueq{\m{4}{3}{3.}} 2^4|\inv{B}| = 16|\inv{B}| \ueq{$|\inv{A}| = \frac{1}{|A|}$} \frac{16}{|B|} = \frac{16}{-4} = -4
\end{align*}
לכן, $\det{\inv{2B}} = \det{B} = -4$.
\br\qed
%%%%% </Q5> %%%%%

\end{document}