\documentclass[11pt, oneside]{article}
\usepackage{geometry}
\geometry{a4paper}
\usepackage[parfill]{parskip}
\usepackage[nodisplayskipstretch]{setspace}
\usepackage{graphicx,titlesec}
\usepackage{amsmath,amssymb,cancel}

% Hebrew Stuff
\usepackage[utf8x]{inputenc}
\usepackage[english,hebrew]{babel}
\usepackage{hebfont}

% Custom commands
\newcommand{\qed}{\R{$\blacksquare$}}
\newcommand{\br}{\\\\\\\\\\\\}
\newcommand{\opr}[1]{\xrightarrow[\text{#1}]{}}
\newcommand{\oprm}[1]{\underset{\substack{#1}}{\longrightarrow}}
\newcommand{\bidiarrow}[1]{\underset{\text{#1}}{\leftrightarrow}}
\newcommand{\ueq}[1]{\underset{\text{#1}}{=}}
\newcommand{\mR}{\mathbb{R}}
\newcommand{\mN}{\mathbb{N}}
\newcommand{\mZ}{\mathbb{Z}}
\newcommand{\mQ}{\mathbb{Q}}
\newcommand{\mC}{\mathbb{C}}
\newcommand{\inv}[1]{#1^{-1}}

% Custom text commands (for Hebrew)
\newcommand{\q}[3]{\R{שאלה #3#2.#1}}
\newcommand{\m}[3]{\R{משפט #3#2.#1}}
\newcommand{\h}[3]{\R{הגדרה #3#2.#1}}
\newcommand{\ms}[3]{\R{מסקנה #3#2.#1}}

% Custom commands for this document
\makeatletter
\newcommand{\cis}[1]{\mathop{\operator@font cis}#1}
\newcommand{\Sp}[1]{\mathop{\operator@font Sp}#1}
\makeatother
\newcommand{\tot}[4]{\begin{bmatrix}#1 & #2\br#3 & #4\end{bmatrix}}
\newcommand{\fcol}[4]{\begin{bmatrix}#1 \br #2\br#3 \br #4\end{bmatrix}}
\newcommand{\poreset}[2]{\item{הקבוצה הפורשת של $#1$: $#2$}}

% Spacing
\titlespacing\subsubsection{0pt}{0pt}{0pt}
\setstretch{0.1}

\title{ממן 31}
\author{יונתן אוחיון}

\begin{document}
\maketitle

%%%%% <Q1> %%%%%
\section{שאלה 1}
ראשית, נפתח את הסוגריים ונגיע לערכו של $z^4$:
\begin{align*}
z^4 & = (1 + i)^6 - (1 - i)^6\\
& = ((1 + i)(1 + i)^2)^2 - ((1 - i)(1 - i)^2)^2\\
& = ((1 + i)(\cancel{1} + 2i - \cancel{1}))^2 - ((1 - i)(\cancel{1} - 2i - \cancel{1}))^2\\
& = (2i - 2)^2 - (2 - 2i)^2\\
& = \cancel{-4} - 8i + \cancel{4} - (\cancel{4} + 8i - \cancel{4})\\
& = -8i - 8i\\
& = -16i
\end{align*}
\def\AT{\frac{3\pi}{2}}
לפיכך, $z^4 = -16i$. כעת נסתכל על מיקום הנקודה $0 - 16i$ על מישור המספרים המרוכבים ונגלה שהיא נמצאת $-16$ יחידות על ציר המרוכבים ו$0$ יחידות על ציר הממשיים, כלומר ההצגה הקוטבית שלה הינה $16\cis{\AT}$ )שכן $\cis{\AT} = \cos{\AT}+i\sin{\AT} = 0 - i = -i$(. כעת, נוכל למצוא את השורשים של $z^4$ בעזרת הנוסחה בעמוד 78:
\begin{align*}
z & = \sqrt[4]{16}(\cis{\frac{\alpha + 2 \pi k}{4}})\\
& = 2\cis{\frac{\AT + 2 \pi k}{4}}\\
& = 2\cis{\frac{3 \pi + 2 \pi k}{8}}
\end{align*}
כעת, נציב $k \in \{0, 1, 2, 3\}$:
\begin{align*}
z_0 = 2\cis{\frac{3\pi}{8}} \quad&\quad z_1 = 2\cis{\frac{7\pi}{8}}\\\\
z_2 = -2\cis{\frac{3\pi}{8}} \quad&\quad z_3 = -2\cis{\frac{7\pi}{8}}
\end{align*}
ואלו הם ערכי $z$.
\br\qed
%%%%% </Q1> %%%%%
\clearpage

%%%%% <Q2> %%%%%
\section{שאלה 2}
\subsection{סעיף א}
\subsubsection{$K$}
כל איברי $K$ שייכים למרחב הלינארי $M^\mR_{2\times2}$ )לפי \q{7}{1}{3.}(. נוכיח ש$K$ הינו תת-מרחב לינארי של $M^\mR_{2\times2}$:
\begin{align*}
K & = \left\{\tot{a - 2c}{c + a}{b}{-c} \mid a,b,c \in \mR\right\}\\
& = \left\{\tot{a}{a}{0}{0} + \tot{-2c}{c}{0}{c} + \tot{0}{0}{b}{0} \mid a,b,c \in \mR \right\}\\
& = \left\{a\tot{1}{1}{0}{0} + c\tot{-2}{1}{0}{1} + b\tot{0}{0}{1}{0} \mid a,b,c \in \mR\right\}\\
& = \Sp\left\{\tot{1}{1}{0}{0}, \tot{-2}{1}{0}{1}, \tot{0}{0}{1}{0}\right\}
\end{align*}
לפיכך, מצאנו קבוצה פורשת ל$K$. לפי \m{7}{5}{1.}, מכיוון שהצלחנו למצוא ל$K$ קבוצה פורשת הוא תת-מרחב לינארי.
\subsubsection{$L$}
ראשית, נגיע לביטוי של $L$ בעזרת $x_2, x_3$:
\begin{eqnarray*}
& x_1 = 2x_1 - 4x_2 - 5 \opr{} -x_1 = -4x_2 - 5 \opr{} x_1 = 4x_2 + 5\\
& \downarrow\\
& L = \left\{(4x_2 + 5, x_2, x_3) \mid x_2,x_3 \in \mR \right\}
\end{eqnarray*}
נניח בשלילה ש$L$ מרחב לינארי. ננסה להוכיח סגירות של הפעולה $+_L$ )שהיא חיבור $n$-יות( ונגיע לסתירה:
\begin{align*}
(4t + 5, t, s) +_L (4x + 5, x, y)
& = (4t + 4x + 10, t + x, s + y)\\
& = (4(t + x) + 10, t + x, s + y)\\
\end{align*}
מכיוון ש$4(t + x) + 10$ אינו ביטוי מהצורה $4x + 5$, הפעולה $+_L$ אינה סגורה ביחס ל$L$ והוא אינו מרחב לינארי.
\subsubsection{$M$}
כל איברי $M$ שייכים למרחב הלינארי $\mR_4[x]$ )לפי \q{7}{1}{9.}(. נוכיח ש$M$ הינו תת-מרחב לינארי של $\mR_4[x]$.
\br
לפי סימון 4.7.6 ולפי הגדרת $M$, ניתן לרשום את $p(x)$ בצורה הבאה:
$$p(x) = a_0 + a_1x + a_2x^2 + a_3x^3$$
כעת, נציב $x = 0$ ונקבל $p(0) = a_0$. באופן דומה, נוכל להציב $x = 1$ ו$x = -1$ ולקבל את המערכת הלינארית הבאה )לפי הגדרת $M$(:
\begin{eqnarray*}
\begin{matrix}
\cancel{a_0} + a_1 + a_2 + a_3 = \cancel{a_0}\br
\cancel{a_0} - a_1 + a_2 - a_3 = \cancel{a_0}
\end{matrix}
\opr{}
\begin{matrix}
0a_0 + a_1 + a_2 + a_3 = 0\br
0a_0 - a_1 + a_2 - a_3 = 0
\end{matrix}
\end{eqnarray*}
\clearpage
כעת, נוכל לדרג אותה עד להגעה למטריצת מדרגות קנונית ולקבל את הצורה הכללית של איבר ב$M$:
\begin{align*}
& \begin{bmatrix}
0 & 1 & 1 & 1\br
0 & -1 & 1 & -1
\end{bmatrix}
\opr{$R_2 \opr{} R_2 + R_1$}
\begin{bmatrix}
0 & 1 & 1 & 1\br
0 & 0 & 2 & 0
\end{bmatrix}
\opr{$R_2 \opr{} \frac{1}{2}R_2$}\\
& \begin{bmatrix}
0 & 1 & 1 & 1\br
0 & 0 & 1 & 0
\end{bmatrix}
\opr{$R_1 \opr{} R_1 - R_2$}
\begin{bmatrix}
0 & 1 & 0 & 1\br
0 & 0 & 1 & 0
\end{bmatrix}
\opr{}
\begin{matrix}
& a_0 = 0\br
& a_1 = -a_3\br
& a_2 = t
\end{matrix}
\end{align*}
לכן, כל $p(x) \in M$ הינו מהצורה $0a_0 + a_1x + a_2x^2 - a_1x^3$. נוכל לסמן את $M$ כך כעת:
\begin{align*}
M
& = \left\{p(x) \in \mR_4[x] \mid p(x) = 0a_0 + a_1x + a_2x^2 - a_1x^3 \right\}\\
& = \left\{p(x) \in \mR_4[x] \mid p(x) = 0a_0 + a_1(x - x^3) + a_2x^2 \right\}\\
& = \Sp\left\{x - x^3, x^2\right\}
\end{align*}
לפיכך, מצאנו קבוצה פורשת ל$M$. לפי \m{7}{5}{1.}, מכיוון שהצלחנו למצוא ל$M$ קבוצה פורשת הוא תת-מרחב לינארי.
\br\qed

\subsection{סעיף ב}
לפי תוצאות סעיף א, נוכל לראות שמצאנו ש$K$ ו$M$ הינם מרחבים לינארים )ובפרט תת-מרחבים של $M^\mR_{2\times2}$ ושל $\mR_4[x]$, בהתאמה(, בעזרת מציאת קבוצות הפורשות אותם. לפיכך הקבוצות הפורשות הן:
\begin{itemize}
\poreset{K}{\left\{\tot{1}{1}{0}{0}, \tot{-2}{1}{0}{1}, \tot{0}{0}{1}{0}\right\}}
\poreset{M}{\left\{x - x^3, x^2\right\}}
\end{itemize}
\qed
\clearpage
%%%%% </Q2> %%%%%

%%%%% <Q3> %%%%%
\section{שאלה 3}
\subsection{סעיף א}
יהי $U$ תת מרחב לינארי של $\mZ^4_5$ מעל $\mZ_5$ הנפרש ע"י קבוצה $P$:
\begin{align*}
& P = \left\{(1, 2, 1, 2), (2, 3, 1, 4), (3, 1, 2, 1)\right\}\\
& U = \Sp{P}
\end{align*}
נרצה להוכיח ש$P$ הינה בסיס של $U$, כלומר פורשת את $U$ ובת"ל בו. מכיוון שנתון ש$P$ פורשת את $U$, נוכיח ש$P$ הינה בת"ל ב$U$:
\begin{align*}
& Px = 0 \opr{} \begin{bmatrix}
1 & 2 & 3 & | & 0\br
2 & 3 & 1 & | & 0\br
1 & 1 & 2 & | & 0\br
2 & 4 & 1 & | & 0
\end{bmatrix}
\oprm{%
R_2 \bidiarrow{} R_3\\
R_2 \opr{} R_2 - R_1\\
R_3 \opr{} R_3 - 2R_1\\
R_4 \opr{} R_4 - 2R_1
}
\begin{bmatrix}
1 & 2 & 3 & | & 0\br
0 & 4 & 4 & | & 0\br
0 & 4 & 0 & | & 0\br
0 & 0 & -5 & | & 0
\end{bmatrix}\\
& \oprm{%
R_2 \bidiarrow{} R_3\\
R_2 \opr{} \frac{1}{4} R_2\\
R_3 \opr{} \frac{1}{4} R_3
}
\begin{bmatrix}
1 & 2 & 3 & | & 0\br
0 & 1 & 0 & | & 0\br
0 & 1 & 1 & | & 0\br
0 & 0 & -5 & | & 0
\end{bmatrix}
\oprm{%
R_1 \opr{} R_1 - 2R_2\\
R_3 \opr{} R_3 - R_2\\
R_1 \opr{} R_1 - 3R_3\\
R_4 \opr{} R_4 + 5R_3
}
\begin{bmatrix}
1 & 0 & 0 & | & 0\br
0 & 1 & 0 & | & 0\br
0 & 0 & 1 & | & 0\br
0 & 0 & 0 & | & 0
\end{bmatrix}
\end{align*}
מכיוון שהפתרון למערכת הלינארית $Px = 0$ הינו הפתרון הטריוויאלי, $P$ הינה בת"ל. לפיכך, $P$ בסיס של $U$ ולפי \h{8}{3}{3.} מתקיים
$$\dim{U} = |P| = 3$$
כנדרש.
\br\qed

\subsection{סעיף ב}
\subsubsection{תת סעיף 1}
נמצא את הבסיס של $\Sp{A}$ מעל $\mC$:
\begin{align*}
\begin{bmatrix}
1 + i & 3 + i & 1 - i\br
1 - i & 1 & -1\br
1 + i & 4i & 0
\end{bmatrix}
\oprm{%
R_1 \opr{} iR_1\\
R_3 \opr{} iR_3\\
R_1 \opr{} R_1 + R_2\\
R_3 \opr{} R_3 + R_2
}
\begin{bmatrix}
0 & 3i & i\br
1 - i & 1 & -1\br
0 & -3 & -1
\end{bmatrix}
\oprm{%
R_1 \bidiarrow{} R_2\\
R_2 \opr{} iR_2\\
R_3 \opr{} R_3 - R_2\\
R_2 \opr{} -R_2
}
\begin{bmatrix}
1 - i & 1 & -1\br
0 & 3 & 1\br
0 & 0 & 0
\end{bmatrix}
\end{align*}
לפיכך, הבסיס של $\Sp{A}$ הינו $((1 - i, 1, -1), (0, 3, 1))$. לפי \h{8}{3}{3.} מתקיים
$$\dim{\Sp{A}} = |B_{\Sp{A}}| = 2$$
כנדרש.
\br\qed

\setcounter{subsubsection}{2}
\subsubsection{תת סעיף 2}
נבדוק את התלות הלינארית של $A$ מעל $\mR$:
\begin{align*}
& \begin{bmatrix}
1 + i & 1 - i & 1 + i\br
3 + i & 1 & 4i\br
1 - i & -1 & 0
\end{bmatrix}
\oprm{%
R_3 \opr{} iR_3\\
R_3 \opr{} R_3 - R_1\\
R_1 \opr{} \frac{(1 - i)}{2}R_1
}
\begin{bmatrix}
1 & -i & 1\br
3 + i & 1 & 4i\br
0 & 1 & 1 + i
\end{bmatrix}
\oprm{%
R_2 \opr{} R_2 - (3 + i)R_1\\
R_2 \opr{} \frac{1}{3}R_2\\
R_3 \opr{} R_3 - iR_2
}
\begin{bmatrix}
1 & -i & 1\br
0 & -i & i - 1\br
0 & 0 & 0
\end{bmatrix}\\
& \oprm{%
R_2 \opr{} -R_2\\
R_1 \opr{} R_1 + iR_2
}
\begin{bmatrix}
1 & 0 & i\br
0 & 1 & 1 + i\br
0 & 0 & 0
\end{bmatrix}
\opr{}
\begin{matrix}
\alpha = -it\\
\beta = -(1 + i)t\\
\gamma = t
\end{matrix}
\end{align*}
והפתרון הכללי:
$$(-it, -(1 + i)t, t),\ t \in \mR$$
לפיכך, לא קיים $0 \neq t \in \mR$ כך ש$\alpha, \beta, \gamma \in \mR$ והווקטורים בת"ל מעל $\mR$. מכיוון ש$A$ גם פורשת וגם בת"ל, היא הינה בסיס ולפי \h{8}{3}{3.} מתקיים
$$\dim{\Sp{A}} = |B_{\Sp{A}}| = 3$$
כנדרש.
\br\qed
\clearpage
%%%%% </Q3> %%%%%

%%%%% <Q4> %%%%%
\section{שאלה 4}
\subsection{סעיף א}
\setcounter{subsubsection}{-1}
\subsubsection{הגדרות}
יהי $E$ הבסיס הסטנדרטי של $M^\mR_{2x2}$:
$$E = \left\{
\tot{1}{0}{0}{0},
\tot{0}{1}{0}{0},
\tot{0}{0}{1}{0},
\tot{0}{0}{0}{1}
\right\}$$

\subsubsection{בסיס ל$U$}
תהי $K$ קבוצה פורשת של $U$:
$$K = \left\{
k_1 = \tot{1}{1}{0}{2},
k_2 = \tot{2}{1}{1}{1},
k_3 = \tot{1}{-1}{0}{2},
k_4 = \tot{1}{-3}{1}{-1}
\right\}$$
נעבור להצגת קואורדינטות בעזרת הבסיס הסטנדרטי $E$:
\begin{center}
\begin{tabular}{c c}
$[k_1]_E = \fcol{1}{1}{0}{2}$ & $[k_2]_E = \fcol{2}{1}{1}{1}$\br
$[k_3]_E = \fcol{1}{-1}{0}{2}$ & $[k_2]_E = \fcol{1}{-3}{1}{-1}$\br
\end{tabular}
\end{center}
נעבור להצגה מטריציונית ונדרג:
\begin{align*}
& \begin{bmatrix}
1 & 1 & 0 & 2\br
2 & 1 & 1 & 1\br
1 & -1 & 0 & 2\br
1 & -3 & 1 & -1
\end{bmatrix}
\oprm{%
R_2 \opr{} R_2 - 2R_1\\
R_3 \opr{} R_3 - R_1\\
R_4 \opr{} R_4 - R_1
}
\begin{bmatrix}
1 & 1 & 0 & 2\br
0 & -1 & 1 & -3\br
0 & -2 & 0 & 0\br
0 & -4 & 1 & -3
\end{bmatrix}
\oprm{%
R_2 \opr{} -R_2\\
R_3 \opr{} R_3 + 2R_2\\
R_4 \opr{} R_4 + 4R_2
}
\begin{bmatrix}
1 & 1 & 0 & 2\br
0 & 1 & -1 & 3\br
0 & 0 & -2 & 6\br
0 & 0 & -3 & 9
\end{bmatrix}\\
& \oprm{%
R_3 \opr{} \frac{1}{2} R_3\\
R_4 \opr{} \frac{1}{4} R_4\\
R_4 \opr{} R_4 - R_3
}
\begin{bmatrix}
1 & 1 & 0 & 2\br
0 & 1 & -1 & 3\br
0 & 0 & -1 & 3\br
0 & 0 & 0 & 0
\end{bmatrix}
\oprm{%
R_3 \opr{} -R_3\\
R_2 \opr{} R_2 + R_3\\
R_1 \opr{} R_1 - R_2
}
\begin{bmatrix}
1 & 0 & 0 & 2\br
0 & 1 & 0 & 0\br
0 & 0 & 1 & -3\br
0 & 0 & 0 & 0
\end{bmatrix}
\end{align*}
נבחר את שורות המטריצה המדורגת השונות מ$0$ ונקבל את הבסיס ל$U$:
$$B_U = \left(\tot{1}{0}{0}{2},\tot{0}{1}{0}{0},\tot{0}{0}{1}{-3}\right)$$
\qed
\clearpage

\subsubsection{בסיס ל$W$}
תהי $G$ קבוצה פורשת של $W$:
$$G = \left\{
g_1 = \tot{4}{2}{1}{0},
g_2 = \tot{2}{1}{0}{-1}
\right\}$$
לפי \q{8}{2}{5.}, נוכל להכפיל את איברי הקבוצה הפורשת במשתנים $\alpha, \beta \in \mR$, לייצג כמערכת משוואות ולדרג:
\begin{eqnarray*}
& \alpha\tot{4}{2}{1}{0} + \beta\tot{2}{1}{0}{-1} = \tot{0}{0}{0}{0} \opr{}
\tot{4\alpha + 2\beta}{2\alpha + 1\beta}{1\alpha+ 0\beta}{0\alpha + (-1)\beta} = \tot{0}{0}{0}{0}\\
& \downarrow\\
& \begin{matrix}
4\alpha + 2\beta = 0\br
2\alpha + 1\beta = 0\br
(*)\ 1\alpha+ 0\beta = 0\br
(*)\ 0\alpha + (-1)\beta = 0
\end{matrix} \opr{}
\begin{matrix}
\alpha = 0\br
-\beta = 0
\end{matrix} \opr{}
\alpha = \beta = 0
\end{eqnarray*}
לפיכך, הפתרון היחידי למערכת המשוואות הינו הפתרון הטריוויאלי והווקטורים בת"ל. מכיוון ש$G$ פורשת את $W$ ובת"ל, היא גם בסיס שלו, כלומר
$$B_W = G = \left(\tot{4}{2}{1}{0}, \tot{2}{1}{0}{-1}\right)$$
\qed

\subsubsection{בסיס ל$U + W$}
לפי \q{7}{6}{5.}, מתקיים $U + W = \Sp B_U \cup B_W$, כלומר הקבוצה
$$Y = \left\{
y_1 = \tot{1}{0}{0}{2},
y_2 = \tot{0}{1}{0}{0},
y_3 = \tot{0}{0}{1}{-3},
y_4 = \tot{4}{2}{1}{0},
y_5 = \tot{2}{1}{0}{-1}
\right\}$$
פורשת את $U + W$. נעבור להצגת קואורדינטות בעזרת הבסיס הסטנדרטי $E$:
\begin{center}
\begin{tabular}{c c c}
$[y_1]_E = \fcol{1}{0}{0}{2}$ & $[y_2]_E = \fcol{0}{1}{0}{0}$ & $[y_3]_E = \fcol{0}{0}{1}{-3}$\br
$[y_4]_E = \fcol{4}{2}{1}{0}$ & & $[y_5]_E = \fcol{2}{1}{0}{-1}$
\end{tabular}
\end{center}
בעמוד הבא נעבור להצגה מטריציונית ונדרג.
\clearpage
\setcounter{subsubsection}{2}
\subsubsection{בסיס ל$U + W$ )המשך(}
נעבור להצגה מטריציונית ונדרג:
\begin{align*}
& \begin{bmatrix}
1 & 0 & 0 & 2\br
0 & 1 & 0 & 0\br
0 & 0 & 1 & -3\br
4 & 2 & 1 & 0\br
2 & 1 & 0 & -1
\end{bmatrix}
\oprm{%
R_3 \opr{} -R_3\\
R_4 \opr{} R_4 - 4R_1\\
R_5 \opr{} R_5 - 2R_1
}
\begin{bmatrix}
1 & 0 & 0 & 2\br
0 & 1 & 0 & 0\br
0 & 0 & -1 & 3\br
0 & 2 & 1 & -8\br
0 & 1 & 0 & -5
\end{bmatrix}
\oprm{%
R_4 \opr{} R_4 - 2R_2\\
R_5 \opr{} R_5 - R_2
}
\begin{bmatrix}
1 & 0 & 0 & 2\br
0 & 1 & 0 & 0\br
0 & 0 & -1 & 3\br
0 & 0 & 1 & -8\br
0 & 0 & 0 & -5
\end{bmatrix}\\
& \oprm{%
R_3 \opr{} R_3 + R_4\\
R_3 \opr{} R_3 - R_5\\
R_5 \opr{} \frac{1}{5} R_5
}
\begin{bmatrix}
1 & 0 & 0 & 2\br
0 & 1 & 0 & 0\br
0 & 0 & 0 & 0\br
0 & 0 & 1 & -8\br
0 & 0 & 0 & 1
\end{bmatrix}
\oprm{%
R_3 \bidiarrow{} R_4\\
R_4 \bidiarrow{} R_5\\
R_4 \opr{} R_4 + 8R_5\\
R_1 \opr{} R_1 -2R_5
}
\begin{bmatrix}
1 & 0 & 0 & 0\br
0 & 1 & 0 & 0\br
0 & 0 & 1 & 0\br
0 & 0 & 0 & 1\br
0 & 0 & 0 & 0
\end{bmatrix}
\end{align*}
נבחר את שורות המטריצה המדורגת השונות מ$0$ ונקבל את הבסיס ל$U + W$:
$$B_{U + W} = \left(\tot{1}{0}{0}{0}, \tot{0}{1}{0}{0}, \tot{0}{0}{1}{0}, \tot{0}{0}{0}{1}\right) = E$$
\qed

\subsection{סעיף ב -- בסיס ל$U \cap W$}
ראשית, נחשב את המימד של החיתוך:
\begin{align*}
\dim{U \cap W}
& = \dim{U} + \dim{W} - \dim{U + W}\\
& = 3 + 2 - 4\\
& = 1
\end{align*}
לפיכך, קיים רק ווקטור אחד בבסיס החיתוך. ידוע לנו שאיברי החיתוך שייכים גם ל$U$ וגם ל$W$, ולכן בסיס האיחוד הינו צירוף לינארי של בסיסי $U$ ו$W$, כלומר מתקיים:
\begin{eqnarray*}
& \alpha\tot{1}{0}{0}{2} + \beta\tot{0}{1}{0}{0} + \gamma\tot{0}{0}{1}{-3} = \delta\tot{4}{2}{1}{0} + \lambda\tot{2}{1}{0}{-1}\\
& \downarrow\\
& \alpha\fcol{1}{0}{0}{2} + \beta\fcol{0}{1}{0}{0} + \gamma\fcol{0}{0}{1}{-3} + \delta\fcol{-4}{-2}{-1}{0} + \lambda\fcol{-2}{-1}{0}{1} = 0\\
\end{eqnarray*}
בעמוד הבא נעביר להצגה מטריציונית ונדרג.
\clearpage
\setcounter{subsection}{1}
\subsection{סעיף ב )המשך(}
נעבור להצגה מטריציונית ונדרג:
\begin{eqnarray*}
& \begin{bmatrix}
1 & 0 & 0 & -4 & -2\br
0 & 1 & 0 & -2 & -1\br
0 & 0 & 1 & -1 & 0\br
2 & 0 & -3 & 0 & 1
\end{bmatrix}
\oprm{%
R_4 \opr{} R_4 - 2R_1\\
R_4 \opr{} R_4 + 3R_3\\
R_4 \opr{} \frac{1}{5}R_4
}
\begin{bmatrix}
1 & 0 & 0 & -4 & -2\br
0 & 1 & 0 & -2 & -1\br
0 & 0 & 1 & -1 & 0\br
0 & 0 & 0 & 1 & 1
\end{bmatrix}
\oprm{%
R_1 \opr{} R_1 + 4R_4\\
R_2 \opr{} R_2 + 2R_4\\
R_3 \opr{} R_3 + R_4
}
\begin{bmatrix}
1 & 0 & 0 & 0 & 2\br
0 & 1 & 0 & 0 & 1\br
0 & 0 & 1 & 0 & 1\br
0 & 0 & 0 & 1 & 1
\end{bmatrix}\\
& \downarrow\\
& \begin{matrix}
\alpha = -2\lambda\\\\\\
\beta = -\lambda\\\\\\
\gamma = -\lambda\\\\\\
\delta = -\lambda\\\\\\
\lambda = t
\end{matrix}
\end{eqnarray*}
על מנת להגיע לווקטור הבסיס, נציב $\lambda = 1$:
\begin{eqnarray*}
& -2\tot{1}{0}{0}{2} - \tot{0}{1}{0}{0} - \tot{0}{0}{1}{-3} = -\tot{4}{2}{1}{0} + \tot{2}{1}{0}{-1}\\\\
& \downarrow\\\\
& \tot{-2}{0}{0}{-4} + \tot{0}{-1}{0}{0} + \tot{0}{0}{-1}{3} = \tot{-4}{-2}{-1}{0} + \tot{2}{1}{0}{-1}\\\\
& \downarrow\\\\
& \tot{-2}{-1}{-1}{-1} = \tot{-2}{-1}{-1}{-1}
\end{eqnarray*}
לפיכך, הבסיס לחיתוך הוא
$$B_{U \cap W} = \left(\tot{-2}{-1}{-1}{-1}\right)$$
כנדרש.
\br\qed
%%%%% </Q4> %%%%%

\end{document}