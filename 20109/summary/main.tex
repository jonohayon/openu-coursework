\documentclass[11pt, oneside]{article}
\usepackage{geometry}
\geometry{a4paper}
\usepackage[parfill]{parskip}
\usepackage[nodisplayskipstretch]{setspace}
\usepackage{graphicx,titlesec}
\usepackage{amsmath,amssymb,cancel}

% Hebrew Stuff
\usepackage[utf8x]{inputenc}
\usepackage[english,hebrew]{babel}
\usepackage{hebfont}
\usepackage{subfiles}

% Custom commands
\newcommand{\qed}{\R{$\blacksquare$}}
\newcommand{\br}{\\\\\\\\\\\\\\}
\newcommand{\opr}[1]{\xrightarrow[\text{#1}]{}}
\newcommand{\logr}[1]{\underset{\text{#1}}{\Longrightarrow}}
\newcommand{\logl}[1]{\underset{\text{#1}}{\Longleftarrow}}
\newcommand{\bidiarrow}[1]{\underset{\text{#1}}{\leftrightarrow}}
\newcommand{\ueq}[1]{\underset{\text{#1}}{=}}
\newcommand{\mC}{\mathbb{C}}
\newcommand{\mR}{\mathbb{R}}
\newcommand{\mN}{\mathbb{N}}
\newcommand{\mZ}{\mathbb{Z}}
\newcommand{\mQ}{\mathbb{Q}}
\newcommand{\inv}[1]{#1^{-1}}

% Custom text commands (for Hebrew)
\newcommand{\q}[3]{\R{שאלה #3#2.#1}}
\newcommand{\m}[3]{\R{משפט #3#2.#1}}
\newcommand{\h}[3]{\R{הגדרה #3#2.#1}}
\newcommand{\ms}[3]{\R{מסקנה #3#2.#1}}

% Custom commands for this document
\newcommand{\definition}[2]{\textbf{#1:} #2.}

% Spacing
\titlespacing\section{0pt}{5pt}{8pt}
\titlespacing\subsubsection{0pt}{5pt}{0pt}
\setstretch{0.1}

\title{אלגברה לינארית 1 -- סיכום\thanks{מבוסס על השיעורים של ד``''ר אסף שרון מסמסטר 7102ג בקמפוס רמת אביב}}
\author{יונתן אוחיון}

\begin{document}
\maketitle

% Fields - introduction to binary operators, groups, abelian groups
\subfile{partials/groups}
\clearpage

% Fields - definition, finite fields
\subfile{partials/fields}
\clearpage

\section*{$n$-יות}
\subsection*{הגדרה}
$n$-יה )קריא: אניה( סדורה היא רשימה של $n$ איברים מקבוצה $A$ המסומנת כך:
\[
(a_{1}, a_{2}, a_{3}, \ldots, a_{n})
\]
האיבר של $A$ המופיע במקום ה$i$ של $n$-יה מכונה הרכיב ה$i$ שלה ומסומן כך: $a_{i}$. אין צורך להקיף בסוגריים $n$-יה של איבר אחד, שכן היא איבר בודד מ$A$. נסמן את אוסף כל ה$n$-יות באורך $k$ מעל $A$ כך: $A^{k}$.

\subsection*{שוויון $n$-יות}
נאמר שה$n$-יה $(a_{1}, \ldots, a_{n})$ שווה ל$n$-יה $(b_{1}, \ldots, b_{m})$ אם ורק אם התנאים הבאים מתקיימים:
\begin{enumerate}
\item $n = m$
\item $\forall 1 \le i \le n, a_{i} = b_{i}$
\end{enumerate}

\subsection*{חיבור $n$-יות}


\end{document}