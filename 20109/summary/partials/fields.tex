\documentclass[../main.tex]{subfiles}

\section*{שדות}
\subsection*{הגדרה}
תהי $\digamma$ קבוצה, $+_{\digamma}, \cdot_{\digamma}$ פעולות על $\digamma$. נגיד שהשלשה $(\digamma, +_{\digamma}, \cdot_{\digamma})$ נקראת שדה אם התכונות הבאות מתקיימות:
\begin{itemize}
\item הזוג $(\digamma, +_{\digamma})$ חבורה אבלית עם איבר ניטרלי המסומן ב$0$ או $0_{\digamma}$.
\item הזוג $(\digamma, \cdot_{\digamma})$ חבורה אבלית עם איבר ניטרלי המסומן ב$1$ או $1_{\digamma}$.
\item הפעולות מקיימות את תכונת הדיסטריבוטיביות )פילוג(: $a \cdot_{\digamma} (b +_{\digamma} c) = (a \cdot_{\digamma} b) +_{\digamma} (a \cdot_{\digamma} c)$
\end{itemize}
איבר בשדה נקרא סקלר )\L{Scalar}(.

\subsection*{דוגמאות}
השלשות $(\mR, +, \cdot)$ ו$(\mQ, +, \cdot)$ עם פעולות הכפל והחיבור הרגילות הן שדה.

\subsection*{שדות סופיים}
יהי $p$ מספר ראשוני. נסמן: $\mZ_{p} = \{0, \dots, p - 1\}$. נגדיר את הפעולות הבאות:
\begin{itemize}
\item שארית החילוק של $a + b$ ב$p$ $a +_{p} b = a \underset{\mod{p}}{+} b = $
\item שארית החילוק של $ab$ ב$p$ $a \cdot_{p} b = a \underset{\mod{p}}{\cdot} b = $
\end{itemize}
מסתבר שהשלשה $(\mZ_{p}, +_{p}, \cdot_{p})$ מקיימת את תכונות השדה.
\subsubsection*{שדות סופיים -- כדאי לזכור}
\begin{enumerate}
\item $\inv{(p - 1)} = p - 1$
\item כאשר $p \ge 2$, $\inv{2} = \frac{p + 1}{2}$
\item אם $\inv{a} = b$, אז:
	\begin{enumerate}
	\item $\inv{b} = a$
	\item $\inv{(-a)} = -b$
	\item $\inv{(-b)} = -a$
	\end{enumerate}
\end{enumerate}