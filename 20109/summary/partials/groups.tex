\documentclass[../main.tex]{subfiles}

\section*{פעולה בינארית על קבוצה}
פעולה בינארית $*$ על קבוצה $A$ הינה כלל התאמה שמתאים לכל זוג סדור $(a, b) \in A \times A$ איבר ב$A$ המסומן ב$a * b$. סימון פורמלי יותר: $*: A \times A \to A$.
\subsubsection*{שאלה}
תהי $*$ פעולה על $\mR$ המוגדרת כך: $a * b = ab - 1$. בדוק האם הפעולה מקיימת את תכונות החלופיות והקיבוציות.

\subsubsection*{פתרון}
נראה שהפעולה מקיימת את תכונת החילופיות. יהיו $a, b \in A$. נוכל לראות שמתקיים $a * b = ab - 1$ וגם $b * a = ba - 1$. מכיוון ש$a*b = b*a \logl{}  ba - 1 = ab - 1 \logl{} ba = ab$ ולכן $*$ חילופית.
נראה שהפעולה לא מקיימת את תכונת הקיבוציות בעזרת דוגמה נגדית:
\begin{align*}
&(1 * 2) * 3 = (2 - 1) * 3 = 1 * 3 = 2\\
&1 * (2 * 3) = 1 * (6 - 1) = 1 * 5 = 4\\
&2 \neq 4 \logr{} (1 * 2) * 3 \neq 1 * (2 * 3)
\end{align*}
ולכן $*$ לא קיבוצית כנדרש.
\br\qed

\section*{הקדמה לשדות -- חבורות}
תהי $G$ קבוצה ו$+$ פעולה עליה. הזוג $(G, +)$ ייקרא חבורה אם $+$ מקיימת את התכונות הבאות:
\begin{itemize}
\item \definition{סגירות}{לכל $a, b \in G$ מתקיים $a * b \in G$}
\item \definition{אסוציאטיביות )קיבוציות(}{לכל $a, b, c \in G$ מתקיים $a * (b * c) = (a * b) * c$}
\item \definition{קיום איבר ניטרלי}{קיים $e \in G$ כך שלכל $a \in G$ מתקיים $a * e = e * a = a$}
\item \definition{קיום איבר הופכי}{לכל $a \in G$ קיים $b \in G$ כך ש$a * b = b * a = e$}
\end{itemize}
בנוסף, $(G, +)$ תיקרא חבורה אבלית/חילופית אם היא חבורה המקיימת את תכונת החילופיות, כלומר אם לכל $a, b \in G$ מתקיים $a * b = b * a$.