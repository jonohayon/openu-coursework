\documentclass[11pt, oneside]{article}
\usepackage{geometry}
\geometry{a4paper}
\usepackage[parfill]{parskip}
\usepackage[nodisplayskipstretch]{setspace}
\usepackage{graphicx,titlesec}
\usepackage{amsmath,amssymb,cancel}

% Hebrew Stuff
\usepackage[utf8x]{inputenc}
\usepackage[english,hebrew]{babel}
\usepackage{hebfont}

% Custom commands
\newcommand{\qed}{\R{$\blacksquare$}}
\newcommand{\br}{\\\\\\\\\\\\\\}
\newcommand{\opr}[1]{\xrightarrow[\text{#1}]{}}
\newcommand{\oprm}[1]{\underset{\substack{#1}}{\longrightarrow}}
\newcommand{\bidiarrow}[1]{\underset{\text{#1}}{\leftrightarrow}}
\newcommand{\ueq}[1]{\underset{\text{#1}}{=}}
\newcommand{\ueqm}[1]{\underset{\substack{#1}}{=}}
\newcommand{\mR}{\mathbb{R}}
\newcommand{\mN}{\mathbb{N}}
\newcommand{\mZ}{\mathbb{Z}}
\newcommand{\mQ}{\mathbb{Q}}
\newcommand{\inv}[1]{#1^{-1}}

% Custom text commands (for Hebrew)
\newcommand{\q}[3]{\R{שאלה #3#2.#1}}
\newcommand{\m}[3]{\R{משפט #3#2.#1}}
\newcommand{\h}[3]{\R{הגדרה #3#2.#1}}
\newcommand{\ms}[3]{\R{מסקנה #3#2.#1}}

% Custom commands for this document
\DeclareMathOperator{\Sp}{Sp}
\DeclareMathOperator{\Ima}{Im}
\newcommand{\fiv}[4]{\begin{bmatrix}#1 \br #2 \br #3 \br #4\end{bmatrix}}
\newcommand{\fir}[4]{\begin{bmatrix}#1 & #2 & #3 & #4\end{bmatrix}}
\newcommand{\fim}[4]{\begin{bmatrix}#1 & #2 \br #3 & #4\end{bmatrix}}
\newcommand{\tiv}[3]{\begin{bmatrix}#1 \br #2 \br #3\end{bmatrix}}
\newcommand{\tir}[3]{\begin{bmatrix}#1 & #2 & #3\end{bmatrix}}

% Spacing
\titlespacing\subsubsection{0pt}{5pt}{3pt}
\setstretch{0.1}

\title{ממן 41}
\author{יונתן אוחיון}

\begin{document}
\maketitle

%%%%% <Q1> %%%%%
\section{שאלה 1}
\subsection{סעיף א}
ראשית, נחשב את הצורה הכללית של $(x + 1)(p'(x))$:
\begin{align*}
(x + 1)(p'(x))
& = x \cdot p'(x) + p'(x)\\
& = (1a_1x^0 + 2a_2x^1 + 3a_3x^2 + \dots) + x \cdot (1a_1x^0 + 2a_2x^1 + 3a_3x^2 + \dots)\\
& = (1a_1x^0 + 2a_2x^1 + 3a_3x^2 + \dots) + (1a_1x^1 + 2a_2x^2 + 3a_3x^3 + \dots)\\
& = 1a_1x^0 + 1a_1x^1 + 2a_2x^1 + 2a_2x^2 + 3a_3x^2 + \dots\\
& = a_1 + (a_1 + 2a_2)x + (2a_2 + 3a_3)x^2 + \dots + ((n - 1)a_{n - 1} + na_n)x^{n - 1}
\end{align*}
נעבור להצגת קואורדינטות בעזרת הבסיס הסטנדרטי $E$:
\[
[(x + 1)(p'(x))]_E =
\left[\sum^n_{i=0}((i - 1)a_{i - 1} + ia_i)x^{i - 1}\right]_E =
\begin{bmatrix}
0\br
a_1\br
a_1 + 2a_2\br
\vdots\br
((n - 1)a_{n - 1} + na_n)x^{n - 1}
\end{bmatrix}
\]

לפיכך, נוכל לייצג את ההעתקה הלינארית $T: \mR_n[x] \opr{} \mR_n[x]$ בעזרת העתקת הקואורדינטות $S: \mR^n \opr{} \mR^n$ המוגדרת כך:
\[
S\left(\begin{bmatrix}
a_0\br
a_1\br
\vdots\br
a_n
\end{bmatrix}\right)
=
\begin{bmatrix}
0\br
a_1\br
a_1 + 2a_2\br
\vdots\br
(n - 1)a_{n - 1} + na_n
\end{bmatrix}
\]
כעת נוכל להוכיח ש$S$ הינה העתקה לינארית )ועקב כך גם $T$( אם מתקיים
\[
S(\alpha [p(x)]_E + \beta [q(x)]_E) = \alpha S([p(x)]_E) + \beta S([q(x)]_E)
\]
 לכל $\alpha, \beta \in \mR$ ו$p(x), q(x) \in \mR_n[x]$. נוכיח בעמוד הבא.
\clearpage

\setcounter{subsection}{0}
\subsection{סעיף א )המשך(}
הוכחה:
\begin{align*}
S\left(\alpha \begin{bmatrix}
a_0\br
\vdots\br
a_n
\end{bmatrix} + \beta \begin{bmatrix}
b_0\br
\vdots\br
b_n
\end{bmatrix}\right)
& = S\left(\begin{bmatrix}
\alpha a_0 + \beta b_0\br
\vdots\br
\alpha a_n + \beta a_n
\end{bmatrix}\right)\\
& = \begin{bmatrix}
0\br
\alpha a_1 + \beta b_1\br
\alpha a_1 + \beta b_1 + 2(\alpha a_2 + \beta b_2)\br
\vdots\br
\alpha (n - 1)a_{n - 1} + \beta (n - 1)b_{n - 1} + n(\alpha a_n + \beta b_n)
\end{bmatrix}\\
& = \begin{bmatrix}
0\br
\alpha a_1 + \beta b_1\br
\alpha (a_1 + 2a_2) + \beta (b_1 + 2b_2)\br
\vdots\br
\alpha ((n-1)a_{n-1} + na_n) + \beta ((n-1)b_n + nb_n)
\end{bmatrix}\\
& = \alpha \begin{bmatrix}
0\br
a_1\br
a_1 + 2a_2\br
\vdots\br
(n - 1)a_{n - 1} + na_n
\end{bmatrix} + \beta \begin{bmatrix}
0\br
b_1\br
b_1 + 2b_2\br
\vdots\br
(n - 1)b_{n - 1} + nb_n
\end{bmatrix}\\
& = \alpha S\left(\begin{bmatrix}
a_0\br
\vdots\br
a_n
\end{bmatrix}\right) + \beta S\left(\begin{bmatrix}
b_0\br
\vdots\br
b_n
\end{bmatrix}\right)
\end{align*}
ו$S$ העתקה לינארית. מכיוון ש$S$ פועלת על ווקטורי הקואורדינטות של $\mR_n[x]$, נוכל להסיק מהעובדה שהיא העתקה לינארית ש$T$ הינה העתקה לינארית גם היא כנדרש.
\br\qed
\clearpage

\subsection{סעיף ב}
נראה ש$T$ אינה טרנספורמציה לינארית ע"י דוגמה נגדית לכפל בסקלר:
\begin{align*}
x = y = &\ 1,\ \alpha = -1\\
\alpha T(x, y) & \overset{?}{=} T(\alpha x, \alpha y)\\
\alpha (2x - y, 3|x|, y) & \overset{?}{=} (2\alpha x - \alpha y, 3|\alpha x|, \alpha y)\\
-1 (2 - 1, 3, 1) & \overset{?}{=} (-2 + 1, 3, -1)\\
(-1, -3, -1) & \neq (-1, 3, -1)
\end{align*}
לפיכך, ההעתקה $T$ אינה מקיימת את תכונת הכפל בסקלר והיא אינה ט"ל.
\br\qed

\subsection{סעיף ג}
נראה ש$T$ אינה טרנספורמציה לינארית ע"י דוגמה נגדית לחיבור:
\begin{eqnarray*}
& X = \begin{bmatrix}
0 & 1 & 1\br
1 & 0 & 1\br
1 & 1 & 0
\end{bmatrix},\ Y = \begin{bmatrix}
1 & 0 & 0\br
0 & 3 & 3\br
0 & 0 & 1
\end{bmatrix}\\
& XY = \begin{bmatrix}
0 & 3 & 4\br
1 & 0 & 1\br
1 & 3 & 3
\end{bmatrix},\ YX = \begin{bmatrix}
0 & 1 & 1\br
6 & 3 & 3\br
1 & 1 & 0
\end{bmatrix}\\
& T(X) + T(Y) \overset{?}{=} T(X + Y)\\
& \cancel{X^2} - \cancel{X} + \cancel{Y^2} - \cancel{Y} \overset{?}{=} \cancel{X^2} + XY + YX + \cancel{Y^2} - \cancel{X} - \cancel{Y}\\
& XY + YX \overset{?}{=} 0\\
& \begin{bmatrix}
0 & 4 & 5\br
7 & 3 & 4\br
2 & 4 & 3
\end{bmatrix} \neq 0
\end{eqnarray*}
לפיכך, ההעתקה $T$ לא מקיימת את תכונת החיבור והיא אינה ט"ל.
\br\qed
\clearpage
%%%%% </Q1> %%%%%

%%%%% <Q2> %%%%%
\section{שאלה 2}
\subsection{מציאת בסיסים ל$\ker T$ ול$\Ima T$}
\subsubsection{$\ker T$}
ראשית, נמצא העתקה $S$ המתאימה ל$T$ בעזרת הצגת הקואורדינטות לפי הבסיס הסטנדרטי של $M^\mR_{2 \times 2}$:
\begin{eqnarray*}
& A = \begin{bmatrix}
a & b\br
c & d
\end{bmatrix}\\
& T: M^\mR_{2 \times 2} \opr{} \mR_n[x],\ T(A) = (a - d)x^2 + (b + c)x + 5(a - d)\\
& \downarrow\\
& S: \mR^4 \opr{} \mR^3,\ S([A]_E) = S\left(\fiv{a}{b}{c}{d}\right) = \tiv{a - d}{b + c}{5(a - d)} = [T(A)]_E
\end{eqnarray*}
לפיכך, נוכל למצוא את גרעין ההעתקה $S$:
\begin{align*}
\ker S
& = \left\{v \mid v \in \mR^4 \land S(v) = 0\right\}\\
& = \left\{[A]_E \mid A \in M^\mR_{2 \times 2} \land [T(A)]_E = 0\right\}\\
& = \left\{v = \fir{a}{b}{c}{d}^t \mid a, b, c, d \in \mR \land S(v) = 0\right\}\\
& = \left\{v = \fir{a}{b}{c}{d}^t \mid \tir{a - d}{b + c}{5(a - d)}^t = 0\right\}\\
\end{align*}
כעת, נוכל לדרג את המערכת הבאה ולקבל את הצורה הכללית של איבר ב$\ker S$:
\begin{align*}
\begin{matrix}
a + 0b + 0c + (-1)d = 0\br
0a + b + c + 0d = 0\br
5a + 0b + 0c + (-5)d = 0
\end{matrix}
\opr{}
\begin{bmatrix}
1 & 0 & 0 & -1\br
0 & 1 & 1 & 0\br
5 & 0 & 0 & -5
\end{bmatrix}
\oprm{%
R_3 \opr{} R_3 - 5R_1
}
\begin{bmatrix}
1 & 0 & 0 & -1\br
0 & 1 & 1 & 0\br
0 & 0 & 0 & 0
\end{bmatrix}
\opr{}
\begin{matrix}
a = d\br
b = -c\br
c = \alpha\br
d = \beta
\end{matrix}
\end{align*}
כלומר, $\ker S = \left\{\fir{\beta}{-\alpha}{\alpha}{\beta} \mid \alpha, \beta \in \mR\right\}$.

נעבור מקואורדינטות בעזרת הבסיס הסטנדרטי:
\begin{align*}
\ker T
& = \left\{\fim{\beta}{-\alpha}{\alpha}{\beta} \mid \alpha, \beta \in \mR\right\}\\
& = \left\{\alpha\fim{0}{-1}{1}{0} + \beta\fim{1}{0}{0}{1} \mid \alpha, \beta \in \mR \right\}\\
& = \Sp\underbrace{\left\{\fim{0}{-1}{1}{0}, \fim{1}{0}{0}{1}\right\}}_P
\end{align*}

כעת, הווקטורים בקבוצה $P$ הפורשת את $\ker{T}$ אינם פרופורציוניים ולכן בלתי תלויים. לפיכך, $P$ הינה פורשת ובת"ל ב$\ker{T}$ ולפיכך הינה בסיס שלו.
\br\qed
\clearpage

\subsubsection{$\Ima{T}$}
נוכל למצוא את $\Ima{T}$:
\[
\Ima{T} = \left\{(a - d)x^2 + (b + c)x  + 5(a- d) \mid a, b, c, d \in \mR\right\}
\]
כעת נעבור להצגת קואורדינטות בעזרת הבסיס הסטנדרטי ונגיע ל$\Ima{S}$ )כאשר $S$ הינה הט"ל אשר מצאנו בסעיף הקודם(:
\begin{align*}
\Ima{S}
& = \left\{\tiv{a - d}{b + c}{5(a - d} \mid a, b, c, d \in \mR\right\}\\
& = \left\{a\tiv{1}{0}{5} + b\tiv{0}{1}{0} + c\tiv{0}{1}{0} + d\tiv{-1}{0}{-5} \mid a, b, c, d \in \mR\right\}\\
& = \Sp\left\{\tiv{1}{0}{5}, \tiv{0}{1}{0}, \cancel{\tiv{0}{1}{0}}, \tiv{-1}{0}{-5}\right\}\\
& = \Sp\underbrace{\left\{p_1 = \tiv{1}{0}{5}, p_2 = \tiv{0}{1}{0}, p_3 = -\tiv{1}{0}{5}\right\}}_P
\end{align*}
כעת, קיבלנו קבוצה $P$ הפורשת את $\Ima{S}$. מכיוון ש$p_3 = -p_1$ והם פרופורציוניים, נצטרך להוציא את $p_3$ מ$P$ על מנת שהיא תהיה בת"ל:
\[
\Ima{S} = \Sp\left\{\tiv{1}{0}{5}, \tiv{0}{1}{0}\right\}
\]
מכיוון שקבוצה זו גם פורשת וגם בת"ל, היא בסיס של $\Ima{S}$. כעת נעבור מהצגת קואורדינטות בעזרת הבסיס הסטנדרטי ונמצא את הבסיס המתאים ל$\Ima{T}$:
\begin{align*}
B_{\Ima{T}}
& = \left(x^2 + 0x + 5, 0x^2 + x + 0\right)\\
& = \left(x^2 + 5, x\right)
\end{align*}
כנדרש.
\br\qed

\subsection{סעיף ב}
אף אחד מהתנאים לא מתקיים, שכן $\Ima{T} \subseteq \mR_n[x]$ ו$\ker{T} \subseteq M^\mR_{2 \times 2}$, וכמובן ש$\mR_n[x] \not\subseteq M^\mR_{2 \times 2}$ ולכן לא יכולה להתקיים הכלה דו כיוונית ע"מ שוויון המרחבים.

לפיכך, $\ker{T} + \Ima{T} \neq M^\mR_{2 \times 2} \land \ker{T} \oplus \Ima{T} \neq M^\mR_{2 \times 2}$ כנדרש.
\br\qed
\clearpage

\subsection{סעיף ג}
\subsubsection{$i$}
ראשית, אנו יודעים שאיברי $\ker{T}$ הינם מטריצות, כלומר
\[
\exists a, b, c, d \in \mR \opr{} A = \fim{a}{b}{c}{d} \in \ker{T}
\]
כעת, די לחשב את $A^2$ ולבדוק אם אכן מתקיים $T(A^2) = 0$. ראשית, נחשב את $A^2$:
\[
A^2 =
A \cdot A =
\fim{a}{b}{c}{d} \cdot \fim{a}{b}{c}{d} =
\fim{a^2 + bc}{ab + bd}{ca + cd}{cb + d^2} =
\fim{a^2 + bc}{b(a + d)}{c(a + d)}{d^2 + bc}
\]
כעת נציב ב$T(A^2) = 0$:
\begin{align*}
T(A^2)
& = (a^2 + \cancel{bc} - (d^2 + \cancel{bc}))x^2 + (b(a + d) + c(a + d))x + 5(a^2 + \cancel{bc} - (d^2 + \cancel{bc}))\\
& = (a + d)(a - d)x^2 + (a + d)(b + c)x + 5(a + d)(a - d)\\
& = (a + d)[(a - d)x^2 + (b + c)x + 5(a - d)]\\
& = (a + d) \cdot T(A) \ueq{$T(A) \in \ker{T}$} 0 \cdot (a + d)\\
& = 0
\end{align*}
כלומר, מתקיים $T(A^2) = 0$ ולפיכך $A^2 \in \ker{T}$ גם הוא.
\br\qed

\subsubsection{$ii$}
הצורה הכללית של פולינום $p(x) \in \Ima{T}$ הינה $(a-d)x^2 + (b+c)x + 5(a-d)$. לפיכך, אם נצרף לו את הפולינום $r(x) = 3x^2 + 2x + 5$ נקבל את $q(x)$ הבא:
\[
q(x) = r(x) + p(x) = (a-d+3)x^2 + (b+c+2)x + 5(a-d+1)
\]
נניח ש$q(x) \in \Ima{T}$ לכל $p(x) \in \Ima{T}$ ונראה דוגמה נגדית לפולינום אשר עבורו $q(x) \not\in \Ima{T}$:
\begin{align*}
\begin{matrix}
a = 1 & b = 2\br
c = 3 & d = 4
\end{matrix}\;
& \opr{}
p(x) = -3x^2 + 3x - 15\\\\
q(x)
& = (-3 + 3)x^2 + (3 + 2)x + 5(-3+1)\\
& = 0x^2 + 5x - 10
\end{align*}
כלומר, מתקיים
\begin{align*}
\begin{matrix}
a - d + 3 = 0\br
a - d + 1 = -10
\end{matrix}
\opr{}
\begin{matrix}
a - d = -3\br
a - d = -11
\end{matrix}
\opr{}
-3 \neq -11
\end{align*}
וסתירה להנחה. לפיכך, לא לכל $p(x) \in \Ima{T}$ מתקיים $q(x) \in \Ima{T}$.
\br\qed
\clearpage
%%%%% </Q2> %%%%%

%%%%% <Q3> %%%%%
\section{שאלה 3}
\subsection{סעיף א}
ראשית, נראה ש$\Ima{T} \subseteq \ker{T}$: לפי הגדרת גרעין ההעתקה, $\ker{T} = \{v \mid T(v) = 0\}$ ולפי הגדרת תמונת ההעתקה, $\Ima{T} = \{T(v) \mid v \in V\}$. לפי הגדרת $T$ מתקיים $T^2 = 0$, כלומר $T(T(v)) = 0$ לכל $v \in V$. כעת, לפי הגדרת הגרעין, $x = T(v) \in \ker{T}$ לכל $v \in V$ )שכן מתקיים $T(x) = 0$(. כעת, כל $x \in \Ima{T}$ שייך גם ל$\ker{T}$ ולכן $\Ima{T} \subseteq \ker{T}$.

כעת, לפי \m{8}{3}{4.},‎ מתקיים $\dim{\Ima{T}} \le \dim{\ker{T}}$, כלומר קיים $\epsilon \in \mN$ כך ש
\[
\dim{\Ima{T}} = \dim{\ker{T}} -‎ \epsilon
\]
לפי \m{9}{6}{1.}, מתקיים
\begin{align*}
& \dim{\ker{T}} + \dim{\Ima{T}} = n\\
& \dim{\ker{T}} + \dim{\ker{T}} - \epsilon = n\\
& 2\dim{\ker{T}} = n + \epsilon\;/:2\\
& \dim{\ker{T}} = \frac{n + \epsilon}{2}
\end{align*}
ומכיוון ש$\epsilon \ge 0$, מתקיים
\[
\dim{\ker{T}} \ge \frac{n}{2}
\]
כנדרש.
\br\qed

\subsection{סעיף ב}
ראשית, לפי סעיף א אנו יודעים ש$\dim{\ker{T}} \ge \frac{n}{2}$, כלומר $\dim{\ker{T}} \ge 1.5$. נוכל להיווכח ש$2 \le \dim{\ker{T}} < n = 3$, ומכיוון שמימד של מרחב ווקטורי הינו מספר טבעי מתקיים $\dim{\ker{T}} = 2$. נסמן את הבסיס לגרעין כך:
\[
k_1, k_2 \in \ker{T} \opr{} B_{\ker{T}} = (k_1, k_2)
\]
כעת, על מנת להשלים את $B_{\ker{T}}$ לבסיס של $V$ נצטרך למצוא $v \in V$ אשר אינו בגרעין ההעתקה, כלומר שמתקיים $T(v) \neq 0$ ולצרפו ל$B_{\ker{T}}$. לפי הנתון, $T \neq 0$ ולכן קיים $v \in V$ כך ש$T(v) \neq 0$. לכן, מתקיים
\begin{eqnarray*}
& B_V = (v, k_1, k_2)\\
& \downarrow\\
& [T(k_1)]_{B_V} = \tir{0}{0}{0}^t \opr{} T(k_1) = 0v + 0k_1 + 0k_2 = 0\\
& [T(k_2)]_{B_V} = \tir{0}{0}{0}^t \opr{} T(k_2) = 0v + 0k_1 + 0k_2 = 0\\
& [T(v)]_{B_V} = \tir{1}{0}{0}^t \opr{} T(v) = 1v + 0k_1 + 0k_2 = v
\end{eqnarray*}
כנדרש.
\br\qed
\clearpage
%%%%% </Q3> %%%%%

%%%%% <Q4> %%%%%
\section{שאלה 4}
\setcounter{subsection}{-1}
\subsection{סימון מטריצת מעבר מ$B'$ ל$B$}
נסמן את מטריצת המעבר מבסיס $B'$ לבסיס $B$ בעזרת $M^{B}_{B'}$.
\subsection{סעיף א}
ראשית, נשלים את הקבוצה הפורשת את $\ker{T}$ לבסיס של $\mR_4[x]$:
\begin{align*}
& b_1 = 1 - x,\ b_2 = x - x^3\\
& [b_1]_E = \fir{0}{0}{-1}{1},\ [b_2]_E = \fir{-1}{0}{1}{0}\\\\
& \begin{bmatrix}
0 & 0 & -1 & 1\br
-1 & 0 & 1 & 0
\end{bmatrix}
\opr{\text{\R{השלמה לבסיס בעזרת $x^2, 1$}}}
\begin{bmatrix}
0 & 0 & -1 & 1\br
-1 & 0 & 1 & 0\br
0 & 1 & 0 & 0\br
0 & 0 & 0 & 1
\end{bmatrix}
\oprm{%
R_1 \opr{} R_1 - R_4\\
R_2 \opr{} R_2 + R_1\\
R_1 \opr{} -R_1\\
R_2 \opr{} -R_2
}
\begin{bmatrix}
0 & 0 & 1 & 0\br
1 & 0 & 0 & 0\br
0 & 1 & 0 & 0\br
0 & 0 & 0 & 1
\end{bmatrix}
\end{align*}
לפיכך, הקבוצה פורשת ובת"ל ב$\mR_4[x]$ ולכן $B$ בסיס כאשר
\[
B = (b_1 = 1 - x, b_2 = x - x^3, b_3 = x^2, b_4 = 1)
\]
כעת, על מנת להגדיר את ההעתקה די להגדיר אותה על איברי הבסיס, שכן איברי המרחב הינם צירוף לינארי שלהם. בנוסף, אנו יודעים שהקבוצה $\{1 - x, x - x^3\}$ פורשת את גרעין ההעתקה ולכן חייב להתקיים $T(1 - x) = T(x - x^3) = 0$. ניקח לדוגמה את ההעתקה $T: \mR_4[x] \opr{} \mR_4[x]$ הבאה:
\[
\begin{matrix}
T(1 - x) = 0 & T(x - x^3) = 0\br
T(x^2) = 1 - x & T(1) = x - x^3
\end{matrix} \opr{}
\begin{matrix}
T(1 - x) = \fir{0}{0}{0}{0}^t & T(x - x^3) = \fir{0}{0}{0}{0}^t\br
T(x^2) = \fir{1}{0}{0}{0}^t & T(1) = \fir{0}{1}{0}{0}^t
\end{matrix}
\]
מכיוון שההעתקה לינארית, נוכל לייצג את $T(ax^3 + bx^2 + cx + d)$ כך:
\[
(*)\quad T(ax^3 + bx^2 + cx + d) = aT(x^3) + bT(x^2) + cT(x) + dT(1)
\]
כעת, לפי \m{01}{6}{1.} מתקיים $[T]_E = M^{E}_{B} \cdot [T]_B \cdot M^{B}_{E} = M^{E}_{B} \cdot [T]_B \cdot (M^{E}_{B})^{-1}$. לפיכך, על מנת למצוא את הנוסחה המפורשת ל$T$ נצטרך למצוא את מטריצת הייצוג שלה בבסיס $B$, מטריצת המעבר מ$E$ ל$B$ והמטריצה ההופכית לה.

ראשית, נמצא את מטריצת הייצוג של ההעתקה:
\[
[T]_B = \fir{[T(1 - x)]_B}{[T(x - x^3)]_B}{[T(x^2)]_B}{[T(1)]_B} = \begin{bmatrix}
0 & 0 & 1 & 0\br
0 & 0 & 0 & 1\br
0 & 0 & 0 & 0\br
0 & 0 & 0 & 0
\end{bmatrix}
\]
בעמוד הבא נמצא את $M^{E}_{B}$ ואת ההופכית לה.
\clearpage
\setcounter{subsection}{0}
\subsection{סעיף א )המשך(}
על מנת למצוא את מטריצת המעבר, ראשית נמצא את הצירופים הלינארים של איברי $E$ בעזרת איברי $B$:
\[
\begin{matrix}
e_1 = -b_1 - b_2 + 0b_3 + b_4 & e_2 = 0b_1 + 0b_2 + b_3 + 0b_4\br
e_3 = -b_1 + 0b_2 + 0b_3 + b_4 & e_4 = 0b_1 + 0b_2 + 0b_3 + b_4
\end{matrix}
\]
כעת, נמצא את מטריצת המעבר:
\[
M^{E}_{B} = \fir{[e_1]_B}{[e_2]_B}{[e_3]_B}{[e_4]_B} = \begin{bmatrix}
-1 & 0 & -1 & 0\br
-1 & 0 & 0 & 0\brי
0 & 1 & 0 & 0\br
1 & 0 & 1 & 1
\end{bmatrix}
\]
כעת, נמצא את המטריצה ההופכית לה ע"י דירוג $[M^{E}_{B} \mid I_4]$ עד להגעה ל$[I_4 \mid (M^{E}_{B})^{-1}]$:
\begin{align*}
& \begin{bmatrix}
-1 & 0 & -1 & 0 & \mid & 1 & 0 & 0 & 0\\
-1 & 0 & 0 & 0 & \mid & 0 & 1 & 0 & 0\\
0 & 1 & 0 & 0 & \mid & 0 & 0 & 1 & 0\\
1 & 0 & 1 & 1 & \mid & 0 & 0 & 0 & 1
\end{bmatrix}
\oprm{%
R_2 \opr{} -R_2\\
R_2 \bidiarrow{} R_1\\
R_2 \opr{} R_2 + R_1\\
R_4 \opr{} R_4 - R_1
}
\begin{bmatrix}
1 & 0 & 0 & 0 & \mid & 0 & -1 & 0 & 0\\
0 & 0 & -1 & 0 & \mid & 1 & -1 & 0 & 0\\
0 & 1 & 0 & 0 & \mid & 0 & 0 & 1 & 0\\
0 & 0 & 1 & 1 & \mid & 0 & 1 & 0 & 1
\end{bmatrix}\\
& \oprm{%
R_4 \opr{} R_4 + R_2\\
R_2 \opr{} -R_2\\
R_2 \bidiarrow{} R_3
}
\begin{bmatrix}
1 & 0 & 0 & 0 & \mid & 0 & -1 & 0 & 0\\
0 & 1 & 0 & 0 & \mid & 0 & 0 & 1 & 0\\
0 & 0 & 1 & 0 & \mid & -1 & 1 & 0 & 0\\
0 & 0 & 0 & 1 & \mid & 1 & 0 & 0 & 1
\end{bmatrix}
\opr{}
M^{B}_{E} = \begin{bmatrix}
0 & -1 & 0 & 0\br
0 & 0 & 1 & 0\br
-1 & 1 & 0 & 0\br
1 & 0 & 0 & 1
\end{bmatrix}
\end{align*}
כעת, נכפול את המטריצות ונגיע למטריצת הייצוג של ההעתקה בעזרת הבסיס הסטנדרטי:
\begin{align*}
& A = M^{B}_{E} \cdot [T]_B = \begin{bmatrix}
0 & -1 & 0 & 0\br
0 & 0 & 1 & 0\br
-1 & 1 & 0 & 0\br
1 & 0 & 0 & 1
\end{bmatrix} \cdot
\begin{bmatrix}
0 & 0 & 1 & 0\br
0 & 0 & 0 & 1\br
0 & 0 & 0 & 0\br
0 & 0 & 0 & 0
\end{bmatrix}
= \begin{bmatrix}
0 & 0 & 0 & -1\br
0 & 0 & 0 & 0\br
0 & 0 & -1 & 1\br
0 & 0 & 1 & 0
\end{bmatrix}\\
& A \cdot M^{E}_{B} = \begin{bmatrix}
0 & 0 & 0 & -1\br
0 & 0 & 0 & 0\br
0 & 0 & -1 & 1\br
0 & 0 & 1 & 0
\end{bmatrix} \cdot \begin{bmatrix}
-1 & 0 & -1 & 0\br
-1 & 0 & 0 & 0\brי
0 & 1 & 0 & 0\br
1 & 0 & 1 & 1
\end{bmatrix} = \begin{bmatrix}
1 & -1 & 0 & 0\br
1 & -1 & 0 & 0\br
1 & 0 & 0 & 1\br
-1 & 1 & 0 & 0
\end{bmatrix}
\end{align*}
כעת, נמצא את $T(e_i)$ לפי הקואורדינטות $[T(e_i)]_E$:
\[
\begin{matrix}
T(e_1) = T(x^3) = -x^3 + x & T(e_2) = T(x^2) = -x + 1\br
T(e_3) = T(x) = 0 & T(e_4) = T(1) = -x^3 + x
\end{matrix}
\]
ונוכל למצוא את הנוסחה המפורשת לפי $(*)$:
\[
T(ax^3 + bx^2 + cx + d) = aT(x^3) + bT(x^2) + cT(x) + dT(1) = (-a-d)x^3 + (a - b + d)x + b
\]
כנדרש.
\br\qed
\clearpage
%%%%% </Q4> %%%%%

%%%%% <Q5> %%%%%
\section{שאלה 5}
\subsection{סעיף א}
ראשית, נמצא את ערכו של $a$. לפי הנתון, ההעתקה לא הפיכה ולפיכך המטריצה המייצגת אותה אינה הפיכה גם היא, כלומר -- הדטרמיננטה שלה שווה ל$0$. כעת, נחשב את הדטרמיננטה של המטריצה המייצגת וכך נגיע לערך $a$:
\begin{align*}
0 = \begin{vmatrix}
1 & 0 & 1\br
1 & a & 2a\br
a & 1 & 2a
\end{vmatrix}
\ueqm{%
C_3 \opr{} C_3 - C_1\\
R_2 \opr{} R_2 - R_1\\
R_3 \opr{} R_3 - aR_1
}
\begin{vmatrix}
1 & 0 & 0\br
0 & a & 2a - 1\br
0 & 1 & 2a
\end{vmatrix}
\ueq{\R{פיתוח לפי $C_1, R_1$}}
a^2 - (2a - 1) \opr{} a^2 - 2a + 1
\end{align*}
כעת, לפי נוסחאת הכפל המקוצר: $(a - 1)^2 = 0$, ולכן $a = 1$. כעת, נמצא את מטריצת המעבר מבסיס $E$ לבסיס $B$ בעזרת מציאת הצירופים הלינארים של $e_i$ בעזרת $b_i$:
\begin{eqnarray*}
& \begin{matrix}
e_1 = 0.5b_1 + 0.5b_2 + 0.5b_3 & e_2 = 0.5b_1 + 0.5b_2 - 0.5b_3\br
e_3 = 0.5b_1 - 0.5b_2 - 0.5b_3
\end{matrix}\\\\
& \downarrow\\\\
& M^{E}_{B} = \tir{[e_1]_B}{[e_2]_B}{[e_3]_B} = \begin{bmatrix}
0.5 & 0.5 & 0.5\br
0.5 & 0.5 & -0.5\br
0.5 & -0.5 & -0.5
\end{bmatrix}
\end{eqnarray*}
כעת, נוכל לייצג את $T(\alpha, \beta, \gamma)$ כך:
\[
T(\alpha, \beta, \gamma) = T(\alpha e_1 + \beta e_2 + \gamma e_3) = \alpha T(e_1) + \beta T(e_2) + \gamma T(e_3)
\]
נמצא את $T(e_i)$ בעזרת המטריצה המייצגת את ההעתקה בבסיס $E$. נמצא את $M^{B}_{E} = (M^{E}_{B})^-1$ ע"י דירוג $[M^{E}_{B} \mid I_4]$ עד להגעה ל$[I_4 \mid (M^{E}_{B})^{-1}]$:
\begin{align*}
& \begin{bmatrix}
0.5 & 0.5 & 0.5 & \mid & 1 & 0 & 0\\
0.5 & 0.5 & -0.5 & \mid & 0 & 1 & 0\\
0.5 & -0.5 & -0.5 & \mid & 0 & 0 & 1\\
\end{bmatrix}
\oprm{%
R_i \opr{} 2R_i\\
R_2 \opr{} R_2 - R_1\\
R_3 \opr{} R_3 - R_1
}
\begin{bmatrix}
1 & 1 & 1 & \mid & 2 & 0 & 0\\
0 & 0 & -2 & \mid & -2 & 2 & 0\\
0 & -2 & -2 & \mid & -2 & 0 & 2\\
\end{bmatrix}
\oprm{%
R_2 \opr{} -0.5R_2\\
R_3 \opr{} -0.5R_3\\
}\\
& \begin{bmatrix}
1 & 1 & 1 & \mid & 2 & 0 & 0\\
0 & 0 & 1 & \mid & 1 & -1 & 0\\
0 & 1 & 1 & \mid & 1 & 0 & -1\\
\end{bmatrix}
\oprm{%
R_2 \bidiarrow{} R_3\\
R_1 \opr{} R_1 - R_2\\
R_2 \opr{} R_2 - R_3
}
\begin{bmatrix}
1 & 0 & 0 & \mid & 1 & 0 & 1\\
0 & 1 & 0 & \mid & 0 & 1 & -1\\
0 & 0 & 1 & \mid & 1 & -1 & 0\\
\end{bmatrix}
\opr{}
M^{B}_{E} = \begin{bmatrix}
1 & 0 & 1\br
0 & 1 & -1\br
1 & -1 & 0
\end{bmatrix}
\end{align*}
בעמוד הבא נחשב את $[T]_E$.
\clearpage
\setcounter{subsection}{0}
\subsection{סעיף א )המשך(}

%%%%% </Q5> %%%%%

\end{document}