\documentclass[11pt, oneside]{article}
\usepackage{geometry}
\geometry{a4paper}
\usepackage[parfill]{parskip}
\usepackage[nodisplayskipstretch]{setspace}
\usepackage{graphicx,titlesec}
\usepackage{amsmath,amssymb,cancel}

% Hebrew Stuff
\usepackage[utf8x]{inputenc}
\usepackage[english,hebrew]{babel}
\usepackage{hebfont}

% Custom commands
\newcommand{\qed}{\R{$\blacksquare$}}
\newcommand{\br}{\\\\\\\\\\\\\\}
\newcommand{\opr}[1]{\xrightarrow[\text{#1}]{}}
\newcommand{\oprm}[2][]{\overset{\substack{#1}}{\underset{\substack{#2}}{\longrightarrow}}}
\newcommand{\bidiarrow}[1]{\underset{\text{#1}}{\leftrightarrow}}
\newcommand{\ueq}[1]{\underset{\text{#1}}{=}}
\newcommand{\ueqm}[2][]{\overset{\substack{#1}}{\underset{\substack{#2}}{=}}}
\newcommand{\mR}{\mathbb{R}}
\newcommand{\mN}{\mathbb{N}}
\newcommand{\mZ}{\mathbb{Z}}
\newcommand{\mQ}{\mathbb{Q}}
\newcommand{\inv}[1]{#1^{-1}}

% Custom text commands (for Hebrew)
\newcommand{\q}[3]{\R{שאלה #3#2.#1}}
\newcommand{\m}[3]{\R{משפט #3#2.#1}}
\newcommand{\h}[3]{\R{הגדרה #3#2.#1}}
\newcommand{\ms}[3]{\R{מסקנה #3#2.#1}}

% Custom commands for this document
\newcommand{\tiv}[3]{\begin{bmatrix}#1 \br #2 \br #3\end{bmatrix}}
\newcommand{\tir}[3]{\begin{bmatrix}#1 & #2 & #3\end{bmatrix}}
\newcommand{\detf}[4]{\begin{vmatrix}#1 & #2\br #3 & #4\end{vmatrix}}

% Spacing
\titlespacing\subsubsection{0pt}{5pt}{3pt}
\setstretch{0.1}

\title{ממן 51}
\author{יונתן אוחיון}

\begin{document}
\maketitle

\setcounter{section}{-1}
\section{סימונים}
\subsection{ריבוי גיאומטרי ואלגברי}
נסמן את הריבוי הגיאומטרי של הערך העצמי $\lambda$ כך: $g_\lambda$ ואת הריבוי האלגברי שלו כך: $a_\lambda$.

%%%%% <Q1> %%%%%
\section{שאלה 1}
\subsection{סעיף א}
\subsubsection{בדיקת לכסינות}
ראשית, נמצא את הפולינום האופייני של מטריצה $A$:
\begin{align*}
P_A(\lambda)
& = |\lambda I - A| = \begin{vmatrix}
\lambda - 3 & 0 & -4\br
1 & \lambda - 1 & -2\br
2 & 0 & \lambda - 1
\end{vmatrix}
\ueq{$R_3 \opr{} R_3 + R_1$} \begin{vmatrix}
\lambda - 3 & 0 & -4\br
1 & \lambda - 1 & -2\br
\lambda - 1 & 0 & \lambda - 1
\end{vmatrix}\\
& \ueq{\R{הוצאת גורם משותף מ$C_1, R_3$}} (\lambda - 1)^2 \begin{vmatrix}
\lambda - 3 & 0 & -4\br
1 & 1 & -2\br
1 & 0 & 1
\end{vmatrix} \ueqm[%
  C_1 \opr{} C_1 - C_2
]{%
  C_1 \opr{} C_1 - C_3\\
  R_1 \bidiarrow{} R_3
}
(\lambda - 1)^2 \begin{vmatrix}
0 & 0 & 1\br
\lambda + 1 & 0 & -4\br
-2 & 1 & -2
\end{vmatrix}\\
& \ueq{\R{פיתוח לפי $R_1$}} (\lambda - 1)^2 \cdot \left[
\cancel{0\detf{1}{-2}{0}{-4}} - \cancel{0\detf{-2}{-2}{\lambda + 1}{-4}} + 1\detf{-2}{1}{\lambda + 1}{0}
\right]\\
& = -(\lambda - 1)^2(\lambda + 1)
\end{align*}
לפי \m{11}{4}{1.}, נוכל להשוות את הפולינום ל$0$ ולקבל את הערכים העצמיים $\lambda = \pm 1$. מכיוון שהפולינום מתפרק לגורמים לינאריים נוכל להמשיך ולבדוק את הריבוי האלגברי והגיאומטרי של כל ע"ע. ראשית, נוכל לראות ש$a_{-1} = 1$, ולפי \q{11}{5}{5.}ב מתקיים $g_{-1} = a_{-1} = 1$. כעת, נוכל לראות שמתקיים $a_1 = 2$. נבדוק מהו ערכו של $g_1$:
\[
g_1 = n - \rho(\lambda I - A) = 3 - 1 = 2
\]
כעת, מכיוון שמתקיים $g_1 = a_1 = 2$, $A$ לכסינה. בעמוד הבא נמצא את המטריצה $P$ המלכסנת אותה ואת המטריצה האלכסונית $B$ הדומה לה.
\clearpage
\subsubsection{מציאת המטריצה המלכסנת}
ראשית, נמצא את הו"ע לכל הע"ע של המטריצה $A$. עבור $\lambda = -1$:
\begin{eqnarray*}
& (A + I)\vec{v} = 0 \opr{}
\begin{bmatrix}
4 & 0 & -4 & \mid & 0\\
1 & 2 & -2 & \mid & 0\\
2 & 0 & -2 & \mid & 0
\end{bmatrix} \oprm[%
  R_1 \opr{} \frac{1}{4}R_1
]{%
  R_2 \opr{} \frac{1}{2}R_2
}
\begin{bmatrix}
1 & 0 & -1 & \mid & 0\\
1 & 2 & -2 & \mid & 0\\
1 & 0 & -1 & \mid & 0
\end{bmatrix} \oprm[%
  R_3 \opr{} R_3 - R_1
]{%
  R_2 \opr{} R_2 - R_1\\
  R_2 \opr{} \frac{1}{2}R_2
}
\begin{bmatrix}
1 & 0 & -1 & \mid & 0\\
0 & 1 & -\frac{1}{2} & \mid & 0\\
0 & 0 & 0 & \mid & 0
\end{bmatrix}\\
& \downarrow\\
& \begin{matrix}
x = z\br
y = \frac{1}{2}z\br
z = 2t
\end{matrix} \opr{} \begin{matrix}
x = 2t\br
y = t\br
\end{matrix} \opr{} t(2, 1, 2)\ \text{\R{הפתרון הכללי:}} \opr{} \vec{v}_{1} = \tiv{2}{1}{2}
\end{eqnarray*}
כעת, נחפש את הווקטורים העצמיים עבור $\lambda = 1$:
\begin{eqnarray*}
& (A - I)\vec{v} = 0 \opr{} \begin{bmatrix}
2 & 0 & -4 & \mid & 0\\
1 & 0 & -2 & \mid & 0\\
2 & 0 & -4 & \mid & 0
\end{bmatrix} \oprm[%
  R_3 \opr{} R_3 - R_1
]{%
  R_2 \bidiarrow{} R_1\\
  R_2 \opr{} R_2 - 2R_1
}
\begin{bmatrix}
1 & 0 & -2 & \mid & 0\\
0 & 0 & 0 & \mid & 0\\
0 & 0 & 0 & \mid & 0
\end{bmatrix}\\
& \downarrow\\
& \begin{matrix}
x = 2z\br
y = t\br
z = s
\end{matrix} \opr{} s(2, 0, 1) + t(0, 1, 0)\ \text{\R{הפתרון הכללי:}} \opr{} \vec{v}_2 = \tiv{2}{0}{1},\ \vec{v}_3 = \tiv{0}{1}{0}
\end{eqnarray*}
לכן, המטריצה המלכסנת $P$ הינה
\[
P = \begin{bmatrix}\\\\\\
\mid & \mid & \mid\br
\vec{v}_1 & \vec{v}_2 & \vec{v}_3\br
\mid & \mid & \mid
\\\\\\\end{bmatrix} = \begin{bmatrix}\\\\\\
2 & 2 & 0\br
1 & 0 & 1\br
2 & 1 & 0
\\\\\\\end{bmatrix}
\]
כנדרש.
\subsubsection{מציאת המטריצה האלכסונית הדומה ל$A$}
ראשית, נמצא את $P^{-1}$:
\begin{align*}
& \begin{bmatrix}
2 & 2 & 0 & \mid & 1 & 0 & 0\\
1 & 0 & 1 & \mid & 0 & 1 & 0\\
2 & 1 & 0 & \mid & 0 & 0 & 1
\end{bmatrix} \oprm[%
  R_1 \opr{} \frac{1}{2}R_1
]{%
  R_3 \opr{} \frac{1}{2}R_3
}
\begin{bmatrix}
1 & 1 & 0 & \mid & \frac{1}{2} & 0 & 0\\\\
1 & 0 & 1 & \mid & 0 & 1 & 0\\\\
1 & \frac{1}{2} & 0 & \mid & 0 & 0 & \frac{1}{2}
\end{bmatrix} \oprm[%
  R_2 \opr{} R_2 - R_1
]{%
  R_3 \opr{} R_3 - R_1
}
\begin{bmatrix}
1 & 1 & 0 & \mid & \frac{1}{2} & 0 & 0\\\\
0 & -1 & 1 & \mid & -\frac{1}{2} & 1 & 0\\\\
0 & -\frac{1}{2} & 0 & \mid & -\frac{1}{2} & 0 & \frac{1}{2}
\end{bmatrix}\\
& \oprm[%
  R_3 \opr{} -2R_3\\
  R_3 \bidiarrow{} R_2
]{%
  R_3 \opr{} R_3 + R_2\\
  R_1 \opr{} R_1 - R_2
} \begin{bmatrix}
1 & 0 & 0 & \mid & \frac{1}{2} & 0 & 1\\\\
0 & 1 & 0 & \mid & 1 & 1 & -1\\\\
0 & 0 & 1 & \mid & \frac{1}{2} & 1 & -1
\end{bmatrix} \opr{}
P^{-1} = \begin{bmatrix}
\frac{1}{2} & 0 & 1\\\\
1 & 1 & -1\\\\
\frac{1}{2} & 1 & -1
\end{bmatrix}
\end{align*}
בדף הבא נמצא את המטריצה האלכסונית $B$ הדומה ל$A$.
\clearpage
\setcounter{subsubsection}{2}
\subsubsection{מציאת המטריצה האלכסונית הדומה ל$A$ )המשך(}
לפי \h{11}{3}{4.}, מתקיים $B = P^{-1}AP$. כעת, נכפיל את המטריצות על מנת להגיע למטריצה האלכסונית $B$:
\begin{align*}
& M =
P^{-1} \cdot A =
\begin{bmatrix}
\frac{1}{2} & 0 & 1\br
1 & 1 & -1\br
\frac{1}{2} & 1 & -1
\end{bmatrix} \cdot \begin{bmatrix}
3 & 0 & -4\br
1 & 1 & -2\br
2 & 0 & 1
\end{bmatrix} =
\begin{bmatrix}
\frac{1}{2} & 0 & -1\br
1 & 0 & -1\br
\frac{1}{2} & 1 & -1
\end{bmatrix}\\
& M \cdot P =
\begin{bmatrix}
\frac{1}{2} & 0 & -1\br
1 & 0 & -1\br
\frac{1}{2} & 1 & -1
\end{bmatrix} \cdot \begin{bmatrix}
2 & 2 & 0\br
1 & 0 & 1\br
2 & 1 & 0
\end{bmatrix} =
\begin{bmatrix}
-1 & 0 & 0\br
0 & 1 & 0\br
0 & 0 & 1
\end{bmatrix}
\end{align*}
לפיכך, מצאנו את המטריצה האלכסונית הדומה ל$A$ והיא הינה
\[
B = \begin{bmatrix}
-1 & 0 & 0\br
0 & 1 & 0\br
0 & 0 & 1
\end{bmatrix}
\]
כנדרש.
\br\qed

\subsection{סעיף ב}
\subsubsection{טענת עזר}
ראשית, נוכיח טענת עזר על מנת להעלות מטריצות לכסינות בחזקה בקלות. לפי \h{11}{3}{4.}, מתקיים
\begin{eqnarray*}
& B = P^{-1}AP \opr{\R{כפל מימין ב$P^{-1}$}}
BP^{-1} = P^{-1}A \opr{\R{כפל משמאל ב$P$}}
PBP^{-1} = A\\
& \text{\R{העלה בחזקה ב$k$}}\downarrow\\
& A^k = (PBP^{-1})^k = \underbrace{PB\cancel{P^{-1}} \cdot \cancel{P}B\cancel{P^{-1}} \cdot \ldots \cdot \cancel{P}B\cancel{P^{-1}} \cdot \cancel{P}BP^{-1}}_{\text{\R{$k$ פעמים}}} = PB^{k}P^{-1}
\end{eqnarray*}
לפיכך, $A^k = PB^{k}P^{-1}$.
\subsubsection{הוכחה}
ראשית, נחשב את המטריצה המייצגת את ההעתקה $T$ לפי הבסיס הסטנדרטי:
\begin{eqnarray*}
& \begin{matrix}
T(\vec{e}_1) = (3, 1, 2) & T(\vec{e}_2) = (0, 1, 0)\br
T(\vec{e}_3) = (-4, -2, -3)
\end{matrix}\\
& \downarrow\\
& [T]_E = \begin{bmatrix}
\mid & \mid & \mid\br
{[T(\vec{e}_1)]_E} & {[T(\vec{e}_2)]_E} & {[T(\vec{e}_3)]_E}\br
\mid & \mid & \mid
\end{bmatrix} = \begin{bmatrix}
3 & 0 & -4\br
1 & 1 & -2\br
2 & 0 & 1
\end{bmatrix} = A\\
\end{eqnarray*}
כעת, מכיוון שהמטריצה המייצגת של $T$ שווה ל$A$ היא לכסינה. בעמוד הבא נחשב את $T^{2020}$ בעזרת טענת העזר.
\clearpage

\setcounter{subsubsection}{1}
\subsubsection{הוכחה )המשך(}
לפי \m{01}{4}{1.}, מתקיים $[T^2]_E = [T]_E \cdot [T]_E$, כלומר $[T^2]_E = A^2$. כעת, נוכל לפרק את $[T^{2020}]_E$ ל$\underbrace{[T^2]_E \cdot \ldots \cdot [T^2]_E}_{\text{\R{$1010$ פעמים}}}$. לפי טענת העזר שהוכחנו, נוכל לראות שמתקיים
\begin{align*}
A^2
& = PB^{2}P^{-1}\\\\
& = P \cdot \cancel{\begin{bmatrix}
-1 & 0 & 0\br
0 & 1 & 0\br
0 & 0 & 1
\end{bmatrix} \cdot \begin{bmatrix}
-1 & 0 & 0\br
0 & 1 & 0\br
0 & 0 & 1
\end{bmatrix}} \cdot P^{-1}\\
& = PP^{-1}\\
& = I
\end{align*}
לפיכך, $[T^{2020}]_E = I$ ומתקיים $T^{2020}(x, y, z) = (x, y, z)$ כנדרש.
\br\qed
%%%%% </Q1> %%%%%

%%%%% <Q2> %%%%%
%%%%% </Q2> %%%%%

\end{document}