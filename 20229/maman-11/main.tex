\documentclass[11pt, oneside]{article}
\usepackage{geometry}
\geometry{a4paper}
\usepackage[parfill]{parskip}
\usepackage[nodisplayskipstretch]{setspace}
\usepackage{graphicx,titlesec}
\usepackage{mathtools,amsmath,amssymb,cancel}

% Hebrew Stuff
\usepackage[utf8x]{inputenc}
\usepackage[english,hebrew]{babel}
\usepackage{hebfont}

% Custom commands
\newcommand{\qed}{\R{$\blacksquare$}}
\newcommand{\br}{\\\\\\\\\\\\\\}
\newcommand{\opr}[1]{\xrightarrow[\text{#1}]{}}
\newcommand{\bidiarrow}[1]{\underset{\text{#1}}{\leftrightarrow}}
\newcommand{\ueq}[1]{\underset{\text{#1}}{=}}
\newcommand{\mC}{\mathbb{C}}
\newcommand{\mR}{\mathbb{R}}
\newcommand{\mN}{\mathbb{N}}
\newcommand{\mZ}{\mathbb{Z}}
\newcommand{\mQ}{\mathbb{Q}}
\newcommand{\inv}[1]{#1^{-1}}

% Custom text commands (for Hebrew)
\newcommand{\q}[3]{\R{שאלה #3#2.#1}}
\newcommand{\m}[3]{\R{משפט #3#2.#1}}
\newcommand{\h}[3]{\R{הגדרה #3#2.#1}}
\newcommand{\ms}[3]{\R{מסקנה #3#2.#1}}

% Custom commands for this document
\newcommand{\ip}[2]{\langle #1,\;#2 \rangle}
\newcommand{\conjm}[1]{#1^{*}}
\newcommand{\ctrans}[2]{\conjm{#1}_{P}(#2)}
\DeclarePairedDelimiter\parens{\lparen}{\rparen}
\DeclareMathOperator{\trnp}{tr}
\newcommand{\tr}[1]{\trnp\parens*{#1}}
\newcommand{\mf}[4]{\begin{bmatrix}#1 & #2\br #3 & #4\end{bmatrix}}

% Spacing
\titlespacing\subsubsection{0pt}{5pt}{3pt}
\setstretch{0.1}

\title{ממן 11}
\author{יונתן אוחיון}

\begin{document}
\maketitle

%%%%% <Q1> %%%%%
\section*{שאלה 1}
\subsection*{סעיף א}
ראשית נוכיח טענת עזר אשר מראה שעבור כל $A, B, C, D \in M^{F}_{n \times n}$ מתקיים $\tr{ABCD} = \tr{DABC}$ באופן הבא:
\begin{align*}
\tr{ABCD}
&= \tr{A(BCD)}\\
\tr{AB} = \tr{BA} \to &= \tr{BCDA}\\
&= \tr{(BC)(DA)}\\
\tr{AB} = \tr{BA} \to &= \tr{DABC}\\
\end{align*}
נחפש את ההעתקה הצמודה $\conjm{T_{P}}: V \opr{} V$ המקיימת לכל $X, Y \in V$:
\[
\ip{T(X)}{Y} = \ip{X}{\ctrans{T}{Y}}
\]
כלומר לפי הגדרת המכפלה הפנימית הסטנדרטית ב$M^{\mC}_{n \times n}$ ולפי הנתון,
\begin{align*}
\tr{\conjm{(\ctrans{T}{Y})} \cdot X}
&= \tr{\conjm{Y} \cdot T(X)}\\
\text{\R{נתון}} \to &= \tr{\conjm{Y} \cdot \inv{P} \cdot X \cdot P}\\
\text{\R{טענת העזר}} \to &= \tr{P \cdot \conjm{Y} \cdot \inv{P} \cdot X}
\end{align*}
כלומר, ההעתקה הצמודה מקיימת $\conjm{(\ctrans{T}{Y})} = P \cdot \conjm{Y} \cdot \inv{P}$, והיא ההעתקה הבאה:
\begin{align*}
\conjm{(\ctrans{T}{X})} &= P \cdot \conjm{X} \cdot \inv{P}\\
\ctrans{T}{X}
&= \conjm{(P \cdot \conjm{X} \cdot \inv{P})}\\
&= \overline{(P \cdot \conjm{X} \cdot \inv{P})^{t}}\\
&= \overline{(\inv{P})^{t} \cdot (\conjm{X})^{t} \cdot P^{t}}\\
&= \overline{(\inv{P})^{t}} \cdot \overline{(\conjm{X})^{t}} \cdot \overline{P^{t}}\\
&= \conjm{(\inv{P})} \cdot \conjm{(\conjm{X})} \cdot \conjm{P}\\
&= \inv{(\conjm{P})} \cdot X \cdot \conjm{P} \ueq{\text{\R{נתון}}} T_{\conjm{P}}
\end{align*}
ולכן $\ctrans{T}{X} = T_{\conjm{P}}(X)$ כנדרש.
\br\qed

\clearpage
\section*{שאלה 1 )המשך(}
\subsection*{סעיף ב}
לפי סעיף א של השאלה, נוכל להיווכח שמתקיים $\ctrans{T}{X} = T_{\conjm{P}}(X)$, כלומר על מנת שנמצא את צירופי הבסיס הסטנדרטי כל מה שעלינו לעשות הוא לחשב את $\conjm{P}, \inv{(\conjm{P})}$ ואת הצורה הכללית של $T_{\conjm{P}}(X)$ ולחשב. נעשה זאת:
\[
\conjm{P} = \overline{P^{t}} = \mf{-i}{-1}{1}{i}, \inv{(\conjm{P})} = \conjm{(\inv{P})} = \frac{1}{2}\mf{i}{1}{-1}{-i}
\]
כעת, נחשב את הצורה הכללית של $T_{\conjm{P}}(X)$:
\begin{align*}
T_{\conjm{P}}(X) =
T_{\conjm{P}}\left(\mf{a}{b}{c}{d}\right) &=
\frac{1}{2} \cdot \mf{-i}{-1}{1}{i} \cdot \mf{a}{b}{c}{d} \cdot \mf{i}{1}{-1}{-i}\\
&= \frac{1}{2} \mf{ib+d+i(-ia-c)}{-ia-c-i(-ib-d)}{-b-id+i(a+ic)}{a+ic-i(b+id)}
\end{align*}
נזכיר מהו הבסיס הסטנדרטי של $M^{\mC}_{2 \times 2}$:
\[
E = \left\{\vec{e_{1}} = \mf{1}{0}{0}{0}, \vec{e_{2}} = \mf{0}{1}{0}{0}, \vec{e_{3}} = \mf{0}{0}{1}{0}, \vec{e_{4}} = \mf{0}{0}{0}{1}\right\}
\]
נציב ונחשב את ערכי ההעתקה על הבסיס:
\begin{align*}
T_{\conjm{P}}(\vec{e_{1}}) = \mf{\frac{1}{2}}{-\frac{i}{2}}{\frac{i}{2}}{\frac{1}{2}} =
\frac{1}{2}\vec{e_{1}} - \frac{i}{2}\vec{e_{2}} + \frac{i}{2}\vec{e_{3}} + \frac{1}{2}\vec{e_{4}}\quad&\quad
T_{\conjm{P}}(\vec{e_{2}}) = \mf{\frac{i}{2}}{-\frac{1}{2}}{-\frac{1}{2}}{-\frac{i}{2}} =
\frac{i}{2}\vec{e_{1}} - \frac{1}{2}\vec{e_{2}} - \frac{1}{2}\vec{e_{3}} - \frac{i}{2}\vec{e_{4}}\\
T_{\conjm{P}}(\vec{e_{3}}) = \mf{-\frac{i}{2}}{-\frac{1}{2}}{-\frac{1}{2}}{\frac{i}{2}} =
-\frac{i}{2}\vec{e_{1}} - \frac{1}{2}\vec{e_{2}} - \frac{1}{2}\vec{e_{3}} + \frac{i}{2}\vec{e_{4}}\quad&\quad
T_{\conjm{P}}(\vec{e_{4}}) = \mf{\frac{1}{2}}{\frac{i}{2}}{-\frac{i}{2}}{\frac{1}{2}} =
\frac{1}{2}\vec{e_{1}} + \frac{i}{2}\vec{e_{2}} - \frac{i}{2}\vec{e_{3}} + \frac{1}{2}\vec{e_{4}}
\end{align*}
לפיכך, מצאנו את צירופי הבסיס הסטנדרטי של הפעלת ההעתקה על הבסיס וכעת נוכל להרכיב את המטריצה המייצגת של ההעתקה:
\[
[\conjm{T_{P}}]_{E} = [T_{\conjm{P}}]_{E} = \frac{1}{2}\begin{bmatrix}
1 & i & -i & 1\br
-i & -1 & -1 & i\br
i & -1 & -1 & -i\br
1 & -i & i & 1
\end{bmatrix}
\]
ומצאנו את המטריצה המייצגת את $\conjm{T_{P}}$ לפי הבסיס הסטנדרטי של $M^{\mC}_{2 \times 2}$ כנדרש.
\br\qed
%%%%% </Q1> %%%%%

\end{document}