\documentclass[11pt, oneside]{article}
\usepackage{geometry}
\geometry{a4paper}
\usepackage[parfill]{parskip}
\usepackage[nodisplayskipstretch]{setspace}
\usepackage{graphicx,titlesec}
\usepackage{amsmath,amssymb,cancel}

% Hebrew Stuff
\usepackage[utf8x]{inputenc}
\usepackage[english,hebrew]{babel}
\usepackage{hebfont}

% Custom commands
\newcommand{\qed}{\R{$\blacksquare$}}
\newcommand{\br}{\\\\\\\\\\\\\\}
\newcommand{\opr}[1]{\xrightarrow[\text{#1}]{}}
\newcommand{\oprm}[1]{\underset{\substack{#1}}{\longrightarrow}}
\newcommand{\bidiarrow}[1]{\underset{\text{#1}}{\leftrightarrow}}
\newcommand{\ueq}[1]{\underset{\text{#1}}{=}}
\newcommand{\mC}{\mathbb{C}}
\newcommand{\mR}{\mathbb{R}}
\newcommand{\mN}{\mathbb{N}}
\newcommand{\mZ}{\mathbb{Z}}
\newcommand{\mQ}{\mathbb{Q}}
\newcommand{\inv}[1]{#1^{-1}}

% Custom text commands (for Hebrew)
\newcommand{\q}[3]{\R{שאלה #3#2.#1}}
\newcommand{\m}[3]{\R{משפט #3#2.#1}}
\newcommand{\h}[3]{\R{הגדרה #3#2.#1}}
\newcommand{\ms}[3]{\R{מסקנה #3#2.#1}}

% Custom commands for this document
\newcommand{\conj}[1]{#1^{*}}
\newcommand{\mfr}[4]{\begin{bmatrix}#1 & #2\br #3 & #4\end{bmatrix}}
\newcommand{\mfd}[4]{\begin{vmatrix}#1 & #2\br #3 & #4\end{vmatrix}}

% Spacing
\titlespacing\subsubsection{0pt}{5pt}{3pt}
\setstretch{0.1}

\title{ממן 21}
\author{יונתן אוחיון}

\begin{document}
\maketitle

%%%%% <Q1> %%%%%
\section*{שאלה 1}
\subsection*{סעיף א}
\subsubsection*{מטריצה $A_{1}$}
ראשית, נחשב את $\conj{A_{1}} = \overline{A^{t}}$:
\[
A_{1} = \mfr{0}{i}{-i}{0}
\opr{} A^{t} = \mfr{0}{-i}{i}{0}
\opr{} \conj{A_{1}} = \mfr{0}{i}{-i}{0}
\]
נוכל לראות שמתקיים $A_{1} = \conj{A_{1}}$. לפיכך, $A_{1}\conj{A_{1}} = \conj{A_{1}}A_{1} = A^{2}_{1}$ ולכן $A_{1}$ מטריצה נורמלית. נמצא מטריצה אוניטרית המלכסנת אותה. ראשית, נמצא את הערכים העצמיים של $A_{1}$:
\[
P_{A_{1}}(\lambda) = \mfd{\lambda}{i}{-i}{\lambda} = \lambda^{2} + 1 = (\lambda - 1)(\lambda + 1)
\]
והערכים העצמיים שמצאנו הינם $\lambda_{1} = 1, \lambda_{2} = -1$. נמצא את הווקטורים העצמיים לכל ע``''ע:
\[
(A_{1} - \lambda_{1}I)\vec{v} = \vec{0} \equiv \mfr{-1}{i}{-i}{-1}
\oprm{R_{1} \to -R_{1}\\R_{2} \to iR_{2}} \mfr{1}{-i}{1}{-i}
\oprm{R_{2} \to R_{2} - R_{1}} \mfr{1}{-i}{0}{0}
\]
ומצאנו את הו``''ע $\vec{v_{1}} = (1, -i)$. נמצא את הווקטור העצמי המשוייך ל$\lambda_{2}$:
\[
(A_{1} - \lambda_{2}I)\vec{v} = \vec{0} \equiv \mfr{1}{i}{-i}{1}
\oprm{R_{2} \to iR_{2}\\R_{2} \to R_{2} - R_{1}} \mfr{1}{i}{0}{0}
\]
ומצאנו את הו``''ע $\vec{v_{2}} = (1, i)$. כעת, ננרמל את שני הווקטורים:
\[
\norm
\]

%%%%% </Q1> %%%%%

\end{document}