\documentclass[11pt, oneside]{article}
\usepackage{geometry}
\geometry{a4paper}
\usepackage[parfill]{parskip}
\usepackage[nodisplayskipstretch]{setspace}
\usepackage{graphicx,titlesec}
\usepackage{amsmath,amssymb,cancel,commath}

% Hebrew Stuff
\usepackage[utf8x]{inputenc}
\usepackage[english,hebrew]{babel}
\usepackage{hebfont}

% Custom commands
\newcommand{\qed}{\R{$\blacksquare$}}
\newcommand{\br}{\\\\\\\\\\\\\\}
\newcommand{\opr}[1]{\xrightarrow[\text{#1}]{}}
\newcommand{\bidiarrow}[1]{\underset{\text{#1}}{\leftrightarrow}}
\newcommand{\ueq}[1]{\underset{\text{#1}}{=}}
\newcommand{\mR}{\mathbb{R}}
\newcommand{\mN}{\mathbb{N}}
\newcommand{\mZ}{\mathbb{Z}}
\newcommand{\mQ}{\mathbb{Q}}
\newcommand{\mC}{\mathbb{C}}
\newcommand{\inv}[1]{#1^{-1}}

% Custom text commands (for Hebrew)
\newcommand{\q}[3]{\R{שאלה #3#2.#1}}
\newcommand{\m}[3]{\R{משפט #3#2.#1}}
\newcommand{\h}[3]{\R{הגדרה #3#2.#1}}
\newcommand{\ms}[3]{\R{מסקנה #3#2.#1}}

% Custom commands for this document
\DeclareMathOperator{\Sp}{Sp}
\newcommand{\mtf}[4]{\begin{bmatrix}#1 & #2\\ #3 & #4\end{bmatrix}}
\newcommand{\ip}[2]{\langle #1, #2 \rangle}

% Spacing
\titlespacing\section{0pt}{0pt}{0pt}
\titlespacing\subsection{0pt}{4pt}{0pt}

\title{אוסף שאלות )8102א(}
\author{אילון בן שמואל}
\date{}

\begin{document}
\maketitle

\section{שאלה 1}
\subsection{סעיף א}
יהי $V$ מרחב \underline{אוקלידי} ויהיו $\vec{u}, \vec{v} \in V$. נסמן ב$\ip{}{}$ מכפלה פנימית ב$V$. הוכח או הפרך:
\begin{enumerate}
\item $\ip{\vec{u}}{\vec{v}} = \frac{1}{4}(\norm{\vec{u} + \vec{v}}^2 - \norm{\vec{u} - \vec{v}}^2)$
\item אם $\norm{\vec{u} + \vec{v}} = \norm{\vec{u} - \vec{v}}$ אז $\vec{u} \bot \vec{v}$
\item אם $\vec{u} + \vec{v} \bot \vec{u} - \vec{v}$ אז $\norm{\vec{u}} = \norm{\vec{v}}$
\end{enumerate}

\subsection{סעיף ב}
חזור על סעיף א3, אך התייחס ל$V$ כמרחב \underline{אוניטרי}.

\subsection{סעיף ג}
יהי $V$ מרחב \underline{אוניטרי} ויהיו $\vec{u}, \vec{v} \in V$. נסמן ב$\ip{}{}$ מכפלה פנימית ב$V$. הוכח או הפרך:\\
הפונקציה $f(\vec{u}, \vec{v}) = Re \ip{\vec{u}}{\vec{v}}$ מהווה מכפלה פנימית במרחב $V$ מעל $\mR$.

\section{שאלה 2}
תהי $\{\vec{g_1}, \dots, \vec{g_n}\} \subseteq \mR^n$. הוכח שלכל $\lambda_1, \dots, \lambda_n \in \mR$ מתקיים
\[
\norm{\sum^{n}_{i = 1} \lambda_i \vec{g_i}} \le
\sqrt{\sum^{n}_{i = 1} \abs{\lambda_i}^2}
\sqrt{\sum^{n}_{i = 1} \norm{\vec{g_i}}^2}
\]
כאשר הנורמות מחושבות לפי המכפלה הפנימית הסטנדרטית ב$\mR^n$.
\clearpage

\section{שאלה 3}
תהי $f: \mR^n \to \mR$ פונקציה:
\[
f(\vec{v}) = \frac{(v_1 + v^2_2 + \dots + v^n_n)^2}{v^2_1 + v^2_2 + \dots + v^2_n}
\]
לכל $\vec{v} = (v_1, \dots, v_n)$ ב$\mR^n$
\subsection{סעיף א}
מצא סקלר $0 < m \in \mR$ \underline{מינימלי} כך ש$f(\vec{u}) \le m$ לכל $\vec{u} \neq \vec0$.
\subsection{סעיף ב}
מצא ווקטור $\vec{v} \neq \vec0$ כך ש$f(\vec{v}) = m$ )שאלה זו פתורה במפגשי "אופק"(.

\section{שאלה 4}
נניח ש$V$ מרחב מכפלה פנימית נוצר סופית מעל $F = \mR$ או $F = \mC$. יהי $B = \{\vec{v_1}, \dots, \vec{v_n}\}$ בסיס של $V$. הוכח ש$B$ בסיס \underline{אורתונורמלי} אם ורק אם לכל $\alpha_1, \dots, \alpha_n \in F$ מתקיים
\[
\norm{\sum^{n}_{k = 1} \alpha_k \vec{v_k}}^2 = \sum^{n}_{k = 1} \abs{\alpha_k}^2
\]

\section{שאלה 5}
נניח ש$V$ מרחב מכפלה פנימית נוצר סופית מעל $F = \mR$ או $F = \mC$ ויהי $B$ בסיס אורתונורמלי $B = \{\vec{v_1}, \dots, \vec{v_n}\} \subseteq V$. נסמן ב$\ip{}{}$ את המכפלה הפנימית במרחב $V$ ונסמן ב$\cdot$ את המכפלה הפנימית הסטנדרטית במרחב $F^n$ מעל $F$. הוכח שלכל $\vec{u}, \vec{w} \in V$ מתקיים
\[
\ip{\vec{u}}{\vec{w}} = [\vec{u}]_B \cdot [\vec{w}]_B
\]

\section{שאלה 6}
האם קיימת מכפלה פנימית במרחב $\mR_3[x]$ מעל $\mR$ כך שהקבוצה הבאה מהווה קבוצה אורתוגונלית?
\begin{enumerate}
\item $\{1 - x, 1 + x + x^2, x^2\}$
\item $\{1, 1 - x^2\}$
\item $\{1 + x, 2 + x, 3 + 2x\}$
\end{enumerate}
\clearpage
\section{שאלה 7}
יהי $V = \mR^4$ עם המכפלה הפנימית הסטנדרטית ויהי $\vec{v} = (1, 2, 3, 4)$. האם קיים תת מרחב של $V$ כך שאורך ההיטל האורתוגונלי של $\vec{v}$ עליו הוא $6$?

\section{שאלה 8}
יהי $V$ מרחב מכפלה פנימית, ויהי $\vec{v} \in V$. יהי $\alpha \in \mR$ כך ש$0 \le \alpha \le \norm{\vec{v}}$. האם קיים תת מרחב של $V$ כך שאורך ההיטל האורתוגונלי של $\vec{v}$ עליו הוא בדיוק $\alpha$?
\section{שאלה 9}
המכפלות הפנימיות בשאלה זו הן המכפלות הפנימיות הסטנדרטיות במרחבים המתאימים.
\begin{enumerate}
\item יהיו $\vec{u} = (1, 2, i, 1, i), \vec{v} = (3 + i, -1, 1 - i, 1, -i)$. האם $\vec{u} \bot \vec{v}$? חשב את המרחק בין $\vec{u}$ ל$\vec{v}$.
\item מהו המרחק בין $\vec{w} = (1, 1, 3)$ לבין תת המרחב $U = \Sp\{(1, 0, 0), (1, 1, 0)\}$ של $\mR^3$?
\item האם המרחק בין $\mtf{1}{2}{3}{4}$ לבין תת המרחב $\Sp\left\{\mtf{1}{2}{0}{0}, \mtf{-2}{-4}{6}{8}\right\}$ הוא $\sqrt{243}$? אם כן, הוכח. אם לא, חשב מהו המרחק.
\end{enumerate}

\section{שאלה 01}
יהי $V = \mR_4[x]$, $F = \mR$, $p, q \in V$. האם הנוסחה הבאה מהווה מכפלה פנימית?
\[
\ip{p}{q} = \sum^{2}_{i = 0} p(i)q(i)
\]

\section{שאלה 11}
יהי $V = M^{\mR}_{2\times2}$, $F = \mR$, $A, B \in V$. האם הנוסחה $\ip{A}{B} = tr(BA)$ מהווה מכפלה פנימית?
\section{שאלה 21}
יהיו $\vec{u} = (x_1, x_2)$ ו$\vec{v} = (y_1, y_2)$ ווקטורים ב$\mR^2$ כמרחב ווקטורי מעל $\mR$. קבע לגבי כל אחת מהנוסחאות הבאות האם היא מהווה מכפלה פנימית ב$\mR^2$:
\begin{enumerate}
\item $\ip{\vec{u}}{\vec{v}} = x_1y_1^2 + x_2y_2$
\item $\ip{\vec{u}}{\vec{v}} = 2x_1y_1 + x_1y_2 + x_2y_1 + x_2y_2$
\item $\ip{\vec{u}}{\vec{v}} = x_1y_1 + 3x_1y_2 + 3x_2y_1 + 9x_2y_2$
\end{enumerate}

\end{document}