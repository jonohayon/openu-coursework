\documentclass[11pt, oneside]{article}
\usepackage{geometry}
\geometry{a4paper}
\usepackage[parfill]{parskip}
\usepackage[nodisplayskipstretch]{setspace}
\usepackage{graphicx,titlesec}
\usepackage{amsmath,amssymb,cancel,mathtools}

% Hebrew Stuff
\usepackage[utf8x]{inputenc}
\usepackage[english,hebrew]{babel}
\usepackage{hebfont}

% Custom commands
\newcommand{\qed}{\R{$\blacksquare$}}
\newcommand{\br}{\\\\\\\\\\\\}
\newcommand{\opr}[1]{\xrightarrow[\text{#1}]{}}
\newcommand{\logr}{\Rightarrow}
\newcommand{\bidiarrow}[1]{\underset{\text{#1}}{\leftrightarrow}}
\newcommand{\ueq}[1]{\underset{\text{#1}}{=}}
\newcommand{\mC}{\mathbb{C}}
\newcommand{\mR}{\mathbb{R}}
\newcommand{\mN}{\mathbb{N}}
\newcommand{\mZ}{\mathbb{Z}}
\newcommand{\mQ}{\mathbb{Q}}
\newcommand{\inv}[1]{#1^{-1}}

% Custom text commands (for Hebrew)
\newcommand{\q}[3]{\R{שאלה #3#2.#1}}
\newcommand{\m}[3]{\R{משפט #3#2.#1}}
\newcommand{\h}[3]{\R{הגדרה #3#2.#1}}
\newcommand{\ms}[3]{\R{מסקנה #3#2.#1}}

% Custom commands for this document
\DeclareMathOperator{\tr}{tr}
\DeclareMathOperator{\Sp}{Sp}
\DeclareMathOperator{\Ima}{Im}
\newcommand{\nimuk}[1]{#1 \text{\R{נימוק: }}}
\newcommand{\mfr}[4]{\begin{bmatrix}\\#1 & #2\br #3 & #4\\\\\end{bmatrix}}
\newcommand{\eon}{\mfr{1}{0}{0}{0}}
\newcommand{\etw}{\mfr{0}{1}{0}{0}}
\renewcommand{\eth}{\mfr{0}{0}{1}{0}}
\newcommand{\efo}{\mfr{0}{0}{0}{1}}
\newcommand{\tot}[1]{#1_{2\times2}}
\def\dotp{\boldsymbol{\cdot}}
\def\M{\mfr{1}{2}{3}{5}}
\def\v{\vec{v}}
\def\w{\vec{w}}

% Spacing
\titlespacing\subsection{0pt}{5pt}{3pt}
\setstretch{0.1}

\title{ממן 31}
\author{יונתן אוחיון}

\begin{document}
\maketitle

%%%%% <Q1> %%%%%
\section*{שאלה 1}
\subsection*{סעיף א}
ראשית, נפשט מעט את הביטוי של $f(A, B)$:
\begin{align*}
f(A, B)
&= \tr{A^{t}MB}\\
\nimuk{\tr{A} = \tr{A^{t}}} &= \tr{(B^{t}M^{t}A)^{t}}\\
&= \tr{B^{t}M^{t}A}
\end{align*}
בנוסף, ידוע לנו שמתקיים $f(B, A) = \tr{B^{t}MA}$, כלומר עלינו למצוא תנאי ש$M$ תקיים על מנת ש$\tr{B^{t}M^{t}A} = \tr{B^{t}MA}$. תנאי זה הוא כמובן תנאי הסימטריות, כלומר $M = M^{t}$.

לפיכך, $f(A, B) = f(B, A)$ אממ $M$ מטריצה סימטרית ומצאנו את התנאי הנדרש.
\br\qed

\subsection*{סעיף ב}
נחשב )בעמוד הזה ובעמוד הבא(:
\subsubsection*{שורה 1}
\begin{eqnarray*}
& f(e_{1}, e_{1}) = \tr{\eon \cdot \M \cdot \eon} = \tr\eon = 1\\\\
& f(e_{1}, e_{2}) = \tr{\eon \cdot \M \cdot \etw} = \tr\mfr{0}{1}{0}{0} = 0\\\\
& f(e_{1}, e_{3}) = \tr{\eon \cdot \M \cdot \eth} = \tr\mfr{2}{0}{0}{0} = 2\\\\
& f(e_{1}, e_{4}) = \tr{\eon \cdot \M \cdot \efo} = \tr\mfr{0}{2}{0}{0} = 0\\\\
& [A]^{R}_{1} = (f(e_{1}, e_{1}), f(e_{1}, e_{2}), f(e_{1}, e_{3}), f(e_{1}, e_{4})) = (1,0,2,0)
\end{eqnarray*}
\clearpage

\section*{שאלה 1 -- המשך}
\subsection*{סעיף ב}
\textbf{שורה 2}
\begin{eqnarray*}
& f(e_{2}, e_{1}) = \tr{\eth \cdot \M \cdot \eon} = \tr\eth = 0\\\\
& f(e_{2}, e_{2}) = \tr{\eth \cdot \M \cdot \etw} = \tr\efo = 1\\\\
& f(e_{2}, e_{3}) = \tr{\eth \cdot \M \cdot \eth} = \tr\mfr{0}{0}{2}{0} = 0\\\\
& f(e_{2}, e_{4}) = \tr{\eth \cdot \M \cdot \efo} = \tr\mfr{0}{0}{0}{2} = 2\\\\
& [A]^{R}_{2} = (f(e_{2}, e_{1}), f(e_{2}, e_{2}), f(e_{2}, e_{3}), f(e_{2}, e_{4})) = (0,1,0,2)
\end{eqnarray*}

\textbf{שורה 3}
\begin{eqnarray*}
& f(e_{3}, e_{1}) = \tr{\etw \cdot \M \cdot \eon} = \tr\mfr{3}{0}{0}{0} = 3\\\\
& f(e_{3}, e_{2}) = \tr{\etw \cdot \M \cdot \etw} = \tr\mfr{0}{3}{0}{0} = 0\\\\
& f(e_{3}, e_{3}) = \tr{\etw \cdot \M \cdot \eth} = \tr\mfr{5}{0}{0}{0} = 5\\\\
& f(e_{3}, e_{4}) = \tr{\etw \cdot \M \cdot \efo} = \tr\mfr{0}{5}{0}{0} = 0\\\\
& [A]^{R}_{3} = (f(e_{3}, e_{1}), f(e_{3}, e_{2}), f(e_{3}, e_{3}), f(e_{3}, e_{4})) = (3,0,5,0)
\end{eqnarray*}

\textbf{שורה 4}
\begin{eqnarray*}
& f(e_{4}, e_{1}) = \tr{\efo \cdot \M \cdot \eon} = \tr\mfr{0}{0}{3}{0} = 0\\\\
& f(e_{4}, e_{2}) = \tr{\efo \cdot \M \cdot \etw} = \tr\mfr{0}{0}{0}{3} = 3\\\\
& f(e_{4}, e_{3}) = \tr{\efo \cdot \M \cdot \eth} = \tr\mfr{0}{0}{5}{0} = 0\\\\
& f(e_{4}, e_{4}) = \tr{\efo \cdot \M \cdot \efo} = \tr\mfr{0}{5}{0}{5} = 5\\\\
& [A]^{R}_{4} = (f(e_{4}, e_{1}), f(e_{4}, e_{2}), f(e_{4}, e_{3}), f(e_{4}, e_{4})) = (0,3,0,5)
\end{eqnarray*}
בעמוד הבא נראה את $[f]_{E}$.
\clearpage

\section*{שאלה 1 -- המשך}
\subsection*{סעיף ב}
כעת, לאחר שחישבנו את שורות $[f]_{E}$, נציב:
\[
[f]_{E} = \begin{bmatrix}\\\\\\
1 & 0 & 2 & 0\br
0 & 1 & 0 & 2\br
3 & 0 & 5 & 0\br
0 & 3 & 0 & 5\\\\\\\\
\end{bmatrix} = \mfr{I}{2I}{3I}{5I}
\]
ומצאנו את המטריצה המייצגת לפי הבסיס הסטנדרטי כנדרש.
\br\qed

\subsection*{סעיף ג}
נסמן ב$f_{1}$ תבנית בילינארית סימטרית וב$f_{2}$ תבנית בילינארית אנטיסימטרית. עלינו למצוא $f_{1}, f_{2}$ כאלו המקיימות $f = f_{1} + f_{2}$. לפי \ms{01}{8}{}, די לנו למצוא שתי מטריצות $A, B \in M^{\mR}_{4\times4}$ המקיימות $A = A^{t}, B = -B^{t}$ שבעזרתן נוכל להגדיר את $f_{1}, f_{2}$ כך שמתקיים
\[
[f]_{E} = [f_{1} + f_{2}]_{E} = [f_{1}]_{E} + [f_{2}]_{E} = A + B
\]

נתבונן במטריצות הבלוקים הבאות:
\[
A = \mfr{I}{K}{K}{5I},\ B = \mfr{0}{M}{-M}{0} \logr A + B = \mfr{I}{K + M}{K - M}{5I}
\]
כעת, נמצא את $K, M$:
\[
K + M = 2I \land K - M = 3I \logr K = 2.5I, M = -0.5I
\]
ומצאנו את המטריצות שחיפשנו והן:
\[
A = \mfr{I}{2.5I}{2.5I}{5I},\ B = \mfr{0}{-0.5I}{0.5I}{0} \logr A + B = \mfr{I}{2I}{3I}{5I}
\]
נראה שהן אכן מקיימות את ההגבלות:
\[
A^{t} = \mfr{I^{t}}{2.5I^{t}}{2.5I^{t}}{5I^{t}} = \mfr{I}{2.5I}{2.5I}{5I} = A,\ -B^{t} = \mfr{0^{t}}{-(0.5I)^{t}}{-(-0.5I)^{t}}{0^{t}} = \mfr{0}{-0.5I}{0.5I}{0} = B
\]

ומצאנו שתי תבניות בילינאריות $f_{1}(v, u) = [v]^{t}_{E}A[u]_{E},\ f_{2}(v, u) = [v]^{t}_{E}B[u]_{E}$ המקיימות\\
 $f = f_{1} + f_{2}$ כנדרש.
\br\qed
\clearpage
%%%%% </Q1> %%%%%

%%%%% <Q2> %%%%%
\section*{שאלה 2}
\subsection*{סימונים}
יהי $B = \{\v_{1}, \ldots, \v_{n}\}$ בסיס ל$V$ ונסמן:
\begin{align*}
\vec{b}, \vec{c}\in \digamma^{n},\ T(\vec{x}) = \sum^{n}_{i=1}b_{i}x_{i} = \vec{b} \dotp [\vec{x}]_{B},\ S(\vec{y}) = \sum^{n}_{j=1}c_{j}y_{j} = \vec{c} \dotp [\vec{y}]_{B}
\end{align*}
כאשר $\dotp$ המכפלה הסקלרית. בנוסף, נסמן את המטריצה המייצגת של $f$ לפי $B$ ב$A$, כלומר $A = [f]_{B}$.
\subsection*{כיוון א -- הצגה $\Leftarrow$ דרגה}
נניח כי ההצגה של $f$ לפי $B$ היא $f(\vec{x}, \vec{y}) = T(\vec{x})S(\vec{y})$ ונוכיח כי $\rho(f) = 1$. ברור כי מתקיים
\[
\v_{i} = 0 \cdot \v_{1} + \ldots + 1 \cdot \v_{i} + \ldots + 0 \cdot \v_{n}
\]
ולכן $f(\v_{i}, \v_{j}) = T(\v_{i})S(\v_{j}) = b_{i}c_{j}$, כלומר
\[
A = \begin{bmatrix}\\
b_{1}c_{1} & \dots & b_{1}c_{n}\br
\vdots & \ddots & \vdots\br
b_{n}c_{1} & \dots & b_{n}c_{n}\\\\\\
\end{bmatrix} \logr
[A]^{R}_{i} = b_{i} \cdot \vec{c}
\]
ולכן $R_{A} \subseteq \Sp{\{\vec{c}\}}$\footnote{$R_{A}$ -- מרחב השורות של $A$}, ולכן $\rho(f) \le 1$. אך מכיוון ש$f \neq 0$, בהכרח $\rho(f) \neq 0$ ולכן $\rho(f) = 1$ כנדרש.

\subsection*{כיוון ב -- דרגה $\Leftarrow$ הצגה}
נניח כי $\rho(A) = 1$. ידוע מלינארית 1 כי קיים $\w \in \digamma^{n}$ כך ש$C_{A} \subseteq \Sp\{\w\}$\footnote{$C_{A}$ -- מרחב העמודות של $A$} או $R_{A} \subseteq \Sp\{\w\}$. נחלק, אם כן, למקרים:\\
אם $R_{A} \subseteq \Sp\{\w\}$, נוכל להניח בהכ כי $\w = \vec{c}$. לכן $\forall 1 \le i \le n \exists b_{i} \in \digamma,\ [A]^{R}_{i} = b_{i} \cdot \vec{c}$.\\
אם $C_{A} \subseteq \Sp\{\w\}$, נוכל להניח בהכ כי $\w = \vec{b}$. לכן $\forall 1 \le i \le n \exists c_{i} \in \digamma,\ [A]^{C}_{i} = c_{i} \cdot \vec{b}$.

בשני המקרים מתקבלת המטריצה הבאה:
\[
A = \begin{bmatrix}\\
b_{1}c_{1} & \dots & b_{1}c_{n}\br
\vdots & \ddots & \vdots\br
b_{n}c_{1} & \dots & b_{n}c_{n}\\\\\\
\end{bmatrix}
\]
לכן לפי \ms{01}{8}{} קיימת תבנית בילינארית המוגדרת באופן הבא:
\[
f(\vec{x}, \vec{y}) = [\vec{x}]^{t}_{B}A[\vec{y}]_{B} = \sum^{n}_{i=1}\sum^{n}_{j=1}b_{i}c_{j}x_{i}y_{j} = \left(\sum^{n}_{i=1}b_{i}x_{i}\right)\left(\sum^{n}_{j=1}c_{j}y_{j}\right) = T(\vec{x})S(\vec{y})
\]
והראינו את ההצגה כמכפלה של שתי תבניות לינאריות כנדרש.
\br\qed
%%%%% </Q2> %%%%%

\end{document}