\documentclass[11pt, oneside]{article}
\usepackage{geometry}
\geometry{a4paper}
\usepackage[parfill]{parskip}
\usepackage[nodisplayskipstretch]{setspace}
\usepackage{graphicx,titlesec}
\usepackage{amsmath,amssymb,cancel,mathtools}

% Hebrew Stuff
\usepackage[utf8x]{inputenc}
\usepackage[english,hebrew]{babel}
\usepackage{hebfont}

% Custom commands
\newcommand{\qed}{\R{$\blacksquare$}}
\newcommand{\br}{\\\\\\\\\\\\}
\newcommand{\opr}[1]{\xrightarrow[\text{#1}]{}}
\newcommand{\bidiarrow}[1]{\underset{\text{#1}}{\leftrightarrow}}
\newcommand{\ueq}[1]{\underset{\text{#1}}{=}}
\newcommand{\mC}{\mathbb{C}}
\newcommand{\mR}{\mathbb{R}}
\newcommand{\mN}{\mathbb{N}}
\newcommand{\mZ}{\mathbb{Z}}
\newcommand{\mQ}{\mathbb{Q}}
\newcommand{\inv}[1]{#1^{-1}}

% Custom text commands (for Hebrew)
\newcommand{\q}[3]{\R{שאלה #3#2.#1}}
\newcommand{\m}[3]{\R{משפט #3#2.#1}}
\newcommand{\h}[3]{\R{הגדרה #3#2.#1}}
\newcommand{\ms}[3]{\R{מסקנה #3#2.#1}}

% Custom commands for this document
\DeclareMathOperator{\tr}{tr}
\newcommand{\nimuk}[1]{#1 \text{\R{נימוק: }}}
\newcommand{\mfr}[4]{\begin{bmatrix}\\#1 & #2\br #3 & #4\\\\\end{bmatrix}}
\newcommand{\eon}{\mfr{1}{0}{0}{0}}
\newcommand{\etw}{\mfr{0}{1}{0}{0}}
\renewcommand{\eth}{\mfr{0}{0}{1}{0}}
\newcommand{\efo}{\mfr{0}{0}{0}{1}}
\def\M{\mfr{1}{2}{3}{5}}

% Spacing
\titlespacing\subsubsection{0pt}{5pt}{3pt}
\setstretch{0.1}

\title{ממן 31}
\author{יונתן אוחיון}

\begin{document}
\maketitle

%%%%% <Q1> %%%%%
\section*{שאלה 1}
\subsection*{סעיף א}
ראשית, נפשט מעט את הביטוי של $f(A, B)$:
\begin{align*}
f(A, B)
&= \tr{A^{t}MB}\\
\nimuk{\tr{A} = \tr{A^{t}}} &= \tr{(B^{t}M^{t}A)^{t}}\\
&= \tr{B^{t}M^{t}A}
\end{align*}
בנוסף, ידוע לנו שמתקיים $f(B, A) = \tr{B^{t}MA}$, כלומר עלינו למצוא תנאי ש$M$ תקיים על מנת ש$\tr{B^{t}M^{t}A} = \tr{B^{t}MA}$. תנאי זה הוא כמובן תנאי הסימטריות, כלומר $M = M^{t}$.

לפיכך, $f(A, B) = f(B, A)$ אממ $M$ מטריצה סימטרית ומצאנו את התנאי הנדרש.
\br\qed

\subsection*{סעיף ב}
נחשב )בעמוד הזה ובעמוד הבא(:
\subsubsection*{שורה 1}
\begin{eqnarray*}
& f(e_{1}, e_{1}) = \tr{\eon \cdot \M \cdot \eon} = \tr\eon = 1\\\\
& f(e_{1}, e_{2}) = \tr{\eon \cdot \M \cdot \etw} = \tr\mfr{0}{1}{0}{0} = 0\\\\
& f(e_{1}, e_{3}) = \tr{\eon \cdot \M \cdot \eth} = \tr\mfr{2}{0}{0}{0} = 2\\\\
& f(e_{1}, e_{4}) = \tr{\eon \cdot \M \cdot \efo} = \tr\mfr{0}{2}{0}{0} = 0\\\\
& [A]^{R}_{1} = (f(e_{1}, e_{1}), f(e_{1}, e_{2}), f(e_{1}, e_{3}), f(e_{1}, e_{4})) = (1,0,2,0)
\end{eqnarray*}
\clearpage

\section*{שאלה 1 -- המשך}
\subsection*{סעיף ב}
\textbf{שורה 2}
\begin{eqnarray*}
& f(e_{2}, e_{1}) = \tr{\eth \cdot \M \cdot \eon} = \tr\eth = 0\\\\
& f(e_{2}, e_{2}) = \tr{\eth \cdot \M \cdot \etw} = \tr\efo = 1\\\\
& f(e_{2}, e_{3}) = \tr{\eth \cdot \M \cdot \eth} = \tr\mfr{0}{0}{2}{0} = 0\\\\
& f(e_{2}, e_{4}) = \tr{\eth \cdot \M \cdot \efo} = \tr\mfr{0}{0}{0}{2} = 2\\\\
& [A]^{R}_{2} = (f(e_{2}, e_{1}), f(e_{2}, e_{2}), f(e_{2}, e_{3}), f(e_{2}, e_{4})) = (0,1,0,2)
\end{eqnarray*}

\textbf{שורה 3}
\begin{eqnarray*}
& f(e_{3}, e_{1}) = \tr{\etw \cdot \M \cdot \eon} = \tr\mfr{3}{0}{0}{0} = 3\\\\
& f(e_{3}, e_{2}) = \tr{\etw \cdot \M \cdot \etw} = \tr\mfr{0}{3}{0}{0} = 0\\\\
& f(e_{3}, e_{3}) = \tr{\etw \cdot \M \cdot \eth} = \tr\mfr{5}{0}{0}{0} = 5\\\\
& f(e_{3}, e_{4}) = \tr{\etw \cdot \M \cdot \efo} = \tr\mfr{0}{5}{0}{0} = 0\\\\
& [A]^{R}_{3} = (f(e_{3}, e_{1}), f(e_{3}, e_{2}), f(e_{3}, e_{3}), f(e_{3}, e_{4})) = (3,0,5,0)
\end{eqnarray*}

\textbf{שורה 4}
\begin{eqnarray*}
& f(e_{4}, e_{1}) = \tr{\efo \cdot \M \cdot \eon} = \tr\mfr{0}{0}{3}{0} = 0\\\\
& f(e_{4}, e_{2}) = \tr{\efo \cdot \M \cdot \etw} = \tr\mfr{0}{0}{0}{3} = 3\\\\
& f(e_{4}, e_{3}) = \tr{\efo \cdot \M \cdot \eth} = \tr\mfr{0}{0}{5}{0} = 0\\\\
& f(e_{4}, e_{4}) = \tr{\efo \cdot \M \cdot \efo} = \tr\mfr{0}{5}{0}{5} = 5\\\\
& [A]^{R}_{4} = (f(e_{4}, e_{1}), f(e_{4}, e_{2}), f(e_{4}, e_{3}), f(e_{4}, e_{4})) = (0,3,0,5)
\end{eqnarray*}
בעמוד הבא נראה את $[f]_{E}$.
\clearpage

\section*{שאלה 1 -- המשך}
\subsection*{סעיף ב}
כעת, לאחר שחישבנו את שורות $[f]_{E}$, נציב:
\[
[f]_{E} = \begin{bmatrix}\\\\\\
1 & 0 & 2 & 0\br
0 & 1 & 0 & 2\br
3 & 0 & 5 & 0\br
0 & 3 & 0 & 5\\\\\\\\
\end{bmatrix}
\]
ומצאנו את המטריצה המייצגת לפי הבסיס הסטנדרטי כנדרש.
\br\qed

\end{document}