\documentclass[11pt, oneside]{article}
\usepackage{geometry}
\geometry{a4paper}
\usepackage[parfill]{parskip}
\usepackage[nodisplayskipstretch]{setspace}
\usepackage{graphicx,titlesec}
\usepackage{amsmath,amssymb,cancel}

% Hebrew Stuff
\usepackage[utf8x]{inputenc}
\usepackage[english,hebrew]{babel}
\usepackage{hebfont}

% Custom commands
\newcommand{\qed}{\R{$\blacksquare$}}
\newcommand{\br}{\\\\\\\\\\\\\\}
\newcommand{\opr}[1]{\xrightarrow[\text{#1}]{}}
\newcommand{\logr}[1]{\xRightarrow[\text{#1}]{}}
\newcommand{\bidiarrow}[1]{\underset{\text{#1}}{\leftrightarrow}}
\newcommand{\ueq}[1]{\underset{\text{#1}}{=}}
\newcommand{\mC}{\mathbb{C}}
\newcommand{\mR}{\mathbb{R}}
\newcommand{\mN}{\mathbb{N}}
\newcommand{\mZ}{\mathbb{Z}}
\newcommand{\mQ}{\mathbb{Q}}
\newcommand{\inv}[1]{#1^{-1}}

% Custom text commands (for Hebrew)
\newcommand{\q}[3]{\R{שאלה #3#2.#1}}
\newcommand{\m}[3]{\R{משפט #3#2.#1}}
\newcommand{\h}[3]{\R{הגדרה #3#2.#1}}
\newcommand{\ms}[3]{\R{מסקנה #3#2.#1}}

% Custom commands for this document
\DeclareMathOperator{\tr}{tr}

% Spacing
\titlespacing\subsubsection{0pt}{5pt}{3pt}
\setstretch{0.1}

\title{סיכום לקראת המבחן בלינארית 2}
\author{יונתן אוחיון}

\begin{document}
\maketitle

\clearpage

\section*{פתרון לשאלה 2ב}
השאלה: תהי $A \in M_{5 \times 5}^{\mR}$ המקיימת $\tr{A} = 0, \rho(A) = 1$. מצאו את הפולינום האופייני שלה ואת הפולינום המינימלי שלה. האם $A$ בהכרח לכסינה מעל $\mR$?

התשובה שלי:

מלינארית 1 נקבל כי $A$ לא הפיכה ולכן $\det{A} = 0$. לכן, 0 ע``''ע של $A$ והריבוב הגיאומטרי שלו )נסמן - $g_{0}$( הינו
\[
g_{0} = 5 - \rho(-A) = 5 - \rho(A) = 4
\]
כעת, נסמן את הריבוב האלגברי של 0 ב$a_{0}$. ידוע כי $g_{0} \le a_{0} \le n$ ולכן נקבל כי
\[
4 \le a_{0} \le 5
\]
כלומר, $a_{0} = 4$ או $a_{0} = 5$. נניח בשלילה כי $a_{0} = 4$. אזי נקבל כי מכיוון ש$P_{A}(\lambda)$ פולינום מתוקן ממעלה 5, יש לו עוד שורש שונה מ0. נסמן את השורש הזה ב$t$. לכן נוכל לראות כי מתקיים
\[
P_{A}(\lambda) = \lambda^{4}(\lambda - t) = \lambda^{5} - \lambda^{4}t
\]
מלינארית 1 ידוע כי המקדם של $\lambda^{4}$ שווה ל$-\tr{A}$, כלומר מתקיים $t = -\tr{A} = 0$ אבל $t \neq 0$ ולכן $a_{0} \neq 4$. לפיכך, נקבל כי $a_{0} = 5$ והפ``''א של $A$ הינו
\[
P_{A}(\lambda) = \lambda^{5}
\]
בנוסף, מכיוון ש$4 = g_{0} \neq a_{0} = 5$, המטריצה $A$ בהכרח אינה לכסינה מעל $\mR$. כעת, נחפש את הפולינום המינימלי של $A$. ידוע לנו כי $\rho(A) = 1$ ולכן הפולינום $m(t) = t$ אינו הפ``''מ של $A$.

\end{document}