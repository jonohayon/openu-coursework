\documentclass[11pt, oneside]{article}
\usepackage{geometry}
\geometry{a4paper}
\usepackage[parfill]{parskip}
\usepackage[nodisplayskipstretch]{setspace}
\usepackage{graphicx,titlesec}
\usepackage{fancyvrb}
\usepackage{amsmath,amssymb,cancel}

% Hebrew Stuff
\usepackage[utf8x]{inputenc}
\usepackage[english,hebrew]{babel}
\usepackage{hebfont}

% Custom commands
\newcommand{\qed}{\R{$\blacksquare$}}
\newcommand{\br}{\\\\\\\\\\\\\\}
\newcommand{\opr}[1]{\xrightarrow[\text{#1}]{}}
\newcommand{\logr}[1]{\xRightarrow[\text{#1}]{}}
\newcommand{\bidiarrow}[1]{\underset{\text{#1}}{\leftrightarrow}}
\newcommand{\ueq}[1]{\underset{\text{#1}}{=}}
\newcommand{\mC}{\mathbb{C}}
\newcommand{\mR}{\mathbb{R}}
\newcommand{\mN}{\mathbb{N}}
\newcommand{\mZ}{\mathbb{Z}}
\newcommand{\mQ}{\mathbb{Q}}
\newcommand{\inv}[1]{#1^{-1}}

% Custom text commands (for Hebrew)
\newcommand{\q}[3]{\R{שאלה #3#2.#1}}
\newcommand{\m}[3]{\R{משפט #3#2.#1}}
\newcommand{\h}[3]{\R{הגדרה #3#2.#1}}
\newcommand{\ms}[3]{\R{מסקנה #3#2.#1}}

% Custom commands for this document

% Spacing
\titlespacing\subsubsection{0pt}{0pt}{3pt}
\setstretch{0.9}

\title{הסבר על קטע הקוד הנתון}
\author{יונתן אוחיון, ת``''ז: 347811523}

\begin{document}
\maketitle
קטע הקוד הנתון )מוקף בפונקציה המשתמשת במצביעים(:
\begin{otherlanguage}{english}
\begin{Verbatim}[
	frame=single,
	framesep=2.5mm
]
void swap (int *A, int *B) {
  *A ^= *B;
  *B ^= *A;
  *A ^= *B;
}
\end{Verbatim}
\end{otherlanguage}

\subsubsection*{הסבר}
קטע הקוד שלעיל מבצע החלפה בין שני משתנים בעזרת פעולת ה\L{XOR} וללא צורך במשתנים נוספים. דבר זה מתאפשר באמצעות תכונותיה של פעולת ה\L{XOR}, אשר מקיימת את תכונות הקומוטטיביות )כלומר מתקיים \L{\verb!A^B == B^A!}(, האסוציאטיביות )\L{\verb!A^(B^C) == (A^B)^C!}( וקיום איבר ניטרלי )\L{\verb!A^0 == A,A^A == 0!}(. הפעולה עובדת באופן הבא: בהינתן שני משתנים \L{\verb!A, B!} יתקיים
\begin{otherlanguage}{english}
\begin{Verbatim}[
	frame=single,
	framesep=2.5mm
]
Initial values: A = a, B = b
Procedure:
  1. A = A ^ B = a ^ b, B = b
  2. B = B ^ A = b ^ (a ^ b) = (b ^ b) ^ a = a, A = a ^ b
  3. A = A ^ B = (a ^ b) ^ a = b ^ (a ^ a) = b
Final values: A = b, B = a
\end{Verbatim}
\end{otherlanguage}

\subsubsection*{דוגמה מספרית}
נפעיל את הפעולה על שני מספרים: \L{\verb!A = 5, B = 6!} )בהצגה בינארית(:
\begin{otherlanguage}{english}
\begin{Verbatim}[
	frame=single,
	framesep=2.5mm
]
1. A = A ^ B = 101 ^ 110 = 011, B = 110
2. B = B ^ A = 110 ^ 011 = 101, A = 011
3. A = A ^ B = 011 ^ 101 = 110, B = 101
\end{Verbatim}
\end{otherlanguage}
ערכם הסופי של המשתנים בהצגה בינארית הינו \L{\verb!A = 110, B = 101!} ובבסיס 01 \L{\verb!A = 6, B = 5!}. כפי שניתן לראות, כל משתנה קיבל את ערכו ההתחלתי של המשתנה השני.

\end{document}