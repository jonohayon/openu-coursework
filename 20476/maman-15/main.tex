\documentclass[11pt, oneside]{article}
\usepackage{geometry}
\geometry{a4paper}
\usepackage[parfill]{parskip}
\usepackage[nodisplayskipstretch]{setspace}
\usepackage{graphicx,titlesec}
\usepackage{amsmath,amssymb,cancel}

% Hebrew Stuff
\usepackage[utf8x]{inputenc}
\usepackage[english,hebrew]{babel}
\usepackage{hebfont}

% Custom commands
\newcommand{\qed}{\R{$\blacksquare$}}
\newcommand{\br}{\\\\\\\\\\\\\\}
\newcommand{\opr}[1]{\xrightarrow[\text{#1}]{}}
\newcommand{\bidiarrow}[1]{\underset{\text{#1}}{\leftrightarrow}}
\newcommand{\ueq}[1]{\underset{\text{#1}}{=}}
\newcommand{\mR}{\mathbb{R}}
\newcommand{\mN}{\mathbb{N}}
\newcommand{\mZ}{\mathbb{Z}}
\newcommand{\mQ}{\mathbb{Q}}
\newcommand{\inv}[1]{#1^{-1}}

% Custom text commands (for Hebrew)
\newcommand{\q}[3]{\R{שאלה #3#2.#1}}
\newcommand{\m}[3]{\R{משפט #3#2.#1}}
\newcommand{\h}[3]{\R{הגדרה #3#2.#1}}
\newcommand{\ms}[3]{\R{מסקנה #3#2.#1}}

% Custom commands for this document

% Spacing
\titlespacing\subsubsection{0pt}{5pt}{3pt}
\setstretch{0.1}

\title{ממן 51}
\author{יונתן אוחיון}

\begin{document}
\maketitle

%%%%% <Q1> %%%%%
\section{שאלה 1}
\subsection{סעיף א}
ראשית, נחשב את $a_0, a_1, a_2$.
כמות המחרוזות באורך $0$ המקיימות את התנאי הינה $1$ )המחרוזת הריקה(. לפיכך, $a_0 = 1$.
כמות המחרוזות באורך $1$ המקיימות את התנאי הינה $3$ )שכן $|\{0, 1, 2\}| = 3$( ולכן $a_1 = 3$.
כמות המחרוזות באורך $2$ המקיימות את התנאי הינה $7$ )שכן כמות המחרוזות באורך $2$ הינה $|A \times A| = 3^2 = 9$, אך יש $2$ מחרוזות באורך $2$ שאינן מקיימות את התנאי ו$9 - 2 = 7$(.

אם מחרוזת באורך $n$ מתחילה ב$0$, אזי היא תהיה חייבת להתחיל ב$02$ והמשך הסדרה הינו באורך $n - 2$.
אם מחרוזת באורך $n$ מתחילה ב$1$ או ב$2$, אזי המשך הסדרה הינו באורך $n - 1$.
לפיכך, זהו יחס הנסיגה:
\[
a_n = 2a_{n - 1} + a_{n - 2}
\]
נראה שהיחס נכון ע"י הצבה של $n = 2$:
\[
7 = a_2 = 2a_1 + a_0 = 2 \cdot 3 + 1 = 7
\]
כנדרש.
\br\qed

\subsection{סעיף ב}
המשוואה האופיינית של הסדרה הינה
\begin{align*}
& \lambda^n = 2\lambda^{n-1} + \lambda^{n-2}\;/:\lambda^{n-2}\\
& \lambda^2 = 2\lambda + 1\\
& \lambda^2 - 2\lambda - 1 = 0
\end{align*}
נפתור:
\[
\lambda = \frac{2 \pm \sqrt{4 + 4}}{2} = \frac{2 \pm 2\sqrt{2}}{2} = 1 \pm \sqrt{2}
\]
כלומר $a_n = A(1 + \sqrt{2})^n + B(1 - \sqrt{2})^n$. בעמוד הבא נחשב את ערכי $A$ ו$B$.
\clearpage

\setcounter{subsection}{0}
\subsection{סעיף א )המשך(}
כעת, נציב את תנאי ההתחלה ונקבל:
\begin{align*}
1 & = a_0 = A + B\\
3
& = a_1\\
& = A(1 + \sqrt{2}) + B(1 - \sqrt{2})\\
& = A + \sqrt{2}A + B - \sqrt{2}B\\
& = A + B + \sqrt{2}(A - B)\\
2 & = \sqrt{2}(A - B)\;/:\sqrt{2}\\
\sqrt{2} & = A - B
\end{align*}
כעת, נוכל למצוא את ערכי $A$ ו$B$ באופן הבא:
\begin{align*}
& A + B + (A - B) = 1 + \sqrt{2}\\
& 2A = 1 + \sqrt{2}\;/:2\\
& A = \frac{1}{2}(1 + \sqrt{2})\\
& A + B - (A - B) = 1 - \sqrt{2}\\
& 2B = 1 - \sqrt{2}\;/:2\\
& B = \frac{1}{2}(1 - \sqrt{2})
\end{align*}
כעת, נוכל להציב בנוסחה שמצאנו לאיברי הסדרה ולקבל את הנוסחה המפורשת ל$a_n$:
\begin{align*}
a_n
& = A(1 + \sqrt{2})^n + B(1 - \sqrt{2})^n\\
& = \frac{1}{2}(1 + \sqrt{2})(1 + \sqrt{2})^n + \frac{1}{2}(1 - \sqrt{2})(1 - \sqrt{2})^n\\
& = \frac{1}{2}(1 + \sqrt{2})^{n + 1} + \frac{1}{2}(1 - \sqrt{2})^{n + 1}\\
& = \frac{1}{2}((1 + \sqrt{2})^{n + 1} + (1 - \sqrt{2})^{n + 1})
\end{align*}
לפיכך, $a_n = \frac{1}{2}((1 + \sqrt{2})^{n + 1} + (1 - \sqrt{2})^{n + 1})$ כנדרש.
\br\qed
\clearpage
%%%%% </Q1> %%%%%

%%%%% <Q2> %%%%%
\section{שאלה 2}
מכיוון שערכי כל המשתנים גדולים מ$1$, נוכל לסמנם ב$x_n = y_n + 2$, כלומר המשוואה הינה
\begin{align*}
& y_1 + y_2 + y_3 + y_4 + y_5 + 10 = 24\\
& y_1 + y_2 + y_3 + y_4 + y_5 = 14
\end{align*}
כעת, שלושה משתנים הינם זוגיים ושניים הינם אי-זוגיים, ויש ${5 \choose 3} = 10$ דרכים לבחור את המשתנים הזוגיים מתוך כלל המשתנים. כעת, נשער בה"כ ששני המשתנים הראשונים הינם אי-זוגיים והשאר זוגיים ונסמן
\[
y_i = \begin{cases}
2z_i + 1, & i \le 2\\\\
2z_i, & \text{\R{אחרת}}
\end{cases}
\]
כעת, נציב ונראה מה המשוואה:
\begin{align*}
& 2z_1 + 1 + 2z_2 + 1 + 2z_3 + 2z_4 + 2z_5 = 14\\
& 2z_1 + 2z_2 + 2z_3 + 2z_4 + 2z_5 = 12\\
& z_1 + z_2 + z_3 + z_4 + z_5 = 6
\end{align*}
וכעת אין שום הגבלה על המשתנים ונוכל לבדוק כמה דרכים לבחור אותם יש. בעיה זו הינה אותה בעיה כמו לזרוק $6$ כדורים ל$5$ תאים, כלומר מספר הדרכים הינו
\[
D(5, 6) = {5 + 6 - 1 \choose 4} = {10 \choose 4}
\]
נכפיל את מספר זה במספר הדרכים לבחור את המשתנים הזוגיים ונקבל
\[
{10 \choose 4} \cdot {5 \choose 3} = 210 \cdot 10 = 2100
\]
לכן, יש $2100$ פתרונות אפשריים למשוואה.
\br\qed
%%%%% </Q2> %%%%%

\end{document}