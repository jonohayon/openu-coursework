\documentclass[11pt, oneside]{article}
\usepackage{geometry}
\geometry{a4paper}
\usepackage[parfill]{parskip}
\usepackage[nodisplayskipstretch]{setspace}
\usepackage{graphicx,titlesec}
\usepackage{amsmath,amssymb,cancel}

% Hebrew Stuff
\usepackage[utf8x]{inputenc}
\usepackage[english,hebrew]{babel}
\usepackage{hebfont}

% Custom commands
\newcommand{\qed}{\R{$\blacksquare$}}
\newcommand{\br}{\\\\\\\\\\\\\\}
\newcommand{\opr}[1]{\xrightarrow[\text{#1}]{}}
\newcommand{\bidiarrow}[1]{\underset{\text{#1}}{\leftrightarrow}}
\newcommand{\ueq}[1]{\underset{\text{#1}}{=}}
\newcommand{\mR}{\mathbb{R}}
\newcommand{\mN}{\mathbb{N}}
\newcommand{\mZ}{\mathbb{Z}}
\newcommand{\mQ}{\mathbb{Q}}
\newcommand{\inv}[1]{#1^{-1}}

% Custom text commands (for Hebrew)
\newcommand{\q}[3]{\R{שאלה #3#2.#1}}
\newcommand{\m}[3]{\R{משפט #3#2.#1}}
\newcommand{\h}[3]{\R{הגדרה #3#2.#1}}
\newcommand{\ms}[3]{\R{מסקנה #3#2.#1}}
\newcommand{\at}[1]{\;\text{\R{)לפי סעיף #1(}}}
\newcommand{\prev}{\;\text{\R{)לפי הסעיף הקודם(}}}

% Custom commands for this document

% Spacing
\titlespacing\subsubsection{0pt}{5pt}{3pt}
\setstretch{0.1}

\title{ממן 31}
\author{יונתן אוחיון}

\begin{document}
\maketitle

%%%%% <Q1> %%%%%
\section{שאלה 1}
\subsection{הקבוצות}
\begin{align*}
& A = [-1, 1]\\
& B = [0, 2]\\
& A - B = [-1, 0)\\
& A \oplus B = \{ x \in \mR \mid x \in [-1, 2] \;\land\; x \not\in (0, 1) \}\\
& A \cup B = [-1, 2]
\end{align*}

\subsection{עוצמת הקבוצות}
לפי \m{4}{5}{}, ניתן להסיק כי העוצמה של קטעים פתוחים וסגורים הינה $C$. לפיכך, כל הקבוצות $A, B, A - B, A \oplus B, A \cup B$ הינם שוות-עוצמה ועוצמתן הינה $C$.
\subsection{שונות הקבוצות}
א. נראה ש$A$ שונה משאר הקבוצות:
\begin{align*}
& -1 \in A \land -1 \not\in B \opr{} A \neq B\\
& 0 \in A \land 0 \not\in A - B \opr{} A \neq A - B\\
& 0.5 \in A \land 0.5 \not\in A \oplus B \opr{} A \neq A \oplus B\\
& 2 \not\in A \land 2 \in A \cup B \opr{} A \neq A \cup B
\end{align*}
ב. נראה ש$B$ שונה משאר הקבוצות:
\begin{align*}
& A \neq B \at{א}\\
& 0 \in B \land 0 \not\in A - B \opr{} B \neq A - B\\
& 0.5 \in B \land 0.5 \not\in A \oplus B \opr{} B \neq A \oplus B\\
& -1 \not\in B \land -1 \in A \cup B \opr{} B \neq A \cup B
\end{align*}
\clearpage

ג. נראה ש$A - B$ שונה משאר הקבוצות:
\begin{align*}
& A \neq A - B \at{א}\\
& B \neq A - B \at{ב}\\
& 2 \not\in A - B \land 2 \in A \oplus B \opr{} A - B \neq A \oplus B\\
& 2 \not\in A - B \land 2 \in A \cup B \opr{} A - B \neq A \cup B\\
\end{align*}
ד. נראה ש$A \oplus B$ שונה משאר הקבוצות:
\begin{align*}
& A \neq A \oplus B \at{א}\\
& B \neq A \oplus B \at{ב}\\
& A - B \neq A \oplus B \at{ג}\\
& 0.5 \not\in A \oplus B \land 0.5 \in A \cup B \opr{} A \oplus B \neq A \cup B
\end{align*}
ה. נראה ש$A \cup B$ שונה משאר הקבוצות:
\begin{align*}
& A \neq A \cup B \at{א}\\
& B \neq A \cup B \at{ב}\\
& A - B \neq A \cup B \at{ג}\\
& A - B \neq A \cup B \at{ד}\\
\end{align*}
לפיכך, הצלחנו למצוא קבוצות $A, B, A - B, A \oplus B, A \cup B$ כך שעוצמותיהן שוות אך הן שונות אחת מהשנייה.
\br\qed
\clearpage
%%%%% </Q1> %%%%%

%%%%% <Q2> %%%%%
\section{שאלה 2}
\subsection{סעיף א}
יהי $n \in \mN \land n > 0$.

תהי $T_n$ קבוצת התת-קבוצות של $\mN$ באורך $n$ )כלומר, $T_n = \left\{X \mid X \in \mathcal{P}(\mN) \land |X| = n\right\}$(. תהי $F_n$ קבוצת הסדרות באורך $n$ מעל $\mN$ )כלומר, $F_n = \mN\times\mN\times\mN\times\dots\times\mN$(. לפיכך, קיימת פונקציה $f: T_n \opr{} F_n$ המתאימה לכל קבוצה ב$T_n$ את סדרת המספרים הנמצאים בה בסדר עולה ב$F_n$.

פונקציה זו הינה חח"ע, שכן לכל קבוצה $P$ ב$T_n$ נוכל להתאים סדרה של איברה בסדר עולה, ולכן $|T_n| \le |F_n|$ )פונקציה זו אינה על, שכן לכל קבוצה תותאם רק סדרה אחת עם איבריה אך קיימות יותר מסדרה אחת שכזו ב$F_n$(. הקבוצה $T_n$ הינה אינסופית, מכיוון שלכל $P \in T_n$ קיימת $M \in T_n$ כך ש$|P| = |M| - 1$, ולכן עוצמתה היא לפחות $\aleph_0$ והיא בת מנייה.
\br\qed

\subsection{סעיף ב}
לפי סעיף א, הוכחנו שלכל $n > 0$, קבוצת התת-קבוצות של $\mN$ באורך $n$ )מסומנת ב$T_n$( היא בת מנייה. מכיוון שאיחוד של קבוצות בנות מננייה הינו קבוצה בת מנייה בעצמו, נוכל לייצג את קבוצת תת הקבוצות בנות המנייה כך:
$$M = \bigcup^\infty_{0 < i \le n} T_i$$
נוכל להבחין ש$i > 0$, ולכן קבוצה זו לא כוללת את קבוצת תת הקבוצות של $\mN$ שעוצמתן שווה ל$0$, כלומר את הקבוצה $\{\emptyset\}$, שהיא כמובן חלק מקבוצות תת הקבוצות הסופיות של $\mN$. לפיכך,
$$M = \bigcup^\infty_{0 \le i \le n} T_i$$
לפיכך, קבוצת תת הקבוצות הסופיות של $\mN$ הינה בת מנייה.
\br\qed

\subsection{סעיף ג}
תהי $L$ קבוצת תת הקבוצות האינסופיות של $\mN$. נניח בשלילה ש$|L| = \aleph_0$ ונגיע לסתירה:

י$\mathcal{P}(\mN)$ היא קבוצת כל תת הקבוצות של $\mN$. לפיכך, $\mathcal{P}(\mN) = M \cup L$. לפיכך,
$$|M| + |L| = |\mathcal{P}(\mN)| \opr{\at{ב}} \aleph_0 + \aleph_0 = C$$
והגענו לסתירה. לפיכך, $L$ היא אינה בת מנייה.
\br\qed

\subsection{סעיף ד}
לפי סעיף ג, נוכל להיווכח בכך ש$L$ אינה בת מנייה. לפיכך, $|L| = C$.
\br\qed

\subsection{סעיף ה}
\subsubsection{$i$}
$$\aleph_0 = |\{\emptyset\} \cup \left\{\left\{X \in \mathcal{P}(\mN) \mid |X| = i \right\} \mid i \in \mN \land i > 0 \right\}|$$
\qed

\subsubsection{$ii$}
$$C = |\left\{X \in \mathcal{P}(\mN) \mid |X| = \aleph_0 \right\}|$$
\qed
%%%%% </Q2> %%%%%

%%%%% <Q3> %%%%%
\section{שאלה 3}
דוגמה נגדית:
\begin{align*}
& A = \{1, 2, 3\},\ B = \{2, 3\},\ A \oplus B = \{1\}\\
& |A| = 3,\ |B| = 2,\ |A \oplus B| = 1\\
& |A| \oplus |B| = (|A| - |B|) + (|B| - |A|) = 1 - 1 = 0\\
& |A| \oplus |B| \neq |A \oplus B|
\end{align*}
\qed
%%%%% </Q3> %%%%%

\end{document}