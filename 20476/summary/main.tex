\documentclass[11pt, oneside]{article}
\usepackage{geometry}
\geometry{a4paper}
\usepackage[parfill]{parskip}
\usepackage[nodisplayskipstretch]{setspace}
\usepackage{graphicx,titlesec}
\usepackage{mathtools,amsmath,amssymb,cancel,commath,nicefrac}

% Hebrew Stuff
\usepackage[utf8x]{inputenc}
\usepackage[english,hebrew]{babel}
\usepackage{hebfont}

% Custom commands
\newcommand{\qed}{\R{$\blacksquare$}}
\newcommand{\br}{\\\\\\\\\\\\\\}
\newcommand{\opr}[1]{\underset{\text{#1}}{\Rightarrow}}
\newcommand{\bidiarrow}[1]{\underset{\text{#1}}{\Leftrightarrow}}
\newcommand{\ueq}[1]{\underset{\text{#1}}{=}}
\newcommand{\mR}{\mathbb{R}}
\newcommand{\mN}{\mathbb{N}}
\newcommand{\mZ}{\mathbb{Z}}
\newcommand{\mQ}{\mathbb{Q}}
\newcommand{\inv}[1]{#1^{-1}}

% Custom text commands (for Hebrew)
\newcommand{\q}[3]{\R{שאלה #3#2.#1}}
\newcommand{\m}[3]{\R{משפט #3#2.#1}}
\newcommand{\h}[3]{\R{הגדרה #3#2.#1}}
\newcommand{\ms}[3]{\R{מסקנה #3#2.#1}}

% Custom commands for this document
\newcommand{\definition}[2]{\textbf{#1}: #2.}
\DeclarePairedDelimiter\parens{\lparen}{\rparen}
\DeclareMathOperator{\dnp}{Domain}
\newcommand{\domain}[1]{\dnp\parens*{#1}}
\DeclareMathOperator{\rnp}{Range}
\newcommand{\range}[1]{\rnp\parens*{#1}}
\DeclareMathOperator{\imnp}{Im}
\newcommand{\image}[1]{\imnp\parens*{#1}}

% Spacing
\titlespacing\section{0pt}{0pt}{0pt}
\titlespacing\subsection{0pt}{0pt}{0pt}
\titlespacing\subsubsection{0pt}{0pt}{0pt}

\title{מתמטיקה בדידה -- סיכום\thanks{נכתב לצורך הקורס באוניברסיטת תל אביב ונערך לתכני הקורס באוניברסיטה הפתוחה.}}
\author{שירה ברמן )נערך ע``''י יונתן אוחיון(}

\begin{document}
\maketitle

\section*{לוגיקה}
\subsection*{הגדרות}
\begin{itemize}
\item \definition{הצרנה}{תרגום משפה טבעית לשפה פורמלית}
\item \definition{טאוטולוגיה}{פסוק המקבל ארך אמת ללא תלות בערך האמת של הפסוקים האטומים שלו}
\item \definition{סתירה}{פסוק המקבל ערך שקרי ללא תלות בערכי האמת של הפסוקים האטומים שלו}
\item \definition{שקילות )טאוטולוגית(}{שני פסוקים בעלי אותה טבלת אמת ייקראו שקולים או שקולים טאוטולוגית}
\end{itemize}
\bigskip

\begin{center}
\textit{טבלה 1 -- הקשרים והכמתים הלוגיים וסימונם}\par
\begin{tabular}{| c | c |}
\hline
\R{הסימן} & \R{שם בעברית}\\
\hline
$\neg$ & \R{שלילה}\\
$\land$ & \R{וגם}\\
$\lor$ & \R{או}\\
$\opr{}, \rightarrow$ & \R{גרירה}\\
$\bidiarrow{}, \leftrightarrow$ & \R{אממ}\\
$\forall$ & \R{לכל}\\
$\exists$ & \R{קיים}\\
$\equiv$ & \R{שקילות לוגית}\\
\hline
\end{tabular}
\end{center}\bigskip

\begin{center}
\textit{טבלה 2 -- טבלת האמת עבור הקשרים}\par
\begin{tabular}{| c | c | c | c | c | c | c |}
\hline
$\alpha$ & $\beta$ & $\neg \alpha$ & $\alpha \land \beta$ & $\alpha \lor \beta$ & $\alpha \opr{} \beta$ & $\alpha \bidiarrow{} \beta$\\
\hline
$T$ & $T$ & $F$ & $T$ & $T$ & $T$ & $T$\\
\hline
$T$ & $F$ & $F$ & $F$ & $T$ & $F$ & $F$\\
\hline
$F$ & $T$ & $T$ & $F$ & $T$ & $T$ & $F$\\
\hline
$F$ & $F$ & $T$ & $F$ & $F$ & $T$ & $T$\\
\hline
\end{tabular}
\end{center}\bigskip

\section*{לוגיקה -- המשך}
\subsection*{שקילויות לוגיות}
\subsubsection*{חוק החילוף )קומוטטיביות(}
\begin{itemize}
\item $a \lor b \equiv b \lor a$
\item $a \land b \equiv b \land a$
\item $a \bidiarrow{} b \equiv b \bidiarrow{} a$
\end{itemize}

\subsubsection*{חוק הקיבוץ )אסוציאטיביות(}
\begin{itemize}
\item $a \lor (b \lor c) \equiv (a \lor b) \lor c$
\item $a \land (b \land c) \equiv (a \land b) \land c$
\end{itemize}

\subsubsection*{חוק הפילוג )דיסטריבוטיביות(}
\begin{itemize}
\item $a \land (b \lor c) \equiv (a \land b) \lor (a \land c)$
\item $a \lor (b \land c) \equiv (a \lor b) \land (a \lor c)$
\end{itemize}

\subsubsection*{חוקי דה-מורגן}
\begin{itemize}
\item $\neg (a \land b) \equiv \neg a \lor \neg b$
\item $\neg (a \lor b) \equiv \neg a \land \neg b$
\end{itemize}

\subsubsection*{גרירה}
\begin{itemize}
\item $(a \lor b) \opr{} c \equiv (a \opr{} c) \land (b \opr{} c)$
\item $a \opr{} (b \land c) \equiv (a \opr{} b) \land (a \opr{} c)$
\end{itemize}

\subsubsection*{כללים נוספים}
\begin{itemize}
\item $\neg (\neg a) \equiv a$
\item $a \opr{} b \equiv \neg b \opr{} \neg a$
\item $a \bidiarrow{} b \equiv b \bidiarrow{} a$
\item $a \opr{} b \equiv (\neg a) \lor b$
\item $a \lor (a \land b) \equiv a \equiv a \land (a \lor b)$
\end{itemize}
\bigskip

\definition{נביעה לוגית}{טענה $b$ נובעת מטענות $a_{1}, a_{2}, \dots , a_{k}$ אם $b$ נכונה בכל פירוש שבו $a_{1}, \dots, a_{k}$ נכונות}
\definition{קבוצה שלמה}{קבוצת קשרים נקראת שלמה אם ניתן לבטא בעזרתה כל פסוק}
\clearpage

\section*{לוגיקה -- המשך}
\subsection*{כמתים}
כשמוכיחים נכונות של פסוק עם $\exists$ )קיים( מתחילים ב -- נבחר $\dots x$\\
כשמוכיחים נכונות של פסוק עם $\forall$ )לכל( מתחילים ב -- יהי $\dots x$
\subsubsection*{שלילת פסוק}
\begin{itemize}
\item $\neg (\forall a, P) \equiv \exists a, \neg P$
\item $\neg (\exists a, P) \equiv \forall a, \neg P$
\end{itemize}

\subsubsection*{שקילויות}
\begin{itemize}
\item $\forall a, (P \land Q) \equiv (\forall a, P) \land (\forall a, Q)$
\item $\exists a, (P \lor Q) \equiv (\exists a, P) \lor (\exists a, Q)$
\end{itemize}

\subsubsection*{החלפת סדר}
\begin{itemize}
\item $\forall a \forall b, P \equiv \forall b \forall a, P \equiv \forall (a, b), P$
\item $\exists a \exists b, P \equiv \exists b \exists a, P \equiv \exists (a, b), P$
\end{itemize}
\clearpage

\section*{תורת הקבוצות}
\subsection*{הגדרות}
\begin{itemize}
\item \definition{קבוצה}{אוסף של עצמים )המהווה עצם בעצמה(. אין חשיבות לסדר האיסרים בקבוצה ואין חשיבות למספר המופעים של איבר בייצוג הקבוצה}
\item \definition{שייכות לקבוצה}{תהי $A$ קבוצה ו$x$ איבר בה. אזי $x$ ייקרא שייך ל$A$ ויסומן כך: $x \in A$}
\item \definition{הכלה}{קבוצה $A$ תיקרא מוכלת בקבוצה $B$ אם כל איבר של $A$ שייך גם ל$B$. פורמלית:\\
$A \subseteq B \bidiarrow{} \forall x, (x \in A \Rightarrow x \in B)$\\
הכלה הינה טרנזיטיבית, כלומר $\forall A, B, C ((A \subseteq B) \land (B \subseteq C) \opr{} A \subseteq C)$}
\item \definition{הכלה ממש}{קבוצה $A$ תיקרא מוכלת ממש בקבוצה $B$ אם היא מוכלת ב$B$ אך אינה שווה לה. פורמלית:\\
$A \subset B \bidiarrow{} (A \subseteq B) \land (A \neq B)$}
\item \definition{שוויון קבוצות}{קבוצה $A$ תיקרא שווה ל$B$ )או $B$ שווה ל$A$( אמ``''מ מתקיימת הכלה דו כיוונית ביניהן, כלומר $A \subseteq B \land B \subseteq A$}
\item \definition{הקבוצה הריקה}{הקבוצה הריקה הנה קבוצה שאין בה איברים והיא מסומנת ב$\emptyset$. הגדרה פורמלית:
$\forall x, x \not\in \emptyset$. יש לציין שהקבוצה הריקה מוכלת בתוך כל קבוצה $A$ )כלומר $\forall A, \emptyset \subseteq A$(}
\item \definition{קבוצת החזקה}{
תהי $A$ קבוצה. קבוצת החזקה של $A$ )המסומנת כך: $\mathcal{P}(A)$( היא קבוצת תת-הקבוצות של $A$: $\mathcal{P}(A) = \left\{X \mid X \subseteq A\right\}$. עוצמת קבוצה זו הינה $\abs{\mathcal{P}(A)} = 2^{|A|}$}
\end{itemize}

\subsection*{פעולות יסודיות על קבוצות}
\begin{itemize}
\item \definition{איחוד קבוצות -- $A \cup B$}{$\forall x, x \in A \cup B \bidiarrow{} x \in A \lor x \in B$}
\item \definition{חיתוך קבוצות -- $A \cap B$}{$\forall x, x \in A \cap B \bidiarrow{} x \in A \land x \in B$}
\item \definition{הפרש קבוצות -- $A - B, A \setminus B$}{$\forall x, x \in A - B \bidiarrow{} x \in A \land x \not\in B$}
\item \definition{הפרש סימטרי של קבוצות -- $A \oplus B$}{$A \oplus B = (A - B) \cup (B - A) = (A \cup B) - (A \cap B )$}
\end{itemize}


\subsubsection*{תכונות הפעולות}
\begin{itemize}
\item \definition{קומוטטיביות}{$A \cup B = B \cup A, A \cap B = B \cap A$}
\item \definition{אסוציאטיביות}{$A \cup (B \cup C) = (A \cup B) \cup C, (A \cap B) \cap C = A \cap (B \cap C)$}
\item \definition{דיסטריבוטיביות 1}{$A \cup (B \cap C) = (A \cup B) \cap (A \cup C)$}
\item \definition{דיסטריבוטיביות 2}{$A \cap (B \cup C) = (A \cap B) \cup (A \cap C)$}
\end{itemize}
\clearpage

\section*{תורת הקבוצות -- המשך}
\subsection*{המשלים}
תהי $A$ קבוצה המוכלת בקבוצת עולם $U$. אזי נגדיר את המשלים של $A$ כך: $\overline{A} = A' = U - A$

\subsection*{חיתוך ואיחוד קבוצות מוכללים}
איחוד:
\[
\bigcup_{i \in \mN} A_{i} = \bigcup^{\infty}_{i = 0} A_{i} = \bigcup \left\{{A_{i} \mid i \in \mN}\right\}
\]
חיתוך:
\[
\bigcap_{i \in \mN} A_{i} = \bigcap^{\infty}_{i = 0} A_{i} = \bigcap \left\{{A_{i} \mid i \in \mN}\right\}
\]

\section*{רלציות )יחסים( -- הקדמה}
\subsection*{זוגות סדורים}
אוסף של שני איברים אשר אחד מהם נקבע כאיבר הראשון והשני כאיבר השני: $(\alpha, \beta)$. במקרה זה, $\alpha$ הוא האיבר הראשון ו$\beta$ הוא השני. שני זוגות סדורים $(\alpha, \beta), (\gamma, \delta)$ שווים זה לזה אמ``''מ $\alpha = \gamma \land \beta = \delta$. ניתן להכליל מושג זה למושג $n$-יה, שהיא אוסף של איברים המסודרים לפי $\mN$:
\[
(\lambda_{1}, \lambda_{2}, \dots, \lambda_{n})
\]
כאשר $\lambda_{1}$ האיבר הראשון ו$\lambda_{n}$ האחרון. שוויון הזוגות הסדורים פועל גם פה: בהינתן שתי $n$-יות $(\alpha_{1}, \dots, \alpha_{n}), (\beta_{1}, \dots, \beta_{n})$, הן ייקראו שוות זו לזו אמ``''מ
\[
\forall i \in \mN, 1 \le i \le n, (\alpha_{i} = \beta_{i})
\]

\subsection*{מכפלה קרטזית}
יהיו $A, B$ קבוצות. המכפלה הקרטזית של $A$ ו$B$ מוגדרת בתור קבוצת כל הזוגות הסדורים של איברי $A, B$ ומסומנת כך:
\[
A \times B = \left\{(\alpha, \beta) \mid \alpha \in A, \beta \in B \right\}
\]
פעולה זו אינה אסוציאטיבית ואם $A \neq B$ היא אינה קומוטטיבית. בנוסף, ניתן לבצע את הפעולה ככל שנרצה ולקבל $n$-יות בגדלים שונים:
\[
A_{1} \times A_{2} \times \dots \times A_{n} = \left\{(\lambda_{1}, \dots, \lambda_{n}) \mid \lambda_{i} \in A_{i}\right\}
\]
כתיבה אחרת היא ``''חזקה``'' של קבוצה והיא נראית כך:
\[
A^{n} = \underbrace{A \times \dots \times A}_{\text{\R{ פעמים}} n} = \left\{(\lambda_{1}, \dots, \lambda_{n}) \mid \lambda_{i} \in A\right\}
\]

\clearpage

\section*{רלציות )יחסים(}
רלציה )יחס( בינארית $R$ מהקבוצה $A$ לקבוצה $B$ הינה תת-קבוצה של $A \times B$ )כלומר $R \in \mathcal{P}(A \times B)$(. נסמן זוג סדור השייך לרלציה $R$ באופנים הבאים: $(\alpha, \beta) \in R \iff \alpha R \beta$. ניתן לתאר רלציות כקבוצות, גרף מכוון )דיגרף( או טבלה. אם $A = B$ אזי הרלציה מעל הקבוצה $A$.

\subsection*{תחום וטווח}
תהי $R$ רלציה מ$A$ ל$B$. אזי התחום של $R$ )מסומן $\domain{R}$( הינו תת-קבוצה של $A$ אשר בתוכה נמצאים כל האיברים של $A$ שמיוחסים לאיבר/ים כלשהם ב$B$, והוא מוגדרת כך:
\[
\domain{R} = \left\{\alpha \in A \mid \exists \beta \in B, (\alpha, \beta) \in R\right\}
\]
בדומה, הטווח של $R$ )מסומן $\range{R}$( הינו תת-קבוצה של $B$ אשר בתוכה נמצאים כל האיברים של $B$ אשר מיוחסים לאיבר/ים כלשהם ב$A$, והוא מוגדרת כך:
\[
\range{R} = \left\{\beta \in B \mid \exists \alpha \in A, (\alpha, \beta) \in R\right\}
\]

\subsection*{הרלציה ההופכית}
תהי $R$ רלציה מ$A$ ל$B$. אזי קיימת רלציה $\inv{R}$ מ$B$ ל$A$ כך שלכל $\alpha R \beta$ מתקיים $\beta \inv{R} \alpha$ והיא מוגדרת כך:
\[
\inv{R} = \left\{(\beta, \alpha) \mid (\alpha, \beta) \in R\right\}
\]

\subsection*{הרכבת / כפל רלציות}
יהיו $S, R$ רלציות, כאשר $R$ מהקבוצה $A$ לקבוצה $B$ ו$S$ מהקבוצה $B$ לקבוצה $C$. אזי מכפלת הרלציות )נקראת גם הרכבת הרלציות( מסומנת $R \circ S$ או $RS$ ומוגדרת כך:
\[
RS = R \circ S = \left\{(\alpha, \gamma) \mid \exists \beta \in B, (\alpha, \beta) \in R \land (\beta, \gamma) \in S\right\}
\]
כפל רלציות הוא אסוציאטיבי, כלומר עבור שלוש רלציות $R, S, T$ )כאשר כמובן מוגדרות המכפלות ביניהן( מתקיים $R(ST) = (RS)T$. בנוסף, הרלציה ההופכית של מכפלת רלציות נראית כך:
\[
\inv{(RS)} = \inv{S}\inv{R}
\]

\subsection*{רלציית הזהות}
יחס הזהות על קבוצה $A$ יסומן ב$I_{A}$ ומוגדר כך:
\[
I_{A} = \left\{(\lambda, \lambda) \mid \lambda \in A\right\} \equiv \forall \alpha, \beta \in A, (\alpha, \beta) \in I_{A} \iff \alpha = \beta
\]
\clearpage

\section*{רלציות -- המשך}

\subsection*{תכונות של רלציות}
\begin{itemize}
\item \definition{רפלקסיביות}{$\forall a \in A,\;(aRa) \equiv I_{A} \subseteq R$}
\item \definition{סימטריות}{$\forall a, b \in A,\;(aRb \bidiarrow{} bRa) \equiv R = \inv{R}$}
\item \definition{אנטי-סימטריות}{$\forall a, b \in A,\;(aRb \land bRa \opr{} a = b)$}
\item \definition{טרנזיטיביות}{$\forall a, b, c \in A,\;(aRb \land bRc \opr{} aRc) \equiv R^{2} \subseteq R$}
\end{itemize}

\subsection*{סגור של רלציה ביחס לתכונה מסויימת}
תהי $R$ רלציה מעל $A$. הסגור של $R$ ביחס לתכונה מסויימת הוא רלציה $S$ מעל $A$ המקיימת את התכונה הזאת, מכילה את $R$ ומוכלת בכל רלציה מעל $A$ המקיימת את התכונה ומכילה את $R$. הסגור הטרנזיטיבי של רלציה $R$ הוא:
\[
S = R \cup R^{2} \cup R^{3} \cup \dots = \bigcup^{\infty}_{1 \le i \in \mN} R^{i}
\]

\subsection*{סוגים שונים של רלציות}
\begin{itemize}
\item \definition{רלציית שקילות}{רלציה רפלקסיבית, טרנזיטיבית וסימטרית}
\item \definition{רלציית קומפטיביליות}{רלציה רפלקסיבית וסימטרית}
\item \definition{רלציית סדר חלקי}{רלציה רפלקסיבית, טרנזיטיבית ואנטיסימטרית. קבוצה עם רלציית סדר חלקי מעליה נקראת קבוצה סדורה חלקית. מסומנת לרוב ב$\le$}
\item \definition{רלציית סדר מלא}{סדר מלא הינו סדר חלקי אשר פועל על כל זוג איברים בקבוצה, כלומר אין איברים בה שאינם ניתנים להשוואה}
\end{itemize}
קבוצה עם סדר חלקי מעליה נקראת קבוצה סדורה חלקית.

\subsection*{איברים מינימליים ומקסימליים, האיבר הקטן ביותר והגדול ביותר}
תהי קבוצה $A$ עם רלציית סדר חלקי מעליה המסומנת ב$\le$. האיבר $\alpha \in A$ ייקרא
\begin{itemize}
\item \definition{איבר מינימלי של $A$}{אם מתקיים $\forall \lambda \in A,\;(\lambda \le \alpha \opr{} \lambda = \alpha)$}
\item \definition{איבר מקסימלי של $A$}{אם מתקיים $\forall \lambda \in A,\;(\alpha \le \lambda \opr{} \lambda = \alpha)$}
\item \definition{האיבר הקטן ביותר ב$A$}{אם $\alpha$ קיים ואם מתקיים $\forall \lambda \in A,\;(\alpha \le \lambda)$}
\item \definition{האיבר הגדול ביותר ב$A$}{אם $\alpha$ קיים ואם מתקיים $\forall \lambda \in A,\;(\lambda \le \alpha)$}
\end{itemize}
בקבוצה סדורה חלקית סופית קיימים איבר מינימלי אחד לפחות ואיבר מקסימלי אחד לפחות. בנוסף, בקבוצה סדורה חלקית יכולים להיות לכל היותר איבר קטן ביותר אחד ואיבר גדול ביותר אחד.

\clearpage
\section*{חלוקות}
\subsection*{חלוקה}
תהי $A$ קבוצה. $\pi \subseteq \mathcal{P}(A) - \{\emptyset\}$ תיקרא חלוקה של $A$ אם איבריה הינן תת-קבוצות זרות זו לזו של $A$ אשר איחודן הוא $A$, או:
\[
\pi = \left\{X \subseteq A \mid \forall Y \in \pi, X \cap Y \neq \emptyset \iff X = Y\right\}
\]
איברי החלוקה $\pi$ )אשר הינן תת-קבוצות של $A$( נקראים המחלקות או הבלוקים של החלוקה. בנוסף, בהינתן $n$ מחלקות של $\pi$ מתקיים $\bigcup^{n}_{i = 1}Q_{i} = A$.

\subsubsection*{מחלקת שקילות וקבוצת מנה}
תהי $E$ רלציית שקילות מעל $A$. אזי מחלקת השקילות של $\alpha \in A$ הינה קבוצת כל האיברים של $A$ הנמצאים ביחס עם $\alpha$, מסומנת ב$\overline{\alpha}$ ומוגדרת כך:
\[
\overline{\alpha} = \left\{\beta \in A \mid \alpha R \beta\right\}
\]
בנוסף, קבוצת מחלקות השקילות של $E$ נקראת קבוצת המנה של $A$ מעל $E$ ומסומנת כך:
\[
\nicefrac{A}{E} = \left\{\overline{\alpha} \mid \alpha \in A\right\}
\]

\subsubsection*{משפט}
כל חלוקה $\pi$ של קבוצה $A$ משרה רלציית שקילות $E$ מעל $A$ המוגדרת כך:
\[
E = \left\{(\alpha, \beta) \mid \exists Q \in \pi, (\alpha, \beta \in Q)\right\}
\]
משפט זה מתקיים גם בכיוון ההפוך, כלומר כל רלציית שקילות $E$ מעל $A$ משרה חלוקה $\pi$ של $A$ למחלקות שקילות.

\subsubsection*{עידון של חלוקה}
יהיו $\pi_{1}, \pi_{2}$ חלוקות של $A$. החלוקה $\pi_{2}$ תיקרא עידון של $\pi_{1}$ אם מתקיים
\[
\forall Q \in \pi_{2}\;\exists G \in \pi_{1},\;Q \subseteq G
\]
כלומר שעבור כל מחלקה של $\pi_{2}$ קיימת מחלקה של $\pi_{1}$ אשר היא מוכלת בה.

\subsubsection*{מחלקת קומפטיביליות, מחלקת קומפטיביליות מקסימלית}
תהי $R$ רלציית קומפטיביליות מעל $A$. אזי נגדיר תת-קבוצה $Q \subseteq A$ להיות מחלקת קומפטיביליות אם כל שניים מאיבריה נמצאים ב$R$, או פורמלית $\forall \alpha, \beta \in Q, \alpha R \beta$. מחלקת קומפטיביליות תיקרא מחלקת קומפטיביליות מקסימלית אם אין אף מחלקת קומפטיביליות אחרת שמכילה אותה באופן אמיתי\footnotemark.

\footnotetext[1]{ככה כתוב בספר. אין לי מושג מה זה אומר ``''הכלה באופן שאינו אמיתי``''.}

\clearpage
\section*{פונקציות )העתקות(}
\subsection*{הגדרה}
פונקציה $f$ )העתקה( מקבוצה $A$ לקבוצה $B$ )מסומנת כך: $f: A \to B$( היא רלציה מ$A$ ל$B$ המקיימת 
\[
\forall \alpha \in A,\;\beta, \gamma \in B,\;((\alpha, \beta) \in f \land (\alpha, \gamma) \in f \opr{} \beta = \gamma)
\]

\subsection*{תחום ותמונה של פונקציה}
תהי $f: A \to B$. אזי התחום של $f$ מוגדר כך:
\[
\domain{f} = \left\{\alpha \in A \mid \exists \beta \in B, f(\alpha) = \beta\right\}
\]
התמונה של $f$ הינה קבוצת האיברים ב$B$ אשר עבורם קיים איבר ב$A$ כך שהם שניהם נמצאים ב$f$, או פורמלית:
\[
\image{f} = \left\{f(\alpha) \mid \alpha \in A\right\}
\]

\subsection*{פונקציות חח``''ע ועל}
תהי $f: A \to B$. הפונקציה $f$ תיקרא חד חד ערכית )בקיצור -- חח``''ע( אם ורק אם מתקיים
\[
\forall \lambda_{1}, \lambda_{2} \in A, (f(\lambda_{1}) = f(\lambda_{2}) \opr{} \lambda_{1} = \lambda_{2})
\]
הפונקציה $f$ תיקרא על אם ורק אם מתקיים
\[
B = \image{f} \equiv \forall \beta \in B \exists \alpha \in A, \beta = f(\alpha)
\]
פונקציה חח``''ע ועל נקראת פונקציית שקילות.

\subsection*{הרכבת / מכפלת פונקציות}
יהיו $f: A \to B, g: B \to C$ פונקציות ונניח ש$\image{f} \subseteq \domain{g}$. אזי ההרכבה של $f, g$ מוגדרת כך:
\[
g \circ f = \left\{(\alpha, \gamma) \mid \exists \beta \in B, \beta = f(\alpha) \land g(\beta) = \gamma\right\} \opr{} (g \circ f)(x) = g(f(x))
\]
תכונות ההרכבה:
\begin{itemize}
\item \definition{אי קומוטטיביות}{בדרך כלל, הרכבת פונקציות אינה קומוטטיבית, כלומר $f \circ g \neq g \circ f$}
\item \definition{אסוציאטיביות}{$(f \circ g) \circ h = f \circ (g \circ h)$}
\item \definition{איבר ניטרלי}{בהינתן העתקות זהות $id_{A}: A \to A, id_{B}: B \to B$ מתקיים\\\
$f \circ id_{A} = id_{B} \circ f = f$}
\end{itemize}

\clearpage
\section*{פונקציות -- המשך}
\subsection*{פונקציה הופכית}
תהי $f: A \to B$ פונקציה חח``''ע. אזי הפונקציה $\inv{f}: B \to A$ קיימת, ומתקיים
\[
\forall \alpha \in A, \beta = f(\alpha) \in B, (\inv{f}(\beta) = \alpha) \equiv f \circ \inv{f} = \inv{f} \circ f = id_{A}
\]
בנוסף, בדומה לרלציות ורלציות הופכיות מתקיימות גם התכונות הבאות:
\begin{itemize}
\item $\inv{(\inv{f})} = f$
\item $\inv{(f \circ g)} = \inv{g} \circ \inv{f}$
\end{itemize}

\subsection*{איזומורפיזם בין קבוצות}
יהיו $(A, \le_{A}), (B, \le_{B})$ שתי קבוצות סדורות חלקית ויחסיהן בהתאמה. $A$ ו$B$ ייקראו איזומורפיות אחת לשנייה אם קיימת העתקה חד חד ערכית $f: A \to B$ כך שמתקיים
\[
\forall \lambda_{1}, \lambda_{2} \in A,\;(\lambda_{1} \le_{A} \lambda_{2} \opr{} f(\lambda_{1}) \le_{B} f(\lambda_{2}))
\]

\subsection*{פונקציות מיוחדות}
\begin{itemize}
\item \definition{פונקצית הזהות}{$id: A \to A, \forall x \in A, id(x) = x$}
\item \definition{פונקציה קבועה}{$f: A \to \{k\}, \forall x \in A, f(x) = k$}
\item \definition{פונקציה אופיינית של $A$}{בהינתן קבוצת עולם $U$ ותת קבוצה שלה $A \subseteq U$,
\[
\forall x \in U, \varphi_{A}(x) = \begin{cases*}
1 & $x \in A$\\
0 & $x \in A'$
\end{cases*}
\]
פונקציה זו נקראת הפונקציה האופיינית של $A$}
\item \definition{סדרות}{סדרה היא פונקציה שתחומה הוא $\mN$ או $\mN^{+}$}
\end{itemize}

\clearpage
\section*{עוצמות}

\end{document}