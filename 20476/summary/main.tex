\documentclass[11pt, oneside]{article}
\usepackage{geometry}
\geometry{a4paper}
\usepackage[parfill]{parskip}
\usepackage[nodisplayskipstretch]{setspace}
\usepackage{graphicx,titlesec,subfiles,adjustbox}
\usepackage{tikz}
\usepackage{tikz-qtree}
\usepackage{mathtools,amsmath,amssymb,cancel,commath,nicefrac}

% Hebrew Stuff
\usepackage[utf8x]{inputenc}
\usepackage[english,hebrew]{babel}
\usepackage{hebfont}

% Custom commands
\newcommand{\qed}{\R{$\blacksquare$}}
\newcommand{\br}{\\\\\\\\\\\\\\}
\newcommand{\opr}[1]{\underset{\text{#1}}{\Rightarrow}}
\newcommand{\bidiarrow}[1]{\underset{\text{#1}}{\Leftrightarrow}}
\newcommand{\ueq}[1]{\underset{\text{#1}}{=}}
\newcommand{\mR}{\mathbb{R}}
\newcommand{\mN}{\mathbb{N}}
\newcommand{\mZ}{\mathbb{Z}}
\newcommand{\mQ}{\mathbb{Q}}
\newcommand{\inv}[1]{#1^{-1}}

% Custom text commands (for Hebrew)
\newcommand{\q}[3]{\R{שאלה #2.#1#3}}
\newcommand{\m}[3]{\R{משפט #3#2.#1}}
\newcommand{\h}[3]{\R{הגדרה #3#2.#1}}
\newcommand{\ms}[3]{\R{מסקנה #3#2.#1}}

% Custom commands for this document
\newcommand{\definition}[2]{\textbf{#1}: #2.}
\newcommand{\defnop}[2]{\textbf{#1}: #2}
\DeclarePairedDelimiter\parens{\lparen}{\rparen}
\DeclareMathOperator{\dnp}{Domain}
\newcommand{\domain}[1]{\dnp\parens*{#1}}
\DeclareMathOperator{\rnp}{Range}
\newcommand{\range}[1]{\rnp\parens*{#1}}
\DeclareMathOperator{\imnp}{Im}
\newcommand{\image}[1]{\imnp\parens*{#1}}
\DeclareMathOperator{\distnp}{dist}
\newcommand{\dist}[1]{\distnp_{G}\parens*{#1}}
\DeclareMathOperator{\card}{card}
\newcommand{\jbox}[1]{\centering{\marginbox{0pt 8pt}{\vbox{#1}}}}

% Spacing
\titlespacing\section{0pt}{0pt}{0pt}
\titlespacing\subsection{0pt}{0pt}{0pt}
\titlespacing\subsubsection{0pt}{0pt}{0pt}

\title{מתמטיקה בדידה -- סיכום\thanks{נכתב לצורך הקורס באוניברסיטת תל אביב ונערך לתכני הקורס באוניברסיטה הפתוחה.}}
\author{שירה ברמן )נערך ע``''י יונתן אוחיון(}

\begin{document}
\maketitle

% Logic - intro
\subfile{partials/logic}
\clearpage

% Set theory - intro
\subfile{partials/sets}

% Set theory - relations
\subfile{partials/relations}
\clearpage

% Set theory - partitions
\subfile{partials/partitions}
\clearpage

% Set theory - functions
\subfile{partials/functions}
\clearpage

% Set theory - cardinal numbers
\subfile{partials/cardinals}
\clearpage

% Combinatorics - intro + Newton's binomial + pigeonhole principle
\subfile{partials/combinatorics/intro}
\clearpage

% Combinatorics - formulas + inclusion-exclusion principle
\subfile{partials/combinatorics/formulas}
\clearpage

% Combinatorics - generating functions + other
\subfile{partials/combinatorics/other}
\clearpage

% Graph theory - intro, important definitions
\subfile{partials/graphs/part1}
\clearpage

% Graph theory - tagged graph, sub-graphs, trees
\subfile{partials/graphs/part2}
\clearpage

% Graph theory - isomorphism, Cayley theorem, Prüfer series, Euler and Hamilton circles, theorems
\subfile{partials/graphs/part3}
\clearpage

% Graph theory - planar graph, graph coloring, Brooks theorem
\subfile{partials/graphs/part4}
\clearpage

% Important stuff for the test
\subfile{partials/test}

\end{document}