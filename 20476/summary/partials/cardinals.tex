\documentclass[../main.tex]{subfiles}

\section*{עוצמות}
\subsection*{הגדרה}
העוצמה )נקראת גם המספר הקרדינלי( של קבוצה $A$ הינה גודל הקבוצה, בין אם היא סופית ובין אם היא ``''אין-סופית``''. העוצמה של הקבוצה מסומנת כך: $\abs{A}$ או כך: $\card{A}$. אם הקבוצה $A$ סופית, אזי העוצמה שלה הינה מספר טבעי, כאשר מתקיים $\abs{\emptyset} = 0$.
\subsection*{שוויון עוצמות}
יהיו $A, B$ קבוצות. $A$ ו$B$ ייקראו שוות עצמה אם ורק אם קיימת פונקציית שקילות מ$A$ על $B$ )כלומר מתקיים $A \sim B$(.

\subsection*{קבוצות בנות מנייה}
קבוצת המספרים הטבעיים $\mN$ נקראת קבוצה בת מנייה ועוצמתה מסומנת ב$\aleph_{0}$. בנוסף, לפי הגדרת שוויון העוצמות, העוצמה של כל קבוצה עבורה מתקיים $A \sim \mN$ הינה גם $\aleph_{0}$. דוגמאות לקבוצות שכאלו: $\mN^{+}, \mN_{even}, \mN_{odd}, \mZ, \mQ, \mN^{k} (k \in \mN)$ וכו'. איחוד קבוצות בנות מנייה הינו קבוצת בת מנייה בעצמו.

\subsection*{קבוצות שאינן בנות מנייה}
קבוצת המספרים הממשיים $\mR$ אינה בת מנייה ואנו מסמנים את עוצמתה באות $C$. בנוסף, לפי הגדרת שוויון העוצמות, העוצמה של כל קבוצה עבורה מתקיים $A \sim \mR$ הינה גם $C$. דוגמאות לקבוצות שכאלו: $[a, b], (a, b), [a, b), (a, b], \mR^{+}, \mR^{-}$ וכו'.

\subsection*{היחס $\le$ לעוצמות}
יהיו $A, B$ קבוצות. אזי נאמר ש$\abs{A} \le \abs{B}$ אם קיימת פונקציה חח``''ע $f: A \to B$.
יחס זה הוא רפלקסיבי, טרנזיטיבי ולפי משפט קנטור-שרדר-ברנשטיין גם אנטיסימטרי.

\subsection*{משפט קנטור-שרדר-ברנשטיין}
יהיו $k, m$ עוצמות. אם $k \le m$ וגם $m \le k$, אז $k = m$.

\subsection*{משפט קנטור}
תהי $A$ קבוצה. אזי $\abs{A} < \abs{\mathcal{P}(A)}$ .
\clearpage

\section*{עוצמות -- המשך}
\subsection*{אריתמטיקה של עוצמות}
\subsubsection*{חיבור}
יהיו $A, B$ קבוצות זרות ו$\abs{A} = k, \abs{B} = m$ . אזי סכום העוצמות יסומן ויוגדר כך:
\[k + m = \abs{A \cup B}\]
דוגמאות לחיבור עוצמות:
\begin{itemize}
\item $k + 0 = k$
\item $\aleph_{0} + 1 = \aleph_{0}$
\item $\aleph_{0} + \aleph_{0} = \aleph_{0}$
\item $C + C = C$
\end{itemize}
בנוסף, בהינתן $k$ עוצמה \textbf{אינסופית} כלשהי מתקיים $k + \aleph_{0} = k$. חיבור עוצמות הוא קומוטטיבי ואסוציאטיבי.

\subsubsection*{כפל}
יהיו $A, B$ קבוצות ו$\abs{A} = k, \abs{B} = m$. אזי כפל העוצמות יסומן ויוגדר כך:
\[k \cdot m = km = \abs{A \times B}\]
דוגמאות לכפל עוצמות:
\begin{itemize}
\item $k \cdot 0 = 0$
\item $k \cdot 1 = 1$
\item $\aleph_{0} \cdot \aleph_{0} = \aleph_{0}$
\item $C \cdot C = C$
\item מממ``''ן 41 של 7102ג נוכל לראות כי מתקיים $C \cdot \aleph_{0} = C$.
\end{itemize}
כפל עוצמות הוא קומוטטיבי ואסוציאטיבי. בנוסף, קיימת דיסטריבוטיביות של הכפל מעל החיבור כך: יהיו $k, m, n$ עוצמות. אזי $k \cdot (m + n) = km + kn$.
\subsubsection*{חזקה}
יהיו $A, B$ קבוצות המקיימות $\abs{A} = k, \abs{B} = m$ . נסמן את קבוצת הפונקציות מ$A$ ל$B$ כך:$B^{A}$. נגדיר את העוצמה $m^{k}$ להיות $\abs{B^{A}}$ . אם הקבוצות הללו סופיות אז עוצמה זו הינה כמות הפונקציות מ$A$ ל$B$. בנוסף, מתקיים
\[
\abs{\mathcal{P}(A)} = 2^{k},\;\abs{\mathcal{P}(\mN)} = 2^{\aleph_{0}} = C,\;C^{\aleph_{0}} = C
\]
ומממ``''ן 41 של 7102ג נוכל לראות כי מתקיים $C^{C} = 2^{C}$.