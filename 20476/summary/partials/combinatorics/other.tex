\documentclass[../../main.tex]{subfiles}

\section*{קומבינטוריקה -- פונקציות יוצרות}
כדי למצוא את מספר הפתרונות בטבעיים של המשוואה $t_{1} + \ldots + t_{n} = k$ המקיימים את ההגבלות $0 \le t_{i} \le b_{i}$, יש לחשב את המקדם של $x^{k}$ בפולינום הבא:
\[
f(x) = (1 + x + \ldots + x^{b_{1}})(1 + x + \ldots + x^{b_{2}})\ldots(1 + x + \ldots + x^{b_{n}})
\]
הפונקציה $f(x)$ היא הפונקציה היוצרת של הסדרה $a_{0}, a_{1}, \ldots, a_{k}, \ldots$, כאשר $a_{k}$ הוא מספר הפתרונות בטבעיים של המשוואה הנ``''ל, המקיימים את ההגבלות הנתונות.
\subsection*{נוסחאות שימושיות}
סכום טור הנדסי סופי: $\sum^{n}_{i = 0} x^{i} = \frac{1-x^{n+1}}{1-x}$ ואינסופי: $\sum^{\infty}_{i = 0}x^{i} = \frac{1}{1 - x}$.

כפל פונקציות יוצרות: אם $f(x) = \sum^{\infty}_{i = 0} a_{i}x^{i}$ ו$g(x) = \sum^{\infty}_{i = 0} b_{i}x^{i}$ ו$f(x) \cdot g(x) = \sum^{\infty}_{i = 0} c_{i}x^{i}$, אז מתקיים $c_{k} = \sum^{k}_{i = 0} a_{i}b_{k - i}$.

$\frac{1}{(1 - x)^{n}} = (1 + x + x^{2} + \ldots)^{n} = \sum^{\infty}_{k = 0} D(n, k)x^{k}$ ובמילים אחרות: המקדם של $x^{k}$ בפיתוח הביטוי $\frac{1}{(1 - x)^{n}}$ הוא $D(n, k)$.

\section*{קומבינטוריקה -- יחסי נסיגה}
\subsection*{פתרון יחסי נסיגה לינאריים הומוגניים}
עבור יחס הנסיגה
\[
a_{n} = k_{1}a_{n - 1} + k_{2}a_{n - 2} + \ldots + k_{k}a_{n - k}
\]
נגדיר את הפולינום הבא עי הצבת $a_{n} = \lambda^{n}$:
\[
\lambda^{n} = k_{1}\lambda^{n - 1} + k_{2}\lambda^{n - 2} + \ldots + k_{k}\lambda^{n - k}
\]
פולינום זה נקרא המשוואה האופיינית / הפולינום האופייני של יחס הנסיגה. נעביר אגפים, נצמצם ב$\lambda^{n - k}$ ונקבל:
\[
\lambda^{k} - k_{1}\lambda^{k - 1} - k_{2}\lambda^{k - 2} - \ldots - k_{k}\lambda^{n - k} = 0
\]
נסמן ב$\lambda_{i}$ את שורשי פולינום זה. אזי נוכל למצוא צירוף לינארי ליחס הנסיגה באופן הבא:
\[
a_{n} = A_{1}\lambda_{1}^{n} + A_{2}\lambda_{2}^{n} + \ldots + A_{k}\lambda_{k}^{n}
\]
כאשר קובעים את המקדמים $A_{i}$ לפי הצבה של התנאים התחיליים )בדר``''כ $a_{0}, a_{1}, a_{2}$(. יש לשים לב שבחרנו נכון את תנאי ההתחלה.