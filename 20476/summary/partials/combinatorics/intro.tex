\documentclass[../../main.tex]{subfiles}

\section*{קומבינטוריקה}
\subsection*{עקרונות ומושגים קומבינטוריים בסיסיים}
\begin{itemize}
  \item \definition{עקרון החיבור}{אם ניתן לבחור עצם $a_{1}$ ב$k_{1}$ דרכים, \ldots, עצם $a_{n}$ ב$k_{n}$ דרכים ואי אפשר לבחור יותר מעצם אחד אז קיימות $\sum^{n}_{i=1} k_{i}$ דרכים לבחירת עצם}
  \item \definition{עקרון הכפל}{אם ניתן לבחור עצם $a_{1}$ ב$k_{1}$ דרכים, עצם $a_{n}$ ב$k_{n}$ דרכים כאשר כל בחירה לא תלויה בבחירה הקודמת אז קיימות $\prod^{n}_{i=1} k_{i}$ דרכים לבחירת עצם}
  \item \definition{מספר האופנים לסידור $n$ אובייקטים בשורה}{מספר זה הינו $n!$, או $\prod^{n}_{i=1} i$}
  \item \defnop{סידור בשורה של $n$ רכיבים המכילים קבוצות של אובייקטים מאותו הסוג}{בהינתן $m$ קבוצות, בכל אחת $k_{m}$ איברים זהים:
    \[\frac{n!}{k_{1}! * \dots * k_{m}!}\]
  }
  \item \defnop{פרמוטציות}{בחירה של $k$ איברים שונים מתוך $n$ איברים שונים, כאשר יש חשיבות לסדר הבחירה:
  \[P(n, k) = \frac{n!}{(n - k)!}\]}
  \item \defnop{קומבינציות}{בחירה של $k$ איברים מתוך $n$ איברים, ללא חשיבות לסדר הבחירה:
  \[C(n, k) = \frac{n!}{k!(n - k)!} = {n \choose k}\]}
\end{itemize}

\subsection*{הבינום של ניוטון}
\[
(a + b)^{n} = \sum^{n}_{i=0} \binom{n}{i} a^{i} b^{n-i}
\]

\subsection*{כמות פונקציות}
כפי שדובר בפרק על עוצמות, מספר הפונקציות $f: A \to B$ הינו $|B|^{|A|}$. בהנחה ו$A = B$, מספר הפונקציות החח``''ע הינו $|A|!$ )בעיה שקולה לסידור איברי $A$ בשורה מול עצמם(, מספר הפונקציות החח``''ע ועל הינו $|A|!$ )אין הבדל שכן הקבוצה $A$ סופית ושווה לעצמה(, מספר הפונקציות שאינן חח``''ע הינו $|A|^{|A|} - |A|!$ )סך כל הפונקציות פחות הפונקציות החח``''ע(.

\subsection*{עקרון שובך היונים}
עקרון שובך היונים מנוסח כך: אם $n + 1$ יונים נכנסות ל$n$ שובכים, אזי בתא אחד לפחות יש יותר מיונה אחת. פורמלית: בחלוקה $\pi$ של קבוצה סופית $A$ ל$n$ מחלקות, קיימת לפחות מחלקה אחת שמספר איבריה גדול או שווה ל$\frac{|A|}{n}$.