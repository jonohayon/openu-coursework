\documentclass[../main.tex]{subfiles}

\section*{חלוקות}
\subsection*{חלוקה}
תהי $A$ קבוצה. $\pi \subseteq \mathcal{P}(A) - \{\emptyset\}$ תיקרא חלוקה של $A$ אם איבריה הינן תת-קבוצות זרות זו לזו של $A$ אשר איחודן הוא $A$, או:
\[
\pi = \left\{X \subseteq A \mid \forall Y \in \pi, X \cap Y \neq \emptyset \iff X = Y\right\}
\]
איברי החלוקה $\pi$ )אשר הינן תת-קבוצות של $A$( נקראים המחלקות או הבלוקים של החלוקה. בנוסף, בהינתן $n$ מחלקות של $\pi$ מתקיים $\bigcup^{n}_{i = 1}Q_{i} = A$.

\subsubsection*{מחלקת שקילות וקבוצת מנה}
תהי $E$ רלציית שקילות מעל $A$. אזי מחלקת השקילות של $\alpha \in A$ הינה קבוצת כל האיברים של $A$ הנמצאים ביחס עם $\alpha$, מסומנת ב$\overline{\alpha}$ ומוגדרת כך:
\[
\overline{\alpha} = \left\{\beta \in A \mid \alpha R \beta\right\}
\]
בנוסף, קבוצת מחלקות השקילות של $E$ נקראת קבוצת המנה של $A$ מעל $E$ ומסומנת כך:
\[
\nicefrac{A}{E} = \left\{\overline{\alpha} \mid \alpha \in A\right\}
\]

\subsubsection*{משפט}
כל חלוקה $\pi$ של קבוצה $A$ משרה רלציית שקילות $E$ מעל $A$ המוגדרת כך:
\[
E = \left\{(\alpha, \beta) \mid \exists Q \in \pi, (\alpha, \beta \in Q)\right\}
\]
משפט זה מתקיים גם בכיוון ההפוך, כלומר כל רלציית שקילות $E$ מעל $A$ משרה חלוקה $\pi$ של $A$ למחלקות שקילות.

\subsubsection*{עידון של חלוקה}
יהיו $\pi_{1}, \pi_{2}$ חלוקות של $A$. החלוקה $\pi_{2}$ תיקרא עידון של $\pi_{1}$ אם מתקיים
\[
\forall Q \in \pi_{2}\;\exists G \in \pi_{1},\;Q \subseteq G
\]
כלומר שעבור כל מחלקה של $\pi_{2}$ קיימת מחלקה של $\pi_{1}$ אשר היא מוכלת בה.

\subsubsection*{מחלקת קומפטיביליות, מחלקת קומפטיביליות מקסימלית}
תהי $R$ רלציית קומפטיביליות מעל $A$. אזי נגדיר תת-קבוצה $Q \subseteq A$ להיות מחלקת קומפטיביליות אם כל שניים מאיבריה נמצאים ב$R$, או פורמלית $\forall \alpha, \beta \in Q, \alpha R \beta$. מחלקת קומפטיביליות תיקרא מחלקת קומפטיביליות מקסימלית אם אין אף מחלקת קומפטיביליות אחרת שמכילה אותה באופן אמיתי\footnotemark.

\footnotetext[1]{ככה כתוב בספר. אין לי מושג מה זה אומר ``''הכלה באופן שאינו אמיתי``''.}
