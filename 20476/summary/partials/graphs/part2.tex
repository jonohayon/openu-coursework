\documentclass[../../main.tex]{subfiles}

\section*{תורת הגרפים -- המשך}
\subsection*{הגדרות וכללים חשובים -- המשך}
\begin{itemize}
\item \definition{תת-גרף}{גרף $G' = (V', E')$ ייקרא תת-גרף של $G$ אם $V' \subseteq V_{G}, E' \subseteq E_{G}$ וכל קשת ב$E'$ מחברת בין שני צמתים של $V'$}
\item \definition{תת-גרף פורש}{תת-גרף $G' = (V', E')$ של $G$ ייקרא תת-גרף פורש אם $V' = V$}
\item \definition{תת-גרף מושרה}{בהניתן תת קבוצה $U \subseteq V$ של צמתי $G$, התת-גרף המושרה על ידי $U$ ב$G$ הוא תת-גרף של $G$ שקבוצת הצמתים שלו היא $U$ וקבוצת הקשתות שלו היא כל הקשתות של $G$ שקצוותיהן ב$U$}
\item \definition{גרף מלא או קליק}{גרף $G$ ייקרא גרף מלא או קליק אם הוא גרף פשוט שכל זוג צמתים בו מחובר על ידי קשת. הגרף המלא על $n$ צמתים יסומן ב$K_{n}$}
\item \definition{גרף משלים}{הגרף המשלים של $G$ יסומן $\overline{G} = (V, \overline{E})$ הוא בעל אותה קבוצת צמתים כמו $G$ וקבוצת הקשתות שלו היא $\overline{E} = \left\{uv \mid uv \not\in E, u \neq v \in V \right\}$, כלומר שני צמתים יהיו מחוברים יהיו מחוברים בקשת ב$\overline{G}$ אם ורק אם הם אינם מחוברים בקשת ב$G$}
\item \definition{גרף דו צדדי}{גרף שניתן לחלק את צמתיו לשתי קבוצות לא ריקות $A, B$ כך שלכל קשת של $G$ יש קצה אחד ב$A$ וקצה אחד ב$B$. שתי הקבוצות נקראות הצדדים של הגרף}
\item \definition{גרף דו צדדי מלא}{גרף דו צדדי פשוט בעל $p$ צמתים בצד אחד ו$q$ צמתים בצד השני, אשר מכיל את כל $p \cdot q$ הקשתות האפשריות. מסומן $K_{p,q}$}
\item \definition{משפט}{גרף $G$ בעל שני צמתים הוא לפחות דו-צדדי אם ורק אם אין בו מעגל באורך אי-זוגי}
\item \definition{יער; עץ; עלה}{גרף ייקרא יער אם אין בו מעגל. גרף ייקרא עץ אם הוא יער קשיר. צומת בעץ נקרא עלה אם דרגתו היא בדיוק $1$}
\item \definition{טענה}{כל גרף קשיר מכיל תת-גרף פורש שהוא עץ}
\item \definition{גרף מתוייג}{גרף $G$ נקרא גרף מתוייג אם לכל צומת בו יש תג $t(u)$ שהוא מספר טבעי, ולכל שני צמתים שונים יש תגים שונים}
\end{itemize}

\subsection*{משפט 5.2}
יהי $G = (V, E)$ גרף. הטענות הבאות שקולות:
\begin{enumerate}
\item $G$ הוא עץ.
\item בין כל שני צמתים של $G$ יש מסלול יחיד.
\item $G$ הוא גרף קשיר מינימלי )במובן זה שהוא גרף קשיר ועם השמטת כל קשת ממנו מתקבל גרף לא קשיר(.
\item $G$ קשיר ו$|E| = |V| - 1$.
\item $G$ אינו מכיל מעגלים ו$|E| = |V| - 1$.
\item $G$ אינו מכיל מעגלים, אבל כל קשת שנוסיף בין הצמתים הקיימים בגרף תיצור מעגל.
\end{enumerate}