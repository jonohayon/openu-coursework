\documentclass[../../main.tex]{subfiles}

\section*{תורת הגרפים}
\subsection*{הגדרות וכללים חשובים}
\begin{itemize}
\item \definition{גרף}{גרף $G = (V, E)$ הוא שלשה המחזיקה בתוכה קבוצה סופית $V$ של איברים הנקראים צמתים, קבוצה סופית $E$ של איברים הנקראים קשתות ופונקציה $f: E \to \mathcal{P}(V)$ המתאימה לכל קשת תת-קבוצה של צמתים מתוך $V$, ובה צומת אחד או שניים}
\item \definition{גרף מכוון}{גרף מכוון הוא גרף אשר הפונקציה שלו היא $f: E \to V \times V$ והיא מתאימה זוג סדור של צמתים לכל קשת )במקום תת-קבוצה של $V$(, ולכן יש הבדל בין סדר הצמתים בגרף מכוון}
\item \definition{צמתים שכנים}{יהיו $v_{1}, v_{2} \in V$. אם קיימת קשת $e \in E$ כך ש$e = v_{1}v_{2}$, כלומר אם קיימת קשת המחברת בין שני הצמתים}
\item \definition{קשת סמוכה}{תהי $e = v_{i}v_{j} \in E$. אזי הקשת $e$ סמוכה לצמתים $v_i$ ו$v_j$}
\item \definition{לולאה}{קשת המחברת בין צומת לעצמו}
\item \definition{קשתות מקבילות}{קשתות המחברות את אותו זוג צמתים}
\item \definition{צומת מבודד}{צומת שאין לו צמתים שכנים}
\item \definition{גרף פשוט}{גרף שאינו מכיל לולאות וקשתות מקבילות}
\item \definition{דרגה של צומת}{מספר הקשתות ב$E$ הסמוכות לצומת, כאשר לולאה נספרת פעמיים. מסומנת ב$\deg_{G}(v)$}
\item \definition{טענה -- סכום הדרגות}{בכל גרף $G = (V, E)$ מתקיים $\sum_{v \in V} \deg_{G}(v) = 2|E|$, כלומר סכום הדרגות בגרף שווה לכפליים מספר הקשתות. בנוסף, בכל גרף מספר הצמתים שדרגתם אי-זוגית הוא זוגי}
\item \definition{מסלול; מעגל}{מסלול בגרף הוא סדרה $P = v_{0}, e_{1}, v_{1}, \ldots, e_{k}, v_{k}$ כאשר $v_{i}$ הם צמתים, $e_{i}$ הם קשתות ו$\forall 1 \le i \le k, e_{i} = v_{i - 1}v_{i}$. מעגל הוא מסלול שבו $v_{0} = v_{k}$}
\item \definition{צמתי קצה; צמתים פנימיים}{במסלול $P$ הקודקודים $v_{0}, v_{k}$ נקראים צמתי הקצה של המסלול ושאר הצמתים נקראים הצמתים הפנימיים}
\item \definition{אורך של מסלול}{נסמן את האורך של מסלול $P$ ב$|P|$ והוא מספר הקשתות במסלול}
\item \definition{מסלול פשוט; מעגל פשוט}{מסלול פשוט הוא מסלול שבו כל הצמתים שונים. מעגל פשוט הוא מסלול פשוט שבו צמתי הקצה שווים}
\item \definition{מרחק}{המרחק $\dist{u, v}$ הוא אורך המסלול הקצר ביותר בין $u$ ל$v$. אם אין מסלול כזה, $\dist{u, v} = \infty$ ואם $u = v$ אז $\dist{u, v} = 0$}
\item \definition{גרף קשיר}{גרף קשיר הוא גרף שבו יש מסלול בין כל שני צמתים}
\item \definition{רכיב קשירות}{תת-קבוצה מקסימלית של $V$ שבין בין כל שני צמתים יש מסלול}
\end{itemize}