\documentclass[../../main.tex]{subfiles}

\section*{תורת הגרפים -- המשך}
\subsection*{איזומורפיזם בין גרפים}
שני גרפים $G = (V, E)$ ו$G' = (V', E')$ נקראים איזומורפיים אם קיימת העתקה $f: V \to V'$ חח``''ע ועל כך שלכל $u, v \in V$ מתקיים $uv \in E$ אם ורק אם $f(u)f(v) \in E'$.\\
הגדרה נוספת: שני גרפים מתוייגים $G = (V, E)$ ו$G' = (V', E')$ נקראים איזומורפיים אם קיימת העתקה $f: V \to V'$ חחע ועל כך שלכל $u \in V$ התג של $u$ שווה לתג של $f(u)$ ובנוסף מתקיים $uv \in E$ אם ורק אם $f(u)f(v) \in E'$.
\subsection*{משפט קיילי}
לכל $n \ge 2$ מספר העצים המתוייגים השונים על קבוצה מתוייגת $V$ של $n$ צמתים הוא $n^{n-2}$.

\subsection*{סדרת פרופר \L{(Pr\"ufer)}}
סדרה $S = (s_{1}, \ldots, s_{n-2})$ באורך $n-2$ הבנויה באופן הבא:
\begin{itemize}
\item \definition{קלט}{עץ $T$ על קבוצת צמתים מתויגת $V$}
\item \definition{אתחול}{$() \to S$}
\item \definition{לולאה}{כל עוד $|V| \ge 2$ בצע:}
\begin{enumerate}
\item אם $|V| = 2$ עצור והחזר את $S$.
\item יהי $v$ העלה בעל התג הקטן ביותר ב$L(T)$ )קבוצת העלים(.
\item הוסף את השכן $s$ של $v$ לסוף הסדרה $S$, והוצא את $v$ מ$T$ ומ$V$.
\end{enumerate}
\end{itemize}

\subsection*{מעגלי אוילר והמילטון}
\begin{itemize}
\item \definition{מסלול )מעגל( אוילר}{מסלול )מעגל( אוילר בגרף $G$ הוא מסלול )מעגל( שבו כל קשת של $G$ מופיעה בדיוק פעם אחת}
\item \definition{מסלול )מעגל( המילטון}{מסלול )מעגל( המילטון בגרף $G$ הוא מסלול )מעגל( שבו כל צומת של $G$ מופיע בדיוק פעם אחת}
\end{itemize}
גרף נקרא אוילרי/המילטוני אם יש בו מעגל אוילר/המילטון.\\
\definition{משפט}{גרף קשיר $G$ הוא אוילרי אם ורק אם דרגת כל צומת בו היא זוגית}\\
\definition{משפט אור \L{(Ore)}}{יהי $G = (V, E)$ גרף פשוט על $|V| = n \ge 3$ צמתים כך שלכל זוג צמתים $u, v \in V$ שאינם שכנים מתקיים $\deg_{G}(u) + \deg_{G}(v) \ge n$. אז $G$ הוא המילטוני}\\
\definition{משפט דירק \L{(Dirac)}}{יהי $G = (V, E)$ גרף פשוט על $|V| = n \ge 3$ צמתים. אם הדרגה של כל צומת היא לפחות $\frac{n}{2}$, אז $G$ המילטוני}