\documentclass[../../main.tex]{subfiles}

\section*{תורת הגרפים -- המשך}
\subsection*{גרפים מישוריים}
\begin{itemize}
\item \definition{הגדרה}{גרף ייקרא מישורי אם ניתן לציירו במישור כך שלא יהיו שתי קשתות שיצטלבו}
\item \definition{פאות}{הפאות של )השיכון המישורי של( $G$ הן חלקי המישור שהגרף מפריד}
\item \definition{נוסחת אוילר}{יהי $G$ גרף מישורי קשיר )לאו דווקא פשוט( בעל $n$ צמתים ו$m$ קשתות. אז מספר הפאות בכל שיכון מישורי של $G$ הוא: $f = m - n + 2$}
\item \definition{עידון של קשת}{עידון של קשת $uv$ של גרף $G$ הוא פעולת ההחלפה של הקשת במסלול $u - x - v$ שאורכו 2, כאשר הצומת $x$ הוא צומת חדש שמוסיפים לגרף}
\item \definition{העדנה של גרף}{גרף $G'$ הוא העדנה של גרף $G$ אם ניתן לקבל את $G'$ מ$G$ על ידי סדרה של עידוני קשתות, כאשר מותר לעדן גם קשתות חדשות שלא היו בגרף ההתחלתי}
\item \definition{טענה}{גרף מישורי הוא מישורי אם ורק אם כל העדנה שלו היא גרף מישורי}
\item \definition{משפט קורטובסקי \L{(Kuratowski)}}{גרף הוא מישורי אם ורק אם הוא לא מכיל כתת-גרף העדנה של $K_{5}$ או של $K_{3,3}$}
\end{itemize}

\subsection*{צביעה של גרף}
\begin{itemize}
\item \definition{צביעה}{צביעה של גרף היא פונקציה מצומתי הגרף לקבוצה שאיבריה נקראים צבעים )או תגיות(}
\item \definition{צביעה נאותה}{צביעה של גרף תיקרא צביעה נאותה אם כל שני צמתים סמוכים צבועים בצבעים שונים}
\item \definition{מספר הצביעה}{מספר הצביעה של גרף $G$ הוא מספר הצבעים המינימלי בצביעה נאותה של $G$, והוא מסומן ב$\chi(G)$. נאמר כי $G$ הוא $k$-צבוע אם $\chi(G) \le k$}
\item \definition{סימון}{נסמן ב$\Delta(G)$ את הדרגה המקסימלית של צומת בגרף $G$}
\item \textbf{כמה טענות:}
	\begin{enumerate}
	\item $\chi(K_{n}) = n$
	\item $\chi(G) = 2$ אם ורק אם $G$ הוא גרף דו-צדדי המכיל לפחות קשת אחת.
	\item אם $G$ מעגל, אזי:
		\begin{itemize}
			\item $\chi(G) = 2$ אם $|V_{G}|$ זוגי
			\item $\chi(G) = 3$ אם $|V_{G}|$ אי-זוגי
		\end{itemize}
	\end{enumerate}
\item \defnop{משפט ברוקס \L{(Brooks)}}{$\chi(G) \le \Delta(G)$ פרט לשני המקרים הבאים שבהם $\chi(G) = \Delta(G) + 1$:}
	\begin{itemize}
		\item ל$G$ יש רכיב קשירות המשרה גרף מלא )קליק( כל $\Delta(G) + 1$ צמתים;
		\item $\Delta(G) = 2$ ויש ל$G$ רכיב קשירות המשרה מעגל באורך אי-זוגי.
	\end{itemize}
\item \definition{משפט ארבעת הצבעים}{כל גרף מישורי $G$ הוא 4-צביע )כלומר $\chi(G) \le 4$(}
\end{itemize}