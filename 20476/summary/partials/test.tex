\documentclass[../main.tex]{subfiles}

\section*{נספח -- דברים חשובים למבחן}
\subsection*{שאלות חשובות בתורת הקבוצות}
\begin{itemize}
\item \definition{\q{2}{81}{א}}{אם $R$ רפלקסיבית, אז גם $\inv{R}, R^{n}$ רפלקסיביות}
\item \definition{\q{2}{81}{ג}}{אם $R, S$ רפלקסיביות אז גם $RS, R \cup S, R \cap S$ רפלקסיביות}
\item \definition{\q{2}{32}{}}{תהי $R$ רלציה. אזי $R \cup \inv{R}, R \cap \inv{R}$ סימטריות}
\item \definition{\q{2}{42}{א}}{אם $R$ אנטיסימטרית אז גם $\inv{R}$ אנטיסימטרית}
\item \definition{\q{2}{72}{}}{אם $R, S$ אנטיסימטריות אז גם $R \cap S$} אנטיסימטרית
\item \definition{\q{2}{92}{א+ב}}{אם $R$ טרנזיטיבית אז גם $\inv{R}, R^{n}$ טרינזיטיביות}
\item \definition{\q{2}{03}{ג}}{אם $R, S$ טרנזיטיביות אז גם $R \cap S$ טרנזיטיבית}
\item \definition{\q{2}{43}{א}}{הסגור הרפלקסיבי של $R$ הוא $R \cup I_{A}$, כאשר $I_{A}$ הוא יחס הזהות מעל $A$}
\item \definition{\q{2}{43}{ב}}{הסגור הסימטרי של $R$ הוא $R \cup \inv{R}$}
\end{itemize}

\subsection*{טענת עזר -- מספר הפונקציות החח``''ע מ$A$ ל$B$}
יהיו $A, B$ קבוצות ונניח כי $|A| = k, |B| = n$. אזי מספר הפונקציות החד-חד-ערכיות מ$A$ ל$B$ הוא $P(n, k)$. בעיה זו שקולה לסידור $k$ איברים ב$n$ תאים, או לבחירת $k$ איברים עם חשיבות לסדר ללא חזרות מתוך $n$ האיברים שבקבוצה $B$.

\subsection*{דגש לפונקציות יוצרות}
לזכור - אם יש לנו פונקציה יוצרת מהסוג הבא:
\[
f(x) = \left(\frac{1 - x^{k}}{1 - x}\right)^{n} = (1 - x^{k})^{n} \cdot \frac{1}{(1 - x)^{n}}
\]
אז עלינו לפתוח את הביטוי השמאלי בעזרת נוסחת הבינום עד החזקה הגדולה ביותר שעדיין קטנה מהחזקה אותה אנו מחפשים, ולכפול בה כל מקדם של $x^{i - m}$ )כאשר $m$ היא החזקה שאנו מחפשים(. דוגמה לכך:
\[
f(x) = x^{3}(1-x^{5}) \cdot \frac{1}{(1-x)^{4}}
\]
נחפש את המקדם של $x^{12-3} = x^{9}$. ראשית לפי נוסחת הבינום, נוכל לראות שהחזקה הגדולה ביותר של הביטוי השמאלי שקטנה מ9 אינה אלא $x^{5}$. נפתח את הביטוי עד $x^{5}$ ונקבל $1 - 4x^{5}$. כעת, כפי שאנו יודעים, המקדם של $x^{k}$ בביטוי מימין הינו $D(4, k)$. נציב ונקבל:
\[
1 \cdot D(4, 9 - 0) - 4 \cdot D(4, 9 - 5) = \framebox{$1 \cdot D(4, 9) - 4 \cdot D(4, 4)$} = \binom{12}{3} - 4\binom{7}{3}
\]