\documentclass[../main.tex]{subfiles}

\section*{רלציות )יחסים( -- הקדמה}
\subsection*{זוגות סדורים}
אוסף של שני איברים אשר אחד מהם נקבע כאיבר הראשון והשני כאיבר השני: $(\alpha, \beta)$. במקרה זה, $\alpha$ הוא האיבר הראשון ו$\beta$ הוא השני. שני זוגות סדורים $(\alpha, \beta), (\gamma, \delta)$ שווים זה לזה אמ``''מ $\alpha = \gamma \land \beta = \delta$. ניתן להכליל מושג זה למושג $n$-יה, שהיא אוסף של איברים המסודרים לפי $\mN$:
\[
(\lambda_{1}, \lambda_{2}, \dots, \lambda_{n})
\]
כאשר $\lambda_{1}$ האיבר הראשון ו$\lambda_{n}$ האחרון. שוויון הזוגות הסדורים פועל גם פה: בהינתן שתי $n$-יות $(\alpha_{1}, \dots, \alpha_{n}), (\beta_{1}, \dots, \beta_{n})$, הן ייקראו שוות זו לזו אמ``''מ
\[
\forall i \in \mN, 1 \le i \le n, (\alpha_{i} = \beta_{i})
\]

\subsection*{מכפלה קרטזית}
יהיו $A, B$ קבוצות. המכפלה הקרטזית של $A$ ו$B$ מוגדרת בתור קבוצת כל הזוגות הסדורים של איברי $A, B$ ומסומנת כך:
\[
A \times B = \left\{(\alpha, \beta) \mid \alpha \in A, \beta \in B \right\}
\]
פעולה זו אינה אסוציאטיבית ואם $A \neq B$ היא אינה קומוטטיבית. בנוסף, ניתן לבצע את הפעולה ככל שנרצה ולקבל $n$-יות בגדלים שונים:
\[
A_{1} \times A_{2} \times \dots \times A_{n} = \left\{(\lambda_{1}, \dots, \lambda_{n}) \mid \lambda_{i} \in A_{i}\right\}
\]
כתיבה אחרת היא ``''חזקה``'' של קבוצה והיא נראית כך:
\[
A^{n} = \underbrace{A \times \dots \times A}_{\text{\R{ פעמים}} n} = \left\{(\lambda_{1}, \dots, \lambda_{n}) \mid \lambda_{i} \in A\right\}
\]

\clearpage

\section*{רלציות )יחסים(}
רלציה )יחס( בינארית $R$ מהקבוצה $A$ לקבוצה $B$ הינה תת-קבוצה של $A \times B$ )כלומר $R \in \mathcal{P}(A \times B)$(. נסמן זוג סדור השייך לרלציה $R$ באופנים הבאים: $(\alpha, \beta) \in R \iff \alpha R \beta$. ניתן לתאר רלציות כקבוצות, גרף מכוון )דיגרף( או טבלה. אם $A = B$ אזי הרלציה מעל הקבוצה $A$.

\subsection*{תחום וטווח}
תהי $R$ רלציה מ$A$ ל$B$. אזי התחום של $R$ )מסומן $\domain{R}$( הינו תת-קבוצה של $A$ אשר בתוכה נמצאים כל האיברים של $A$ שמיוחסים לאיבר/ים כלשהם ב$B$, והוא מוגדרת כך:
\[
\domain{R} = \left\{\alpha \in A \mid \exists \beta \in B, (\alpha, \beta) \in R\right\}
\]
בדומה, הטווח של $R$ )מסומן $\range{R}$( הינו תת-קבוצה של $B$ אשר בתוכה נמצאים כל האיברים של $B$ אשר מיוחסים לאיבר/ים כלשהם ב$A$, והוא מוגדרת כך:
\[
\range{R} = \left\{\beta \in B \mid \exists \alpha \in A, (\alpha, \beta) \in R\right\}
\]

\subsection*{הרלציה ההופכית}
תהי $R$ רלציה מ$A$ ל$B$. אזי קיימת רלציה $\inv{R}$ מ$B$ ל$A$ כך שלכל $\alpha R \beta$ מתקיים $\beta \inv{R} \alpha$ והיא מוגדרת כך:
\[
\inv{R} = \left\{(\beta, \alpha) \mid (\alpha, \beta) \in R\right\}
\]

\subsection*{הרכבת / כפל רלציות}
יהיו $S, R$ רלציות, כאשר $R$ מהקבוצה $A$ לקבוצה $B$ ו$S$ מהקבוצה $B$ לקבוצה $C$. אזי מכפלת הרלציות )נקראת גם הרכבת הרלציות( מסומנת $R \circ S$ או $RS$ ומוגדרת כך:
\[
RS = R \circ S = \left\{(\alpha, \gamma) \mid \exists \beta \in B, (\alpha, \beta) \in R \land (\beta, \gamma) \in S\right\}
\]
כפל רלציות הוא אסוציאטיבי, כלומר עבור שלוש רלציות $R, S, T$ )כאשר כמובן מוגדרות המכפלות ביניהן( מתקיים $R(ST) = (RS)T$. בנוסף, הרלציה ההופכית של מכפלת רלציות נראית כך:
\[
\inv{(RS)} = \inv{S}\inv{R}
\]

\subsection*{רלציית הזהות}
יחס הזהות על קבוצה $A$ יסומן ב$I_{A}$ ומוגדר כך:
\[
I_{A} = \left\{(\lambda, \lambda) \mid \lambda \in A\right\} \equiv \forall \alpha, \beta \in A, (\alpha, \beta) \in I_{A} \iff \alpha = \beta
\]
\clearpage

\section*{רלציות -- המשך}

\subsection*{תכונות של רלציות}
\begin{itemize}
\item \definition{רפלקסיביות}{$\forall a \in A,\;(aRa) \equiv I_{A} \subseteq R$}
\item \definition{סימטריות}{$\forall a, b \in A,\;(aRb \bidiarrow{} bRa) \equiv R = \inv{R}$}
\item \definition{אנטי-סימטריות}{$\forall a, b \in A,\;(aRb \land bRa \opr{} a = b)$}
\item \definition{טרנזיטיביות}{$\forall a, b, c \in A,\;(aRb \land bRc \opr{} aRc) \equiv R^{2} \subseteq R$}
\end{itemize}

\subsection*{סגור של רלציה ביחס לתכונה מסויימת}
תהי $R$ רלציה מעל $A$. הסגור של $R$ ביחס לתכונה מסויימת הוא רלציה $S$ מעל $A$ המקיימת את התכונה הזאת, מכילה את $R$ ומוכלת בכל רלציה מעל $A$ המקיימת את התכונה ומכילה את $R$. הסגור הטרנזיטיבי של רלציה $R$ הוא:
\[
S = R \cup R^{2} \cup R^{3} \cup \dots = \bigcup^{\infty}_{1 \le i \in \mN} R^{i}
\]

\subsection*{סוגים שונים של רלציות}
\begin{itemize}
\item \definition{רלציית שקילות}{רלציה רפלקסיבית, טרנזיטיבית וסימטרית}
\item \definition{רלציית קומפטיביליות}{רלציה רפלקסיבית וסימטרית}
\item \definition{רלציית סדר חלקי}{רלציה רפלקסיבית, טרנזיטיבית ואנטיסימטרית. קבוצה עם רלציית סדר חלקי מעליה נקראת קבוצה סדורה חלקית. מסומנת לרוב ב$\le$}
\item \definition{רלציית סדר מלא}{סדר מלא הינו סדר חלקי אשר פועל על כל זוג איברים בקבוצה, כלומר אין איברים בה שאינם ניתנים להשוואה}
\end{itemize}
קבוצה עם סדר חלקי מעליה נקראת קבוצה סדורה חלקית.

\subsection*{איברים מינימליים ומקסימליים, האיבר הקטן ביותר והגדול ביותר}
תהי קבוצה $A$ עם רלציית סדר חלקי מעליה המסומנת ב$\le$. האיבר $\alpha \in A$ ייקרא
\begin{itemize}
\item \definition{איבר מינימלי של $A$}{אם מתקיים $\forall \lambda \in A,\;(\lambda \le \alpha \opr{} \lambda = \alpha)$}
\item \definition{איבר מקסימלי של $A$}{אם מתקיים $\forall \lambda \in A,\;(\alpha \le \lambda \opr{} \lambda = \alpha)$}
\item \definition{האיבר הקטן ביותר ב$A$}{אם $\alpha$ קיים ואם מתקיים $\forall \lambda \in A,\;(\alpha \le \lambda)$}
\item \definition{האיבר הגדול ביותר ב$A$}{אם $\alpha$ קיים ואם מתקיים $\forall \lambda \in A,\;(\lambda \le \alpha)$}
\end{itemize}
בקבוצה סדורה חלקית סופית קיימים איבר מינימלי אחד לפחות ואיבר מקסימלי אחד לפחות. בנוסף, בקבוצה סדורה חלקית יכולים להיות לכל היותר איבר קטן ביותר אחד ואיבר גדול ביותר אחד.
