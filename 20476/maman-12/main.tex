\documentclass[11pt, oneside]{article}
\usepackage{geometry}
\geometry{a4paper}
\usepackage[parfill]{parskip}
\usepackage[nodisplayskipstretch]{setspace}
\usepackage{graphicx,titlesec}
\usepackage{amsmath,amssymb,cancel}

% Hebrew Stuff
\usepackage[utf8x]{inputenc}
\usepackage[english,hebrew]{babel}
\usepackage{hebfont}

% Custom commands
\newcommand{\qed}{\R{$\blacksquare$}}
\newcommand{\br}{\\\\\\\\\\\\\\}
\newcommand{\opr}[1]{\xrightarrow[\text{\R{#1}}]{}}
\newcommand{\bidiarrow}[1]{\underset{\text{\R{#1}}}{\leftrightarrow}}
\newcommand{\ueq}[1]{\underset{\text{\R{#1}}}{=}}
\newcommand{\q}[3]{שאלה #3#2.#1}
\newcommand{\mR}{\mathbb{R}}
\newcommand{\mN}{\mathbb{N}}
\newcommand{\inv}[1]{#1^{-1}}

% Spacing
\titlespacing\subsubsection{0pt}{5pt}{3pt}
\setstretch{0.1}

\title{ממ"ן 21}
\author{יונתן אוחיון}

\begin{document}
\maketitle

%%%%% <Q1> %%%%%
\section{שאלה 1}

\subsection{סעיף א}
$$|A| = 3 \rightarrow |A \times A| = 3^2 \rightarrow |\mathcal{P}(A \times A)| = 2^{3^2} = 512$$
\qed

\subsection{סעיף ב}
ננסה להוכיח ש$S$ יחס שקילות ונגיע לסתירה:
\subsubsection{רפלקסיביות}
רלציה $R$ על $A$ הינה רפלקסיבית אם מתקיים $I_A \subseteq R$ )כלומר $(x, x) \in R$(. נראה ש$S$ רפלקסיבית: 

$$\forall R \in M \opr{} RR = R^2 \opr{} R^2 = R^2 \opr{} (R, R) \in S$$

\subsubsection{סימטריות}
לפי הגדרת $S$, $(R_1, R_2) \in S$ אמם $R_1R_2 = R_2R_1$. לפיכך, $(R_2, R_1) \in S$ גם כן שכן $R_2R_1 = R_1R_2$ לפי ההגדרה.

\subsubsection{טרנזיטיביות}
נראה ש$S$ אינו טרנזיטיבי באמצעות דוגמה נגדית:
\begin{align*}
& R_1 = \{(2, 3)\},\ R_2 = \emptyset,\ R_3 = \{(3,2)\}\\
& R_1R_2 = R_2R_1 = \emptyset \opr{} (R_1, R_2) \in S\\
& R_3R_2 = R_2R_3 = \emptyset \opr{} (R_3, R_2) \in S\\
& R_1R_3 = \{(2, 2)\},\ R_3R_1 = \{(3, 3)\}\\
& R_3R_1 \neq R_1R_3 \opr{} (R_1, R_3) \notin S
\end{align*}

לכן, $S$ אינו יחס שקילות.
\br\qed
%%%%% </Q1> %%%%%
\clearpage

%%%%% <Q2> %%%%%
\section{שאלה 2}
\subsection{סעיף א}
לא נכון. דוגמה נגדית:
\begin{align*}
& R_1 = \{(2, 3)\}\\
& R_2 = \{(3, 2)\}\\
& s(R_1) = s(R_2) = \{(2, 3), (3, 2)\}\\
& R_1 \neq R_2
\end{align*}
\qed

\subsection{סעיף ב}
לא נכון, מכיוון שכל $R \in Range(S)$ הינו יחס סימטרי אך היחס מוגדר על $M$ )שאיבריה אינם בהכרח יחסים סימטריים(.
\br\qed

\subsection{סעיף ג}
לא נכון. דוגמה נגדית:
\begin{align*}
& R_1 = \{(3, 2)\}\\
& R_2 = \{(2, 3)\}\\
& R_1R_2 = \{(3, 3)\}\\
& s(R_1) = \{(2, 3), (3, 2)\}\\
& s(R_2) = \{(3, 2), (2, 3)\}\\
& s(R_1R_2) = \{(3, 3)\}\\
& s(R_1)s(R_2) = \{(2, 2), (3, 3)\}\\
& s(R_1R_2) \neq s(R_1)s(R_2)
\end{align*}
\qed

\subsection{סעיף ד}
נכון. הוכחה:
\begin{align*}
& s(R_1) \ueq{\q{2}{34}{ב}} R \cup \inv{R} \\
& s(s(R)) \ueq{סעיף קודם} s(R) \cup \inv{(s(R))} \ueq{הצבה} R \cup \inv{R} \cup \inv{(R \cup \inv{R})}\\
& s(s(R)) \ueq{\q{2}{6}{ג3.}} R \cup \inv{R} \cup R \cup \inv{R} \ueq{קומוטטיביות} R \cup R \cup \inv{R} \cup \inv{R}\\
& s(s(R)) \ueq{אידמפוטנטיות} R\ \cancel{\cup\ R} \cup \inv{R}\ \cancel{\cup\ \inv{R}} = R \cup \inv{R}\\
& s(s(R)) = s(R))
\end{align*}
\qed
%%%%% </Q2> %%%%%
\clearpage

%%%%% <Q3> %%%%%
\section{שאלה 3}
\subsection{סעיף א}
נוכיח ש$K$ סדר חלקי מעל $F$:
\subsubsection{רפלקסיביות}
נראה ש$K$ רפלקסיבי:
$$\forall f \in F \opr{} f(n) = f(n) \opr{הגדרת גדול שווה} f(n) \leq g(n) \opr{הגדרת היחס} (f, f) \in K$$

\subsubsection{טרנזיטיביות}
נניח שקיימים $f, g, h \in F \opr{} (f, g) \in K \land (g, f) \in K$ ונראה ש$(f, h) \in K$. ע"פ הגדרת היחס $K$,
מכיוון ש$(f, g) \in K$, בהכרח $f(n) \leq g(n)$. בנוסף, מכיוון ש$(g, h) \in K$ בהכרח $g(n) \leq h(n)$. לפיכך ולפי הגדרת
גדול שווה, $f(n) \leq g(n) \leq h(n) \opr{} f(n) \leq h(n)$ ולכן $(f, h) \in K$ גם כן.
\subsubsection{אנטיסימטריות}
אם קיימות $f, g \in F$ כך ש$(f, g) \in K \land (g, f) \in K$, הרי $f(n) \leq g(n) \land g(n) \leq f(n)$, מה שאומר שבהכרח $f(n) = g(n)$. לפיכך, היחס $K$ אנטיסימטרי.

לכן, $K$ יחס חלקי מעל $F$.
\br\qed

\subsection{סעיף ב}
נניח ש$K$ סדר מלא, ניתן דוגמה נגדית ונגיע לסתירה:
\begin{align*}
& f(n) = n\\
& g(n) = 2
\end{align*}
לפי ההנחה, $K$ סדר מלא ועבור כל $f, g \in F$ מתקייים $(f, g) \in K \lor (g, f) \in K$. אך מכיוון שהפונקציה $g$ מחזירה
לכל $n \in \mN$ את המספר 2, לא מתקיים $(f, g) \in K$ וגם לא $(g, f) \in K$ )שכן הראשון מתקיים רק כאשר $f(n) \leq 2$ והשני כאשר $f(n) \geq 2$ אך הם אינם מוגדרים כך(. לפיכך, הגענו לסתירה להנחה ו$K$ אינו סדר מלא.
\br\qed
\clearpage

\setcounter{section}{2}
\section{שאלה 3}
\setcounter{subsection}{2}

\subsection{סעיף ג}
לא נכון, מכיוון שלכל $f, g \in F$ הפונקציה $g(n) = f(n) + 1$ תמיד מוגדרת, ללא קשר ל$f$ )שכן $\mN$ קבוצה אינסופית ותמיד מתקיים $f(n) \leq g(n)$(. לפיכך, אין ב$F$ איברים מקסימליים לגבי $K$.
\br\qed

\subsection{סעיף ד}
כן. לפי הגדרת $F$, כל איברי $F$ הם פונקציות מ$\mN$ ל$\mN$. לפיכך, ומכיוון שהמספר הקטן ביותר ב$\mN$ הוא $0$, האיבר המינימלי ב$F$ לגבי $K$ הוא $f(n) = 0$, שכן תמיד מתקיים $f(n) \leq g(n)$ לכל פונקציה $g \in F$.
\br\qed

\subsection{סעיף ה}
לכל $f \in F$ קיים $g \in F$ שמכסה אותו:
\begin{equation*}
g(n) =
\begin{cases}
f(n), & n \neq 1\ \text{\R{אם}} \br
f(n) + 1, & \text{\R{אחרת}}
\end{cases}
\end{equation*}
אך אם נשים לב, נוכל להחליף את התנאי $n = 1$ ואת היפוכו )במקרה הזה -- $\neg(n \neq 1) \opr{} n = 1$( בכל תנאי אחר והפונקציה $g$ עדיין תכסה את $f$.
\br\qed

\clearpage

%%%%% </Q3> %%%%%

%%%%% <Q4> %%%%%
\section{שאלה 4}
\subsection{סעיף א}
מכיוון שנתונים לנו שני מקרי בסיס, נרצה לבדוק את נכונות שניהם )ואז נוכל להשתמש גם ב$n = k$ וגם ב$n = k - 1$ בהוכחה(.

\subsubsection{$n = 0$}
$$2 * 3^0 + (-2)^1 = 2 * 1 - 2 = 0 = f(0)$$

\subsubsection{$n = 1$}
$$2 * 3^1 + (-2)^2 = 6 + 4 = 10 = f(1)$$

כעת, נוכל להניח שהתנאי מתקיים ל$n = k$ וגם ל$n = k - 1$ ונוכיח שהוא מתקיים גם ל$n = k + 1$.
\begin{align*}
& f(n) = 2 * 3^n + (−2)^{n+1}\\
& f(n-1) = 2 * 3^{n-1} + (-2)^n
\end{align*}
מההגדרה הרקורסיבית נובע:
$$f(n + 1) = f(n) + 6f(n-1)$$
עכשיו נוכל להציב את ערכי $f(n)$ ו$f(n-1)$ בפונקציה הרקורסיבית ולהוכיח:
\begin{align*}
f(n+1) & = 2 * 3^n + (−2)^{n+1} + 6(2 * 3^{n-1} + (-2)^n)\\
& = 6 * 3^{n-1} + -2(-2)^n + 12 * 3^{n-1} + 6(-2)^n\\
& = 6 * 3^{n-1} + 12 * 3^{n-1} + -2(-2)^n + 6(-2)^n\\
& = 18 * 3^{n-1} + 4(-2)^n\\
& = 2 * 3^{n+1} + (-2)^{n+2}
\end{align*}
לפי עקרון האינדוקציה השלמה, הבדיקה והמעבר, התנאי נכון לכל $n$ טבעי.
\br\qed

\subsubsection{סעיף ב}
הפונקציה $f$ אינה על מכיוון שהיא מתאימה לכל $n$ מספר זוגי, ו$\mN$ כולל בתוכו את כל המספרים הטבעיים )ולא רק את הזוגיים(:
\begin{align*}
f(n) & = 2 * 3^n + (-2)^n -2\\
& = 2 * 3^n - 2 * (-2)^n\\
& = 2 * (3^n - (-2)^n)
\end{align*}
\qed


\end{document}