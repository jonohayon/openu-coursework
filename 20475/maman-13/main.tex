\documentclass[11pt, oneside]{article}
\usepackage{geometry}
\geometry{a4paper}
\usepackage[parfill]{parskip}
\usepackage[nodisplayskipstretch]{setspace}
\usepackage{graphicx,titlesec,commath}
\usepackage{amsmath,amssymb,cancel}

% Hebrew Stuff
\usepackage[utf8x]{inputenc}
\usepackage[english,hebrew]{babel}
\usepackage{hebfont}

% Custom commands
\newcommand{\qed}{\R{$\blacksquare$}}
\newcommand{\br}{\\\\\\\\\\}
\newcommand{\opr}[1]{\xrightarrow[\text{#1}]{}}
\newcommand{\logr}[1]{\underset{\text{#1}}{\Longrightarrow}}
\newcommand{\bidiarrow}[1]{\underset{\text{#1}}{\leftrightarrow}}
\newcommand{\ueq}[1]{\underset{\text{#1}}{=}}
\newcommand{\mC}{\mathbb{C}}
\newcommand{\mR}{\mathbb{R}}
\newcommand{\mN}{\mathbb{N}}
\newcommand{\mZ}{\mathbb{Z}}
\newcommand{\mQ}{\mathbb{Q}}
\newcommand{\inv}[1]{\frac{1}{#1}}

% Custom text commands (for Hebrew)
\newcommand{\q}[3]{\R{שאלה #3#2.#1}}
\newcommand{\m}[3]{\R{משפט #3#2.#1}}
\newcommand{\h}[3]{\R{הגדרה #3#2.#1}}
\newcommand{\ms}[3]{\R{מסקנה #3#2.#1}}
\renewcommand{\l}[3]{\R{למה #3#2.#1}}

% Custom commands for this document
\newcommand{\defi}[3]{\int_{#1}^{#2} #3}
% \newcommand{\evi}[3]{\left.#1\right\rvert_{#2}^{#3}}
\newcommand{\evi}[3]{\eval[2]{\;#1\;}_{#2}^{#3}}
\newcommand{\sub}[2]{\begin{bmatrix} #1\br #2\end{bmatrix}}
\newcommand{\subo}[1]{\begin{bmatrix} #1\end{bmatrix}}

% Spacing
\titlespacing\subsection{0pt}{3pt}{3pt}
\setstretch{0.2}

\title{ממן 31}
\author{יונתן אוחיון}

\begin{document}
\maketitle
\begin{abstract}
בממן זה החלטתי להחליף את שאלה 5 בשאלת הרשות.
\end{abstract}
\clearpage

%%%%% <Q1a> %%%%%
\section*{שאלה 1א}
ראשית, נפשט את האינטגרנד בעזרת הצבה ונפרק את האינטגרל לשלושה אינטגרלים שונים:
\[
I = \defi{-1}{\infty}{x^{2}\cos{x^{5}}dx}
= \sub{t = x^{5}}{dx = \frac{1}{5t^{\frac{2}{5}}}dt}
= \frac{1}{5} \defi{-1}{\infty}{\frac{\cancel{t^{\frac{2}{5}}}\cos{t}}{t^{\frac{\not4}{5}}}dt}
= \frac{1}{5}
	\underbrace{
		\defi{-1}{0}{\frac{\cos{t}}{t^{\frac{2}{5}}}dt}
	}_{I_{1}} +
	\frac{1}{5}
	\underbrace{
		\defi{0}{\pi}{\frac{\cos{t}}{t^{\frac{2}{5}}}dt}
	}_{I_{2}} +
	\frac{1}{5}
	\underbrace{
		\defi{\pi}{\infty}{\frac{\cos{t}}{t^{\frac{2}{5}}}dt}
	}_{I_{3}}
\]
כעת, על מנת להראות את ההתכנסות של $I$, עלינו להראות את ההתכנסות של כל אחד מהאינטגרלים $I_{1}, I_{2}$ ו$I_{3}$. נתחיל בהוכחת טענת עזר:
\subsection*{טענת עזר}
נתבונן בפונקציות $f, g$ הבאות:
\[
\abs{f(u)} = \abs{\frac{\cos{u}}{u^{\frac{2}{5}}}} = \frac{\abs{\cos{u}}}{u^{\frac{2}{5}}}, g(u) = \frac{1}{u^{\frac{2}{5}}}
\]
נרצה להוכיח טענת עזר שתשמש אותנו בהוכחת ההתכנסות של $I_{1}$ ו$I_{2}$ -- יהי $0 < b \in \mR$. נוכיח ש$\defi{0}{b}{f(u)du}$ מתכנס בהחלט בתחום. מתקיים כמובן $0 \le \abs{\cos{u}} \le 1$ לכל $u \in \mR$ ולכן גם $0 \le \abs{f(u)} \le g(u)$ לכל $u \in (0, b]$. כעת, מכיוון ש$\frac{2}{5} < 1$, לפי \l{3}{2}{} מתקיים ש$\defi{0}{b}{g(u)du}$ מתכנס. לכן לפי מבחן ההשוואה $\defi{0}{b}{f(t)dt}$ מתכנס בהחלט כנדרש.

\subsection*{האינטגרלים הראשונים -- $I_{1}, I_{2}$}
ראשית, נבצע הצבה:
\[
\defi{-1}{0}{\frac{\cos{t}}{t^{\frac{2}{5}}}dt}
= \sub{u = -t}{dt = -du}
= -\defi{1}{0}{\frac{\cos({-u})}{(-u)^{\frac{2}{5}}}du}
= \defi{0}{1}{\frac{\cos{u}}{u^{\frac{2}{5}}}du}
\]
מטענת העזר נוכל לראות שאם נבחר $b = 1$ נקבל ש$I_{1}$ מתכנס בהחלט ולפיכך מתכנס כנדרש. כעת, אם נבחר $b = \pi$ נקבל שלפי מבחן ההשוואה $I_{2}$ מתכנס בהחלט ולפיכך מתכנס כנדרש.

\subsection*{האינטגרל השלישי -- $I_{3}$}
נרצה להשתמש במבחן דיריכלה עבור הוכחת התכנסות $I_{3}$ עם $f(t) = t^{-\frac{2}{5}}$ ו$g(t) = \cos{t}$ בתחום $[\pi, \infty)$. נראה שתנאי המשפט מתקיימים: ראשית, נגזור את $f$: $f^{\prime}(t) = -\frac{2}{5}t^{-\frac{7}{5}}$, אשר היא כמובן פונקציה רציפה. בנוסף כמובן שמתקיים $\lim_{t \to \infty} f(t) = 0$. כעת נתבונן ב$G(t)$:
\[
G(t) = \defi{\pi}{t}{g(x)dx} = \evi{-\sin{x}}{x = t}{x = \pi} = -\sin{t}
\]
כמובן שמתקיים $\abs{G(t)} \le 1$ לכל $t \in [\pi, \infty)$ ולכן היא חסומה בקטע. כעת, מכיוון שמתקיימים כל תנאי המשפט, האינטגרל $\defi{\pi}{\infty}{f(t)g(t)dt}$ מתכנס, כלומר $I_{3}$ מתכנס כנדרש.

מכיוון ש$I_{1}, I_{2}, I_{3}$ מתכנסים מתקיים ש$I$ מתכנס בתנאי כנדרש.
\br\qed
\clearpage
%%%%% </Q1a> %%%%%

%%%%% <Q1b> %%%%%
\section*{שאלה 1ב}
נגדיר:
\[
f(x) = \frac{\sin({x^{2} - 1})}{(x - 1)^{2}}, g(x) = \frac{1}{x - 1}
\]
נחשב את הגבול $\lim_{x \to 1} \frac{f(x)}{g(x)}$:
\[
\lim_{x \to 1} \frac{f(x)}{g(x)}
= \lim_{x \to 1} \frac{\sin({x^{2} - 1})(x + 1)}{x^{2} - 1}
= \subo{t = x^{2} - 1}{}
= \lim_{t \to 0} \frac{\sin{t}}{t} \cdot \lim_{x \to 1} (x + 1) = 2
\]
לכל $x \in (1, 2]$ מתקיים $f(x) > 0 \land g(x) > 0$ וגם האינטגרל $\defi{1}{2}{g(x)dx} = \defi{1}{2}{\frac{1}{x - 1}dx}$ מתבדר ולכן כל תנאי \m{3}{5}{} מתקיימים והאינטגרל $\defi{1}{2}{f(x)dx}$ מתדבר.

לכן, גם האינטגרל $\defi{0}{\infty}{f(x)dx} = \defi{0}{\infty}{\frac{\sin({x^{2} - 1})}{(x - 1)^{2}}dx}$ מתדבר כנדרש.
\br\qed
\clearpage
%%%%% </Q1b> %%%%%

%%%%% <Q2> %%%%%
\section*{שאלה 2}
\subsection*{גבול עזר}
ראשית, נחשב את הגבול הבא אשר יעזור לנו בפתרון השאלה:
\[
\lim_{x \to 0} \frac{x - \sin{x}}{x^{3}}
\ueq{L'H}
\lim_{x \to 0} \frac{1 - \cos{x}}{3x^{2}}
\ueq{L'H}
\lim_{x \to 0} \frac{\sin{x}}{6x}
= \frac{1}{6} \lim_{x \to 0} \frac{\sin{x}}{x} = \frac{1}{6}
\]
בנוסף, מכיוון ש$\lim_{x \to \infty} \frac{1}{x} = 0$ ו$\sin{x}$ חסומה, נקבל מכלל האפסה כפול חסומה מאינפי 1 ש$\lim_{x \to \infty} \frac{\sin{x}}{x} = 0$.

\subsection*{פתרון השאלה}
נסמן $f(x) = \frac{x - \sin{x}}{x^{\alpha} + x^{\beta}}$ ונוכל לראות שמתקיים $f(x) > 0$ לכל $x > 0$. ראשית, נפצל את האינטגרל לשניים ונמצא את התנאים עבור שניהם:
\[
I = \defi{0}{\infty}{f(x)dx}
= \underbrace{\defi{0}{1}{f(x)dx}}_{I_{1}} +
\underbrace{\defi{1}{\infty}{f(x)dx}}_{I_{2}}
\]
כעת, נעבור לבדיקות ההתכנסות לערכי $\alpha, \beta \in \mR$ המקיימים $\beta > \alpha$:

\subsubsection*{התכנסות $I_{1}$}
על מנת להוכיח את התכנסות $I_{1}$ נרצה להשתמש במבחן ההשוואה הגבולי. ניקח $g(x) = \frac{x^{3}}{x^{\alpha}} = \frac{1}{x^{3 - \alpha}}$ ונחשב את הגבול $\lim_{x \to 0} \frac{f(x)}{g(x)}$:
\[
\lim_{x \to 0} \frac{f(x)}{g(x)}
= \lim_{x \to 0} \frac{x^{\alpha}(x - \sin{x})}{x^{3}(x^{\alpha} + x^{\beta})}
= \lim_{x \to 0} \frac{x - \sin{x}}{x^{3}} \cdot \lim_{x \to 0} \frac{1}{1 + x^{\beta - \alpha}}
= \frac{1}{6}
\]
כעת, לפי \l{3}{2}{} נוכל לראות ש$\defi{0}{1}{g(x)dx}$ מתכנס אמ``''מ $\alpha - 3 < 1$, או אם $\alpha < 4$. לכן ממבחן ההשוואה הגבולי נקבל ש$I_{1}$ מתכנס אמ``''מ $\alpha < 4$ כנדרש.

\subsection*{התכנסות $I_{2}$}
נרצה להוכיח את התכנסות $I_{2}$ גם בעזרת מבחן ההשוואה הגבולי. ניקח $h(x) = \frac{x}{x^{\beta}} = \frac{1}{x^{\beta - 1}}$ ונחשב את הגבול $\lim_{x \to \infty} \frac{f(x)}{h(x)}$:
\[
\lim_{x \to \infty} \frac{f(x)}{h(x)}
= \lim_{x \to \infty} \frac{x^{\beta}(x - \sin{x})}{x(x^{\alpha} + x^{\beta})}
= \left(1 - \lim_{x \to \infty} \frac{\sin{x}}{x}\right) \cdot \lim_{x \to \infty} \frac{1}{1 + x^{\alpha - \beta}} = 1
\]
לפי \l{3}{21}{} נראה ש$\defi{1}{\infty}{h(x)dx}$ מתכנס אמ``''מ $\beta - 1 > 1$, כלומר אם $\beta > 2$. לכן נקבל ממבחן ההשוואה הגבולי ש$I_{2}$ מתכנס אממ $\beta > 2$ כנדרש.

לסיכום, אם נאחד את שתי התנאים ונניח כי $\beta > \alpha$, נקבל כי $I$ מתכנס אמ``''מ $\beta > 2$ וגם $\alpha < 4$ כנדרש.
\br\qed
\clearpage
%%%%% </Q2> %%%%%

%%%%% <Q3> %%%%%
\section*{שאלה 3}
מהנתון $\lim_{x \to \infty} x^{2}f(x) = 1$ נקבל שקיים $x_{0} > 0$ כך ש$f(x) > 0$ לכל $x \in (x_{0}, \infty)$. בנוסף, אם נבחר $g(x) = \frac{1}{x^{2}}$ נקבל ש$\lim_{x \to \infty} \frac{f(x)}{g(x)} = 1$.

לכן, מכיוון ש$\defi{x_{0}}{\infty}{\frac{dx}{x^{2}}}$ מתכנס, לפי מבחן ההשוואה הגבולי נקבל ש$\defi{x_{0}}{\infty}{f(x)dx}$ מתכנס. נסמן: $\defi{x_{0}}{\infty}{f(x)dx} = F \in \mR$. כעת, נתבונן בגבול שאנו צריכים לחשב:
\[
\lim_{n \to \infty} \defi{0}{1}{f(nx)dx}
= \sub{t = nx}{dx = \frac{1}{n}dt}
= \lim_{n \to \infty} \defi{0}{n}{\frac{1}{n}f(t)dt}
= \lim_{n \to \infty} \frac{1}{n}\defi{0}{n}{f(t)dt}
\]
מהיינה נקבל ש$\lim_{n \to \infty} \defi{0}{n}{f(t)dt} = F$ וכמובן $\lim_{n \to \infty} \frac{1}{n} = 0$, ולכן מתקיים:
\[
\lim_{n \to \infty} \defi{0}{1}{f(nx)dx}
= \lim_{n \to \infty} \frac{1}{n}\defi{0}{n}{f(t)dt}
= F \cdot 0 = 0
\]
וחישבנו את הגבול בשאלה כנדרש.
\br\qed
%%%%% </Q3> %%%%%

%%%%% <Q4> %%%%%
\section*{שאלה 4}
נרצה להשתמש במבחן דיריכלה על מנת להוכיח את ההתכנסות. נסמן $g(x) = f^{\prime}(x)\sin({f(x)})$. נשים לב כי מתקיים
\[
f^{\prime}(x)\sin({f(x)}) = \left(-\cos(f(x))\right)^{\prime} \logr{} \int g(x)dx = -\cos(f(x)) + C
\]
ולכן $g$ רציפה ב$[a, \infty)$, הפונקציה $G(x) = \defi{a}{x}{g(t)dt}$ מקיימת
\[
G(x) = \evi{-\cos(f(x))}{a}{t} = \cos(f(a)) - \cos(f(t))
\]
ולכן כמובן חסומה )מכיוון ש$\cos$ חסומה ו$\cos(f(a))$ קבוע( ב$[a, \infty)$. כעת, נתבונן ב$h(x) = \frac{1}{f(x)}$. נגזור:
\[
h^{\prime}(x) = -\frac{f^{\prime}(x)}{f^{2}(x)}
\]
לכן מאינפי 1 נקבל ש$h$ מונוטונית יורדת בתחום, וכמובן ש$h^{\prime}$ רציפה בתחום )כי $f, f^{\prime}, x^{2}$ רציפות בתחום ו$h^{\prime}$ הנה מנה והרכבה שלהן(. בנוסף, מהנתון נקבל ש$\lim_{x \to \infty} h(x) = \frac{1}{\lim_{x \to \infty} f(x)} = 0$, ולכן ממבחן דיריכלה נקבל כי $\defi{a}{\infty}{h(x)f(x)dx}$ מתכנס, או כלומר שהאינטגרל
\[
\defi{a}{\infty}{\frac{f^{\prime}(x)}{f(x)}\sin(f(x))dx}
\]
מתכנס כנדרש.
\br\qed
\clearpage
%%%%% </Q4> %%%%%

%%%%% <Q6> %%%%%
\section*{שאלת הרשות}
נניח בשלילה שהטענה לא נכונה. אזי קיים $\varepsilon > 0$ כך שלכל $M \in \mR$ קיים $c > M$ המקיים $\abs{f(c)} \ge \varepsilon$. מהנתון $f$ רציפה במ``''ש ולכן קיים $\delta \in (0, c - M)$ כך שלכל $a < x < y < x + \delta$ מתקיים $\abs{f(y) - f(x)} < \frac{\varepsilon}{2}$.

לפיכך נסיק מאי-שוויון המשולש שלכל $x \in (c - \delta, c + \delta)$ מתקיים $\abs{f(x)} \ge \abs{f(c)} - \abs{f(x) - f(c)} > \varepsilon - \frac{\varepsilon}{2} = \frac{\varepsilon}{2}$. לכן לפי משפט הערך הממוצע קיים $x_{0} \in [c - \delta, c + \delta]$ כך שמתקיים
\[
\abs{\defi{c - \delta}{c + \delta}{f(x)dx}} = \abs{2\delta \cdot f(x_{0})} \ge 2\delta \cdot \frac{\varepsilon}{2} = \delta\epsilon
\]

לפיכך, קיבלנו כי קיים $\varepsilon_{1} = \delta\epsilon$ כך שלכל $M$ קיימים $c + \delta > c - \delta > M$ כך ש$\abs{\defi{c - \delta}{c + \delta}{f(x)dx}} \ge \varepsilon_{1}$ ולכן לפי מבחן קושי האינטגרל $\defi{a}{\infty}{f(x)dx}$ מתבדר, בסתירה לנתון.

הגענו לסתירה והוכחנו את נכונות הטענה בשאלה בדרך השלילה כנדרש.
\br\qed
%%%%% </Q6> %%%%%

\end{document}