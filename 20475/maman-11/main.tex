\documentclass[11pt, oneside]{article}
\usepackage{geometry}
\geometry{a4paper}
\usepackage[parfill]{parskip}
\usepackage[nodisplayskipstretch]{setspace}
\usepackage{graphicx,titlesec,commath}
\usepackage{amsmath,amssymb,cancel}

% Hebrew Stuff
\usepackage[utf8x]{inputenc}
\usepackage[english,hebrew]{babel}
\usepackage{hebfont}

% Custom commands
\newcommand{\qed}{\R{$\blacksquare$}}
\newcommand{\br}{\\\\\\\\\\\\\\}
\newcommand{\opr}[1]{\xrightarrow[\text{#1}]{}}
\newcommand{\logr}[1]{\underset{\text{#1}}{\Longrightarrow}}
\newcommand{\bidiarrow}[1]{\underset{\text{#1}}{\leftrightarrow}}
\newcommand{\ueq}[1]{\underset{\text{#1}}{=}}
\newcommand{\mC}{\mathbb{C}}
\newcommand{\mR}{\mathbb{R}}
\newcommand{\mN}{\mathbb{N}}
\newcommand{\mZ}{\mathbb{Z}}
\newcommand{\mQ}{\mathbb{Q}}
\newcommand{\inv}[1]{\frac{1}{#1}}

% Custom text commands (for Hebrew)
\newcommand{\q}[3]{\R{שאלה #3#2.#1}}
\newcommand{\m}[3]{\R{משפט #3#2.#1}}
\newcommand{\h}[3]{\R{הגדרה #3#2.#1}}
\newcommand{\ms}[3]{\R{מסקנה #3#2.#1}}

% Custom commands for this document
\newcommand{\defi}[3]{\int^{#2}_{#1} #3}

% Spacing
\titlespacing\subsubsection{0pt}{5pt}{3pt}
\setstretch{0.1}

\title{ממן 11}
\author{יונתן אוחיון}

\begin{document}
\maketitle
\clearpage

\section*{שאלת הרשות}
נסמן $f^{n}(x) = (f(x))^{n}$. תהי $f$ פונקציה רציפה ואי-שלילית בתחום $[0, 1]$. נוכיח כי הסדרה $a_{n} = \defi{0}{1}{f^{n}(x)dx}$ מתכנסת אמ``''מ $f(x) \le 1$ לכל $x$ בתחום.
\subsection*{כיוון א}
נניח כי $0 \le f(x) \le 1$ לכל $x \in [0, 1]$. נראה כעת באינדוקציה כי $0 \le f^{n + 1}(x) \le f^{n}(x) \le 1$ לכל $n$ טבעי:

נוכיח את מקרה הבסיס בו $n = 1$. ידוע כי $0 \le f(x) \le 1$, ולכן נוכל לכפול את אי השוויון ב$f(x)$ ולקבל ש$0 \le f^{2}(x) \le f(x) \le 1$ כנדרש.\\
נניח כעת כי הטענה נכונה עבור $n = k$ ונוכיח עבור $n = k + 1$:
\[
0 \le f^{k + 1}(x) \le f^{k}(x) \le 1 \logr{$0 \le f(x)$} f(x) \cdot f^{k + 1}(x) \le f(x) \cdot f^{k}(x) \le f(x) \logr{} 0 \le f^{k + 2}(x) \le f^{k + 1}(x) \le 1
\]
לכן לפי עקרון האינדוקציה השלמה מתקיים $0 \le f^{n + 1}(x) \le f^{n}(x) \le 1$ לכל $n$ טבעי.

כעת, ממונוטוניות האינטגרל נובע כי $0 \le \defi{0}{1}{f^{n + 1}(x)dx} \le \defi{0}{1}{f^{n}(x)dx} \le 1$ לכל $n$ טבעי, כלומר לכל $n$ טבעי מתקיים $0 \le a_{n + 1} \le a_{n} \le 1$.לפיכך, הסדרה $(a_{n})$ מונוטונית יורדת וחסומה, ולכן ממשפט באינפי 1 נובע כי $(a_{n})$ מתכנסת כנדרש.

\subsection*{כיוון ב}
נניח כי $\lim_{n \to \infty} a_{n} =א L \in \mR$ ונניח בשלילה כי $f(x) > 1$ לכל $x$ בתחום. נגדיר את הסדרה $b_{n} = f^{n}(x)$. נראה כי $b_{n + 1} > b_{n} > 1$ לכל $n$ טבעי: ידוע כי $f(x) > 1$ לכל $x \in [0, 1]$. נכפול את אי השוויון ב$f(x)$ ונקבל $f^{2}(x) > f(x) > 1$. באינדוקציה נקבל כי $f^{n + 1}(x) > f^{n}(x) > 1$ לכל $n$ טבעי כנדרש. ממונוטוניות האינטגרל נובע כי $a_{n + 1} > a_{n}$ לכל $n$ טבעי ומכך נובע כי
\[
\abs{\frac{a_{n + 1}}{a_{n}}} > 1 \logr{} \lim_{n \to \infty} \abs{\frac{a_{n + 1}}{a_{n}}} > 1
\]
ולכן ממבחן המנה לגבולות נובע כי $\lim_{n \to \infty} a_{n} = \infty$, ובפרט $\lim_{n \to \infty} a_{n} \not\in \mR$, בסתירה להנחה. לפיכך, $f(x) \le 1$ לכל $x \in [0, 1]$ כנדרש.

לכן, לסיכום, הוכחנו שהסדרה $a_{n} = \defi{0}{1}{f^{n}(x)dx}$ מתכנסת אמ``''מ $f(x) \le 1$ לכל $x \in [0, 1]$.
\br\qed

\end{document}