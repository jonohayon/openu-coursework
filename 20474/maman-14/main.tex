\documentclass[11pt, oneside]{article}
\usepackage{geometry}
\geometry{a4paper}
\usepackage[parfill]{parskip}
\usepackage[nodisplayskipstretch]{setspace}
\usepackage{graphicx,titlesec}
\usepackage{amsmath,amssymb,cancel,mathtools}

% Hebrew Stuff
\usepackage[utf8x]{inputenc}
\usepackage[english,hebrew]{babel}
\usepackage{hebfont}

% Custom commands
\newcommand{\qed}{\R{$\blacksquare$}}
\newcommand{\br}{\\\\\\\\\\\\\\}
\newcommand{\opr}[1]{\xrightarrow[\text{#1}]{}}
\newcommand{\logr}[1]{\xRightarrow[\text{#1}]{}}
\newcommand{\bidiarrow}[1]{\underset{\text{#1}}{\leftrightarrow}}
\newcommand{\ueq}[1]{\underset{\text{#1}}{=}}
\newcommand{\mC}{\mathbb{C}}
\newcommand{\mR}{\mathbb{R}}
\newcommand{\mN}{\mathbb{N}}
\newcommand{\mZ}{\mathbb{Z}}
\newcommand{\mQ}{\mathbb{Q}}
\newcommand{\inv}[1]{#1^{-1}}

% Custom text commands (for Hebrew)
\newcommand{\q}[3]{\R{שאלה #3#2.#1}}
\newcommand{\m}[3]{\R{משפט #3#2.#1}}
\newcommand{\h}[3]{\R{הגדרה #3#2.#1}}
\newcommand{\ta}[3]{\R{טענה #3#2.#1}}
\newcommand{\ms}[3]{\R{מסקנה #3#2.#1}}

% Custom commands for this document
\usepackage[inline]{enumitem}
\newcommand{\finv}[1]{\frac{1}{#1}}
\DeclareMathOperator{\Ima}{Im}
\DeclarePairedDelimiter{\floor}{\lfloor}{\rfloor}
\DeclarePairedDelimiter{\abs}{\lvert}{\rvert}

% Spacing
\titlespacing\subsubsection{0pt}{5pt}{3pt}
\setstretch{0.1}

\title{ממ``''ן 41}
\author{יונתן אוחיון}

\begin{document}
\maketitle

%%%%% <Q1> %%%%%
\section*{שאלה 1}
\subsection*{סעיף א}
לא נכון. דוגמה נגדית תהיה $f = h, g = k$ )כאשר הפונקציות $h, k$ הן הפונקציות המוגדרות בממן(. נוכל לחשב ולראות ש$h(0) = h(1) = 0$ ולכן היא לא חח``''ע, אך גם ש$(h \circ k)(x) = x$:
\[
(h \circ k)(x)
= \begin{cases}
k(x) & k(x) \le 0\\\\\\
k(x) - 1 & k(x) > 0
\end{cases}
= \begin{cases}
x & x \le 0\\\\\\
x + 1 - 1 & x > 0
\end{cases}
= \begin{cases}
x & x \le 0\\\\\\
x & x > 0
\end{cases} = x
\]
ולכן אם $(f \circ g) = id$, $f$ אינה בהכרח חח``''ע כנדרש.
\br\qed

\subsection*{סעיף ב}
נכון. נניח בשלילה ש$g$ לא חח``''ע, כלומר מתקיים
\[
\exists a, b \in \mR,\ g(a) = g(b) \land a \neq b
\]
נפעיל את $f$ על שני הצדדים ונקבל:
\[
f(g(a)) = f(g(b)) \equiv (f \circ g)(a) = (f \circ g)(b) \logr{$(f \circ g) = id$} a = b
\]
בסתירה להנחה. לפיכך, $g$ חח``''ע כנדרש.
\br\qed
\clearpage

\section*{שאלה 1 -- המשך}
\subsection*{סעיף ג + ד}
לא נכון. יהיו $f, g: \mR \to \mR$ מוגדרות באופן הבא: $f(x) = g(x) = \finv{x}$. שתי הפונקציות לא מוגדרות בנקודה $x = 0$, ולכן $\Ima{f} \neq \mR \land \Ima{g} \neq \mR$ ולכן לא על. נשים לב ש$(f \circ g) = id$:
\[
(f \circ g)(x) = f\left(\finv{x}\right) = \finv{\finv{x}} = x
\]
נוכל לשים לב שהדוגמה הזאת עובדת גם עבור המקרה הנדרש בסעיף ד, שכן גם $g(x)$ לא על במקרה זה. לפיכך, אם $(f \circ g) = id$, לא $f$ ולא $g$ בהכרח על כנדרש.
\br\qed

\subsection*{סעיף ה}
לא נכון. ניתן בתור דוגמה נגדית את אותה הדוגמה מסעיף א. ראשית, נגדיר את פונקציית ההרכבה $k \circ h$:
\[
(k \circ h)(x) = \begin{cases}
h(x) & h(x) \le 0\br
h(x) - 1 & h(x) > 0
\end{cases}
\]
אזי $(f \circ g)(x) = (h \circ k)(x) = x$ כפי שהראינו בסעיף א, אך מחישוב נובע $h(0) = h(1) = 0$, ולכן לפי ההגדרה, $(k \circ h)(0) = (k \circ h)(1) = 0$ ולא מתקיים $(k \circ h) = id$ כנדרש.
\br\qed

\subsection*{סעיף ו}
נכון. נניח ש$g$ על. אזי לפי הנתון מתקיים $f(g(x)) = x \logr{} g(f(g(x))) = g(x)$. מכיוון ש$g$ על, לכל $y \in \mR$ קיים $x \in \mR$ כך ש$y = g(x)$. לפיכך, $g(f(y)) = y$ לכל $y$, כלומר $\forall y \in \mR, (g \circ f)(y) = y$ כנדרש.
\br\qed
\clearpage
%%%%% </Q1> %%%%%

%%%%% <Q2> %%%%%
\section*{שאלה 2}
\subsection*{סעיף א}
יהי $0 < \varepsilon$. נבחר $\delta = \frac{1}{\pi}$. לפיכך, מתקיים
\[
0 < \abs*{x - \tfrac{2}{\pi}} < \delta \logr{} x \in N^{*}_{\delta}\left(\tfrac{2}{\pi}\right) = \left(\tfrac{1}{\pi}, \tfrac{3}{\pi}\right)
\]
לכן:
\[
\tfrac{1}{x} \in (\tfrac{\pi}{3}, \pi) \land \tfrac{1}{x} \neq \tfrac{\pi}{2}
\logr{}
\sin{\tfrac{1}{x}} \in \left(0, \sin{\tfrac{\pi}{3}}\right)  \land \sin{\tfrac{1}{x}} \neq 1\ (= \sin{\tfrac{\pi}{2}})
\]
ולכן $\sin{\tfrac{1}{x}} < 1$ ובפרט $0 < \sin{\tfrac{1}{x}} < 1$ )בטווח זה, $\sin{x} > 0$( ולכן
\[
\forall x \in N^{*}_{\delta}\left(\tfrac{2}{\pi}\right),\ \floor{\sin{\tfrac{1}{x}}} = 0
\logr{} \abs*{\floor{\sin{\tfrac{1}{x}}}} = 0
\logr{} \abs*{\floor{\sin{\tfrac{1}{x}}}} < \varepsilon
\]
לכל $0 < \varepsilon$ ולכן $f(x) \opr{$x \to \tfrac{2}{\pi}$} 0$ כנדרש.
\br\qed

\subsection*{סעיף ב}
נסמן: $f(x) = \sqrt{2x - \sin{3x}}$. יהי $M_{1} \in \mR$. נבחר $M_{2} = \frac{M^{2}_{1} + 1}{2}$. אזי:
\begin{align*}
x > M_{2}
& \logr{} x > \frac{M^{2}_{1} + 1}{2}\\
\sin{x} \le 1 \text{\R{נימוק: }}& \logr{} x > \frac{M^{2}_{1} + \sin{3x}}{2}\\
& \logr{} 2x > M^{2}_{1} + \sin{3x}\\
& \logr{} 2x - \sin{3x} > M^{2}_{1}\\
& \logr{} \sqrt{2x - \sin{3x}} > M_{1} \logr{} f(x) > M_{1}
\end{align*}
לכן, לפי \h{4}{55}{}, $f(x) \opr{$x \to \infty$} \infty$ כנדרש.
\br\qed
\clearpage
%%%%% </Q2> %%%%%

%%%%% <Q3> %%%%%
\section*{שאלה 3}
\subsection*{סעיף א1}
ההגדרה: נגיד כי $\lim_{x \to \infty} f(x) = L$ אם קיים $L \in \mR$ כך שלכל $0 < \varepsilon$ קיים $M \in \mR$ כך שלכל $x > M$ מתקיים $\abs{f(x) - L} < \varepsilon$.

נשלול: נגיד כי ל$f(x)$ לא קיים גבול ממשי כש$x \to \infty$ אם לכל $L \in \mR$, קיים $0 < \varepsilon$ כך שלכל $M \in \mR$ קיים $x > M$ כך ש$\abs{f(x) - L} \ge \varepsilon$. שללנו את ההגדרה כנדרש.
\br\qed

\subsection*{סעיף ב1}
יהיו $L, M \in \mR$. נבחר $\varepsilon = \tfrac{1}{6}$ ונניח כי $x > M$. ידוע לנו ש$-1 \le \cos{x} \le 1$ עבור כל $x \in \mR$ ולכן מתקיים
\begin{align*}
4 \le 5 + \cos{x} \le 6
& \logr{} \frac{1}{6} \le \frac{1}{5 + \cos{x}}\\
\forall x \in \mR, \abs{x} \ge 0 \text{\R{נימוק: }}& \logr{} \frac{1}{6} \le \frac{\abs{4 - 5L - L\cos{x}}}{6} \le \frac{\abs{4 - 5L - L\cos{x}}}{5 + \cos{x}} = \abs*{\frac{4 - L(5 + \cos{x})}{5 + \cos{x}}}\\
& \;\qquad = \abs*{\frac{4}{5 + \cos{x}} - L} = \abs*{f(x) - L}
\logr{} \abs{f(x) - L} \ge \frac{1}{6}
\end{align*}
ומצאנו $\varepsilon$ כך ש$\abs{f(x) - L} \ge \varepsilon$ לכל $L \in \mR$ ולכן לא קיים ל$f(x)$ גבול ממשי לפי סעיף א1 כנדרש.
\br\qed

\clearpage

\section*{שאלה 3 -- המשך}
\subsection*{סעיף א2}
ההגדרה: נגיד כי $\lim_{x \to \infty} f(x) = L$ אם לכל סדרה $(x_{n})^{\infty}_{n = 1}$ המקיימת $\lim_{n \to \infty}  x_{n} = \infty$, מתקיים $\lim_{n \to \infty} f(x_{n}) = L$.

נשלול: נגיד כי ל$f(x)$ לא קיים גבול ממשי כש$x \to \infty$ אם קיימת סדרה $(x_{n})^{\infty}_{n = 1}$ המקיימת $\lim_{n \to \infty} x_{n} = \infty$ כך שהסדרה $\left(f(x_{n})\right)^{\infty}_{n = 1}$ מתבדרת במובן הצר. שללנו את ההגדרה כנדרש.
\br\qed

\subsection*{סעיף ב2}
תהי $(x_{n})$ סדרה המוגדרת כך: $x_{n} = n\pi$. לפיכך לפי \m{2}{34}{}ד, $\lim_{n \to \infty} x_{n} = \infty$.
ממחזוריות פונקציית הקוסינוס ניתן לראות כי מתקיים
\begin{align*}
& \cos{x_{2n}} = \cos{2\pi n} = 1\\
& \cos{x_{2n-1}} = \cos{(2n-1)\pi} = -1
\end{align*}
נסמן: $a_{n} = f(x_{n})$. נבחר שתי תת-סדרות מכסות של $(a_{n})$: הסדרה $a_{2n}$ והסדרה $a_{2n-1}$. נוכל לראות ששתי סדרות אלו הינן סדרות קבועות:
\begin{align*}
& a_{2n} = \frac{4}{5 + \cos{2\pi n}} = \frac{4}{6} = \frac{2}{3} \logr{} \lim_{n \to \infty} a_{2n} = \frac{2}{3}\\
& a_{2n - 1} = \frac{4}{5 + \cos{(2n - 1)\pi}} = \frac{4}{4} = 1 \logr{} \lim_{n \to \infty} a_{2n - 1} = 1\\
\end{align*}
מכיוון ששתי תת-סדרות אלו הינן תת-סדרות מכסות וגבולותיהן שונים זה מזה, לפי \m{3}{13}{} הסדרה $a_{n} = f(x_{n})$ מתבדרת. לכן לפי ההגדרה בסעיף א2 ל$f(x)$ לא קיים גבול ממשי כנדרש.
\br\qed

\clearpage
%%%%% </Q3> %%%%%

%%%%% <Q4> %%%%%
\section*{שאלה 4}
\subsection*{סעיף א}
נחשב את הגבול:
\begin{align*}
\lim_{x \to 0} \frac{1 - \cos{x}}{x^{2}}
& = \lim_{x \to 0} \frac{(1 - \cos{x})(1 + \cos{x})}{x^{2}(1 + \cos{x})}\\
& = \lim_{x \to 0} \frac{1 - \cos^{2}{x}}{x^{2}(1 + \cos{x})}\\
& = \lim_{x \to 0} \frac{\sin^{2}{x}}{x^{2}(1 + \cos{x})}\\
& = \left(\lim_{x \to 0} \frac{\sin{x}}{x}\right)^{2} \cdot \lim_{x \to 0} \frac{1}{1 + \cos{x}}\\
\text{\R{\m{4}{54}{}}} & = \frac{1}{1 + \lim_{x \to 0} \cos{x}}\\
\text{\R{\ta{4}{44}{}}} & = \framebox{$\dfrac{1}{2}$}\\
\end{align*}
וחישבנו את הגבול כנדרש. 
\br\qed

\subsection*{טענות עזר}
נוכיח כי $\lim_{x \to \infty} \tfrac{1}{x} = 0$ וכי $\lim_{x \to 0^{+}} \tfrac{1}{x} = \infty$.

ראשית, נוכיח כי $\lim_{x \to \infty} x = \infty$: יהי $M_{1} \in \mR$ ונבחר $M_{2} = M_{1}$. לפיכך, לכל $x > M_{1}$ מתקיים $x = f(x) > M_{2} = M_{1}$ וסיימנו. לפיכך, לפי \m{4}{35}{}ה מתקיים $\lim_{x \to \infty} \tfrac{1}{x} = 0$ כנדרש.

שנית, נוכיח את הגבול השני בעזרת הגדרת הגבול לפי היינה: תהי $(x_{n})^{\infty}_{n = 1}$ סדרה אפסה החסומה מלרע ע``''י 0, כלומר $x_{n} > 0$ לכל $n$. לפיכך לפי \m{2}{34}{}ו, $\lim_{n \to \infty} \tfrac{1}{x_{n}} = \infty$. לכן, לפי \h{4}{15}{} מתקיים $\lim_{x \to 0^{+}} \tfrac{1}{x} = \infty$ כנדרש.
\br\qed

בעמוד הבא נראה את החישובים לסעיפים ב וג.
\clearpage
\section*{שאלה 4 -- המשך}
\subsection*{סעיף ב}
נראה שהגבול אינו קיים. נפשט מעט את ביטוי הגבול:
\begin{align*}
\lim_{x \to x_{0}} \frac{\sin^{4}{x}}{x^{7}}
& = \lim_{x \to x_{0}} \frac{\sin^{4}{x}}{x^{4}} \cdot \lim_{x \to x_{0}} \frac{1}{x^{3}}\\
& = \left(\lim_{x \to x_{0}} \frac{\sin{x}}{x}\right)^{4} \cdot \lim_{x \to x_{0}} \frac{1}{x^{3}}\\
\text{\R{\m{4}{84}{}}} & = \lim_{x \to x_{0}} \frac{1}{x^{3}}\\
\end{align*}
לפיכך:
\[
\lim_{x \to 0^{+}} \frac{1}{x^{3}} = 
\left(\lim_{x \to 0^{+}} \frac{1}{x}\right)^{3} = \infty^{3} = \infty \logr{} \lim_{x \to 0^{+}} \frac{\sin^{4}{x}}{x^{7}} = \infty
\]
כעת, נוכיח ש$\lim_{x \to 0^{-}} \tfrac{1}{x^{3}} = -\infty$ לפי \q{4}{57}{}ב. יהי $M \in \mR$ ונבחר $0 < \delta = \abs{M}$. בנוסף נוכל לראות כי מכיוון שאנו מתקרבים ל0 בסביבת $\delta$ \underline{\emph{נקובה}} משמאל, $M$ בהכרח שונה מ0. נניח בנוסף כי $-\delta < x < 0$ ונחלק למקרים:

אם $M > 0$ אז $\delta = \abs{M} = M$ ולכן מתקיים:
\begin{align*}
-M < x < 0
& \logr{} M > -x \logr{} M > \frac{1}{M} > \frac{1}{-x} > \frac{1}{x^{2}} > \frac{1}{x^{3}}\\
& \logr{} M > \frac{1}{x^{3}} \logr{} f(x) < M
\end{align*}
אם $M < 0$ אז $\delta = \abs{M} = -M$ ולכן מתקיים:
\begin{align*}
M < x < 0
& \logr{} M > \frac{1}{M} > \frac{1}{x} > \frac{1}{x^{3}}\\
& \logr{} M > \frac{1}{x^{3}} \logr{} f(x) < M
\end{align*}
והראנו שמתקיים $f(x) < M$ לכל $x$ בסביבה שמאלית של 0 ולכן $\lim_{x \to 0^{-}} \tfrac{1}{x^{3}} = -\infty$. כעת הראינו שהגבול של $f(x)$ בנקודה 0 משמאל שונה מהגבול שלה בנקודה מימין ולכן לא קיים לה גבול כנדרש.
\br\qed
\clearpage

\section*{שאלה 4 -- המשך}
\subsection*{סעיף ג}
נחשב את הגבול:
\begin{align*}
\lim_{x \to \infty} \frac{-3x^{5} + 5x^{3} + 1}{5x^{5} + 3x^{3} - 1}
& = \lim_{x \to \infty} \frac{\cancel{x^{5}}(-3 + \tfrac{5}{x^{2}} + \tfrac{1}{x^{5}})}{\cancel{x^{5}}(5 + \tfrac{3}{x^{2}} - \tfrac{1}{x^{5}})}\\
& = \frac{-3 + \lim_{x \to \infty}\tfrac{5}{x^{2}} + \lim_{x \to \infty}\tfrac{1}{x^{5}}}{5 + \lim_{x \to \infty}\tfrac{3}{x^{2}} - \lim_{x \to \infty}\tfrac{1}{x^{5}}}\\
\text{\R{לפי טענת העזר}} & = \frac{-3 + 5 \cdot 0 + 0}{5 + 3 \cdot 0 - 0} = \framebox{$-\dfrac{3}{5}$}
\end{align*}
וחישבנו את הגבול כנדרש.
\br\qed

\subsection*{טענת עזר לסעיף ד}
תהי $f: \mR \to \mR$ כך ש$\lim_{x \to -\infty} f(x)$ קיים. אזי $\lim_{x \to -\infty} f(x) = \lim_{x \to \infty} f(-x)$. נוכיח בעזרת הגדרת הגבול בסדרות:

תהי $(x_{n})^{\infty}_{n = 1}$ סדרה כך ש$\lim_{n \to \infty} x_{n} = -\infty$ ונסמן $y_{n} = -x_{n}$. אזי לפי \ta{2}{93}{} מתקיים $\lim_{n \to \infty} -x_{n} = \infty$, או במילים אחרות, $\lim_{n \to \infty} y_{n} = \infty$.

לפי הגדרת הגבול בסדרות, הגבול $\lim_{n \to \infty} f(x_{n}) = L$ קיים. בנוסף, מכיוון ש$x_{n} = -y_{n}$, גם הגבול $\lim_{n \to \infty} f(-y_{n})$ קיים ובפרט שווה ל$L$.

לכן, לכל סדרה $(y_{n})^{\infty}_{n = 1}$ המקיימת $\lim_{n \to \infty} y_{n} = \infty$ מתקיים $\lim_{n \to \infty} f(-y_{n}) = L$ ולכן לפי הגדרת הגבול בסדרות מתקיים $\lim_{x \to \infty} f(-x) = \lim_{x \to -\infty} f(x)$ כנדרש.
\br\qed

\subsection*{סעיף ד}
נסמן:
\begin{align*}
f(x) = \sqrt{x^{2} - \sin{x}} - x
& \logr{} f(-x) = \sqrt{x^{2} + \sin{x}} + x\\
& \logr{} g(x) = \sqrt{x} + \frac{x}{x^{2} + \sin{x}},\ h(x) = x^{2} + \sin{x}\\
& \logr{} f(-x) = g(h(x))
\end{align*}
נראה ש$\lim_{x \to -\infty} f(x) = \infty$, אך לפי טענת העזר, די לנו לחשב את $\lim_{x \to \infty} g(h(x))$. בעמוד הבא ניעזר ב\m{4}{93}{} על מנת להראות שגבול זה אכן קיים.
\clearpage

\section*{שאלה 4 -- המשך}
\subsection*{סעיף ד -- המשך}
לפי \m{4}{93}{}, עלינו להראות שהגבולות הבאים קיימים:
\[
\lim_{x \to \infty} g(x) = \infty \qquad\lim_{t \to \infty} h(t) = \infty
\]
נראה זאת בעזרת אריתמטיקה:
\begin{align*}
\lim_{x \to \infty} g(x)
& = \lim_{x \to \infty} \sqrt{x} + \frac{x}{x^{2} + \sin{x}}\\
& = \lim_{x \to \infty} \sqrt{x} + \lim_{x \to \infty} \frac{x}{x^{2} + \sin{x}}\\
& = \lim_{x \to \infty} \sqrt{x} + \lim_{x \to \infty} \frac{1}{x + \frac{\sin{x}}{x}}
\end{align*}
כעת, מכיוון ש$\sin{x}$ חסומה ו$\lim_{x \to \infty} \frac{1}{x} = 0$, מתקיים $\lim_{x \to \infty} \frac{\sin{x}}{x} = 0$ )חסומה כפול אפסה( ולכן $\lim_{x \to \infty} x + \frac{\sin{x}}{x} = \infty$ ולכן $\lim_{x \to \infty} \frac{1}{x + \frac{\sin{x}}{x}} = 0$. לפיכך,
\[
\lim_{x \to \infty} g(x) = \lim_{x \to \infty} \sqrt{x} + \lim_{x \to \infty} \frac{1}{x + \frac{\sin{x}}{x}} = \lim_{x \to \infty} \sqrt{x} = \infty
\]
כנדרש. כעת, נחשב את הגבול השני:
\begin{align*}
\lim_{t \to \infty} h(t)
& = \lim_{t \to \infty} t^{2} + \sin{t}\\
& = \lim_{t \to \infty} t^{2}\left(1 + \frac{\sin{t}}{t^{2}}\right)\\
& = \lim_{t \to \infty} t^{2} \cdot \left(1 + \lim_{t \to \infty} \frac{\sin{t}}{t^{2}}\right)
\end{align*}
כעת, מכיוון ש$\sin{t}$ חסומה ו$\lim_{t \to \infty} \frac{1}{t^{2}} = 0$, מתקיים $\lim_{t \to \infty} \frac{\sin{t}}{t^{2}} = 0$ )חסומה כפול אפסה(. לפיכך,
\[
\lim_{t \to \infty} h(t) = \lim_{t \to \infty} t^{2} \cdot \left(1 + \lim_{t \to \infty} \frac{\sin{t}}{t^{2}}\right) = \lim_{t \to \infty} t^{2} = \infty
\]
כנדרש. לכן, לפי \m{4}{93}{} מתקיים $\lim_{x \to \infty} g(h(x)) = \infty$, כלומר $\lim_{x \to \infty} f(-x) = \infty$. לכן, לפי טענת העזר, $\lim_{x \to -\infty} f(x) = \infty$ וחישבנו את הגבול כנדרש.
\br\qed
\clearpage

\section*{שאלה 4 -- המשך}
\subsection*{סעיף ה -- $k = 0$}
נראה ש$\lim_{x \to 0} \sin{\frac{x}{2}}\floor{\sin{x}} = 0$. ראשית, נוכיח את הגבול מימין. יהי $\varepsilon \in \mR$ ונבחר $\delta = \frac{\pi}{2}$. אזי מתקיים
\begin{align*}
\forall 0 < x < \frac{\pi}{2},\ 0 < \sin{x} < 1
& \logr{} \floor{\sin{x}} = 0 \logr{} \floor{\sin{x}} \in N_{\varepsilon}(0)\\
& \logr{} \lim_{x \to 0^{+}} \floor{\sin{x}} = 0\\
& \logr{} \lim_{x \to 0^{+}} \sin{\frac{x}{2}} \cdot \lim_{x \to 0^{+}} \floor{\sin{x}} = 0 \cdot \lim_{x \to 0^{+}} \sin{\frac{x}{2}} = 0
\end{align*}
כעת, נוכיח את הגבול משמאל ונתחיל מהגבול משמאל של הפונקציה $g_{1}(x) = \floor{\sin{x}}$. יהי $\varepsilon \in \mR$ ונבחר $\delta = \frac{\pi}{2}$. לפיכך,
\begin{align*}
\forall -\frac{\pi}{2} < x < 0,\ -1 < \sin{x} < 0
& \logr{} \floor{\sin{x}} = -1 \logr{} \floor{\sin{x}} \in N_{\varepsilon}(-1)\\
& \logr{} \lim_{x \to 0^{-}} \floor{\sin{x}} = -1\\
& \logr{} \lim_{x \to 0^{-}} \sin{\frac{x}{2}} \cdot \lim_{x \to 0^{-}} \floor{\sin{x}} = -\lim_{x \to 0^{-}} \sin{\frac{x}{2}}
\end{align*}
כעת, נשתמש ב\m{4}{93}{} על מנת להוכיח\; את\; הגבול\; הדרוש,\; כלומר\; עלינו\; להוכיח\quad את\quad הטענות\\\\\\ הבאות:
\begin{center}
\begin{enumerate*}
\item $\sin{x} \opr{$x \to 0$} 0$
\item $\frac{t}{2} \opr{$t \to 0$} 0$
\item קיימת סביבה נקובה $N^{*}_{\delta}(0)$ כך ש$g(t)$ נמצא בתוכה
\end{enumerate*}
\end{center}
הטענה הראשונה נמצאת ב\m{4}{44}{}, ואת הטענה השנייה נראה מאריתמטיקה:
\[
\lim_{t \to 0} \frac{t}{2} = \frac{1}{2} \lim_{t \to 0} t = \frac{1}{2} \cdot 0 = 0
\]
ובנוסף הראינו, לפי הגדרת הגבול, כי בהכרח קיימת סביבה נקובה המקיימת את התנאי הדרוש. לפיכך, לפי \m{4}{93}{} מתקיים:
\[
0 = \lim_{x \to 0} \sin{\frac{x}{2}} = -\lim_{x \to 0^{-}} \sin{\frac{x}{2}}
\]
ולכן מתקיים $\lim_{x \to 0} \sin{\frac{x}{2}} \floor{\sin{x}} = 0$ כנדרש.
\br\qed
\clearpage

\section*{שאלה 4 -- המשך}
\subsection*{סעיף ה -- $k = 1$}
נתבונן בגבול $\lim_{x \to \frac{\pi}{2}} \floor{\sin{x}}$. נבחר $\delta = \frac{\pi}{2}$. עבור כל $0 < x < \pi$ חוץ מ$x = \frac{\pi}{2}$ מתקיים $0 < \sin{x} < 1$, ולכן גם $\floor{\sin{x}} = 0$. לכן, לכל $x \in N^{*}_{\delta}\left(\frac{\pi}{2}\right)$ מתקיים $\floor{\sin{x}} \in N_{\varepsilon}(0)$, כלומר $\lim_{x \to \frac{\pi}{2}} = 0$. לפיכך, מתקיים
\[
\lim_{x \to \frac{\pi}{2}} \sin{\frac{x}{2}}\floor{\sin{x}} = \lim_{x \to \frac{x}{2}} \sin{\frac{x}{2}} \cdot \lim_{x \to \frac{x}{2}} \floor{\sin{x}} = 0 \cdot \lim_{x \to \frac{x}{2}} \sin{\frac{x}{2}} = 0
\]
כנדרש.
\br\qed

\subsection*{סעיף ה -- $k = 2$}
נתבונן בפונקציה $g_{1}(x) = \floor{\sin{x}}$. נוכיח ש$\lim_{x \to \pi^{+}}g_{1}(x) = -1$. נבחר $\delta = \pi$. אזי לכל $\pi < x < 2\pi$ מתקיים $-1 < \sin{x} < 0$, כלומר $\floor{\sin{x}} = -1$ ולכן $\lim_{x \to \pi^{+}} g_{1}(x) = -1$.
כעת, נראה כי $\lim_{x \to \pi^{-}} g_{1}(x) = 0$. נבחר $\delta = \frac{\pi}{2}$. אזי לכל $\frac{\pi}{2} < x < \pi$ מתקיים $0 < \sin{x} < 1$, כלומר $\floor{\sin{x}} = 0$ ולכן $\lim_{x \to \pi^{-}} g_{1}(x) = 0$.

בנוסף, נתבונן בפונקציה $g_{2}(x) = \sin{\frac{x}{2}}$. נוכיח כי $\lim_{x \to \pi} g_{2}(x) = 1$ בעזרת \m{4}{93}{}. ראשית, הפונקציה $\frac{t}{2} = \frac{1}{2}t$ הינה פונקציה לינארית ולכן $\lim_{t \to \pi} \frac{t}{2} = \frac{\pi}{2}$. בנוסף, לפי \q{4}{77}{} מתקיים $\lim_{x \to \frac{\pi}{2}} \sin{x} = 1$. לבסוף, ברור שקיימת סביבה נקובה $N^{*}_{\delta}(\frac{\pi}{2})$ כך ש$\frac{t}{2}$ נמצא בתוכה לפי הגדרת הגבול. לפיכך, $\lim_{t \to \pi} \sin{\frac{t}{2}} = 1$.

כעת, נוכל להראות שהגבול של הפונקציה בשאלה אינו קיים. הגבול מימין הינו
\[
\lim_{x \to \pi^{+}} \sin{\frac{x}{2}}\floor{\sin{x}} = \lim_{x \to \pi^{+}} \sin{\frac{x}{2}} \cdot \lim_{x \to \pi^{+}} \floor{\sin{x}} = 1 \cdot -1 = -1
\]
והגבול משמאל הינו
\[
\lim_{x \to \pi^{-}} \sin{\frac{x}{2}}\floor{\sin{x}} = \lim_{x \to \pi^{-}} \sin{\frac{x}{2}} \cdot \lim_{x \to \pi^{-}} \floor{\sin{x}} = 1 \cdot 0 = 0
\]
כמובן ש$-1 \neq 0$ ולכן אין לפונקציה גבול בנקודה $\pi$ כנדרש.
\br\qed
%%%%% </Q4> %%%%%

\end{document}