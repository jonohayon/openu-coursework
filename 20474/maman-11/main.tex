\documentclass[11pt, oneside]{article}
\usepackage{geometry}
\geometry{a4paper}
\usepackage[parfill]{parskip}
\usepackage[nodisplayskipstretch]{setspace}
\usepackage{graphicx,titlesec}
\usepackage{amsmath,amssymb,cancel,commath,mathtools}

% Hebrew Stuff
\usepackage[utf8x]{inputenc}
\usepackage[english,hebrew]{babel}
\usepackage{hebfont}

% Custom commands
\newcommand{\qed}{\R{$\blacksquare$}}
\newcommand{\br}{\\\\\\\\\\}
\newcommand{\opr}[1]{\xrightarrow[\text{#1}]{}}
\newcommand{\logr}[1]{\xRightarrow[\text{#1}]{}}
\newcommand{\bidiarrow}[1]{\underset{\text{#1}}{\leftrightarrow}}
\newcommand{\ueq}[1]{\underset{\text{#1}}{=}}
\newcommand{\mR}{\mathbb{R}}
\newcommand{\mN}{\mathbb{N}}
\newcommand{\mZ}{\mathbb{Z}}
\newcommand{\mQ}{\mathbb{Q}}
\newcommand{\inv}[1]{#1^{-1}}

% Custom text commands (for Hebrew)
\newcommand{\q}[3]{\R{שאלה #3#2.#1}}
\newcommand{\m}[3]{\R{משפט #3#2.#1}}
\newcommand{\h}[3]{\R{הגדרה #3#2.#1}}
\newcommand{\ms}[3]{\R{מסקנה #3#2.#1}}
\newcommand{\ta}[3]{\R{טענה #3#2.#1}}

% Custom commands for this document
\newcommand{\floor}[1]{\left\lfloor {#1} \right\rfloor}
\newcommand{\ceil}[1]{\left\lceil {#1} \right\rceil}

% Spacing
\titlespacing\subsubsection{0pt}{0pt}{3pt}
\setstretch{0.1}

\title{ממן 11}
\author{יונתן אוחיון}

\begin{document}
\maketitle

%%%%% <Q1> %%%%%
\section{שאלה 1}
\subsection{סעיף א}
נניח ש$a$ רציונלי ונגיע לסתירה. ראשית, מכיוון שמתקיים $\mN \subset \mQ$, גם $k$ ו$l$ רציונליים. כעת, נפשט את הביטוי מעט:
\[
a = k + l\sqrt{2} \logr{$() - k$} a - k = l\sqrt{2} \logr{$\frac{()}{l}$} \sqrt{2} = \frac{a - k}{l}
\]
כעת, מכיוון ש$a, k$ רציונליים, הפרשם רציונלי גם הוא. לפיכך, מכיוון ש$a - k, l$ רציונליים, המנה שלהם רציונלית גם היא ולכן $\frac{a - k}{l} = \sqrt{2}$ רציונלי, וזוהי כמובן סתירה. לפיכך, $a$ אינו רציונלי כנדרש.
\br\qed

\subsection{סעיף ב}
נפתח את הביטוי בעזרת נוסחת הבינום:
\[
(1 + \sqrt{2})^n
= \sum^{n}_{i = 0} {n \choose i} 1^{n - i} \sqrt{2}^i
= \sum^{n}_{i = 0} {n \choose i} \sqrt{2}^i
\]
נוכל לשים לב שעבור הביטוי $\sqrt{2}^{2m}$, התוצאה הינה למעשה $2^m$. לפיכך, נוכל לפצל את הסכומים לחזקות זוגיות ואי-זוגיות:
\begin{align*}
\sum^{n}_{i = 0} {n \choose i} \sqrt{2}^i
&= \sum^{n}_{i = 0} {n \choose 2i} \sqrt{2}^{2i} + \sum^{n}_{j = 0} {n \choose 2j + 1} \sqrt{2}^{2j + 1}\\
&= \sum^{n}_{i = 0} {n \choose 2i} 2^i + \sum^{n}_{j = 0} {n \choose 2j + 1} 2^j\sqrt{2}\\
&= \underbrace{\sum^{n}_{i = 0} {n \choose 2i} 2^i}_{a} + \underbrace{\sqrt{2}\sum^{n}_{j = 0} {n \choose 2j + 1} 2^j}_{b}
\end{align*}
%נסמן:
%\[
%a = \sum^{n}_{i = 0} {n \choose 2i} 2^i, \; b = \sqrt{2}\sum^{n}_{j = 0} {n \choose 2j + 1} 2^j
%\]
כעת, $a$ הוא כמובן מספר רציונלי )כפל רציונליים וחיבור רציונליים(, ולפי \q{1}{16}{}א נוכל לראות ש$b$ אינו מספר רציונלי )חיבור רציונליים וכפל אי רציונלי ברציונלי(. כעת, שוב לפי \q{1}{16}{}א נוכל לראות שהסכום $a + b$ אי רציונלי גם הוא, ולכן $(1 + \sqrt{2})^n$ אי רציונלי כנדרש.
\br\qed

\clearpage
%%%%% </Q1> %%%%%

%%%%% <Q2> %%%%%
\section{שאלה 2}
\setcounter{subsection}{-1}
\subsection{טענת עזר}
ראשית, נוכיח טענת עזר אשר אומרת שעבור כל $1 \le x, n \in \mR$ מתקיים $\frac{x}{n} \le x$. לפי \ta{1}{24}{2.}, לכל $1 \le n \in \mR$ מתקיים $\frac{1}{n} \le 1$. נכפיל את שני האגפים ב$x$ ונקבל $\frac{x}{n} \le x$ כנדרש.
\br\qed

\subsection{סעיף א}
ראשית, נוכיח ש$|a| + 1 > 0$ בה"כ. לפי תכונות הערך המוחלט, $|a| \ge 0$. אם $|a| > 0$, בהכרח גם $|a| + 1 > 0$ )שכן $1 > 0$( וסיימנו. אם $|a| = 0$ אזי $|a| + 1 = 1 > 0$ כנדרש. לפיכך, $|a| + 1, |b| + 1 > 0$ ונוכל להשתמש בנוסחה הנתונה במטלה על מנת לפשט מעט את הביטוי:
\begin{align*}
\abs{\sqrt{|a| + 1} - \sqrt{|b| + 1}}
&= \abs{\frac{(\sqrt{|a| + 1} + \sqrt{|b| + 1})(\sqrt{|a| + 1} - \sqrt{|b| + 1})}{\sqrt{|a| + 1} + \sqrt{|b| + 1}}}\\
\text{\R{נוסחאות הכפל המקוצר}} \to &= \abs{\frac{|a| + \not1 - |b| - \not1}{\sqrt{|a| + 1} + \sqrt{|b| + 1}}}\\
\text{\R{חיסור}} \to &= \abs{\frac{|a| - |b|}{\sqrt{|a| + 1} + \sqrt{|b| + 1}}}\\
\text{\R{מתכונות הערך המוחלט}} \to &= \frac{\abs{|a| - |b|}}{\abs{\sqrt{|a| + 1} + \sqrt{|b| + 1}}}
\end{align*}
כעת, לפי טענת העזר שהוכחנו מתקיים
\[
\abs{\sqrt{|a| + 1} - \sqrt{|b| + 1}} = \frac{\abs{|a| - |b|}}{\abs{\sqrt{|a| + 1} + \sqrt{|b| + 1}}} \le \abs{|a| - |b|}
\]
וגם
\[
\frac{\abs{a - b}}{2} \le \abs{a - b}
\]
לפי אי-שוויון המשולש מתקיים $\abs{|a| - |b|} \le \abs{a - b}$ ולכן גם
\[
\abs{\sqrt{|a| + 1} - \sqrt{|b| + 1}} \le \abs{|a| - |b|} \le \frac{\abs{a - b}}{2} \le \abs{a - b} \logr{\R{טרנזיטיביות}} \abs{\sqrt{|a| + 1} - \sqrt{|b| + 1}} \le \frac{\abs{a - b}}{2}
\]
כנדרש.
\br\qed
\clearpage

\setcounter{section}{1}
\section{שאלה 2 )המשך(}
\setcounter{subsection}{1}
\subsection{סעיף ב}
נחלק את הביטוי לשלושה תתי מקרים:
\subsubsection{מקרה א -- $a = 0$}
לפי תכונות הערך המוחלט, עבור $a = 0$ מתקיים $|a| = 0$. לפיכך,
\[
\left(\frac{a + |a|}{2}\right)^2 = \left(\frac{a - |a|}{2}\right)^2 = a^2 = 0
\]
כנדרש.
\subsubsection{מקרה ב -- $a > 0$}
לפי תכונות הערך המוחלט, עבור $a > 0$ מתקיים $|a| = a$. לפיכך,
\[
\left(\frac{a + |a|}{2}\right)^2 + \left(\frac{a - |a|}{2}\right)^2
= \left(\frac{\cancel{2}a}{\cancel{2}}\right)^2 + \left(\frac{0}{2}\right)^2
= a^2 + 0
= a^2
\]
כנדרש.
\subsubsection{מקרה ג -- $a < 0$}
לפי תכונות הערך המוחלט, עבור $a < 0$ )כלומר קיים $m \in \mR$ כך ש$a = -m$( מתקיים $|a| = -a = m$. לפיכך,
\begin{align*}
\left(\frac{a + |a|}{2}\right)^2 + \left(\frac{a - |a|}{2}\right)^2
&= \left(\frac{\cancel{-m + m}}{2}\right)^2 + \left(\frac{-m - m}{2}\right)^2\\
&= \left(\frac{0}{2}\right)^2 + \left(\frac{-\cancel{2}m}{\cancel{2}}\right)^2\\
&= (-m)^2 = a^2\\
\end{align*}
כנדרש.
\br\qed

\clearpage
%%%%% </Q2> %%%%%

%%%%% <Q3> %%%%%
\section{שאלה 3}
\subsection{סעיף א}
מתכונות הערך השלם מתקיים
\[
\floor{x} \le x < \floor{x} + 1,\;
\floor{y} \le y < \floor{y} + 1
\]
נחבר את אי השוויונות ונקבל
\[
\floor{x} + \floor{y} \le x + y < \floor{x} + \floor{y} + 2
\]
כעת, אם מתקיים $\floor{x} + \floor{y} \le x + y < \floor{x} + \floor{y} + 1$, אזי מתכונות הערך השלם מתקיים שהערך השלם של $x + y$ הינו $n = \floor{x} + \floor{y}$, או בקיצור $\floor{x + y} = \floor{x} + \floor{y}$.

אם מתקיים $\floor{x} + \floor{y} + 1 \le x + y < \floor{x} + \floor{y} + 2$, אזי מתכונות הערך השלם מתקיים $\floor{x + y} = \floor{x} + \floor{y} + 1$. מכך נובע ש$\floor{x + y} > \floor{x} + \floor{y}$. "נשלב" בין שני המקרים ונקבל ש$\floor{x} + \floor{y} \le \floor{x + y}$ כנדרש.
\br\qed

\subsection{סעיף ב}
\subsubsection{$(i)$}
נפתח את הביטוי:
\[
\floor{x - \frac{1}{2}}^2 = 25 \logr{$\sqrt{()}$} \floor{x - \frac{1}{2}} = 5 \lor \floor{x - \frac{1}{2}} = -5
\]

כעת, לפי תכונות הערך השלם, נוכל לקבל את אי השוויונות הבאים: 
\[
-5 \le x - \frac{1}{2} < -4, 5 \le x - \frac{1}{2} < 6
\]
כעת, נפשט את הביטויים שבתוך אי השוויונות:
\begin{eqnarray*}
& 5 \le x - \frac{1}{2} < 6 \lor -5 \le x - \frac{1}{2} < -4\\
& \Downarrow\\
& (5 \le x - \frac{1}{2} \land x - \frac{1}{2} < 6) \lor (-5 \le x - \frac{1}{2} \land x - \frac{1}{2} < -4)\\
& \Downarrow\\
& (5.5 \le x \land x < 6.5) \lor (-4.5 \le x \land x < -3.5)\\
& \Downarrow\\
& 5.5 \le x < 6.5 \lor -4.5 \le x < -3.5\\
& \Downarrow\\
& x \in [5.5, 6.5) \lor x \in [-4.5, -3.5) \Rightarrow x \in [5.5, 6.5) \cup [-4.5, -3.5)
\end{eqnarray*}
ולכן קבוצת הפתרונות הינה $[5.5, 6.5) \cup [-4.5, -3.5)$ כנדרש.
\br\qed

\clearpage
\setcounter{section}{2}
\section{שאלה 3 )המשך(}
\setcounter{subsection}{1}
\subsection{סעיף ב )המשך(}
\setcounter{subsubsection}{1}
\subsubsection{$(ii)$}
לפי תכונות הערך השלם, נוכל לקבל את אי השוויון הבא:
\[
9 \le x^2 < 10 \logr{} \overbrace{9 \le x^2}^{(1)} \land \overbrace{x^2 < 10}^{(2)}
\]

ראשית, נפתור את $(1)$:
\[
9 \le x^2 \Rightarrow x^2 - 9 \ge 0 \Rightarrow (x - 3)(x + 3) \ge 0
\]
כעת, אנו יודעים שזוהי למעשה משוואה ריבועית ושורשיה הינם 
\[x_1 = 3,\;x_2 = -3\]
ולכן ערכי המשוואה הגדולים מ$0$ או שווים לו הינם $-3 \le x \lor x \ge 3$

כעת, נפתור את $(2)$:
\[
x^2 < 10 \logr{} x^2 - 10 < 0 \logr{} (x - \sqrt{10})(x + \sqrt{10}) < 0
\]
גם זוהי משוואה ריבועית ושורשיה הינם
\[x_1 = \sqrt{10},\;x_2 = -\sqrt{10}\]
ולכן ערכי המשוואה הקטנים מ$0$ הינם $-\sqrt{10} < x < \sqrt{10}$. לפיכך, ערכי ה$x$ הפותרים את המשוואה $\floor{x^2} = 9$ הינם
\begin{eqnarray*}
& (x \le -3 \lor x \ge 3) \land (-\sqrt{10} < x < \sqrt{10})\\
& \Downarrow\\
& -\sqrt{10} < x \le -3 \land 3 \le x < \sqrt{10}\\
& \Downarrow\\
& x \in (-\sqrt{10}, -3] \land x \in [3, \sqrt{10})\\
& \Downarrow\\
& x \in (-\sqrt{10}, -3] \cup [3, \sqrt{10})
\end{eqnarray*}
לפיכך, קבוצת הפתרונות הינה $(-\sqrt{10}, -3] \cup [3, \sqrt{10})$ כנדרש.
\br\qed
\clearpage
%%%%% </Q3> %%%%%

%%%%% <Q4> %%%%%
\section{שאלה 4}
\subsection{סעיף א}
לפי צפיפות הממשיים ב$\mR$, עבור כל $x, y \in \mR$ המקיימים $x < y$ קיים $q \in \mQ$ כך ש$x < q < y$. נבחר $x, y \in \mR$ המקיימים $(*)\;0 \le x < y \le 1$. נכפול את אי השוויון ב$\frac{1}{\sqrt{3}}$ ונקבל
\[
0 \le \frac{x}{\sqrt{3}} < \frac{y}{\sqrt{3}} \le \frac{1}{\sqrt{3}}
\]
לפי טענת העזר שהוכחנו בשאלה 2, נוכל להיווכח שמתקיים
\[
0 \le \frac{x}{\sqrt{3}} < \frac{y}{\sqrt{3}} \le 1
\]
אזי קיים $q \in \mQ$ המקיים
\[
0 \le \frac{x}{\sqrt{3}} < q < \frac{y}{\sqrt{3}} \le 1
\]
נכפול את אי השוויון בחזרה ב$\sqrt{3}$ ונקבל $0 \le x < q\sqrt{3} < y \le 1$ )לפי $(*)$(. לפיכך, עבור כל $x, y \in [0, 1]$ קיים $0 < q\sqrt{3} \in \mQ$ כך ש$x < q\sqrt{3} < y$ כנדרש.
\br\qed

\subsection{סעיף ב}
נזכיר מהי הגדרת הצפיפות: קבוצה $A$ של מספרים צפופה בקטע $I$ אם לכל $x, y \in I$ כך ש$x < y$ קיים $a \in A$ כך ש$x < a < y$.

נצרין את ההגדרה ונעבור לכתב כמתים:
\[
p = \forall x, y \in I, x < y \logr{} \exists a \in A, x < a \land a < y
\]
כעת, נשלול את הפסוק:
\[
\neg p = \exists x, y \in I, x < y \logr{} \forall a \in A, x \ge a \lor a \ge y
\]
נחזור לביטוי: קבוצה $A$ של מספרים אינה צפופה בקטע $I$ אם קיימים $x, y \in I$ המקיימים $x < y$ כך שלכל $a \in A$ מתקיים $x \ge a$ או $a \ge y$. וזוהי השלילה של הגדרת הצפיפות כנדרש.
\br\qed

\subsection{סעיף ג}
נניח בשלילה שקבוצת השברים העשרוניים שלא מופיעה בהם הספרה 3 צפופה בקטע $[-1, 1]$, כלומר עבור כל $x, y \in [-1, 1]$ המקיימים $x < y$ קיים שבר עשרוני שלא מופיעה בו הספרה 3 המסומן ב$q$ כך ש$x < q < y$. נוכל לקחת $x = 0.3, y = 0.35$. אזי $x < y$, אך לא קיים ביניהם שבר אשר בפיתוחו העשרוני לא מופיעה הספרה 3 )שכן כל שבר ביניהם מתחיל ב$0.3\dots$( בסתירה לנתון שהקבוצה צפופה. לפיכך, הקבוצה אינה צפופה בקטע כנדרש.
\br\qed
%%%%% </Q4> %%%%%

\end{document}