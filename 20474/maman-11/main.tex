\documentclass[11pt, oneside]{article}
\usepackage{geometry}
\geometry{a4paper}
\usepackage[parfill]{parskip}
\usepackage[nodisplayskipstretch]{setspace}
\usepackage{graphicx,titlesec}
\usepackage{amsmath,amssymb,cancel,commath,mathtools}

% Hebrew Stuff
\usepackage[utf8x]{inputenc}
\usepackage[english,hebrew]{babel}
\usepackage{hebfont}

% Custom commands
\newcommand{\qed}{\R{$\blacksquare$}}
\newcommand{\br}{\\\\\\\\\\\\\\}
\newcommand{\opr}[1]{\xrightarrow[\text{#1}]{}}
\newcommand{\logr}[1]{\xRightarrow[\text{#1}]{}}
\newcommand{\bidiarrow}[1]{\underset{\text{#1}}{\leftrightarrow}}
\newcommand{\ueq}[1]{\underset{\text{#1}}{=}}
\newcommand{\mR}{\mathbb{R}}
\newcommand{\mN}{\mathbb{N}}
\newcommand{\mZ}{\mathbb{Z}}
\newcommand{\mQ}{\mathbb{Q}}
\newcommand{\inv}[1]{#1^{-1}}

% Custom text commands (for Hebrew)
\newcommand{\q}[3]{\R{שאלה #3#2.#1}}
\newcommand{\m}[3]{\R{משפט #3#2.#1}}
\newcommand{\h}[3]{\R{הגדרה #3#2.#1}}
\newcommand{\ms}[3]{\R{מסקנה #3#2.#1}}
\newcommand{\ta}[3]{\R{טענה #3#2.#1}}

% Custom commands for this document

% Spacing
\titlespacing\subsubsection{0pt}{5pt}{3pt}
\setstretch{0.1}

\title{ממן 11}
\author{יונתן אוחיון}

\begin{document}
\maketitle

%%%%% <Q1> %%%%%
\section{שאלה 1}
\subsection{סעיף א}
נניח ש$a$ רציונלי ונגיע לסתירה. ראשית, מכיוון שמתקיים $\mN \subset \mQ$, גם $k$ ו$l$ רציונליים. כעת,
\[
(*)\qquad a = k + l\sqrt{2} \opr{$() - k$} a - k = l\sqrt{2}
\]
מכיוון ש$a, k \in \mQ \land a - k = a + (-k)$ ופעולת החיבור מעל הרציונליים הינה פעולה סגורה, מתקיים גם $a - k \in \mQ$. לפיכך, נוכל לייצגו בעזרת $\alpha, \beta \in \mZ \land \beta \neq 0$ כך: $a - k = \frac{\alpha}{\beta}$ ונוכל להניח שהצגה זו הינה ההצגה המצומצמת ביותר של מספר זה )כלומר, הגורם המשותף היחידי בין $\alpha$ ו$\beta$ הוא $1$(. לכן לפי $(*)$ מתקיים
\[
\frac{\alpha}{\beta} = l\sqrt{2} \opr{$()^2$}
\frac{\alpha^2}{\beta^2} = 2l^2 \opr{$() * \beta^2$}
\alpha^2 = 2l^2\beta^2
\]
מכיוון ש$2 \mid 2l^2\beta^2$ מתקיים $2 \mid \alpha^2$, ומכיוון שריבוע של מספר אי-זוגי הינו מספר אי-זוגי מתקיים גם $2 \mid \alpha$, כלומר קיים $m \in \mN$ כך ש$\alpha = 2m$. לכן
\[
4m^2 = 2l^2\beta^2 \opr{$\frac{()}{2}$}
l^2\beta^2 = 2m^2 \opr{$\frac{()}{l^2}$}
\beta^2 = 2 \cdot \frac{m^2}{l^2}
\]
לכן $2 \mid \beta^2$ ומכיוון שריבוע של מספר אי-זוגי הינו מספר אי-זוגי, מתקיים גם $2 \mid \beta$ והגענו לסתירה להנחה שהגורם המשותף היחידי בין $\alpha$ ו$\beta$ הינו $1$. לפיכך, $a = k + l\sqrt{2}$ הינו מספר אי-רציונלי כנדרש.
\br\qed

\clearpage
\setcounter{section}{0}
\section{שאלה 1 )המשך(}
\subsection{סעיף ב}
נוכיח את הטענה באינדוקציה עבור $1 \le n \in \mN$. מקרה הבסיס הוא כמובן $n = 1$, כלומר
\[
a = (1 + \sqrt{2})^1 = 1 + \sqrt{2}
\]
מספר זה הינו מהצורה $a = k + l\sqrt{2}$ )עבור $k = l = 1$( ולפי סעיף א של השאלה אנו יודעים שזהו מספר אי רציונלי. כעת, נניח נכונות עבור $n = k$ ונוכיח עבור $n = k + 1$:
\begin{align*}
a
&= (1 + \sqrt{2})^{k + 1}\\
&= (1 + \sqrt{2}) \cdot (1 + \sqrt{2})^k\\
&= (1 + \sqrt{2})^k + \sqrt{2}(1 + \sqrt{2})^k
\end{align*}
לפי הנחת האינדוקציה $(1 + \sqrt{2})^k$ הינו מספר אי רציונלי, אך לפי \q{1}{16}{}ד נוכל לראות שמכפלת מספרים אי רציונליים אינה בהכרח מספר אי רציונלי בעצמה ולכן $\sqrt{2}(1 + \sqrt{2})^k$ אינו בהכרח אי רציונלי. לפיכך, נוכל לחלק את הביטוי לשני מקרים:
\subsubsection{מקרה א -- $\sqrt{2}(1 + \sqrt{2})^k \in \mQ$}
לפי הנחת האינדוקציה $(1 + \sqrt{2})^k$ מספר אי רציונלי. לפי \q{1}{16}{}א נוכל לראות שסכום מספר אי רציונלי ומספר רציונלי הינו מספר אי רציונלי בעצמו ולכן $a \not\in \mQ$ כנדרש.
\subsubsection{מקרה ב -- $\sqrt{2}(1 + \sqrt{2})^k \not\in \mQ$}

\clearpage
%%%%% </Q1> %%%%%

%%%%% <Q2> %%%%%
\section{שאלה 2}
\setcounter{subsection}{-1}
\subsection{טענת עזר}
ראשית, נוכיח טענת עזר אשר אומרת שעבור כל $1 \le x, n \in \mR$ מתקיים $\frac{x}{n} \le x$. לפי \ta{1}{24}{2.}, לכל $1 \le n \in \mR$ מתקיים $\frac{1}{n} \le 1$. נכפיל את שני האגפים ב$x$ ונקבל $\frac{x}{n} \le x$ כנדרש.
\br\qed

\subsection{סעיף א}
ראשית, נוכיח ש$|a| + 1 > 0$ בה"כ. לפי תכונות הערך המוחלט, $|a| \ge 0$. אם $|a| > 0$, בהכרח גם $|a| + 1 > 0$ )שכן $1 > 0$( וסיימנו. אם $|a| = 0$ אזי $|a| + 1 = 1 > 0$ כנדרש. לפיכך, $|a| + 1, |b| + 1 > 0$ ונוכל להשתמש בנוסחה הנתונה במטלה על מנת לפשט מעט את הביטוי:
\begin{align*}
\abs{\sqrt{|a| + 1} - \sqrt{|b| + 1}}
&= \abs{\frac{(\sqrt{|a| + 1} + \sqrt{|b| + 1})(\sqrt{|a| + 1} - \sqrt{|b| + 1})}{\sqrt{|a| + 1} + \sqrt{|b| + 1}}}\\
\text{\R{נוסחאות הכפל המקוצר}} \to &= \abs{\frac{|a| + \not1 - |b| - \not1}{\sqrt{|a| + 1} + \sqrt{|b| + 1}}}\\
\text{\R{חיסור}} \to &= \abs{\frac{|a| - |b|}{\sqrt{|a| + 1} + \sqrt{|b| + 1}}}\\
\text{\R{מתכונות הערך המוחלט}} \to &= \frac{\abs{|a| - |b|}}{\abs{\sqrt{|a| + 1} + \sqrt{|b| + 1}}}
\end{align*}
כעת, לפי טענת העזר שהוכחנו מתקיים
\[
\abs{\sqrt{|a| + 1} - \sqrt{|b| + 1}} = \frac{\abs{|a| - |b|}}{\abs{\sqrt{|a| + 1} + \sqrt{|b| + 1}}} \le \abs{|a| - |b|}
\]
וגם
\[
\frac{\abs{a - b}}{2} \le \abs{a - b}
\]
לפי אי-שוויון המשולש מתקיים $\abs{|a| - |b|} \le \abs{a - b}$ ולכן גם
\[
\abs{\sqrt{|a| + 1} - \sqrt{|b| + 1}} \le \abs{|a| - |b|} \le \frac{\abs{a - b}}{2} \le \abs{a - b} \logr{\R{טרנזיטיביות}} \abs{\sqrt{|a| + 1} - \sqrt{|b| + 1}} \le \frac{\abs{a - b}}{2}
\]
כנדרש.
\br\qed
\clearpage

\setcounter{section}{1}
\section{שאלה 2 )המשך(}
\setcounter{subsection}{1}
\subsection{סעיף ב}
נחלק את הביטוי לשלושה תתי מקרים:
\subsubsection{מקרה א -- $a = 0$}
לפי תכונות הערך המוחלט, עבור $a = 0$ מתקיים $|a| = 0$. לפיכך,
\[
\left(\frac{a + |a|}{2}\right)^2 = \left(\frac{a - |a|}{2}\right)^2 = a^2 = 0
\]
כנדרש.
\subsubsection{מקרה ב -- $a > 0$}
לפי תכונות הערך המוחלט, עבור $a > 0$ מתקיים $|a| = a$. לפיכך,
\[
\left(\frac{a + |a|}{2}\right)^2 + \left(\frac{a - |a|}{2}\right)^2
= \left(\frac{\cancel{2}a}{\cancel{2}}\right)^2 + \left(\frac{0}{2}\right)^2
= a^2 + 0
= a^2
\]
כנדרש.
\subsubsection{מקרה ג -- $a < 0$}
לפי תכונות הערך המוחלט, עבור $a < 0$ )כלומר קיים $m \in \mR$ כך ש$a = -m$( מתקיים $|a| = -a = m$. לפיכך,
\begin{align*}
\left(\frac{a + |a|}{2}\right)^2 + \left(\frac{a - |a|}{2}\right)^2
&= \left(\frac{\cancel{-m + m}}{2}\right)^2 + \left(\frac{-m - m}{2}\right)^2\\
&= \left(\frac{0}{2}\right)^2 + \left(\frac{-\cancel{2}m}{\cancel{2}}\right)^2\\
&= (-m)^2 = a^2\\
\end{align*}
כנדרש.
\br\qed
%%%%% </Q2> %%%%%

\end{document}