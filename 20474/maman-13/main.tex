\documentclass[11pt, oneside]{article}
\usepackage{geometry}
\geometry{a4paper}
\usepackage[parfill]{parskip}
\usepackage[nodisplayskipstretch]{setspace}
\usepackage{graphicx,titlesec}
\usepackage{amsmath,amssymb,cancel,mathtools}

% Hebrew Stuff
\usepackage[utf8x]{inputenc}
\usepackage[english,hebrew]{babel}
\usepackage{hebfont}

% Custom commands
\newcommand{\qed}{\R{$\blacksquare$}}
\newcommand{\br}{\\\\\\\\\\\\\\}
\newcommand{\opr}[1]{\xrightarrow[\text{#1}]{}}
\newcommand{\logr}[1]{\underset{\text{#1}}{\Rightarrow}}
\newcommand{\bidiarrow}[1]{\underset{\text{#1}}{\leftrightarrow}}
\newcommand{\ueq}[1]{\underset{\text{#1}}{=}}
\newcommand{\mR}{\mathbb{R}}
\newcommand{\mN}{\mathbb{N}}
\newcommand{\mZ}{\mathbb{Z}}
\newcommand{\mQ}{\mathbb{Q}}
\newcommand{\inv}[1]{#1^{-1}}

% Custom text commands (for Hebrew)
\newcommand{\q}[3]{\R{שאלה #3#2.#1}}
\newcommand{\m}[3]{\R{משפט #2.#1#3}}
\newcommand{\h}[3]{\R{הגדרה #3#2.#1}}
\newcommand{\ms}[3]{\R{מסקנה #3#2.#1}}
\renewcommand{\t}[3]{\R{טענה #2.#1#3}}
\renewcommand{\d}[3]{\R{דוגמה #2.#1#3}}

% Custom commands for this document
\DeclarePairedDelimiter\ceil{\lceil}{\rceil}
\DeclarePairedDelimiter\floor{\lfloor}{\rfloor}
\DeclarePairedDelimiter\quo{\langle}{\rangle}
\newcommand{\finv}[1]{\frac{1}{#1}}

% Spacing
\titlespacing\subsubsection{0pt}{5pt}{3pt}
\setstretch{0.1}

\title{ממ``''ן 31}
\author{יונתן אוחיון}

\begin{document}
\maketitle

%%%%% <Q1> %%%%%
\section*{שאלה 1}
ראשית, נוכיח שהסדרה $(a_{n})$ מוגדרת לכל $n$. נוכל לראות שהסדרה מוגדרת אמ``''מ
\[
4(1 - a_{n}) \neq 0
\logr{} 4 - 4a_{n} \neq 0
\logr{} 4a_{n} \neq 4
\logr{} a_{n} \neq 1
\]
$(*)$ נוכיח באינדוקציה שמתקיים $0 \le a_{n} < \frac{1}{2}$ לכל $n$. עבור מקרה הבסיס $n = 1$ אי-שוויון זה ברור מהגדרת הסדרה )שכן $a_{1} = 0$(. כעת, נניח נכונות עבור $n = k$ ונוכיח עבור $n = k + 1$:
\begin{align*}
0 \le a_{k} < \frac{1}{2}
& \logr{} -\frac{1}{2} < -a_{k} \le 0\\
& \logr{} \frac{1}{2} < 1 - a_{k} \le 1\\
& \logr{} 1 \le \frac{1}{1 - a_{k}} < 2\\
& \logr{} \frac{1}{4} \le \frac{1}{4(1 - a_{k})} < \frac{1}{2} \logr{} 0 \le a_{k + 1} < \frac{1}{2}
\end{align*}
לפיכך ולפי הגדרת הסדרה מתקיים $a_{n} \neq 1$ ולכן היא מוגדרת לכל $n$. כעת, נוכיח באינדוקציה שהסדרה מונוטונית עולה. עבור מקרה הבסיס $n = 1$, נחשב ונראה שאי-השוויון מתקיים:
\[
a_{1} = 0, a_{2} = \frac{1}{4 - a_{1}} = \frac{1}{4} \logr{} a_{1} < a_{2}
\]
כעת,  נניח נכונות עבור $n = k$ ונוכיח עבור $n = k + 1$:
\begin{align*}
a_{k} < a_{k + 1}
& \logr{} 1 - a_{k} > 1 - a_{k + 1}\\
& \logr{} \frac{1}{1 - a_{k}} < \frac{1}{1 - a_{k + 1}}\\
& \logr{} \frac{1}{4(1 - a_{k})} < \frac{1}{4(1 - a_{k + 1})}\\
& \logr{} a_{k + 1} < \frac{1}{4(1 - a_{k + 1})} \logr{} a_{k + 1} < a_{k + 2}
\end{align*}
לכן הסדרה מונוטונית עולה. לפיכך ולפי $(*)$ הסדרה מונוטונית וחסומה, ולכן לפי \m{3}{61}{} היא מתכנסת. בעמוד הבא נחשב את גבולה.
\clearpage
\section*{שאלה 1 -- המשך}

כעת, נחשב את גבול הסדרה. נסמן $\lim_{n \to \infty} a_{n} = L$. לפי \m{2}{92}{} מתקיים
\begin{align*}
L = \lim_{n \to \infty} a_{n}
& = \lim_{n \to \infty} a_{n + 1}\\
& = \lim_{n \to \infty} \frac{1}{4 - 4a_{n}}\\
& = \frac{1}{4 - 4\lim_{n \to \infty} a_{n}}\\
& = \frac{1}{4 - 4L}
\end{align*}
נכפול את שני הצדדים ב$4 - 4L$ ונקבל:
\begin{align*}
-4L^{2} + 4L = 1
& \logr{} -4L^{2} + 4L - 1 = 0\\
& \logr{} 4L^{2} - 4L + 1 = 0\\
& \logr{} (2L - 1)^{2} = 0\\
& \logr{} L = \frac{1}{2}
\end{align*}
וזהו גבול הסדרה כנדרש.
\br\qed
\clearpage
%%%%% </Q1> %%%%%

%%%%% <Q2> %%%%%
\section*{שאלה 2}
\subsection*{סעיף א}
נחשב את הגבול:
\begin{align*}
\lim_{n \to \infty} \frac{(-5)^{n} + 2(-3)^{n} + 3}{(-4)^{n} + 2(-2)^{n} + 3}
& = \lim_{n \to \infty} \frac{\cancel{(-4)^{n}}((\frac{5}{4})^{n} + 2(\frac{3}{4})^{n} + \frac{3}{(-4)^{n}})}{\cancel{(-4)^{n}}(1 + 2(\frac{2}{4})^{n} + \frac{3}{(-4)^{n}})}\\
& = \frac{\lim_{n \to \infty} ((\frac{5}{4})^{n} + 2(\frac{3}{4})^{n} + \frac{3}{(-4)^{n}})}{\lim_{n \to \infty} (1 + 2(\frac{2}{4})^{n} + \frac{3}{(-4)^{n}})}\\
& = \frac{\infty + 2 \cdot 0 +0}{1 + 2 \cdot 0 + 0} = \infty
\end{align*}
וחישבנו את הגבול כנדרש.
\br\qed

\subsection*{סעיף ב}
ראשית, נסמן
\[
a_{n} = \frac{(-5)^{n} + 2(-3)^{n} + 3}{(-4)^{n} + 2(-2)^{n} + 3},\;b_{n} = \frac{(-4)^{n} + 2(-2)^{n} + 3}{(-5)^{n} + 2(-3)^{n} + 3}
\]
נוכל לשים לב ש$a_{n} = \frac{1}{b_{n}}$. לפיכך ולפי סעיף א, מתקיים 
\[
\lim_{n \to \infty} b_{n} = \frac{1}{\lim_{n \to \infty} a_{n}} \ueq{\m{2}{34}{ה}} 0
\]
כנדרש.
\br\qed
\clearpage

\section*{שאלה 2 -- המשך}
\subsection*{סעיף ג}
\subsubsection*{טענת עזר -- $\lim_{n \to \infty} (1 - \frac{1}{n})^{n} = \frac{1}{e}$:}
ראשית, נוכיח ש$\lim_{n \to \infty} (1 - \frac{1}{n})^{n} = \frac{1}{e}$. נגדיר את הסדרה $b_{n} = (1 + \frac{1}{n})^{n} = (\frac{n + 1}{n})^{n}$. אזי $\lim_{n \to \infty} b_{n} = e$. נשים לב שלמעשה מתקיים
\begin{align*}
b_{n} = \left(\frac{n + 1}{n}\right)^{n}
\logr{} b_{n - 1}
& = \left(\frac{n}{n - 1}\right)^{n - 1}\\
& = \left(\frac{n}{n - 1}\right)^{-1}\left(\frac{n}{n - 1}\right)^{n}\\
& = \left(\frac{n - 1}{n}\right)\left(\frac{n}{n - 1}\right)^{n}\\
& = \left(1 - \frac{1}{n}\right)\left(\frac{n}{n - 1}\right)^{n}
\end{align*}
לפי \m{2}{92}{} מתקיים $\lim_{n \to \infty} b_{n} = \lim_{n \to \infty} b_{n - 1}$. נחשב את הגבול ונקבל:
\begin{align*}
e = \lim_{n \to \infty} \left(1 - \frac{1}{n}\right)\left(\frac{n}{n - 1}\right)^{n}
& = \lim_{n \to \infty} \left(1 - \frac{1}{n}\right) \cdot \lim_{n \to \infty}\left(\frac{n}{n - 1}\right)^{n}\\
& = \left(\lim_{n \to \infty} 1 - \lim_{n \to \infty} \frac{1}{n}\right) \cdot \lim_{n \to \infty}\left(\frac{n}{n - 1}\right)^{n}\\
& = \lim_{n \to \infty} \left(\frac{n}{n - 1}\right)^{n}
\end{align*}
ולכן $\lim_{n \to \infty} \left(\frac{n}{n - 1}\right)^{n} = e$. כעת, נוכל לראות שמתקיים
\[
\left(1 - \frac{1}{n}\right)^{n}
= \left(\frac{n - 1}{n}\right)^{n}
= \left(\left(\frac{n}{n - 1}\right)^{n}\right)^{-1}
\]
ולכן מתקיים
\begin{align*}
\lim_{n \to \infty} \left(1 - \frac{1}{n}\right)^{n}
& = \lim_{n \to \infty} \left(\left(\frac{n}{n - 1}\right)^{n}\right)^{-1}\\
& = \left(\lim_{n \to \infty} \left(\frac{n}{n - 1}\right)^{n}\right)^{-1}\\
& = e^{-1} = \frac{1}{e}
\end{align*}
כנדרש.
\br\qed
\clearpage

\section*{שאלה 2 -- המשך}
\subsection*{סעיף ג -- המשך}
כעת נחשב את גבול הסדרה. ראשית, נפשט את הביטוי:
\[
a_{n}
= \left(\frac{1}{n} - 1\right)^{n}
= (-1)^{n}\left(1 - \frac{1}{n}\right)^{n}
\]
נניח בשלילה ש$(a_{n})$ מתכנסת לגבול $L$. לכן, לפי \m{3}{52}{} כל הגבולות החלקיים שלה שווים ל$L$. נסתכל על שתי תת-סדרות של $(a_{n})$:
\begin{eqnarray*}
& a_{2n} = (-1)^{2n}\left(1 - \frac{1}{2n}\right)^{2n}, a_{2n - 1} = (-1)^{2n - 1}\left(1 - \frac{1}{2n - 1}\right)^{2n - 1}\\
& \Downarrow\\
& a_{2n} = \left(1 - \frac{1}{2n}\right)^{2n}, a_{2n - 1} = -\left(1 - \frac{1}{2n - 1}\right)^{2n - 1}\\
\end{eqnarray*}
נשים לב שמתקיים $a_{2n} = b_{2n}$ ולכן היא תת-סדרה של הסדרה $b_{n}$ )סדרה זו הוגדרה בטענת העזר(. הסדרה $b_{n}$ מתכנסת, ולכן לפי \m{3}{52}{} כל הגבולות החלקיים שלה שווים לגבול שלה. לפיכך, $\lim_{n \to \infty} a_{2n} = \frac{1}{e}$.

בנוסף, נוכל לשים לב שמתקיים $a_{2n - 1} = -b_{2n - 1}$ ולכן היא תת-סדרה של $-b_{n}$. מכיוון שהיא מתכנסת, לפי \m{3}{52}{} כל הגבולות החלקיים שלה שווים לגבול שלה ולכן $\lim_{n \to \infty} a_{2n - 1} = -\frac{1}{e}$.
לסיכום:
\[
\lim_{n \to \infty} a_{2n} = \frac{1}{e},\;\lim_{n \to \infty} a_{2n - 1} = -\frac{1}{e}
\]

כמובן ש$\frac{1}{e} \neq -\frac{1}{e}$, ולכן מצאנו שני גבולות חלקיים שונים של $(a_{n})$ בסתירה להנחה. לפיכך, הסדרה $(a_{n})$ לא מתכנסת.

כעת, נסמן ב$\hat{L}$ את קבוצת הגבולות החלקיים של $(a_{n})$ ונמצא אותה: לפי ממצאינו, $\left\{\frac{1}{e}, -\frac{1}{e}\right\} \subseteq \hat{L}$. נוכיח כעת ששני הגבולות החלקיים הללו הינם הגבולות החלקיים היחידים של הסדרה. נוכל לראות ששתי תת-סדרות אלו מכסות את הסדרה $(a_{n})$, ולכן לפי \m{3}{03}{}, מתקיים $\hat{L} = \left\{\frac{1}{e}, -\frac{1}{e}\right\}$ ומצאנו את כל הגבולות החלקיים של $(a_{n})$ כנדרש.
\\\\\\\\\\\qed
\subsection*{סעיף ד}
נתון ש$(a_{n})$ סדרה עולה ממש של מספרים שלמים. לפיכך, החל ממקום מסוים $N$, לכל $n > N$ מתקיים $a_{n} > 0$. מכיוון שהסדרה הינה סדרה עולה ממש של מספרים שלמים, ובמקרה זה -- גם חיוביים, היא תת סדרה של הסדרה $b_{n} = n$. לפיכך, הסדרה $c_{n} = (1 + \frac{1}{a_{n}})^{a_{n}}$ תת-סדרה של הסדרה $d_{n} = (1 + \frac{1}{n})^{n}$, ולכן גבולן שווה. לפי \d{3}{5}{}, מתקיים $\lim_{n \to \infty} d_{n} = e$, ולפי \m{3}{03}{}, גם $\lim_{n \to \infty} c_{n} = e$ כנדרש.
\\\\\\\\\\\qed
%%%%% </Q2> %%%%%

\clearpage
%%%%% <Q3> %%%%%
\section*{שאלה 3}
\subsection*{סעיף א}
נוכיח כי $0 \le a_{n} \le 1$ לכל $n$. מתכונות הערך השלם נובע כי
\begin{align*}
\floor{\sqrt{n}} \le \sqrt{n} < \floor{\sqrt{n}} + 1
& \logr{} \floor{\sqrt{n}} - \floor{\sqrt{n}} \le \sqrt{n} - \floor{\sqrt{n}} < \floor{\sqrt{n}} + 1 - \floor{\sqrt{n}}\\
& \logr{} 0 \le \sqrt{n} - \floor{\sqrt{n}} < 1\\
& \logr{} 0 \le \quo{\sqrt{n}} < 1 \logr{} 0 \le a_{n} < 1
\end{align*}
לפיכך, הסדרה $(a_{n})$ חסומה כנדרש.
\br\qed

\subsection*{סעיף ב}
ראשית, נראה שתת הסדרה $(a_{n^{2}})^{\infty}_{n = 1}$ מתכנסת ל0:
\[
a_{n^{2}} = \quo{\sqrt{n^{2}}} = \quo{n} \ueq{$n \in \mN$} 0 \logr{} \lim_{n \to \infty} a_{n^{2}} = 0
\]
ומצאנו תת-סדרה המתכנסת ל0. נוכיח כעת כי זהו הגבול התחתון של $(a_{n})$. נניח שקיימת תת-סדרה אחרת, $(a_{n_{k}})^{\infty}_{k = 1}$ )כאשר $\sqrt{n_{k}} \not\in \mZ$( המתכנסת לגבול קטן יותר מ0, או
\[
\lim_{k \to \infty} a_{n_{k}} < \lim_{n \to \infty} a_{n^{2}}
\]
לפי תכונות החלק השברי ולפי הגדרת הסדרה, בהכרח $0 < \quo{\sqrt{n_{k}}}$. לפיכך,
\[
a_{n^{2}} < a_{n_{k}} \logr{\m{2}{13}{}} \lim_{n \to \infty} a_{n^{2}} < \lim_{k \to \infty} a_{n_{k}}
\]
בסתירה להנחה. לפיכך, לא קיימת תת-סדרה אחרת המתכנסת לגבול קטן יותר מ0, ו
\[\liminf_{n \to \infty} a_{n} = 0\]
כנדרש.
\br\qed
\clearpage

\section*{שאלה 3 -- המשך}
\subsection*{סעיף ג}
ראינו בסעיף א שלכל $n$ מתקיים $0 \le a_{n} < 1$, כלומר 0 חסם מלרע של $(a_{n})$. בנוסף, ראינו בסעיף ב שהתת-סדרה $(a_{n^{2}})$ היא הסדרה הקבועה $0$, כלומר לכל $n$ מתקיים $a_{n^{2}} = 0$, ולכן זהו $\min{\{a_{n}\}^{\infty}_{n = 1}}$ ובפרט $\inf{a_{n}}$ כנדרש.
\br\qed

\subsection*{סעיף ד}
נוכיח כי $\quo{\sqrt{n^{2} - 1}} = \sqrt{n^{2} - 1} - n + 1$ לכל $n \in \mN$. לפי הגדרת החלק השברי מתקיים
\[\quo{\sqrt{n^{2} - 1}} = \sqrt{n^{2} - 1} - \floor{\sqrt{n^{2} - 1}}\]
נפשט את הטענה שעלינו להוכיח עבור כל $n \in \mN$:
\[
\sqrt{n^{2} - 1} - \floor{\sqrt{n^{2} - 1}} = \sqrt{n^{2} - 1} - n + 1 \logr{} \floor{\sqrt{n^{2} - 1}} = n - 1
\]
לפי תכונות הערך השלם, $\sqrt{n^{2} - 1} - 1 < \floor{\sqrt{n^{2} - 1}} \le \sqrt{n^{2} - 1}$, כלומר שעלינו להוכיח שמתקיים
\begin{eqnarray*}
& \sqrt{n^{2} - 1} - 1 < n - 1 \le \sqrt{n^{2} - 1}\\\\
& \Updownarrow\\\\
& \sqrt{n^{2} - 1} - 1 < n - 1 \land n - 1 \le \sqrt{n^{2} - 1}\\\\
& \Updownarrow\\\\
& \cancel{n^{2}} - 1 < \cancel{n^{2}} \land \cancel{n^{2}} - 2n + 1 \le \cancel{n^{2}} - 1\\\\
& \Updownarrow\\\\
& 2n - 1 \ge 1 \iff 2n \ge 2 \iff n \ge 1
\end{eqnarray*}
וזה כמובן נכון, שכן $n \in \mN \logr{} n \ge 1$, ולכן עבור כל $n \in \mN$ מתקיים $\quo{\sqrt{n^{2} - 1}} = \sqrt{n^{2} - 1} - n + 1$ כנדרש.
\br\qed
\clearpage

\section*{שאלה 3 -- המשך}
\subsection*{סעיף ה}
ידוע לנו שעבור כל $1 < n \in \mN$ מתקיים גם $n^{2} - 1 \in \mN$. לפיכך, הסדרה $\sqrt{n^{2} - 1}$ היא תת-סדרה של $\sqrt{n}$, ולפי \m{3}{52}{} גבולן שווה. לפי \d{2}{52}{} נוכל לראות שגבולן בפרט שווה ל$\infty$. בנוסף, אנו יודעים ש$\lim_{n \to \infty} n = \infty$ וש$\lim_{n \to \infty} 1 = 1$. נחשב, אם כן, את גבול הסדרה:
\begin{align*}
\lim_{n \to \infty} \sqrt{n^{2} - 1} - n + 1
& = \lim_{n \to \infty} \sqrt{n^{2} - 1} - n + \lim_{n \to \infty} 1\\
(\footnotemark[1]\text{\R{נימוק בהערת שוליים}}) & = \lim_{n \to \infty} \frac{\cancel{n^{2}} - 1 - \cancel{n^{2}}}{\sqrt{n^{2} - 1} + n} + 1\\
& = -\lim_{n \to \infty} \finv{\sqrt{n^{2} - 1} + n} + 1
\end{align*}
כעת, לפי \m{2}{34}{א}, מתקיים $\sqrt{n^{2} - 1} + n \opr{$n \to \infty$} \infty$, ולכן לפי \m{2}{34}{ה} מתקיים
\[
\lim_{n \to \infty} \finv{\sqrt{n^{2} - 1} + n} = 0
\logr{} -\lim_{n \to \infty} \finv{\sqrt{n^{2} - 1} + n} = -0 = 0
\]
ולכן $-\lim_{n \to \infty} \inv{(\sqrt{n^{2} - 1} + n)} + 1 = 0 + 1 = 1$ כנדרש.
\br\qed

\footnotetext[1]{
יש לציין שניתן לעשות את המעבר הזה רק מכיוון שעבור $1 < n \in \mN$, מתקיים $0 < n^{2} - 1 \in \mN$.
}

\subsection*{סעיף ו}
כפי שהוכחנו בסעיף הקודם, עבור $n > 1$ טבעי מתקיים $n^{2} - 1 \in \mN$, ולכן אנו יכולים להגדיר תת-סדרה $(a_{n^{2} - 1})^{\infty}_{n = 2}$. נראה שהיא מתכנסת ל1:
\begin{align*}
n^{2} - 1 \in \mN
& \logr{} \lim_{n \to \infty} a_{n^{2} - 1} = \lim_{n \to \infty} \quo{\sqrt{n^{2} - 1}}\\
\text{\R{לפי סעיף ד}} & \logr{} \lim_{n \to \infty} \quo{\sqrt{n^{2} - 1}} = \lim_{n \to \infty} \sqrt{n^{2} - 1} - n + 1\\
\text{\R{לפי סעיף ה}} & \logr{} \lim_{n \to \infty} \sqrt{n^{2} - 1} - n + 1 = 1\\
& \logr{} \lim_{n \to \infty} a_{n^{2} - 1} = 1
\end{align*}
ומצאנו תת-סדרה המתכנסת לגבול 1. לפיכך, 1 גבול חלקי של $(a_{n})$ כנדרש.
\br\qed
\clearpage

\section*{שאלה 3 -- המשך}
\subsection*{סעיף ז}
נניח שקיים גבול חלקי של $(a_{n})$ הגדול ממש מ1 ונסמן אותו ב$c$. כלומר, קיימת תת-סדרה $(a_{n_{k}})^{\infty}_{k = 1}$ המתכנסת ל$c > 1$.
כפי שהוכחנו בסעיף א, נוכל לראות שהסדרה $(a_{n})$ חסומה מלעיל ע``''י 1, ולכן לפי משפט 5.04.3 מתקיים $\limsup_{n \to \infty} a_{n} \le 1$, והגענו לסתירה )מצאנו תת-סדרה ששואפת לגבול גדול יותר מהגבול העליון(. לפיכך, $\limsup_{n \to \infty} a_{n} = 1$ כנדרש.
\br\qed

\subsection*{סעיף ח}
כפי שהראינו בסעיף ו, קיימת תת-סדרה $(a_{n_{k}})^{\infty}_{k = 1}$ המתכנסת לגבול 1. כלומר, לפי הגדרת הגבול מתקיים
\begin{align*}
\forall 0 < \varepsilon\;\exists N \in \mN\;\forall N < n,\;|a_{n} - 1| < \varepsilon
& \logr{} a_{n} \in N_{\varepsilon}(1)\\
& \logr{} a_{n} \in (1 - \varepsilon, 1 + \varepsilon)\\
& \logr{} 1 - \varepsilon < a_{n} < 1 + \varepsilon\\
\text{\R{בפרט}} & \logr{} 1 - \varepsilon < a_{n}
\end{align*}
בנוסף, 1 חסם מלעיל של $(a_{n})$ לפי סעיף א, ולכן לפי \t{3}{9}{} מתקיים $\sup{a_{n}} = 1$. לסדרה $(a_{n})$ אין מקסימום, מכיוון שלפי הגדרת החסמים בסעיף א, עבור כל $n$ טבעי מתקיים $a_{n} < 1$, כלומר לכל $n$ טבעי מתקיים $a_{n} \le 1 \land a_{n} \neq 1$ ולכן לסדרה אין מקסימום כנדרש.
\br\qed
%%%%% </Q3> %%%%%

\end{document}