\documentclass[11pt, oneside]{article}
\usepackage{geometry}
\geometry{a4paper}
\usepackage[parfill]{parskip}
\usepackage[nodisplayskipstretch]{setspace}
\usepackage{graphicx,titlesec}
\usepackage{amsmath,amssymb,cancel}

% Hebrew Stuff
\usepackage[utf8x]{inputenc}
\usepackage[english,hebrew]{babel}
\usepackage{hebfont}

% Custom commands
\newcommand{\qed}{\R{$\blacksquare$}}
\newcommand{\br}{\\\\\\\\\\\\\\}
\newcommand{\opr}[1]{\xrightarrow[\text{#1}]{}}
\newcommand{\logr}[1]{\Longrightarrow}
\newcommand{\bidiarrow}[1]{\underset{\text{#1}}{\leftrightarrow}}
\newcommand{\ueq}[1]{\underset{\text{#1}}{=}}
\newcommand{\mC}{\mathbb{C}}
\newcommand{\mR}{\mathbb{R}}
\newcommand{\mN}{\mathbb{N}}
\newcommand{\mZ}{\mathbb{Z}}
\newcommand{\mQ}{\mathbb{Q}}
\newcommand{\inv}[1]{#1^{-1}}

% Custom text commands (for Hebrew)
\newcommand{\q}[3]{\R{שאלה #3#2.#1}}
\newcommand{\m}[3]{\R{משפט #3#2.#1}}
\newcommand{\h}[3]{\R{הגדרה #3#2.#1}}
\newcommand{\ms}[3]{\R{מסקנה #3#2.#1}}
\newcommand{\ta}[3]{\R{טענה #3#2.#1}}

% Custom commands for this document

% Spacing
\titlespacing\subsubsection{0pt}{5pt}{3pt}
\setstretch{0.1}

\title{נספח לממן 61-- הוכחות ארוכות ומייגעות}
\author{יונתן אוחיון}

\begin{document}
\maketitle

\begin{abstract}
נספח זה כולל בתוכו שתי הוכחות ארוכות שכתבתי לשאלות 1א ו2ב בממן 61 של חשבון אינפינטסימלי 1. הוכחות אלו נכתבו במהלך הלילה / שעות הלימודים שלי בתיכון ולכן יכול להיות שהן אינן מדויקות. \textbf{נספח זה אינו חלק מהמטלה והוא צורף רק מכיוון שאני מעוניין לראות אם הוכחות אלו נכונות.} אשמח אם תבדוק אותן אם יש לך זמן.
\end{abstract}

\section*{השאלות}
\subsection*{שאלה 1א}
חשב את הגבול הבא:
\[
\lim_{n \to \infty} \left(1 + \sin\frac{1}{n}\right)^{n}
\]
\subsection*{שאלה 2ב}
תהי $f(x) = e^{-x} + \sin^{2}{x}$. הוכח כי $\inf f\left([0, \infty)\right) = 0$.

\clearpage

\section*{פתרון -- שאלה 1א}
\subsection*{טענת עזר}
תהי $f$ כך ש$\lim_{t \to \infty} f(t) = x, 0 \neq x \in \mR$. נראה כי מתקיים
\begin{equation}
\lim_{t \to \infty} \left(1 + \frac{f(t)}{t}\right)^{t} = e^{x}
\end{equation}
ראשית, נעבור לסדרות. תהי $(x_{n})^{\infty}_{n=1}$ סדרה המקיימת $\lim_{n \to \infty} x_{n} = \infty$. אזי לפי היינה,
\[
\lim_{t \to \infty} \left(1 + \frac{f(t)}{t}\right)^{t} = \lim_{n \to \infty} \left(1 + \frac{f(x_{n})}{x_{n}}\right)^{x_{n}}
\]
נתבונן בסדרה $y_{n} = \frac{f(x_{n})}{x_{n}}$. נוכל לראות שמתקיים $\lim_{n \to \infty} y_{n} = 0$. נציב ב)1( ונראה כי
\begin{align*}
\lim_{n \to \infty} \left(1 + \frac{f(x_{n})}{x_{n}}\right)^{x_{n}}
&= \lim_{n \to \infty} (1 + y_{n})^{\frac{f(x_{n})}{y_{n}}}\\
&= \lim_{n \to \infty} \left((1 + y_{n})^\frac{1}{y_{n}}\right)^{f(x_{n})}\\
\text{\R{\ta{6}{51}{}}} &= {\lim_{n \to \infty} (1 + y_{n})^\frac{1}{y_{n}}}^{\lim_{n \to \infty} f(x_{n})}\\
\text{\R{היינה}} &= {\lim_{u \to 0} (1 + u)^\frac{1}{u}}^{\lim_{t \to \infty} f(t)}\\
\text{\R{\ta{6}{81}{}}} &= e^{\lim_{t \to \infty} f(t)} = \framebox{$e^{x}$}
\end{align*}
והוכחנו כי מתקיים
\[
\lim_{t \to \infty} \left(1 + \frac{f(t)}{t}\right)^{t} = e^{x}
\]
כנדרש.
\br\qed
\clearpage

\section*{פתרון -- שאלה 1א )המשך(}
\subsection*{הוכחה}
ראשית, נזכר בהוכחה של הגבול המפורסם $\lim_{x \to 0} \frac{\sin{x}}{x} = 1$. בתור חלק מההוכחה, הוכח אי השוויון הבא עבור $x$ בקרבת 0:
\[
\cos{x} \le \frac{\sin{x}}{x} \le 1 \equiv x\cos{x} \le \sin{x} \le x
\]
כלומר, נוכל לראות שעבור $x$ בקרבת 0 מתקיים
\begin{equation}
(1 + x\cos{x})^{\frac{1}{x}}
\le (1 + \sin{x})^{\frac{1}{x}}
\le (1 + x)^{\frac{1}{x}}
\end{equation}
בנוסף, מכלל הסנדוויץ' נובע כי מתקיים
\begin{equation}
\lim_{x \to 0} (1 + x\cos{x})^{\frac{1}{x}}
\le \lim_{x \to 0} (1 + \sin{x})^{\frac{1}{x}}
\le \lim_{x \to 0} (1 + x)^{\frac{1}{x}}
\end{equation}
ידוע כי $\lim_{n \to \infty} \frac{1}{n} = 0$. מ)2(, )3( והיינה נובע כי
\begin{equation}
\lim_{n \to \infty} \left(1 + \frac{\cos{\frac{1}{n}}}{n}\right)^{n}
\le \lim_{n \to \infty} \left(1 + \sin{\frac{1}{n}}\right)^{n}
\le \lim_{n \to \infty} \left(1 + \frac{1}{n}\right)^{n}
\end{equation}
לפיכך, כל שעלינו הוא לחשב את הגבול הימני והשמאלי ולהשתמש בכלל הסנדוויץ'. הגבול הימני הוא כמובן הגבול הידוע $\lim_{n \to \infty} \left(1 + \frac{1}{n}\right)^{n} = e$. בנוסף, מטענת העזר נובע כי
\begin{equation}
\lim_{n \to \infty} \left(1 + \frac{\cos{\frac{1}{n}}}{n}\right)^{n} = e^{\lim_{n \to \infty} \cos{\frac{1}{n}}}
\end{equation}

מכיוון ש$\cos$ רציפה בכל נקודה ו$\lim_{n \to \infty} \frac{1}{n} = 0$, מתקיים $\lim_{n \to \infty} \cos{\frac{1}{n}} = \cos{0} = 1$. לפיכך ולפי )5(,
\[
\lim_{n \to \infty} \left(1 + \frac{\cos{\frac{1}{n}}}{n}\right)^{n} = e^{\lim_{n \to \infty} \cos{\frac{1}{n}}} = e^{1} = e
\]
לכן, נוכל לכתוב את )4( מחדש בתור
\[
e \le \lim_{n \to \infty} \left(1 + \sin{\frac{1}{n}}\right)^{n} \le e
\]
ומכלל הסנדוויץ' נובע כי $\lim_{n \to \infty} \left(1 + \sin{\frac{1}{n}}\right)^{n} = e$ כנדרש.
\br\qed
\clearpage

\section*{פתרון -- שאלה 2ב}
נסמן $A = [0, \infty)$ ונסמן $\inf_{S} f = \inf f(S)$ )כמו כן, $\min_{S} f = \min f(S)$(. ראשית נראה כי 0 חסם מלרע של $f(A)$. יהי $x \in A$. אזי:
\[
0 < e \logr 0 < \frac{1}{e} \logr 0 = 0^{x} < \left(\frac{1}{e}\right)^{x} = \frac{1}{e^{x}} = e^{-x} \logr 0 < e^{-x}
\]
בנוסף, ידוע כי לכל $y \in \mR$ מתקיים $0 \le y^{2}$. לפיכך, $0 \le \sin^{2}{x}$. לפיכך, עבור $x \in A$ מתקיים
\[
0 < e^{-x} + \sin^{2}{x}
\]
והוכחנו ש0 חסם מלרע של $f(A)$. כעת עלינו להראות שמתקיים $\inf_{A} f = 0$. נסמן $g(x) = e^{-x}$ ונתבונן בגבול $\lim_{x \to \infty} g(x)$. ידוע לנו שמתקיים
\[
\lim_{x \to \infty} e^{-x} = \frac{1}{\lim_{x \to \infty} e^{x}} = 0
\]
לפיכך לפי הגדרת הגבול באינסוף מתקיים
\begin{align*}
\forall \varepsilon > 0 \exists M \in \mR \forall x \in (M, \infty),\;e^{-x} \in N_{\varepsilon}(0)
&\iff \forall \varepsilon > 0 \exists M \in \mR \forall x > M,\;e^{-x} < \varepsilon\\
\text{\R{נבחר $M = 0$}} &\iff \forall \varepsilon > 0 \forall x > 0, e^{-x} < \varepsilon\\
\text{\R{בפרט קיים $x > 0$ המקיים זאת}} &\iff \forall \varepsilon > 0 \exists x > 0, e^{-x} < \varepsilon
\end{align*}
בנוסף, $e^{-0} = e^{0} = 1$ ולכן לפי הגדרת האינפימום מתקיים $\inf_{A} g = 0$.

בנוסף, ידוע כי $h(x) = \sin^{2}(x) \in [0, 1]$ ולכן $\min_{A} h = 0$ ובפרט $\inf_{A} h = 0$.

כעת מהגדרת $f$ נובע כי
\[
f(A) = \left\{ e^{-x} + \sin^{2}{x} \mid x \in A \right\} = \left\{ e^{-x} \mid x \in A \right\} + \left\{ \sin^{2}{x} \mid x \in A \right\} = g(A) + h(A)
\]
לכן מתקיים
\[
\inf_{A} f = \inf_{A} g + \inf_{A} h = 0 + 0 = 0
\]
 והוכחנו כי $\inf_{A} f = 0$ כנדרש.
\br\qed
\end{document}