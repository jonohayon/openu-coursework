\documentclass[11pt, oneside]{article}
\usepackage{geometry}
\geometry{a4paper}
\usepackage[parfill]{parskip}
\usepackage[nodisplayskipstretch]{setspace}
\usepackage{graphicx,titlesec}
\usepackage{amsmath,amssymb,cancel,commath}

% Hebrew Stuff
\usepackage[utf8x]{inputenc}
\usepackage[english,hebrew]{babel}
\usepackage{hebfont}

% Custom commands
\newcommand{\qed}{\R{$\blacksquare$}}
\newcommand{\br}{\\\\\\\\\\\\\\}
\newcommand{\opr}[1]{\xrightarrow[\text{#1}]{}}
\newcommand{\logr}{\Longrightarrow}
\newcommand{\bidiarrow}[1]{\underset{\text{#1}}{\leftrightarrow}}
\newcommand{\ueq}[1]{\underset{\text{#1}}{=}}
\newcommand{\mC}{\mathbb{C}}
\newcommand{\mR}{\mathbb{R}}
\newcommand{\mN}{\mathbb{N}}
\newcommand{\mZ}{\mathbb{Z}}
\newcommand{\mQ}{\mathbb{Q}}
\newcommand{\inv}[1]{#1^{-1}}

% Custom text commands (for Hebrew)
\newcommand{\q}[3]{\R{שאלה #3#2.#1}}
\newcommand{\m}[3]{\R{משפט #3#2.#1}}
\newcommand{\h}[3]{\R{הגדרה #3#2.#1}}
\newcommand{\ms}[3]{\R{מסקנה #3#2.#1}}
\newcommand{\ta}[3]{\R{טענה #3#2.#1}}

% Custom commands for this document
\renewcommand{\sf}{\sin\frac{1}{n}}

% Spacing
\titlespacing\subsubsection{0pt}{5pt}{3pt}
\setstretch{0.1}

\title{ממ``''ן 61}
\author{יונתן אוחיון}

\begin{document}
\maketitle

%%%% <Q1a> %%%%
\section*{שאלה 1א}
ראשית, נחשב את גבול הסדרה $a_{n} = n\sin{\frac{1}{n}}$:
\begin{align*}
1
&= \lim_{x \to 0} \frac{\sin{x}}{x}\\
\text{$x_{n} \opr{$n \to \infty$} 0$ \R{היינה: לכל}}
&= \lim_{n \to \infty} \frac{\sin{x_{n}}}{x_{n}}\\
\text{$x_{n} = \frac{1}{n}$ \R{בפרט עבור}}
&= \lim_{n \to \infty} \frac{\sf}{\frac{1}{n}}\\
&= \lim_{n \to \infty} n\sf
\end{align*}
לפיכך, הראינו כי $\lim_{n \to \infty} n\sf = 1$. כעת, נראה ש$\lim_{n \to \infty} \sf = 0$:
\begin{align*}
0
&= \lim_{x \to 0} \sin{x}\\
\text{$x_{n} \opr{$n \to \infty$} 0$ \R{היינה: לכל}}
&= \lim_{n \to \infty} \sin{x_{n}}\\
\text{$x_{n} = \frac{1}{n}$ \R{בפרט עבור}}
&= \lim_{n \to \infty} \sf
\end{align*}
לפיכך הראינו כי $\lim_{n \to \infty} \sf = 0$. כעת, נראה כי הגבול הבא שווה ל$e$:
\begin{align*}
\lim_{n \to \infty} \left(1 + \sf\right)^{n}
&= \lim_{n \to \infty} \left(1 + \sf\right)^{n \cdot \frac\sf\sf}\\
&= \lim_{n \to \infty} \left(\left(1 + \sf\right)^{\frac{1}{\sf}}\right)^{n\sf}\\
\text{\R{\ta{6}{51}{}}}
&= \lim_{n \to \infty} \left(1 + \sf\right)^{\frac{1}{\sf}}\\
\text{\R{היינה}}
&= \lim_{x \to 0} (1 + x)^{\frac{1}{x}} = e
\end{align*}
לפיכך מתקיים $(1 + \sf)^{n} \opr{$n \to \infty$} e$ כנדרש.
\br\qed
%%%% </Q1a> %%%%
\clearpage

%%%% <Q1b> %%%%
\section*{שאלה 1ב}
נתבונן בפונקציה $f(x) = \abs{x}^{x^{2}}$. נוכל לראות שמתקיים
\[
\lim_{x \to \infty} \abs{x}
\le \lim_{x \to \infty} f(x)
= \lim_{x \to \infty} \abs{x}^{x^{2}}
\]
בנוסף, ידוע כי $\lim_{x \to \infty} \abs{x} = \infty$. לכן, מקריטריון ההשוואה לאינסוף באנלוגיה לפונקציות נובע כי $\lim_{x \to \infty} f(x) = \infty$.

מכך נובע כי
\[
0 = \lim_{x \to \infty} \frac{1}{\abs{x}^{x^{2}}}
= \lim_{x \to \infty} \frac{1}{\abs{x}}^{x^{2}}
= \lim_{x \to \infty} \abs{\frac{1}{x}}^{x^{2}}
\]
כעת, נציב $t = \frac{1}{x}$ )כמובן ש$t \to 0$( ונקבל את הגבול הבא:
\[
0 = \lim_{x \to \infty} \abs{\frac{1}{x}}^{x^{2}}
= \lim_{t \to 0} \abs{t}^{\frac{1}{t^{2}}}
\]
כלומר, הוכחנו כי מתקיים
\[
\lim_{t \to 0} \abs{t}^{\frac{1}{t^{2}}} = 0
\]
כנדרש.
\br\qed
%%%% </Q1> %%%%
\clearpage

%%%% <Q2a> %%%%
\section*{שאלה 2א}
נתבונן בגבול $\lim_{x \to \infty} e^{-x}$:
\begin{equation}
\lim_{x \to \infty} e^{x} = \infty \logr \lim_{x \to \infty} \frac{1}{e^{x}} = 0 = \lim_{x \to \infty} e^{-x}
\end{equation}
בנוסף, $\lim_{n \to \infty} \pi n = \infty$. לפיכך, לפי )6( ומהיינה מתקיים
\[
\lim_{n \to \infty} e^{-\pi n} = 0
\]
כעת נתבונן בשתי תת סדרות המכסות את $\sin^{2}{\pi n}$: $\sin^{2}{2\pi n}$ ו$\sin^{2}{2\pi n + \pi}$. נתבונן בגבולותיהן:
\[
\lim_{n \to \infty} \sin^{2}{2\pi n} \ueq{\R{מחזוריות $\sin$}} \sin^{2}{2\pi} = \sin^{2}{0} = 0
\]
בנוסף,
\[
\lim_{n \to \infty} \sin^{2}{2\pi n + \pi} \ueq{\R{מחזוריות $\sin$}} \sin^{2}{2\pi + \pi} = \sin^{2}{\pi} = 0
\]
מכיוון שתת סדרות אלו מכסות את $\sin^{2}{\pi n}$, מתקיים
\[
\lim_{n \to \infty} \sin^{2}{\pi n} = 0
\]

לפיכך, מתקיים
\[
0 = \lim_{n \to \infty} e^{-\pi n} + \lim_{n \to \infty} \sin^{2}{\pi n} = \lim_{n \to \infty} e^{-\pi n} + \sin^{2}{\pi n} = \lim_{n \to \infty} f(\pi n)
\]
והוכחנו כי $\lim_{n \to \infty} f(\pi n) = 0$ כנדרש.
\br\qed
%%%% </Q2a> %%%%
\clearpage

%%%% <Q2b> %%%%
\section*{שאלה 2ב}
נסמן $A = [0, \infty)$. ראשית נראה כי 0 חסם מלרע של $f(A)$. יהי $x \in A$. אזי:
\[
0 < e \logr 0 < \frac{1}{e} \logr 0 = 0^{x} < \left(\frac{1}{e}\right)^{x} = \frac{1}{e^{x}} = e^{-x} \logr 0 < e^{-x}
\]
בנוסף, ידוע כי לכל $y \in \mR$ מתקיים $0 \le y^{2}$. לפיכך, $0 \le \sin^{2}{x}$. לפיכך, עבור $x \in A$ מתקיים
\[
0 < e^{-x} + \sin^{2}{x}
\]
והוכחנו ש0 חסם מלרע של $f(A)$. כעת, מהסעיף הקודם נובע כי $\exists n \in \mN \forall c > 0, f(\pi n) < c$. בנוסף, מכיוון ש$f(\pi n) \opr{$n \to \infty$} 0$, אי שוויון זה נשמר לכל $n$ טבעי ולכן כל $c > 0$ אינו חסם מלרע של $f$. לפיכך $\inf f(A) = 0$ כנדרש.
\br\qed
%%%% </Q2b> %%%%

%%%% <Q2c> %%%%
\section*{שאלה 2ג}
נניח ש$f$ מקבלת מינימום ב$A$, כלומר קיים $x_{0} \in A$ כך שלכל $x \in A$ מתקיים $f(x_{0}) \le f(x)$. לפיכך לפי \ta{3}{31}{} והסעיף הקודם מתקיים
\[
f(x_{0}) = 0 \logr e^{-x_{0}} + \sin^{2}{x} = 0 \logr \sin^{2}{x_{0}} = -e^{-x_{0}}
\]
אבל כפי שראינו בסעיף הקודם $e^{-x} > 0 \logr -e^{-x} < 0$ וגם $\sin^{2}{x} > 0$ והגענו לסתירה. לפיכך, לא קיים $x_{0}$ שכזה ולכן לא קיים $f(x_{0})$ המקיים את הטענה ולכן $f$ לא מקבלת מינימום ב$A$ כנדרש.
\br\qed
%%%% </Q2c> %%%%
\clearpage

%%%% <Q3a> %%%%
\section*{שאלה 3א}
\subsection*{תחום הגדרה}
מכיוון ש$\sin$ ו$\sin^{2}$ מוגדרות בכל $\mR$ ו$\frac{1}{x}$ מוגדרת לכל $x \neq 0$, הפונקציה $g(x) = \sin^{2}x\sin\frac{1}{x}$ מוגדרת לכל $x \neq 0$. מכיוון שהגדרנו את $f$ כך ש$f(0) = 0$, הפונקציה $f(x)$ מוגדרת בכל $\mR$.
\br\qed

\subsection*{תחום רציפות}
נראה ש$f$ רציפה בכל $\mR$. נניח ש$x_{0} \in \mR$ נק' אי רציפות מסוג כלשהו של $f$ ונחלק למקרים:

אם $x_{0} \neq 0$, קיימת סביבה של $x_{0}$ שבה $f$ מתלכדת עם $\sin^{2}x\sin\frac{1}{x}$. מכיוון שפונקציה זו הינה כפל של הרכבה של פונקציות אלמנטריות, היא רציפה בכל נקודה בתחום הגדרתה. כפי שראינו לעיל, פונקציה זו מוגדרת לכל $x \neq 0$ ולכן גם רציפה ב$x_{0}$, בסתירה להנחה שזו נק' אי רציפות.

נשאר לנו להראות ש$f$ רציפה ב0. מכיוון ש$\sin\frac{1}{x}$ חסומה ו$\sin^{2}x$ אפסה בסביבת 0, מתקיים
\[
\lim_{x \to 0} \sin^{2}x\sin\frac{1}{x} = 0
\]
בנוסף, ידוע לנו ש$f(0) = 0$ לפי הגדרתה של $f$ ומכיוון ש$\lim_{x \to 0} f(x) = f(0) = 0$, $f$ רציפה ב0 בסתירה להנחה.

לפיכך, הראינו של$f$ לא קיימות נק' אי רציפות משום סוג ולכן היא רציפה בכל $\mR$ כנדרש.
\br\qed

\subsection*{תחום גזירות}
נראה ש$f$ גזירה בכל $\mR$ ע``''י חלוקה למקרים. ראשית, נתבונן בפונקציות המרכיבות את $f$ כאשר $x \neq 0$:
\begin{itemize}
\item $\frac{1}{x}$ הינה פונקציה רציונלית ולכן גזירה בכל נק' בתחום הגדרתה, כלומר בכל $x \neq 0$. נגזרתה בתחום זה )לפי כלל החזקה(: $-\frac{1}{x^{2}}$.
\item $\sin$ גזירה בכל $\mR$ ולכן גם $\sin^{2}$ גזירה בכל $\mR$. נגזרותיהן )בהתאמה(: $2\sin{x}\cos{x}, \cos{x}$.
\item לפיכך, ההרכבה $\sin\frac{1}{x}$ גזירה בכל נקודה בתחום הגדרתה לפי \m{7}{12}{} ונגזרתה בתחום זה היא $-\frac{\cos{x}}{x^{2}}$.
\end{itemize}
לפיכך, לפי כלל המכפלה $f$ גזירה בכל $x \neq 0$ ונגזרתה בתחום זה היא:
\begin{align*}
f^{\prime}(x)
&= \left(\sin^{2}x\sin\frac{1}{x}\right)^{\prime}\\
&= (\sin^{2}x)^{\prime}\sin\frac{1}{x} + \left(\sin\frac{1}{x}\right)^{\prime}\sin^{2}x\\
&= 2\sin{x}\cos{x}\sin\frac{1}{x} - \frac{\cos\frac{1}{x}\sin^{2}x}{x^{2}}
 = \sin{2x}\sin\frac{1}{x} - \frac{\cos\frac{1}{x}\sin^{2}x}{x^{2}}
\end{align*}
ומצאנו את הנגזרת של $f$ בכל $x \neq 0$. בעמוד הבא נראה ש$f$ גזירה ב0 ומתקיים $f^{\prime}(0) = 0$.
\clearpage

\section*{שאלה 3א -- המשך}
\subsection*{תחום גזירות -- המשך}
נתבונן במקרה שנותר לנו להוכיח, בו $x = 0$. נתבונן בהגדרת הנגזרת בנקודה:
\begin{align*}
f^{\prime}(0)
&= \lim_{h \to 0} \frac{f(h) - f(0)}{h}\\
f(0) = 0 \text{\R{נימוק: }} &= \lim_{h \to 0} \frac{f(h)}{h}\\
&= \lim_{h \to 0} \frac{\sin^{2}{h}\sin\frac{1}{h}}{h}\\
&= \lim_{h \to 0} \sin{h}\sin\frac{1}{h} \cdot \lim_{h \to 0} \frac{\sin{h}}{h}\\
&= \lim_{h \to 0} \sin{h}\sin\frac{1}{h}\\
\text{\R{$\sin{h}$ אפסה, $\sin\frac{1}{h}$ חסומה}}&= 0
\end{align*}
לכן, $f$ גזירה ב0 ובפרט מתקיים $f^{\prime}(0) = 0$. לפיכך, נוכל להגדיר את הנגזרת של $f$ באופן הבא:
\[
f^{\prime}(x) =
\begin{cases}
\sin{2x}\sin\frac{1}{x} - \frac{\cos\frac{1}{x}\sin^{2}x}{x^{2}} & x \neq 0\br
0 & x = 0\\\\\\\\\\
\end{cases}
\]
ומצאנו את תחום הגזירות וערך הנגזרת של $f$ בכל $\mR$ כנדרש.
\br\qed
%%%% </Q3a> %%%%
\clearpage

%%%% <Q3b> %%%%
\section*{שאלה 3ב}
\subsection*{תחום הגדרה}
מכיוון ש$\abs{x}$ מוגדרת לכל $x \in \mR$ ו$\ln{x}$ מוגדרת לכל $x > 0$, הפונקציה $g(x) = \abs\ln{x}$ מוגדרת לכל $x > 0$.
\br\qed
\subsection*{תחום רציפות}
מכיוון ש$\ln{x}$ רציפה בכל נקודה בתחום הגדרתה )כלומר בכל $x > 0$( ו$\abs{x}$ רציפה בכל $x \in \mR$, $g$ רציפה בכל $x > 0$, כלומר בכל נקודה בתחום הגדרתה.
\br\qed
\subsection*{תחום גזירות}
ראשית, $g$ אינה גזירה בכל $x \le 0$, שכן היא אינה רציפה שם. הפונקציה $g$ הינה הרכבה של שתי פונקציות: $h(x) = \abs{x}, i(x) = \ln{x}$, כלומר $g(x) = h(i(x))$. לפיכך, לפי \m{7}{12}{}, $g$ גזירה ב$0 < x_{0}$ אמ``''מ $i$ גזירה ב$x_{0}$ ו$h$ גזירה ב$i(x)$. נראה, אם כן, ש$g$ גזירה בכל $0 < x \neq 1$. יהי $0 < x_{0}$ ונחלק למקרים:

אם $0 < x_{0} < 1$, $i$ גזירה ב$x_{0}$ )שכן היא נמצאת בתחום ההגדרה שלה ו$i$ גזירה בכל נקודה בתחום הגדרתה(. בנוסף, $\ln{x_{0}} < 0$ ו$h$ גזירה ב$x_{0}$ )שכן היא גזירה בכל נקודה בתחום הגדרתה שאינה 0(. לפיכך, $g$ גזירה ב$x_{0}$ ולפי כלל השרשרת נגזרתה בנקודה זו היא:
\[
f^{\prime}(x_{0}) = -\ln{x} \cdot \frac{1}{x} = \frac{-\ln{x}}{x}
\]

אם $1 < x_{0}$, $i$ כמובן גזירה ב$x_{0}$. בנוסף, $0 < \ln{x_{0}}$ ו$h$ גזירה ב$i(x_{0})$. לפיכך, $g$ גם היא גזירה ב$x_{0}$ ולפי כלל השרשרת נגזרתה בנקודה זו היא:
\[
f^{\prime}(x_{0}) = \ln{x} \cdot \frac{1}{x} = \frac{\ln{x}}{x}
\]

נותר לנו כעת להוכיח ש$g$ אינה גזירה ב$x_{0} = 1$. כפי שנאמר לעיל, $g$ הינה הרכבה של $\abs{x} $ ושל $\ln{x}$ ולכן גזירה ב$1$ אממ $\ln{x}$ גזירה ב1 ו$\abs{x}$ גזירה ב$\ln{1}$. כפי שאנו יודעים, $\ln{x}$ גזירה ב1 שכן היא גזירה בכל נקודה בתחום הגדרתה, אבל $\ln{1} = 0$ ו$\abs{x}$ אינה גזירה ב0 ולכן $g$ לא גזירה ב1.

לפיכך, $g$ גזירה בכל $0 < x \neq 1$ ונגזרתה היא:
\[
g(x) = \begin{cases}\\\\
\frac{-\ln{x}}{x} & 0 < x < 1\br
\frac{\ln{x}}{x} & 1 < x\\\\\\
\end{cases}
\]
ומצאנו את תחום הגזירות של $g$ כנדרש.
\br\qed
%%%% </Q3b> %%%%
\clearpage

%%%% <Q4> %%%%
\section*{שאלה 4}
תהי $f$ פונקציה זוגית ב$\mR$, כלומר $f(-x) = f(x)$ לכל $x \in \mR$. נניח בנוסף כי $f$ גזירה ב0. נתבונן בנגזרת השמאלית של $f$ ב0:
\begin{align*}
f^{\prime}_{-}(0)
&= \lim_{h \to 0^{-}} \frac{f(h) - f(0)}{h}\\
&= \lim_{h \to 0^{-}} \frac{f(h)}{h} - \lim_{h \to 0^{-}} \frac{f(0)}{h}\\
\text{\R{זוגיות $f$}} &= \lim_{h \to 0^{-}} \frac{f(-h)}{h} + \lim_{h \to 0^{+}} \frac{f(0)}{h}\\
&= -\lim_{h \to 0^{+}} \frac{f(h)}{h} + \lim_{h \to 0^{+}} \frac{f(0)}{h}\\
\text{(1)\quad\quad\quad} &= -\left(\lim_{h \to 0^{+}} \frac{f(h) - f(0)}{h}\right)
\end{align*}
ובנגזרת הימנית ב0:
\[
f^{\prime}_{+}(0) = \lim_{h \to 0^{+}} \frac{f(h) - f(0)}{h}\\
\]
לפי )1( והנתון מתקיים
\[
f^{\prime}_{+}(0) = -f^{\prime}_{+}(0) \Longrightarrow
2f^{\prime}_{+}(0) = 0 \Longrightarrow
f^{\prime}_{+}(0) = 0
\]
לפי הנתון, $f^{\prime}_{-}(0) = f^{\prime}_{+}(0) = 0$, כלומר הנגזרת הימנית והנגזרת השמאלית שוות ובפרט שוות ל0 ולכן $f^{\prime}(0) = 0$ כנדרש.
\br\qed
%%%% </Q4> %%%%
\clearpage

%%%% <Q5a1> %%%%
\section*{שאלה 5א}
\subsection*{רציפות משמאל ב0}
נראה ש$f$ רציפה משמאל ב0. לפי הגדרת $f$, עבור $x < 0$ היא מתלכדת עם הפונקציה $g(x) = x + xe^{\frac{1}{x}}$. לפיכך, על מנת לחשב את הגבול של $f$ ב0 משמאל עלינו לחשב את הגבול של $g$ ב0 משמאל:
\begin{align*}
\lim_{x \to 0^{-}} g(x)
&= \lim_{x \to 0^{-}} x + xe^{\frac{1}{x}}\\
&= \lim_{x \to 0^{-}} x + \lim_{x \to 0^{-}} xe^{\frac{1}{x}}\\
&= \lim_{x \to 0^{-}} xe^{\frac{1}{x}}\\
\text{$t = -\frac{1}{x}, t \to \infty$ \R{הצבה:}}
&= \lim_{t \to \infty} \frac{e^{-t}}{t}\\
&= \lim_{t \to \infty} \frac{1}{t} \cdot \lim_{t \to \infty} e^{-t}\\
&= 0 \cdot 0 = 0
\end{align*}
לפיכך,
\[
\lim_{x \to 0^{-}} f(x)
 = \lim_{x \to 0^{-}} g(x)
 = \lim_{x \to 0^{-}} x + xe^{\frac{1}{x}} = 0
\]
לפי הגדרת $f$ מתקיים $f(0) = 0$ ולכן $\lim_{x \to 0^{-}} f(x) = f(0)$ ו$f$ רציפה משמאל ב0 כנדרש.
\br\qed
%%%% </Q5a1> %%%%

%%%% <Q5a2> %%%%
\subsection*{רציפות מימין ב0 -- $a = 2$}
נראה ש$f$ רציפה מימין כאשר  $a = 2$. לפי הגדרת $f$ היא מתלכדת עם $h(x) = \frac{2 - 2\cos{x}}{\sin{x}}$ עבור $x > 0$ ולכן עלינו לחשב את הגבול של $h$ ב0 מימין:
\begin{align*}
\lim_{x \to 0^{+}} h(x)
&= \lim_{x \to 0^{+}} \frac{2 - 2\cos{x}}{\sin{x}}\\
&= 2\lim_{x \to 0^{+}} \frac{1 - \cos{x}}{\sin{x}}\\
&= 2\lim_{x \to 0^{+}} \frac{\sin^{\not2}{x}}{\cancel{\sin{x}}(1 + \cos{x})}\\
&= 2\lim_{x \to 0^{+}} \sin{x} \cdot \frac{1}{1 + \lim_{x \to 0^{+}} \cos{x}}\\
&= 2 \cdot 0 \cdot \frac{1}{2} = 0 = f(0)
\end{align*}
ומכיוון ש$\lim_{x \to 0^{+}} f(x) = f(0)$ כאשר $a = 2$, היא רציפה מימין ב0. בפרט מכיוון שהיא תמיד רציפה משמאל ב0 היא רציפה ב0 כאשר $a = 2$ כנדרש.
\br\qed
%%%% </Q5a2> %%%%
\clearpage

%%%% </Q5a3> %%%%
\section*{שאלה 5א -- המשך}
\subsection*{אי-רציפות מימין ב0 -- $a \neq 2$}
נראה ש$f$ אינה רציפה מימין ב0 כאשר $a \neq 2$. נתבונן בגבולות הבא:
\[
\lim_{x \to 0^{+}} a - 2\cos{x} = a - 2\lim_{x \to 0^{+}} \cos{x} = a - 2
\]
גבול זה שונה מ0 כאשר $a \neq 2$, קטן מ0 כאשר $a < 2$ וגדול מ0 כאשר $a > 2$. בנוסף, מכיוון ש$\lim_{x \to 0} \sin{x} = 0$ מתקיים $\lim_{x \to 0^{+}} \frac{1}{\sin{x}} = \infty$. 

נחלק למקרים. נתבונן בגבול של $h(x)$ במקרה שבו $a > 2$ ב0 מימין:
\begin{align*}
\lim_{x \to 0^{+}} h(x)
&= \lim_{x \to 0^{+}} \frac{a - 2\cos{x}}{\sin{x}}\\
&= \lim_{x \to 0^{+}} (a - 2\cos{x}) \cdot \lim_{x \to 0^{+}} \frac{1}{\sin{x}}\\
&= (a - 2) \cdot \lim_{x \to 0^{+}} \frac{1}{\sin{x}}\\
\text{$0 < k = a - 2$ \R{נסמן:}}&= k \cdot \lim_{x \to 0^{+}} \frac{1}{\sin{x}}\\
&= k \cdot \infty = \infty
\end{align*}
לפיכך כאשר $a > 2$, $f$ הולכת לאינסוף כאשר $x \to 0$ מימין ובפרט לא הולכת ל0. לפיכך, $f$ לא רציפה מימין ב0 כאשר $a > 2$ ולכן לא רציפה ב0.

אם $a < 2$ אז קיים $m \in \mR$ כך ש$-m = a - 2$, או $m = 2 - a$. בנוסף, לפי \m{4}{35}{}ז,  נתבונן, אם כן, בגבול מימין של $-h(x)$ כאשר $x \to 0$:
\begin{align*}
\lim_{x \to 0^{+}} -h(x)
&= \lim_{x \to 0^{+}} \frac{2\cos{x} - a}{\sin{x}}\\
&= \lim_{x \to 0^{+}} (2\cos{x} - a) \cdot \lim_{x \to 0^{+}} \frac{1}{\sin{x}}\\
&= (2 - a) \cdot \lim_{x \to 0^{+}} \frac{1}{\sin{x}}\\
\text{$0 < m = 2 - a$ \R{נסמן:}} &= m \cdot \lim_{x \to 0^{+}} \frac{1}{\sin{x}}\\
&= m \cdot \infty  = \infty
\end{align*}
לפיכך לפי \m{4}{35}{}ז מתקיים $\lim_{x \to 0^{+}} f(x) = -\infty$ ובפרט הגבול אינו 0. לפיכך, $f$ אינה רציפה מימין ב0 כאשר $a < 2$ ולכן לא רציפה ב0.

לפיכך, כאשר $a \neq 2$, $f$ אינה רציפה ב0 כנדרש.
\br\qed
%%%% </Q5a3> %%%%
\clearpage

%%%% <Q5b> %%%%
\section*{שאלה 5ב}
לפי הסעיף הקודם, $f$ רציפה ב0 אמ``''מ $a = 2$, ולכן היא אינה גזירה עבור כל $a \neq 2$. נראה שהיא כן גזירה ב$a = 2$. ראשית נגדיר מחדש את $f$:
\[
f(x) = \begin{cases}
x + xe^{\frac{1}{x}} & x < 0\br
0 & x = 0\br
\frac{2 - 2\cos{x}}{\sin{x}} & x > 0
\end{cases}
\]
נחשב את הנגזרת השמאלית של $f$ ב0:
\begin{align*}
f^{\prime}_{-}(0)
&= \lim_{h \to 0^{-}} \frac{f(h)}{h}\\
&= \lim_{h \to 0^{-}} \frac{\cancel{h}(1 + e^{\frac{1}{h}})}{\cancel{h}}\\
&= 1 + \lim_{h \to 0^{-}} e^{\frac{1}{h}}\\
\text{$t = -\frac{1}{h}, t \to \infty$ \R{הצבה:}}
&= 1 + \lim_{t \to \infty} e^{-t} = 1
\end{align*}
לפיכך $f$ גזירה משמאל ב0 ובפרט ערך הנגזרת שלה בנקודה זו שווה ל1. נחשב את הנגזרת הימנית של $f$ ב0:
\begin{align*}
f^{\prime}_{+}(0)
&= \lim_{h \to 0^{+}} \frac{f(h)}{h}\\
&= \lim_{h \to 0^{+}} \frac{2 - 2\cos{h}}{h\sin{h}}\\
&= 2\lim_{h \to 0^{+}} \frac{1 - \cos{h}}{h\sin{h}}\\
&= 2\lim_{h \to 0^{+}} \frac{\sin^{\not2}{h}}{h\cancel{\sin{h}}(1 + \cos{h})}\\
&= 2\lim_{h \to 0^{+}} \frac{\sin{h}}{h} \cdot \frac{1}{1 + \lim_{h \to 0^{+}} \cos{h}}\\
&= 2 \cdot 1 \cdot \frac{1}{2} = 1
\end{align*}
לפיכך $f$ גזירה מימין ב0 ובפרט ערך נגזרתה הימנית בנקודה זו שווה ל1.

לפיכך, מכיוון ש$f$ גזירה מימין ומשמאל ב0 ומתקיים $f^{\prime}_{-}(0) = f^{\prime}_{+}(0) = 1$, מתקיים $f^{\prime}(0) = 1$ ומצאנו את ערכי $a$ עבורם $f$ גזירה ב0 כנדרש.
\br\qed
%%%% </Q5b> %%%%

\end{document}