\documentclass[11pt, oneside]{article}
\usepackage{geometry}
\geometry{a4paper}
\usepackage[parfill]{parskip}
\usepackage[nodisplayskipstretch]{setspace}
\usepackage{graphicx,titlesec}
\usepackage{amsmath,amssymb,cancel,commath}

% Hebrew Stuff
\usepackage[utf8x]{inputenc}
\usepackage[english,hebrew]{babel}
\usepackage{hebfont}

% Custom commands
\newcommand{\qed}{\R{$\blacksquare$}}
\newcommand{\br}{\\\\\\\\\\\\\\}
\newcommand{\opr}[1]{\xrightarrow[\text{#1}]{}}
\newcommand{\logr}[1]{\xRightarrow[\text{#1}]{}}
\newcommand{\bidiarrow}[1]{\underset{\text{#1}}{\leftrightarrow}}
\newcommand{\ueq}[1]{\underset{\text{#1}}{=}}
\newcommand{\mC}{\mathbb{C}}
\newcommand{\mR}{\mathbb{R}}
\newcommand{\mN}{\mathbb{N}}
\newcommand{\mZ}{\mathbb{Z}}
\newcommand{\mQ}{\mathbb{Q}}
\newcommand{\inv}[1]{#1^{-1}}

% Custom text commands (for Hebrew)
\newcommand{\q}[3]{\R{שאלה #3#2.#1}}
\newcommand{\m}[3]{\R{משפט #3#2.#1}}
\newcommand{\h}[3]{\R{הגדרה #3#2.#1}}
\newcommand{\ms}[3]{\R{מסקנה #3#2.#1}}
\newcommand{\ta}[3]{\R{טענה #3#2.#1}}

% Custom commands for this document

% Spacing
\titlespacing\subsubsection{0pt}{5pt}{3pt}
\setstretch{0.1}

\title{ממ``''ן 61}
\author{יונתן אוחיון}

\begin{document}
\maketitle

%%%% <Q1> %%%%
\section*{שאלה 1 -- טענת עזר}
תהי $f$ כך ש$\lim_{t \to \infty} f(t) = x, 0 \neq x \in \mR$. נראה כי מתקיים
\begin{equation}
\lim_{t \to \infty} \left(1 + \frac{f(t)}{t}\right)^{t} = e^{x}
\end{equation}
ראשית, נעבור לסדרות. תהי $(x_{n})^{\infty}_{n=1}$ סדרה המקיימת $\lim_{n \to \infty} x_{n} = \infty$. אזי לפי היינה,
\[
\lim_{t \to \infty} \left(1 + \frac{f(t)}{t}\right)^{t} = \lim_{n \to \infty} \left(1 + \frac{f(x_{n})}{x_{n}}\right)^{x_{n}}
\]
נתבונן בסדרה $y_{n} = \frac{f(x_{n})}{x_{n}}$. נוכל לראות שמתקיים $\lim_{n \to \infty} y_{n} = 0$. נציב ב)1( ונראה כי
\begin{align*}
\lim_{n \to \infty} \left(1 + \frac{f(x_{n})}{x_{n}}\right)^{x_{n}}
&= \lim_{n \to \infty} (1 + y_{n})^{\frac{f(x_{n})}{y_{n}}}\\
&= \lim_{n \to \infty} \left((1 + y_{n})^\frac{1}{y_{n}}\right)^{f(x_{n})}\\
\text{\R{\ta{6}{51}{}}} &= {\lim_{n \to \infty} (1 + y_{n})^\frac{1}{y_{n}}}^{\lim_{n \to \infty} f(x_{n})}\\
\text{\R{היינה}} &= {\lim_{u \to 0} (1 + u)^\frac{1}{u}}^{\lim_{t \to \infty} f(t)}\\
\text{\R{\ta{6}{81}{}}} &= e^{\lim_{t \to \infty} f(t)} = \framebox{$e^{x}$}
\end{align*}
והוכחנו כי מתקיים
\[
\lim_{t \to \infty} \left(1 + \frac{f(t)}{t}\right)^{t} = e^{x}
\]
כנדרש.
\br\qed
\clearpage

\section*{שאלה 1א -- הוכחה}
ראשית, נזכר בהוכחה של הגבול המפורסם $\lim_{x \to 0} \frac{\sin{x}}{x} = 1$. בתור חלק מההוכחה, הוכח אי השוויון הבא עבור $x$ בקרבת 0:
\[
\cos{x} \le \frac{\sin{x}}{x} \le 1 \equiv x\cos{x} \le \sin{x} \le x
\]
כלומר, נוכל לראות שעבור $x$ בקרבת 0 מתקיים
\begin{equation}
(1 + x\cos{x})^{\frac{1}{x}}
\le (1 + \sin{x})^{\frac{1}{x}}
\le (1 + x)^{\frac{1}{x}}
\end{equation}
בנוסף, מכלל הסנדוויץ' נובע כי מתקיים
\begin{equation}
\lim_{x \to 0} (1 + x\cos{x})^{\frac{1}{x}}
\le \lim_{x \to 0} (1 + \sin{x})^{\frac{1}{x}}
\le \lim_{x \to 0} (1 + x)^{\frac{1}{x}}
\end{equation}
ידוע כי $\lim_{n \to \infty} \frac{1}{n} = 0$. מ)2(, )3( והיינה נובע כי
\begin{equation}
\lim_{n \to \infty} \left(1 + \frac{\cos{\frac{1}{n}}}{n}\right)^{n}
\le \lim_{n \to \infty} \left(1 + \sin{\frac{1}{n}}\right)^{n}
\le \lim_{n \to \infty} \left(1 + \frac{1}{n}\right)^{n}
\end{equation}
לפיכך, כל שעלינו הוא לחשב את הגבול הימני והשמאלי ולהשתמש בכלל הסנדוויץ'. הגבול הימני הוא כמובן הגבול הידוע $\lim_{n \to \infty} \left(1 + \frac{1}{n}\right)^{n} = e$. בנוסף, מטענת העזר נובע כי
\begin{equation}
\lim_{n \to \infty} \left(1 + \frac{\cos{\frac{1}{n}}}{n}\right)^{n} = e^{\lim_{n \to \infty} \cos{\frac{1}{n}}}
\end{equation}

מכיוון ש$\cos$ רציפה בכל נקודה ו$\lim_{n \to \infty} \frac{1}{n} = 0$, מתקיים $\lim_{n \to \infty} \cos{\frac{1}{n}} = \cos{0} = 1$. לפיכך ולפי )5(,
\[
\lim_{n \to \infty} \left(1 + \frac{\cos{\frac{1}{n}}}{n}\right)^{n} = e^{\lim_{n \to \infty} \cos{\frac{1}{n}}} = e^{1} = e
\]
לכן, נוכל לכתוב את )4( מחדש בתור
\[
e \le \lim_{n \to \infty} \left(1 + \sin{\frac{1}{n}}\right)^{n} \le e
\]
ומכלל הסנדוויץ' נובע כי $\lim_{n \to \infty} \left(1 + \sin{\frac{1}{n}}\right)^{n} = e$ כנדרש.
\br\qed

\section*{שאלה 1ב}
נתבונן בפונקציה $f(x) = \abs{x}^{x^{2}}$. נוכל לראות שמתקיים
\[
\lim_{x \to \infty} f(x)
= \lim_{x \to \infty} \abs{x}^{x^{2}}
= \infty
\]
מכך נובע כי
\[
0 = \lim_{x \to \infty} \frac{1}{\abs{x}^{x^{2}}}
= \lim_{x \to \infty} \frac{1}{\abs{x}}^{x^{2}}
= \lim_{x \to \infty} \abs{\frac{1}{x}}^{x^{2}}
\]
כעת, נציב $t = \frac{1}{x}$ )כמובן ש$t \to 0$( ונקבל את הגבול הבא:
\[
0 = \lim_{x \to \infty} \abs{\frac{1}{x}}^{x^{2}}
= \lim_{t \to 0} \abs{t}^{\frac{1}{t^{2}}}
\]
כלומר, הוכחנו כי מתקיים
\[
\lim_{t \to 0} \abs{t}^{\frac{1}{t^{2}}} = 0
\]
כנדרש.
\br\qed
%%%% </Q1> %%%%
\clearpage

%%%% <Q3a> %%%%
\section*{שאלה 3א}
\subsection*{תחום הגדרה}
מכיוון ש$\sin$ ו$\sin^{2}$ מוגדרות בכל $\mR$ ו$\frac{1}{x}$ מוגדרת לכל $x \neq 0$, הפונקציה $g(x) = \sin^{2}x\sin\frac{1}{x}$ מוגדרת לכל $x \neq 0$. מכיוון שהגדרנו את $f$ כך ש$f(0) = 0$, הפונקציה $f(x)$ מוגדרת בכל $\mR$.
\br\qed

\subsection*{תחום רציפות}
נראה ש$f$ רציפה בכל $\mR$. נניח ש$x_{0} \in \mR$ נק' אי רציפות מסוג כלשהו של $f$ ונחלק למקרים:

אם $x_{0} \neq 0$, קיימת סביבה של $x_{0}$ שבה $f$ מתלכדת עם $\sin^{2}x\sin\frac{1}{x}$. מכיוון שפונקציה זו הינה כפל של הרכבה של פונקציות אלמנטריות, היא רציפה בכל נקודה בתחום הגדרתה. כפי שראינו לעיל, פונקציה זו מוגדרת לכל $x \neq 0$ ולכן גם רציפה ב$x_{0}$, בסתירה להנחה שזו נק' אי רציפות.

נשאר לנו להראות ש$f$ רציפה ב0. מכיוון ש$\sin\frac{1}{x}$ חסומה ו$\sin^{2}x$ אפסה בסביבת 0, מתקיים
\[
\lim_{x \to 0} \sin^{2}x\sin\frac{1}{x} = 0
\]
בנוסף, ידוע לנו ש$f(0) = 0$ לפי הגדרתה של $f$ ומכיוון ש$\lim_{x \to 0} f(x) = f(0) = 0$, $f$ רציפה ב0 בסתירה להנחה.

לפיכך, הראינו של$f$ לא קיימות נק' אי רציפות משום סוג ולכן היא רציפה בכל $\mR$ כנדרש.
\br\qed

\subsection*{תחום גזירות}
נראה ש$f$ גזירה בכל $\mR$ ע``''י חלוקה למקרים. ראשית, נתבונן בפונקציות המרכיבות את $f$ כאשר $x \neq 0$:
\begin{itemize}
\item $\frac{1}{x}$ הינה פונקציה רציונלית ולכן גזירה בכל נק' בתחום הגדרתה, כלומר בכל $x \neq 0$. נגזרתה בתחום זה )לפי כלל החזקה(: $-\frac{1}{x^{2}}$.
\item $\sin$ גזירה בכל $\mR$ ולכן גם $\sin^{2}$ גזירה בכל $\mR$. נגזרותיהן )בהתאמה(: $2\sin{x}\cos{x}, \cos{x}$.
\item לפיכך, ההרכבה $\sin\frac{1}{x}$ גזירה בכל נקודה בתחום הגדרתה לפי \m{7}{12}{} ונגזרתה בתחום זה היא $-\frac{\cos{x}}{x^{2}}$.
\end{itemize}
לפיכך, לפי כלל המכפלה $f$ גזירה בכל $x \neq 0$ ונגזרתה בתחום זה היא:
\begin{align*}
f^{\prime}(x)
&= \left(\sin^{2}x\sin\frac{1}{x}\right)^{\prime}\\
&= (\sin^{2}x)^{\prime}\sin\frac{1}{x} + \left(\sin\frac{1}{x}\right)^{\prime}\sin^{2}x\\
&= 2\sin{x}\cos{x}\sin\frac{1}{x} - \frac{\cos\frac{1}{x}\sin^{2}x}{x^{2}}
\end{align*}
ומצאנו את הנגזרת של $f$ בכל $x \neq 0$. בעמוד הבא נראה ש$f$ גזירה ב0 ומתקיים $f^{\prime}(0) = 0$.
\clearpage

\section*{שאלה 3א -- המשך}
\subsection*{תחום גזירות -- המשך}
נתבונן במקרה שנותר לנו להוכיח, בו $x = 0$. נתבונן בהגדרת הנגזרת בנקודה:
\begin{align*}
f^{\prime}(0)
&= \lim_{h \to 0} \frac{f(h) - f(0)}{h}\\
f(0) = 0 \text{\R{נימוק: }} &= \lim_{h \to 0} \frac{f(h)}{h}\\
&= \lim_{h \to 0} \frac{\sin^{2}{h}\sin\frac{1}{h}}{h}\\
&= \lim_{h \to 0} \sin{h}\sin\frac{1}{h} \cdot \lim_{h \to 0} \frac{\sin{h}}{h}\\
&= \lim_{h \to 0} \sin{h}\sin\frac{1}{h}\\
\text{\R{$\sin{h}$ אפסה, $\sin\frac{1}{h}$ חסומה}}&= 0
\end{align*}
לכן, $f$ גזירה ב0 ובפרט מתקיים $f^{\prime}(0) = 0$. לפיכך, נוכל להגדיר את הנגזרת של $f$ באופן הבא:
\[
f^{\prime}(x) =
\begin{cases}
2\sin{x}\cos{x}\sin\frac{1}{x} - \frac{\cos\frac{1}{x}\sin^{2}x}{x^{2}} & x \neq 0\br
0 & x = 0\\\\\\\\\\
\end{cases}
\]
ומצאנו את תחום הגזירות וערך הנגזרת של $f$ בכל $\mR$ כנדרש.
\br\qed
%%%% </Q3a> %%%%
\clearpage

%%%% <Q3b> %%%%
\section*{שאלה 3ב}
\subsection*{תחום הגדרה}
מכיוון ש$\abs{x}$ מוגדרת לכל $x \in \mR$ ו$\ln{x}$ מוגדרת לכל $x > 0$, הפונקציה $g(x) = \abs\ln{x}$ מוגדרת לכל $x > 0$.
\subsection*{תחום רציפות}
מכיוון ש$\ln{x}$ רציפה בכל נקודה בתחום הגדרתה )כלומר בכל $x > 0$( ו$\abs{x}$ רציפה בכל $x \in \mR$, $g$ רציפה בכל $x > 0$, כלומר בכל נקודה בתחום הגדרתה.
\subsection*{תחום גזירות}
%%%% </Q3b> %%%%
\clearpage

%%%% <Q4> %%%%
\section*{שאלה 4}
תהי $f$ פונקציה זוגית ב$\mR$, כלומר $f(-x) = f(x)$ לכל $x \in \mR$. נניח בנוסף כי $f$ גזירה ב0. נתבונן בנגזרת השמאלית של $f$ ב0:
\begin{align*}
f^{\prime}_{-}(0)
&= \lim_{h \to 0^{-}} \frac{f(h) - f(0)}{h}\\
&= \lim_{h \to 0^{-}} \frac{f(h)}{h} - \lim_{h \to 0^{-}} \frac{f(0)}{h}\\
\text{\R{זוגיות $f$}} &= \lim_{h \to 0^{-}} \frac{f(-h)}{h} + \lim_{h \to 0^{+}} \frac{f(0)}{h}\\
&= -\lim_{h \to 0^{+}} \frac{f(h)}{h} + \lim_{h \to 0^{+}} \frac{f(0)}{h}\\
\text{(1)\quad\quad\quad} &= -\left(\lim_{h \to 0^{+}} \frac{f(h) - f(0)}{h}\right)
\end{align*}
ובנגזרת הימנית ב0:
\[
f^{\prime}_{+}(0) = \lim_{h \to 0^{+}} \frac{f(h) - f(0)}{h}\\
\]
לפי )1( והנתון מתקיים
\[
f^{\prime}_{+}(0) = -f^{\prime}_{+}(0) \Longrightarrow
2f^{\prime}_{+}(0) = 0 \Longrightarrow
f^{\prime}_{+}(0) = 0
\]
לפי הנתון, $f^{\prime}_{-}(0) = f^{\prime}_{+}(0) = 0$, כלומר הנגזרת הימנית והנגזרת השמאלית שוות ובפרט שוות ל0 ולכן $f^{\prime}(0) = 0$ כנדרש.
\br\qed
%%%% </Q4> %%%%

\end{document}