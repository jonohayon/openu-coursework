\documentclass[11pt, oneside]{article}
\usepackage{geometry}
\geometry{a4paper}
\usepackage[parfill]{parskip}
\usepackage[nodisplayskipstretch]{setspace}
\usepackage{graphicx,titlesec,multicol}
\usepackage{amsmath,amssymb,mathtools,commath,enumitem}

% Hebrew Stuff
\usepackage[utf8x]{inputenc}
\usepackage[english,hebrew]{babel}
\usepackage{hebfont}

% Custom commands
\newcommand{\qed}{\R{$\blacksquare$}}
\newcommand{\br}{\\\\\\\\\\\\\\}
\newcommand{\opr}[1]{\xrightarrow[\text{#1}]{}}
\newcommand{\logr}[1]{\xRightarrow[\text{#1}]{}}
\newcommand{\bidiarrow}[1]{\underset{\text{#1}}{\leftrightarrow}}
\newcommand{\ueq}[1]{\underset{\text{#1}}{=}}
\newcommand{\mC}{\mathbb{C}}
\newcommand{\mR}{\mathbb{R}}
\newcommand{\mN}{\mathbb{N}}
\newcommand{\mZ}{\mathbb{Z}}
\newcommand{\mQ}{\mathbb{Q}}
\newcommand{\inv}[1]{#1^{-1}}

% Custom text commands (for Hebrew)
\newcommand{\q}[3]{\R{שאלה #3#2.#1}}
\newcommand{\m}[3]{\R{משפט #3#2.#1}}
\newcommand{\h}[3]{\R{הגדרה #3#2.#1}}
\newcommand{\ms}[3]{\R{מסקנה #3#2.#1}}

% Custom commands for this document
\renewcommand{\b}[1]{\textbf{#1}}
\newtheorem{axiom}{אקסיומה}
\newtheorem{attribute}{תכונה}
\newtheorem{theorem}{משפט}
\DeclarePairedDelimiter\ceil{\lceil}{\rceil}
\DeclarePairedDelimiter\floor{\lfloor}{\rfloor}
\DeclarePairedDelimiter\pq{\langle}{\rangle}
\newenvironment{definition}{\textbf{הגדרה:}}{\\}
\newenvironment{marking}{\textbf{סימון:} \begin{equation*}}{\end{equation*}}
\newenvironment{examples}{\textbf{דוגמאות:} \begin{itemize}}{\end{itemize}}

% Spacing
\titlespacing\subsection{0pt}{5pt}{3pt}

\title{חשבון אינפיניטסימלי 1 -- סיכום\thanks{נכתב לפי ההרצאות של אורי ברזנר באליאנס בסמסטר 8102א + תוספות מהספרים החדשים}}
\author{יונתן אוחיון}

\begin{document}
\maketitle

%%%%% <real-numbers> %%%%%
\begin{multicols}{2}[
\section*{המספרים הממשיים}
\subsection*{אקסיומות השדה}
השלשה $(\mR, \cdot, +)$ הינה \b{שדה} עם הפעולות $(\cdot, +)$ מעל $\mR$, כלומר מתקיימות התכונות הבאות:
]
\subsection*{פעולת הכפל )$\cdot$(}
\begin{itemize}[leftmargin=*]
\item קומוטטיביות )חילופיות(:\\
$\forall x, y \in \mR, x \cdot y = y \cdot x$

\item אסוציאטיביות )קיבוץ(:\\
$\forall x, y, z \in \mR, x \cdot (y \cdot z) = (x \cdot y) \cdot z$

\item קיום איבר ניטרלי )1(:\\
$\forall x \in \mR, x \cdot 1 = 1 \cdot x = x$

\item קיום איבר הופכי:\\
$\forall 0 \neq x \in \mR\ \exists y \in \mR, x \cdot y = 1$
\end{itemize}

\columnbreak

\subsection*{פעולת החיבור )$+$(}
\begin{itemize}[leftmargin=*]
\item קומוטטיביות )חילופיות(:\\
$\forall x, y \in \mR, x + y = y + x$

\item אסוציאטיביות )קיבוץ(:\\
$\forall x, y, z \in \mR, x + (y + z) = (x + y) + z$

\item קיום איבר ניטרלי )1(:\\
$\forall x \in \mR, x + 0 = 0 + x = x$

\item קיום איבר נגדי:\\
$\forall x \in \mR\ \exists y \in \mR, x + y = 0$
\end{itemize}
\end{multicols}
בנוסף, שתי הפעולות ביחד מקיימות את תכונת הדיסטריבוטיביות )פילוג(, המוגדרת כך:
\[
\forall x, y, z \in \mR, x \cdot (y + z) = x \cdot y + x \cdot z
\]
\clearpage

\section*{המספרים הממשיים -- המשך}
\subsection*{אקסיומות ותכונות}
\begin{axiom}[אקסיומת השלמות]
יהיו $A, B \subseteq \mR$ כך שלכל $a \in A$ ו$b \in B$ מתקיים $a \le b$. אזי קיים $c \in \mR$ כך ש$a \le c \le b$ לכל $a \in A, b \in B$.
\end{axiom}

\begin{attribute}[תכונת ארכימדס]
לכל $x \in \mR$ קיים $n \in \mN$ כך ש$x < n$ $\Longleftarrow$ לכל $x \in \mR$ קיים $n \in \mN$ כך ש$\frac{1}{n} < x$.
\end{attribute}

\begin{attribute}[צפיפות הרציונליים בממשיים]
לכל $x, y \in \mR$ )$x \neq y$( קיים $q \in \mQ$ כך ש$x < q < y$.
\end{attribute}

\subsection*{הערך השלם, הערך השברי ותכונותיהם}
\begin{itemize}
\item $\floor{x}$ -- הערך השלם ה``''תחתון``''
\item $\ceil{x}$ -- הערך השלם ה``''עליון``''
\item $\pq{x}$ -- החלק השברי, מוגדר כך: $\pq{x} = x - \floor{x}$
\item $\forall x \in \mR, \floor{x} \le x < \floor{x} + 1$
\item $\forall x \in \mR, x - 1 < \floor{x} \le x$
\end{itemize}

%%%%% </real-numbers> %%%%%

%%%%% <sequences-1> %%%%%
\section*{סדרות}
\begin{definition}
סדרה היא קבוצת מספרים )ממשיים(, המסודרת לפי $\mN$.
\end{definition}

\begin{marking}
a_{1}, a_{2}, a_{3}, \ldots \equiv (a_{n})^{\infty}_{n = 1} \equiv (a_{n}) \text{\R{במקרים בהם ידוע ההקשר, }}
\end{marking}

\begin{examples}
\item $a_{n} = 17\ (17, 17, 17, \ldots)$, סדרה קבועה
\item $a_{n} = n\ (1, 2, 3, \ldots)$
\item $a_{n} = \frac{1}{n}\ (1, \frac{1}{2}, \frac{1}{3}, \ldots)$, הסדרה ההרמונית, שואפת לאפס
\end{examples}
\clearpage

\section*{גבול של סדרה}
\begin{definition}
תהי $(a_{n})$ סדרה. אזי $L$ ייקרא גבול הסדרה אם לכל $\varepsilon > 0$ קיים $N \in \mN$ כך שלכל $n > N$ מתקיים $\abs{a_{n} - L} < \varepsilon$ .
\end{definition}
בכתיב כמתים, ההגדרה נראית כך:
\[
\forall \varepsilon > 0 \exists N \in \mN \forall n > N, \abs{a_{n} - L} < \varepsilon
\]
שלילת ההגדרה נראית כך:
\[
\exists \varepsilon > 0 \forall N \in \mN \exists n > N, \abs{a_{n} - L} \ge \varepsilon
\]
שלילה זו למעשה מראה ש$L$ נתון אינו גבול של הסדרה. אם ברצוננו להראות שהסדרה מתבדרת )כלומר אין לה גבול(, נצטרך להראות שהביטוי מתקיים \underline{לכל} $L \in \mR$.

\begin{marking}
\lim_{n \to \infty} a_{n} = L,\ a_{n} \opr{$n \to \infty$} L
\end{marking}

\begin{theorem}[משפט הסנדוויץ']
יהיו $(a_{n}), (b_{n}), (c_{n})$ שלוש סדרות כך ש$a_{n} \le b_{n} \le c_{n}$ כמעט לכל $n$ וגם $\lim_{n \to \infty} a_{n} = \lim_{n \to \infty} b_{n} = L$. אזי גם $b_{n}$ מתכנסת ו$\lim_{n \to \infty} b_{n} = L$.
\end{theorem}

\subsection*{סדרות חסומות, סדרות אפסות}
\begin{definition}
סדרה $(a_{n})$ תיקרא סדרה חסומה אם קיים $0 < M$ כך ש$\abs{a_{n}} < M$ .
\end{definition}
\begin{definition}
סדרה $(a_{n})$ תיקרא סדרה אפסה אם $a_{n} \opr{$n \to \infty$} 0$.
\end{definition}
\begin{theorem}
כל סדרה מתכנסת היא חסומה, כלומר אם סדרה אינה חסומה היא אינה מתכנסת. הערה: לא כל סדרה חסומה מתכנסת.
\end{theorem}
\begin{theorem}
תהי $(a_{n})$ סדרה חסומה, $(b_{n})$ סדרה אפסה. אזי $\lim_{n \to \infty} a_{n}b_{n} = 0$.
\end{theorem}

\subsection*{אריתמטיקה של גבולות}
יהיו $(a_{n}), (b_{n})$ סדרות כך שמתקיים $a_{n} \opr{$n \to \infty$} L,\ b_{n} \opr{$n \to \infty$} M$. אזי מתקיימים:
\begin{eqnarray}
\lim_{n \to \infty} (a_{n} + b_{n}) = L + M\\
\lim_{n \to \infty} (a_{n} \cdot b_{n}) = L \cdot M
\end{eqnarray}
%%%%% </sequences-1> %%%%%

\end{document}