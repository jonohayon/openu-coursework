\documentclass[../main.tex]{subfiles}

\section*{סדרות}
\begin{definition}
סדרה היא קבוצת מספרים )ממשיים(, המסודרת לפי $\mN$.
\end{definition}

\begin{marking}
a_{1}, a_{2}, a_{3}, \ldots \equiv (a_{n})^{\infty}_{n = 1} \equiv (a_{n}) \text{\R{במקרים בהם ידוע ההקשר, }}
\end{marking}

\begin{examples}
\item $a_{n} = 17\ (17, 17, 17, \ldots)$, סדרה קבועה
\item $a_{n} = n\ (1, 2, 3, \ldots)$
\item $a_{n} = \frac{1}{n}\ (1, \frac{1}{2}, \frac{1}{3}, \ldots)$, הסדרה ההרמונית, שואפת לאפס
\end{examples}
\clearpage

\section*{גבול של סדרה}
\begin{definition}
תהי $(a_{n})$ סדרה. אזי $L$ ייקרא גבול הסדרה אם לכל $\varepsilon > 0$ קיים $N \in \mN$ כך שלכל $n > N$ מתקיים $\abs{a_{n} - L} < \varepsilon$ .
\end{definition}
בכתיב כמתים, ההגדרה נראית כך:
\[
\forall \varepsilon > 0 \exists N \in \mN \forall n > N, \abs{a_{n} - L} < \varepsilon
\]
שלילת ההגדרה נראית כך:
\[
\exists \varepsilon > 0 \forall N \in \mN \exists n > N, \abs{a_{n} - L} \ge \varepsilon
\]
שלילה זו למעשה מראה ש$L$ נתון אינו גבול של הסדרה. אם ברצוננו להראות שהסדרה מתבדרת )כלומר אין לה גבול(, נצטרך להראות שהביטוי מתקיים \underline{לכל} $L \in \mR$.

\begin{marking}
\lim_{n \to \infty} a_{n} = L,\ a_{n} \opr{$n \to \infty$} L
\end{marking}

\begin{theorem}[משפט הסנדוויץ']
יהיו $(a_{n}), (b_{n}), (c_{n})$ שלוש סדרות כך ש$a_{n} \le b_{n} \le c_{n}$ כמעט לכל $n$ וגם $\lim_{n \to \infty} a_{n} = \lim_{n \to \infty} b_{n} = L$. אזי גם $b_{n}$ מתכנסת ו$\lim_{n \to \infty} b_{n} = L$.
\end{theorem}

\subsection*{סדרות חסומות, סדרות אפסות}
\begin{definition}
סדרה $(a_{n})$ תיקרא סדרה חסומה אם קיים $0 < M$ כך ש$\abs{a_{n}} < M$ .
\end{definition}
\begin{definition}
סדרה $(a_{n})$ תיקרא סדרה אפסה אם $a_{n} \opr{$n \to \infty$} 0$.
\end{definition}
\begin{theorem}
כל סדרה מתכנסת היא חסומה, כלומר אם סדרה אינה חסומה היא אינה מתכנסת. הערה: לא כל סדרה חסומה מתכנסת.
\end{theorem}
\begin{theorem}
תהי $(a_{n})$ סדרה חסומה, $(b_{n})$ סדרה אפסה. אזי $\lim_{n \to \infty} a_{n}b_{n} = 0$.
\end{theorem}

\subsection*{אריתמטיקה של גבולות}
יהיו $(a_{n}), (b_{n})$ סדרות כך שמתקיים $a_{n} \opr{$n \to \infty$} L,\ b_{n} \opr{$n \to \infty$} M$. אזי מתקיימים:
\begin{eqnarray}
\lim_{n \to \infty} (a_{n} + b_{n}) = L + M\\
\lim_{n \to \infty} (a_{n} \cdot b_{n}) = L \cdot M
\end{eqnarray}