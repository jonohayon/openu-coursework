\documentclass[12pt, oneside]{article}
\usepackage{geometry}
\geometry{letterpaper}
\usepackage{amssymb}
\usepackage{amsmath}
\usepackage{hyperref}
\usepackage{cancel}

% Initializing graphicx
\usepackage{graphicx}
\graphicspath{{assets/}}

% Custom Commands
\newcommand{\img}[1] {
	\includegraphics[scale=0.33]{#1}
}

\title{Maman 14}
\author{Jonathan Ohayon}
% No date cuz I'm living on the 3dge

\begin{document}
\maketitle

\section{Question 1}
\setcounter{subsection}{-1}
\subsection{Declarations}
Let $A$ and $B$ be the following sets:
\begin{equation*}
A = \{a, b\}, B = \{1, 2, 3\}
\end{equation*}

\subsection{A}
\begin{itemize}
\item $f(\{a, b\}) = \{1, 2\}$ - Injective
\item $f(\{a, b\}) = \{1, 3\}$ - Injective
\item $f(\{a, b\}) = \{2, 1\}$ - Injective
\item $f(\{a, b\}) = \{2, 3\}$ - Injective
\item $f(\{a, b\}) = \{3, 1\}$ - Injective
\item $f(\{a, b\}) = \{3, 2\}$ - Injective
\item $f(\{a, b\}) = \{1, 1\}$ - Not injective
\item $f(\{a, b\}) = \{2, 2\}$ - Not injective
\item $f(\{a, b\}) = \{3, 3\}$ - Not injective
\end{itemize}
\clearpage

\subsection{B}
\begin{itemize}
\item $f(\{1, 2, 3\}) = \{a, a, a\}$ - Not surjective
\item $f(\{1, 2, 3\}) = \{b, b, b\}$ - Not surjective
\item $f(\{1, 2, 3\}) = \{a, a, b\}$ - Surjective
\item $f(\{1, 2, 3\}) = \{a, b, b\}$ - Surjective
\item $f(\{1, 2, 3\}) = \{a, b, a\}$ - Surjective
\item $f(\{1, 2, 3\}) = \{b, b, a\}$ - Surjective
\item $f(\{1, 2, 3\}) = \{b, a, b\}$ - Surjective
\item $f(\{1, 2, 3\}) = \{b, a, a\}$ - Surjective
\end{itemize}

\subsection{C}
There isn't any $g \circ f$ that's inverse because in order for a function to be inverse it needs to be subjective and there isn't an $f$ that's subjective.
\clearpage

\section{Question 2}
\setcounter{subsection}{-1}
\subsection{Declarations}
Let $f: A \rightarrow B$ and $C, D \subseteq A$ so that $C \neq D$.

\subsection{A}
We can show a counter example where $f(C) \neq f(D)$ and $C \neq D$ and $f$ isn't injective.
\begin{eqnarray*}
& A = \{1, 2, 3, 4\}\\
& B = \{5, 6, 7\}\\
& f = 
	\begin{cases}
		1 & 5\\
		2 & 6\\
		3 & 6\\
		4 & 7
	\end{cases}\\
& C = \{1, 2\}\\
& D = \{3, 4\}
\end{eqnarray*}

\subsection{B}
Let $f$ be an injective function.
We can show that $f(C) \neq f(D)$ by assuming falsely that $f(C) = f(D)$ and showing that because of the declaration that $f$ is an injective function, $C = D$ and then we'll have a contradiction to the declaration that says that $C \neq D$. Therefore, if $f$ is an injective function then $f(C) \neq f(D)$.

\subsection{C}
Example:
\begin{eqnarray*}
& A = \{4\}, B = \{5\}, C = \{4\}, D = \emptyset\\
& f = 
	\begin{cases}
	1 & 2
	\end{cases}\\
& \framebox{$f(C) = f(C) \cup f(D) = \{1\} \cup \emptyset = \{1\}$}
\end{eqnarray*}

\clearpage

\section{Question 3}
\setcounter{subsection}{-1}
\subsection{Declarations}
Let $f$ and $g$ be the following functions:
\begin{eqnarray*}
& f, g: \mathbb{N} \rightarrow \mathbb{N}\\
& f(n) = 
  \begin{cases}
  	n - 1 & \text{Even numbers}\\
		\dfrac{n + 1}{2} & \text{Odd numbers}
  \end{cases}\\
& g(n) = 2n - 1
\end{eqnarray*}

\subsection{A}
In order to show that $f$ is not an injective function, we would need to show $x, y \in \mathbb{N}$ that are $f(x) = f(y)$. If we'll substitute 1 and 2, we can see that $f(1) = 1$ and $f(2) = 1$. Therefore, $f$ is not an injective function.

\subsection{B}
Let $a, b \in \mathbb{N}$.
We can show that $g$ is an injective function by assuming that $g(a) = g(b)$ and showing that $a = b$.
\begin{equation*}
g(a) = g(b) \Rightarrow
2a - \cancel{1} = 2b - \cancel{1} \Rightarrow
\cancel{2}a = \cancel{2}b \Rightarrow
\framebox{$a = b$}
\end{equation*}

\subsection{C}
We can show that $f$ is a surjective function by showing that for every $y \in \mathbb{N}$, there's $x \in \mathbb{N}$ of which $f(x) = y$. In this case though, we will need to check that twice - one time for even numbers, and one time for odd numbers.\\
Even numbers:
\begin{equation*}
x, y \in \mathbb{N} \Rightarrow
y = x - 1 \Rightarrow
x = y + 1 \Rightarrow
y = y + \cancel{1 - 1} \Rightarrow
\framebox{$y = y$}
\end{equation*}
Odd numbers:
\begin{equation*}
x, y \in \mathbb{N} \Rightarrow
y = \dfrac{x + 1}{2} \Rightarrow
2y = x + 1 \Rightarrow
x = 2y - 1 \Rightarrow
y = \dfrac{\cancel{2}y \cancel{- 1 + 1}}{\cancel{2}} \Rightarrow
\framebox{$y = y$}
\end{equation*}
Therefore, $f$ is surjective.
\clearpage

\subsection{D}
We can use a counter example to show that $g$ isn't surjective.
\begin{equation*}
x, y \in \mathbb{N} \Rightarrow
f(x) = y = 2 \Rightarrow
2 = 2x - 1 \Rightarrow
2x = 3 \Rightarrow
x = 1.5 \Rightarrow
\framebox{$x \not\in \mathbb{N}$}
\end{equation*}
Therefore, $g$ isn't surjective.

\subsection{E}
We can show that $f \circ g = id_\mathbb{N}$ in the following proof (we need only one this time because 2 times something minus 1 is always an odd number):\\
\begin{eqnarray*}
& x \in \mathbb{N}\\
& (f \circ g)(x) = f(g(x))\\
& f(g(x)) = \dfrac{\cancel{2}x \cancel{- 1 + 1}}{\cancel{2}}\\
& \framebox{$f(g(x)) = x$}
\end{eqnarray*}
Therefore, $f \circ g = id_\mathbb{N}$.

\subsection{F}
We can show that $g \circ f \neq id_\mathbb{N}$ using the following counter example:
\begin{eqnarray*}
& 6 \in \mathbb{N}\\
& (g \circ f)(x) = g(f(x))\\
& g(f(x)) = 2(6 - 1) - 1 \neq 2
\end{eqnarray*}
Therefore, $g \circ f \neq id_\mathbb{N}$.
\clearpage

\section{Question 4}
\subsection{A}
Let $h, g, f$ be function from $A$ to $A$, $f$ is subjective and $g \circ f = h \circ f$.
We can show that $g = h$ with the following proof:
\begin{eqnarray*}
& x, y \in A \\
& f(x) = y \\
& (g \circ f)(x) = g(f(x)) = g(y)\\
& (h \circ f)(x) = h(f(x)) = h(y)\\
& g \circ f = h \circ f\\
& g(y) = h(y)\\
& \framebox{$g = y$}
\end{eqnarray*}
Therefore, $g = h$.

\subsection{B}
Example:
\begin{eqnarray*}
& f(n) = 2n - 1\\
& g(n) = 
	\begin{cases}
	1& \text{Odd numbers}\\
	3& \text{Even numbers}\\
	\end{cases}\\
& h(n) = 
	\begin{cases}
	1& \text{Odd numbers}\\
	2& \text{Even numbers}\\
	\end{cases}\\
& g \circ f = 1\\
& h \circ f = 1\\
& \framebox{$g \circ f = h \circ f$ and $g \neq h$}
\end{eqnarray*}

\end{document}