\documentclass[12pt, oneside]{article}
\usepackage{geometry}
\geometry{letterpaper}
\usepackage{amssymb}
\usepackage{amsmath}
\usepackage{hyperref}
\usepackage{cancel}

% Initializing graphicx
\usepackage{graphicx}
\graphicspath{{assets/}}

% Custom Commands
\newcommand{\img}[1] {
	\includegraphics[scale=0.33]{#1}
}

\title{Maman 13}
\author{Jonathan Ohayon}
% No date cuz I'm living on the 3dge

\begin{document}
\maketitle

\section{Question 1}
\setcounter{subsection}{-1}
\subsection{Declarations}
Let \emph{A} be a set with a binary operation \emph{*} which implements closure and reduction rules over it, and $e \in A$ so that for every $x \in A$, $x * e = x$.

\subsection{A}
This following table of an example of $(A, *)$ shows that it is, in fact, possible that $e$ won't be a neutral element in the declared $A$.
\begin{center}
\begin{tabular}{c | c | c | c | c}
* & a & b & c & e\\
\hline
a & b & c & e & a \\
\hline
b & a & e & c & b \\
\hline
c & e & a & b & c \\
\hline
e & c & b & a & e \\
\end{tabular}
\end{center}

\subsection{B}
As declared in the question, $x * e = x$, so in order to show that $e$ is a neutral element in $A$, we need to show that $x * e = e * x = x$. Since we're told that in this section of the question that $(A, *)$ has the associative property, we can show that $e * x = x$ using the following explanation:
\begin{equation*}
e, x, y \in A \Rightarrow
x * (e * b) = (x * e) * b \Rightarrow
\cancel{x} * (e * b) = \cancel{x} * b \Rightarrow
e * b = b \Rightarrow
\framebox{$b * e = e * b = b$}
\end{equation*}

\clearpage

\section{Question 2}
\setcounter{subsection}{-1}
\subsection{Declarations}
Let $*$ be a binary over group $G$:
\begin{eqnarray*}
& G = \{e, a, b, c\}\\
& e \neq a \neq b \neq c\\
& (a * a) * a = b\\
& e\;\text{is the neutral element in}\;G
\end{eqnarray*}

\subsection{A}
We can proof every one of these statements by using a contradiction to the $(a * a) * a = b$ statement which was defined in this question's declarations.

\begin{center}
\begin{tabular}{c | c}
a * a = e & a * a = b\\
e * a = b & b * a = b\\
\framebox{$e * a = e \neq b$} & \framebox{$e \neq a$, therefore $b * a \neq b$}\\
\hline
a * a = a & \\
a * a = b & \\
\framebox{$a * a = a \neq b$}
\end{tabular}
\end{center}

\subsection{B}
According to section $A$, $a * a \neq a, b, e$. According to the group's closure property, $a * a = a, b, c, e$. Therefore, $a * a = c$. Also, according to the question's declarations, $(a * a) * a = b$. Therefore, $c * a = (a * a) * a = b$.

\subsection{C}
According to sections $A$ and $B$, $e * e = a, c * a = b$ and $a * a = c$. Therefore, according to the reduction laws $b * a = e$. Therefore $b * a \neq a, b * a \neq b$ and $b * a \neq c$.
\clearpage

\subsection{D}
The operation table:
\begin{center}
\begin{tabular}{c | c | c | c | c}
* & a & b & c & e\\
\hline
a & c & e & b & a\\
\hline
b & e & c & a & b\\
\hline
c & b & a & e & c\\
\hline
e & a & b & c & e\\
\end{tabular}
\end{center}
$e$'s line has been completed as it's the neutral element in $(G, *)$ and $a$'s line has been completed according to the results of section C. All of the other squares has been completed according to the reduction laws ("the sudoko property").

\section{Question 3}
\subsection{A}
\setcounter{subsubsection}{-1}
\subsubsection{Declarations}
Let $(A, *)$ be the following set and binary operation:
\begin{eqnarray*}
& A = \{2n\;|\;n \in \mathbb{Z}\}\\
& \text{For every}\;a,b \in A, a * b = a + b - ab
\end{eqnarray*}

\subsubsection{Closure}
In order to show that $(A, *)$ operation implements closure, we need to show that it’s results belong to the A set. Since multiplication, addition and subtraction of even numbers is always even, everything we will assign to $a$ or $b$ will always return a value that belongs to $\mathbb{Z}$.

\subsubsection{Associative Property}
In order to show that the * operation has the associative property over A, we need to show that the results of $a * (b * c)$ and $(a * b) * c$ are equal. We can show that using the following explanation:
\begin{eqnarray*}
& a, b, c \in A\\
& a * (b * c) = a * (b + c - bc) = a + (b + c - bc) - a(b + c - bc)\\
& (a * b) * c = (a + b - ab) * c = (a + b - ab) + c - (a + b - ab)c\\
& a + (b + c - bc) - a(b + c - bc) = (a + b - ab) + c - (a + b - ab)c\\
& \cancel{a} + \cancel{b} + \cancel{c} - \cancel{bc} - \cancel{ab} - \cancel{ac} + \cancel{abc} = \cancel{a} + \cancel{b} - \cancel{ab} + \cancel{c} - \cancel{ac} - \cancel{bc} + \cancel{abc}\\
& \framebox{$0 = 0$}
\end{eqnarray*}

\subsubsection{Neutral Element}
In order to show that there's a neutral element in $(A, *)$, we need to show an element of $A$ that, when being used under the * operation, returns the other element of $A$ (aka $x * e = e * x = x$ for every $x \in A$). In this context, 0 is the neutral element in $A$, and we can show that using the following explanation:
\begin{equation*}
a, 0 \in A \Rightarrow
a * 0 = a + 0 - 0 = a \Rightarrow
0 * a = 0 + a - 0 = a \Rightarrow
\framebox{$a * 0 = 0 * a = a$}
\end{equation*}

\subsubsection{Inverse Element}
In order to show that there's an inverse element in $(A, *)$, we need to show an element of $A$ that, when being used under the * operation, returns the neutral element (aka $a * b = e$). We can show that there \emph{isn't} an inverse element in $A$ using the following counter-example:
\begin{eqnarray*}
& 8, b, 0 \in A\\
& 8 * b = 0 \Rightarrow\\
& 8 + b - 8b = 0 \Rightarrow\\
& 8 - 7b = 0 \Rightarrow\\
& 7b = 8\;/\;:7 \Rightarrow\\
& \framebox{$b = \dfrac{8}{7} \not\in A$}
\end{eqnarray*}

\subsection{B}
\setcounter{subsubsection}{-1}
\subsubsection{Declarations}
Let $(A, *)$ be the following set and binary operation:
\begin{eqnarray*}
& A = \mathbb{Q}\backslash\{1\}\\
& \text{For every}\;a, b \in A, a * b = a + b - ab
\end{eqnarray*}

\subsubsection{Closure}
In order to show that $(A, *)$ operation implements closure, we need to show that it’s results belong to the A set. In order for $(A, *)$ to \emph{not} show closure, we need to find $a, b \in \mathbb{Q}\backslash\{1\}$ that when they are being used under the $*$ operation, they will result in the number 1 (as it's the only rational number that is excluded from $A$). Since there aren't any two numbers that their sum is bigger than their multiplication without the number 1, we can show that there is, in fact, closure in $(A, *)$.

\subsubsection{Associative Property}
\framebox{Same explanation as in 3.1.2}

\subsubsection{Neutral Element}
\framebox{Same explanation as in 3.1.3}

\subsubsection{Inverse Element}
In order to show that there's an inverse element in $(A, *)$, we need to show an element of $A$ that, when being used under the * operation, returns the neutral element (aka $a * b = e$). We can show that there is an inverse element in $A$ using a proof by contradiction (should find $a, b, 0 \in A$ that $a * b \neq 0$).:
\begin{eqnarray*}
& 4, 1\frac{1}{3}, 0 \in A\\
& 4 * 1\frac{1}{3} = 0\\
& 4 + 1\frac{1}{3} - 5\frac{1}{3} = 0\\
\end{eqnarray*}
Therefore, if we will assign the values of $(4, 1\frac{1}{3})$ to $(a, b)$ then there will be a contradiction. Therefore, there is an inverse element in $(A, *)$.

\section{Question 4}
\setcounter{subsection}{-1}
\subsection{Declarations}
Let $(G, *)$ be a group and a binary operation.

\subsection{A}
\begin{eqnarray*}
& a, a^{-1} \in G\\
& (a * a^{-1}) * b = a * (a^{-1} * b)\;\text{- because of the associative property} \Rightarrow\\
& a * (a^{-1} * b) = b\\
& x = (a^{-1} * b) \in G\;\text{- because of closure}
\end{eqnarray*}
Therefore, $a * x = b$: for every $a, b \in G$ there's an element $x \in G$ so that $x = a^{-1} * b$, implying that $a * x = b$.

\subsection{B}
According to section A, we can say that $(a * b) * x = b * a$ (assigning $a * b$ and $b * a$). Therefore, because of the associative property, $a * (b * x) = b * a$ and according to the declarations then $b = b * x$. Therefore $x = e$ (the neutral element). Therefore $a * b = b * a$ because $(a * b) * x = b * a$ and the set implements commutativity.

\end{document}